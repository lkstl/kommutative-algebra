\section{Artinsche Ringe}

\subsection{Elementare Eigenschaften Artinscher Ringe}

\begin{frame}{Primideale in artinschen Ringen}
	Wir erinnern an folgende Möglichkeiten, einen Ring \(A\) als artinsch zu
	charakterisieren:
	\begin{enumerate}[<+->]
	\item
		Jede nicht leere Menge von Idealen von \(A\) besitzt ein minimales Element.
	\item
		Jede absteigende Kette von Idealen in \(A\) ist stationär.
	\end{enumerate}
	\mode<article>{Die anscheinende Symmetrie mit noetherschen Ringen ist
	irreführend. Und zwar werden wir sehen, daß jeder artinsche Ring
	notwendigerweise ein noetherscher Ring und zudem von besonders einfacher
	Gestalt ist.}
	\begin{proposition}<+->
		Sei \(A\) ein artinscher kommutativer Ring. Dann ist jedes Primideal
		in \(A\) maximal.
	\end{proposition}
	\begin{proof}<+->
		Ist \(\ideal p\) ein Primideal in \(A\), so ist
		\(B \coloneqq A/\ideal p\) ein artinscher Integritätsbereich.
		Sei \(x \in B \setminus \{0\}\). Dann ist \((x^n) = (x^{n + 1})\) für
		ein \(n \gg 0\), da \(B\) artinsch ist, also \(x^n = x^{n + 1} y\)
		für \(y \in B\). Kürzen mit \(x^n\) liefert \(1 = xy\), also ist
		\(x \in B^\units\). Damit ist \(B\) ein Körper und \(\ideal p\) ein
		maximales Ideal.
	\end{proof}
\end{frame}

\begin{frame}{Maximale Ideale in artinschen Ringen}
	\begin{corollary}<+->
		In einem artinschen kommutativen Ring ist das Nilradikal gleich dem
		Jacobsonschen Radikal.
		\qed
	\end{corollary}
	\begin{proposition}<+->
		Ein artinscher kommutativer Ringe hat nur endlich viele maximale Ideale.
	\end{proposition}
	\begin{proof}<+->
		\begin{enumerate}[<+->]
		\item<.->
			Die Menge aller endlichen Schnitte maximaler Ideale hat ein minimales
			Element, etwa \(\ideal m_1 \cap \dotsb \cap \ideal m_n\), wobei die
			\(\ideal m_i\) maximale Ideale sind.
		\item
			Für jedes weitere maximale Ideal
			\(\ideal m\) haben wir damit \(\ideal m \supset
			\ideal m_1 \cap \dotsb \cap \ideal m_n\).
		\item
			Da \(\ideal m\) ein Primideal
			ist, folgt \(\ideal m \supset \ideal m_i\) für ein \(i\), aufgrund
			der Maximalität also \(\ideal m = \ideal m_i\).
			\qedhere
		\end{enumerate}
	\end{proof}
\end{frame}

\begin{frame}{Nilpotentes Nilradikal in artinschen Ringen}
	\begin{proposition}<+->
		In einem kommutativen artinschen Ring ist das Nilradikal \(\ideal n\)
		nilpotent.
	\end{proposition}
	\begin{proof}<+->
		\begin{enumerate}[<+->]
		\item<.->
			Da der Ring artinsch ist, existiert ein \(k \gg 0\) mit
			\(\ideal a \coloneqq \ideal n^k = \ideal n^{k + 1} = \dotsb\).
			Angenommen, \(\ideal a \neq (0)\).
		\item
			Die Menge der Ideale \(\ideal b\) mit \(\ideal a \ideal b \neq (0)\)
			ist nicht leer, da insbesondere \(\ideal a\) ein solches Ideal ist,
			besitzt also ein minimales Element \(\ideal c\).
		\item
			Es existiert ein \(x \in \ideal c\) mit \(x \ideal a \neq (0)\).
			Weiter ist \((x) \subset \ideal c\), also \((x) = \ideal c\)
			aufgrund der Minimalität von \(\ideal c\).
		\item
			Weiter ist \((x \ideal a) \ideal a = x \ideal a^2 = x \ideal a
			\neq (0)\) und sowieso \(x \ideal a \subset (x)\), also
			\(x \ideal a = (x)\) aufgrund der Minimalität von \((x)\).
		\item
			Damit ist \(x = xy\) für ein \(y \in \ideal a\), also
			\(x = xy = xy^2 = \dotsb = xy^n\). Da \(y\) insbesondere nilpotent
			ist, ist folglich \(x = 0\), Widerspruch.
			\qedhere
		\end{enumerate}
	\end{proof}
\end{frame}

\subsection{Der Struktursatz für artinsche kommutative Ringe}

\begin{frame}{Dimension kommutativer Ringe}
	Sei \(A\) ein kommutativer Ring.
	\begin{definition}<+->
		Eine \emph{Primidealkette in \(A\)} ist eine streng aufsteigende
		Kette \(\ideal p_0 \subsetneq \ideal p_1 \subsetneq \dotsb
		\subsetneq \ideal p_n\) von Primidealen in \(A\). Es heißt \(n\)
		die \emph{Länge} der Kette.
	\end{definition}
	\begin{definition}<+->
		Die \emph{Dimension \(\dim A\) von \(A\)} ist das Supremum der Längen
		aller Primidealketten in \(A\).
	\end{definition}
	\begin{example}<+->
		Es ist \(\dim A \ge 0\) genau dann, wenn \(A\) nicht der Nullring ist.
	\end{example}
\end{frame}

\begin{frame}{Beispiele für Dimensionen}
	\begin{example}<+->
		Die Dimension eines Körpers ist \(0\).
	\end{example}
	\begin{example}<+->
		Für den Ring der ganzen Zahlen gilt \(\dim \set Z = 1\).
	\end{example}
\end{frame}

\begin{frame}{Das Nullideal in artinschen kommutativen Ringen}
	\begin{lemma}<+->
		\label{lem:zero_is_prod_of_max}
		Seien \(\ideal m_1, \dotsc, \ideal m_n\) die maximalen Ideale
		eines artinschen kommutativen Ringes \(A\). Dann ist
		\(\prod\limits_{i = 1}^n \ideal m_i^k = (0)\) für \(k \gg 0\).
	\end{lemma}
	\begin{proof}<+->
		\begin{enumerate}[<+->]
		\item<.->
			Da jedes Primideal in \(A\) maximal ist, gilt \(\ideal n = \ideal m_1
			\cap \dotsb \cap \ideal m_n\) für das Nilradikal \(\ideal n\) von
			\(A\).
		\item
			Da das Nilradikal in \(A\) nilpotent ist, gilt
			\(\prod\limits_{i = 1}^n \ideal m_i^k
			\subset (\bigcap\limits_{i = 1}^n \ideal m_i)^k = \ideal n^k = (0)\)
			für \(k \gg 0\).
			\qedhere
		\end{enumerate}
	\end{proof}
\end{frame}

\begin{frame}{Dimensionscharakterisierung artinscher kommutativer Ringe}
	\begin{theorem}<+->
		Ein kommutativer Ring \(A \neq 0\) ist genau dann artinsch, wenn er
		noethersch ist und \(\dim A = 0\) gilt.
	\end{theorem}
\end{frame}

\begin{frame}{Beweis der Dimensionscharakterisierung}
	\begin{proof}<+->
		\begin{enumerate}[<+->]
		\item<.->
			Ist \(A\) artinsch, so sind alle Primideale maximal, also \(\dim
			A = 0\). Weiter ist nach dem letzten Lemma das Nullideal Produkt
			von maximalen Idealen. Damit sind äquivalent, daß \(A\) artinsch und
			noethersch ist.
		\item
			Sei \(A\) noethersch mit \(\dim A = 0\). Da das Nullideal eine
			Primärzerlegung besitzt, gibt es nur endlich viele minimale
			Primideale. Wegen \(\dim A = 0\) sind alle diese maximal.
		\item
			Das Nilradikal ist also Schnitt endlich vieler maximaler Ideale.
			Da das Nilradikal nilpotent in noetherschem \(A\) ist, ist
			das Nullideal Produkt endlich vieler
			maximaler Ideale. Damit sind äquivalent, daß \(A\) artinsch und
			noethersch ist.
			\qedhere
		\end{enumerate}
	\end{proof}
\end{frame}

\begin{frame}{Ein kommutativer Ring mit nur einem Primideal}
	\begin{example}<+->
		Ein kommutativer Ring mit genau einem Primideal muß nicht
		notwendigerweise noethersch (und damit auch nicht artinsch) sein: Sei
		etwa \(A = K[x_1, x_2, \dotsc]\) der Polynomring in unendlich vielen
		Variablen über einem Körper \(K\).
		\\
		Sei \(\ideal a \coloneqq (x_1, x_2^2, \dotsc)\). Dann besitzt
		\(B \coloneqq A/\ideal a\) genau ein Primideal, nämlich das Bild von
		\((x_1, x_2, \dotsc)\).
		\\
		Damit ist \(B\) ein lokaler Ring der Dimension \(0\). Allerdings ist
		\(B\) nicht noethersch, denn z.B.~sein Primideal ist nicht endlich
		erzeugt.
	\end{example}
\end{frame}

\begin{frame}{Artinsche lokale Ringe}
	Ist \((A, \ideal m)\) ein artinscher lokaler Ring, so ist \(\ideal m\)
	das einzige Primideal von \(A\) und damit gleich dem Nilradikal von \(A\).
	\\
	Insbesondere ist jedes Element aus \(\ideal m\) nilpotent, und \(\ideal m\)
	selbst ist nilpotent.
	\\
	Jedes Element aus \(A\) ist damit entweder eine Einheit oder nilpotent.
	\begin{example}<+->
		Sei \(p\) ein Primzahl und \(n \ge 1\). Dann ist \(\set Z/(p^n)\) ein
		artinscher lokaler Ring mit maximalem Ideal \(\set Z/(p^n) (p)\).
	\end{example}
\end{frame}

\begin{frame}{Nilpotenz des maximalen Ideals charakterisiert lokale artinsche Ringe}
	\begin{proposition}<+->
		Sei \((A, \ideal m)\) ein noetherscher lokaler Ring. Dann tritt genau
		einer der beiden folgenden Fälle ein:
		\begin{enumerate}[<+->]
		\item<.->
			Für alle \(n \in \set N_0\) ist \(\ideal m^n \supsetneq
			\ideal m^{n + 1}\).
		\item
			Es ist \(\ideal m^n = (0)\) für \(n \gg 0\), in welchem Falle
			\(A\) artinsch ist.
		\end{enumerate}
	\end{proposition}
	\begin{proof}<+->
		Ist \(\ideal m^n = \ideal m^{n + 1}\), so folgt mit dem
		Nakayamaschen Lemma, daß \(\ideal m^n = 0\).
		\\
		Insbesondere ist dann
		\(\ideal m^n \subset \ideal p\) für ein Primideal \(\ideal p\). Nach
		Wurzelziehen erhalten wir \(\ideal m = \ideal p\), also besitzt \(A\)
		dann nur ein Primideal, ist also artinsch.
	\end{proof}
\end{frame}

\begin{frame}{Der Struktursatz für artinsche kommutative Ringe}
	\begin{theorem}[Struktursatz für artinsche kommutative Ringe]<+->
		\label{thm:artin_structure}
		Ein artinscher kommutativer Ring ist (eindeutig bis auf Isomorphie
		der Faktoren) ein direktes Produkt artinscher lokaler
		Ringe.
	\end{theorem}
	\begin{proof}[Beweis der Existenz]<+->
		\renewcommand{\qedsymbol}{}
		\begin{enumerate}[<+->]
		\item<.->
			Seien \(\ideal m_1, \dotsb, \ideal m_n\) die verschiedenen maximalen
			Ideale eines artinschen Ringes \(A\). Nach dem letzten Hilfssatz
			ist \(\prod\limits_{i = 1}^n \ideal m_i^k = (0)\) für \(k \gg 0\).
			Da die \(\ideal m_i = \sqrt{\ideal m_i^k}\) paarweise koprim sind,
			sind auch die \(\ideal m_i^k\) paarweise koprim.
		\item
			Damit gilt \(\bigcap\limits_i \ideal m_i^k
			= \prod\limits_i \ideal m_i^k = (0)\). Damit ist insbesondere
			der kanonische Homomorphismus \(A \to \prod\limits A/\ideal m_i^k\)
			ein Isomorphismus. Da die \(A/\ideal m_i^k\) jeweils lokale
			artinsche Ringe sind, ist \(A\) direktes Produkt artinscher lokaler
			Ringe.
			\qedhere
		\end{enumerate}
	\end{proof}
\end{frame}

\begin{frame}{Eindeutigkeit im Struktursatz}
	\begin{proof}[Beweis der Eindeutigkeit]<+->
		\begin{enumerate}[<+->]
		\item<.->
			Sei umgekehrt \(A \cong \prod\limits_{i = 1}^n A_i\) für einen
			kommutativen Ring \(A\), wobei die
			\(A_i\) lokale artinsche kommutative Ringe sind. Seien
			\(\pi_i\colon A \to A_i\) die Projektionen und \(\ideal a_i
			\coloneqq \ker\pi_i\). Die \(\ideal a_i\) sind paarweise koprim
			mit \(\bigcap\limits_i \ideal a_i = (0)\).
		\item
			Sei \(\ideal q_i\) das einzige Primideal in \(A_i\), und sei
			\(\ideal p_i \coloneqq A \cap \ideal q_i\). Dann ist \(\ideal p_i\)
			ein maximales Primideal in \(A\).
		\item
			Da \((0)\) in \(A_i\) ein \(\ideal q_i\)-primäres Ideal ist, ist \(\ideal a_i\) in \(A\)
			ein \(\ideal p_i\)-primäres Ideal. Damit ist \(\bigcap\limits_i
			\ideal a_i = (0)\) eine Primärzerlegung.
		\item
			Da die \(\ideal a_i\) paarweise koprim sind, sind die \(\ideal p_i\)
			ebenso paarweise koprim, also die zu \((0)\)
			assoziierten isolierten Primideale.
		\item
			Damit sind alle Primkomponenten von \(A\) isoliert, also nach
			dem zweiten Eindeutigkeitssatz durch \(A\) eindeutig bestimmt.
			Also sind die \(A_i \cong A/\ideal a_i\) durch \(A\) eindeutig
			bestimmt.
			\qedhere
		\end{enumerate}
	\end{proof}
\end{frame}

\subsection{Artinsche lokale Ringe}

\begin{frame}{Der Zariskische Kotangentialraum}
	Sei \((A, \ideal m, F)\) ein lokaler Ring. Dann ist
	die spezielle Faser \(\ideal m/\ideal m^2\) von \(\ideal m\) ein \(F\)-Vektorraum.
	\begin{definition}<+->
		Der \(F\)-Vektorraum \(\ideal m/\ideal m^2\) ist der
		\emph{Zariskische Kotangentialraum von \(A\)}.
	\end{definition}
	\begin{visibleenv}<+->
		Ist \(\ideal m\) endlich erzeugt (z.B.~weil \(A\) noethersch ist),
		ist auch \(\ideal m/\ideal m^2\) als \(F\)-Vektorraum endlich erzeugt.
		Es folgt, daß in diesem Falle \(\dim_F \ideal m/\ideal m^2 < \infty\).
	\end{visibleenv}
	
	\begin{visibleenv}<+->
		Wir erinnern an die Tatsache, daß sich jede Basis des Zariskischen
		Kotangentialraums \(\ideal m/\ideal m^2\) zu einem Erzeugendensystem von \(\ideal m\)
		hochheben läßt.
	\end{visibleenv}
\end{frame}

\begin{frame}{Artinsche lokale Ringe, deren maximales Ideal ein Hauptideal ist}
	\begin{lemma}<+->
		Sei \((A, \ideal m)\) ein artinscher lokaler Ring, so daß \(\ideal m\) ein
		Hauptideal ist. Dann ist jedes Ideal \(\ideal a\) von \(A\) ein Hauptideal.
	\end{lemma}
	\begin{proof}<+->
		\begin{enumerate}[<+->]
		\item<.->
			Sei \(\ideal m = (x)\) für ein \(x \in A\).
			Wir können \(\ideal a \neq (0)\) annehmen. Da \(\ideal m\) auch das Nilradikal
			ist, ist \(\ideal m\) nilpotent. Damit existiert ein \(r \in \set N_0\) mit
			\(\ideal a \subset \ideal m^r\), aber \(\ideal a \subsetneq \ideal m^{r + 1}\).
		\item
			Folglich existiert ein \(a \in A\) mit \(y \coloneqq ax^r \in \ideal a\), aber
			\(y \notin (x^{r + 1})\). Es folgt \(a \notin (x)\), also ist \(a\) eine Einheit in
			\(A\).
		\item
			Damit ist \(x^r \in \ideal a\), also \(\ideal m^r = (x^r) \subset \ideal a\),
			also \(\ideal a = \ideal m^r = (x^r)\). Also ist \(\ideal a\) ein Hauptideal.
			\qedhere
		\end{enumerate}
	\end{proof}
\end{frame}

\begin{frame}{Artinsche lokale Ringe}
	\begin{proposition}<+->
		Für einen artinschen lokalen Ring \((A, \ideal m, F)\) sind folgende Aussagen äquivalent:
		\begin{enumerate}[<+->]
		\item<.->
			Jedes Ideal in \(A\) ist ein Hauptideal.
		\item
			Das maximale Ideal \(\ideal m\) ist ein Hauptideal.
		\item
			Es ist \(\dim_F \ideal m/\ideal m^2 \leq 1\).
		\end{enumerate}
	\end{proposition}
	\begin{proof}<+->
		\begin{enumerate}[<+->]
		\item<.->
			Daß aus der ersten die zweite und aus der zweite die dritte Aussage folgt ist offensichtlich.
		\item
			Ist \(\dim_F \ideal m/\ideal m^2 = 0\), also \(\ideal m = \ideal m^2\), so folgt \(\ideal m = 0\)
			nach dem Nakayamaschen Lemma. Damit ist \(A\) ein Körper und nichts weiter zu zeigen.
		\item
			Ist \(\dim_F \ideal m/\ideal m^2 = 1\), so wird \(\ideal m\) von einem Element \(x \in A\) erzeugt.
			Dann schließen wir mit dem Hilfssatz.
			\qedhere
		\end{enumerate}
	\end{proof}
\end{frame}

\begin{frame}{Beispiele zu lokalen artinschen Ringen}
	Sei \(K\) ein Körper.
	\begin{example}<+->
		Die Ringe \(\set Z/(p^n)\) und \(K[x]/(f^n)\), wobei \(p\) eine Primzahl ist und \(f\) ein
		irreduzibles Polynom, sind artinsche lokale Ringe, deren maximales Ideal
		von einem Element (nämlich dem Bild von \(p\) beziehungsweise \(f\)) erzeugt wird.
	\end{example}
	\begin{example}<+->
		Im artinschen lokalen Ring \(K[x^2, x^3]/(x^4)\) ist das maximale Ideal \(\ideal m\)
		von den zwei Elementen \(x^2\) und \(x^3\) modulo \(x^4\) erzeugt. Damit ist
		\(\ideal m^2 = (0)\), also \(\dim_K \ideal m/\ideal m^2 = 2\).
	\end{example}
\end{frame}


