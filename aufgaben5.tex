\subsection{Ganzheit und Bewertungen}

\begin{exercise}
	\label{exer:extension_to_integral_extension}
	Sei \(A \subset B\) eine ganze Erweiterung kommutativer Ringe.
	Sei \(\phi\colon A \to L\) ein Ringhomomorphismus in einen algebraisch
	abgeschlossenen Körper \(L\). Zeige, daß \(\phi\) zu einem
	Ringhomomorphismus \(\psi\colon B \to L\) fortgesetzt werden kann.
	
	(Tip:~\prettyref{thm:existence_of_primes_in_integral_extensions}.)
\end{exercise}

\begin{exercise}
	Sei \(A\) ein kommutativer Ring.
	Sei \(\phi\colon B \to B'\) ein Homomorphismus kommutativer \(A\)-Algebren.
	Sei \(C\) eine weitere \(A\)-Algebra. Zeige: Ist \(\phi\) ganz, so ist
	auch \(\phi \otimes \id_C\colon B \otimes_A C \to B' \otimes_A C\)
	ganz.
\end{exercise}

\begin{exercise}
	Sei \(A \subset B\) eine ganze Erweiterung kommutativer Ringe. Sei \(\ideal
	n\) ein maximales Ideal von \(B\) und \(\ideal m \coloneqq A \cap
	\ideal n\) das entsprechende maximale Ideal von \(A\). Ist \(B_{\ideal n}\)
	in jedem Falle ganz über \(A_{\ideal m}\)?
	
	(Tip: Betrachte die Ringerweiterung \(K[x^2 - 1] \subset K[x]\) für einen
	Körper \(K\), und sei \(\ideal n = (x - 1)\). Kann das Element \(\frac
	1 {x + 1}\) ganz sein?)
\end{exercise}

\begin{exercise}
	Sei \(A \subset B\) eine ganze Erweiterung kommutativer Ringe. Zeige:
	\begin{enumerate}
	\item
		Ist \(x \in A\) eine Einheit in \(B\), so ist \(x\) auch eine Einheit
		in \(A\).
	\item
		Ist \(\ideal k\) das Jacobsonsche Radikal von \(B\), so ist die
		Kontraktion
		\(\ideal j \coloneqq A \cap \ideal k\) das Jacobsonsche Radikal von
		\(A\).
	\end{enumerate}
\end{exercise}

\begin{exercise}
	Sei \(A\) ein kommutativer Ring. Seien \(B_1, \dotsc, B_n\) ganze
	kommutative \(A\)-Algebren. Zeige, daß \(\prod\limits_{i = 1}^n B_i\)
	eine ganze \(A\)-Algebra ist.
\end{exercise}

\begin{exercise}
	Sei \(A \subset B\) eine Erweiterung kommutativer Ringe, so daß
	\(S \coloneqq B \setminus A\) in \(B\) multiplikativ abgeschlossen ist.
	Zeige, daß dann \(A\) ganz abgeschlossen in \(B\) ist.
\end{exercise}

\begin{exercise}
	\label{exer:product_poly_in_closure}
	Sei \(A \subset B\) eine Erweiterung kommutativer Ringe. Sei \(C\) der
	ganze Abschluß von \(A\) in \(B\). Seien \(f, g \in B[x]\) normierte
	Polynome mit \(fg \in C[x]\). Zeige, daß dann auch \(f, g \in C[x]\).
	
	(Tip: Sei \(B \subset D\) eine Ringerweiterung, in dem \(f\) und \(g\)
	in Linearfaktoren zerfallen, etwa \(f = \prod (x - a_i)\) und \(g = \prod
	(x - b_j)\). Die \(a_i, b_j\) sind Wurzeln von \(fg\) und damit ganz über
	\(C\). Damit sind die Koeffizienten von \(f, g\) ganz über \(C\).)
\end{exercise}

\begin{exercise}
	Sei \(A \subset B\) eine Erweiterung kommutativer Ringe. Sei \(C\) der
	ganze Abschluß von \(A\) in \(B\). Zeige, daß dann \(C[x]\) der ganze
	Abschluß von \(A[x]\) in \(B[x]\) ist.

	(Tip: Ist \(f \in B[x]\) ganz über \(A[x]\), so ist \[f^m + g_1 f^{m - 1}
	+ \dotsb + g_m = 0\] für gewisse \(g_i \in A[x]\). Sei \(r \gg 0\) eine ganze
	Zahl. Sei \(f_1 \coloneqq f - x^r\), also
	\[(f_1 + x^r)^m + g_1 (f + x^r)^{m - 1} + \dotsb + g_m = 0,\]
	das heißt \[f_1^m + h_1 f_1^{m - 1} + \dotsb + h_m = 0\]
	für gewisse \(h_i \in A[x]\), wobei \(h_m = (x^r)^m + g_1(x^r)^{m - 1}
	+ \dotsb + g_m\). Wende jetzt~\prettyref{exer:product_poly_in_closure} auf
	die Polynome \(-f_1\) und \(f_1^{m - 1} + h_1 f_1^{m - 2} + \dotsb +
	h_{m - 1}\) an.
\end{exercise}

\begin{exercise}
	Sei \(G\) eine endliche Gruppe von Automorphismen eines kommutativen Ringes
	\(A\). Sei \(A^G\) der Unterring der \(G\)-Invarianten, das heißt derjenigen
	Elemente \(x \in A\) für die \(g(x) = x\) für alle \(g \in G\).
	\begin{enumerate}
	\item
		Zeige, daß \(A\) ganz über \(A^G\) ist.
	
		(Tip: Sei \(x \in A\). Überlege Dir, daß \(x\) Wurzel des Polynoms
		\(\prod\limits_{g \in G} (t - g(x))\) in \(t\) ist.)
	\item
		Sei \(S\) eine \(G\)-invariante multiplikativ abgeschlossene Teilmenge
		in \(A\), das heißt \(g(S) \subset S\) für alle \(g \in G\). Sei
		\(S^G \coloneqq S \cap A^G\). Zeige, daß sich die Wirkung von \(G\)
		auf \(A\) zu einer Wirkung auf \(S^{-1} A\) fortsetzen läßt und daß
		\((S^G)^{-1} A^G \cong (S^{-1} A)^G\).
	\end{enumerate}
\end{exercise}

\begin{exercise}
	Sei \(G\) eine endliche Gruppe von Automorphismen eines kommutativen Ringes
	\(A\). Sei \(\ideal p\) ein Primideal in \(A^G\). Sei \(\mathfrak Q\) die
	Menge aller Primideale \(\ideal q\) von \(A\) mit \(\ideal p = A^G \cap
	\ideal q\). Zeige, daß \(G\) auf \(\mathfrak Q\) transitiv operiert, und
	folgere, daß \(\mathfrak Q\) endlich ist.

	(Tip: Seien \(\ideal q_1, \ideal q_2 \in \mathfrak Q\) und \(x \in \ideal
	q_1\). Dann ist \(\prod\limits_{g \in G} g(x) \in \ideal A^G \cap \ideal q_1
	= \ideal p \subset \ideal q_2\), also \(g(x) \in \ideal q_2\) für ein
	\(g \in G\). Folgere, daß \(\ideal q_1 \subset \bigcup\limits_{g \in G}
	g(\ideal p_2)\), und wende dann~\prettyref{prop:ideal_in_union_of_primes}
	und~\prettyref{cor:equality_of_primes_in_integral_extension} an.)
\end{exercise}

\begin{exercise}
	\label{exer:global_noether_norm}
	Sei \(A \subset B\) eine endlich erzeugte Erweiterung von
	Integritätsbereichen. Zeige, daß ein \(s \in A \setminus \{0\}\) und eine
	\(A\)-Algebra \(B' \subset B\) mit \(B' \cong A[y_1, \dotsc, y_n]\)
	existieren, so daß \(B[s^{-1}]\) ganz über \(B'[s^{-1}]\) ist.
	
	(Tip: Sei \(S \coloneqq A \setminus \{0\}\), das heißt \(K = S^{-1} A\) ist
	der Quotientenkörper von \(A\). Dann ist \(S^{-1} B\) eine endlich erzeugte
	\(K\)-Algebra, und nach~\prettyref{prop:noether_norm} existieren damit
	\(x_1, \dotsc, x_n \in S^{-1} B\), so daß \(K[x_1, \dotsc, x_n] \subset
	S^{-1} B\) der Polynomring über \(K\) in \(n\) Variablen ist und daß
	\(S^{-1} B\) ganz über \(K[x_1, \dotsc, x_n]\) ist. Seien
	\(z_1, \dotsc, z_m\) Erzeuger von \(B\) als kommutative \(A\)-Algebra. Dann
	sind die \(z_i\), aufgefaßt als Element in \(S^{-1} B\) ganz über
	\(K[x_1, \dotsc, x_n]\). Schreibe Ganzheitsbedingungen für die \(z_j\) hin,
	und zeige, daß ein \(s \in S\) existiert, so daß \(x_i = \frac{y_i} s\)
	mit
	\(y_i \in B\) und so daß \(s z_j\) ganz über \(B'\) ist. Folgere, daß dieses
	\(s\) die geforderte Eigenschaft hat.)
\end{exercise}

\begin{exercise}
	\label{exer:global_extension_to_ac}
	Seien \(A \subset B\) eine endlich erzeugte Erweiterung von
	Integritätsbereichen. Zeige, daß ein \(s \in A \setminus \{0\}\) existiert,
	so daß jeder Ringhomomorphismen \(\phi\colon A \to L\) in einen algebraisch
	abgeschlossenen Körper \(L\) mit \(\phi(s) \neq 0\) zu einem
	Ringhomomorphismus \(B \to L\) fortgesetzt werden kann.
	
	(Tip: Mit den Bezeichnungen des
	von~\prettyref{exer:global_noether_norm} kann \(\phi\) zunächst auf
	\(B'\) fortgesetzt werden, etwa indem alle \(y_i\) auf \(0\) geschickt
	werden. Sodann kann \(\phi\) weider auf \(B'[s^{-1}]\) fortgesetzt werden,
	da \(\phi(s) \in L^\units\), schlich auf \(B[s^{-1}]\)
	nach~\prettyref{exer:extension_to_integral_extension},
	da \(B[s^{-1}]\) ganz über \(B'[s^{-1}]\) ist.)
\end{exercise}

\begin{exercise}
	\label{exer:trivial_jacobson_ideal}
	Sei \(A \subset B\) eine endlich erzeugte Erweiterung von
	Integritätsbereichen. Zeige: Ist das Jacobsonsche Radikal von \(A\) das
	Nullideal, so ist auch das Jacobsonsche Radikal von \(B\) das Nullideal.
	
	(Tip: Sei \(v \in B \setminus \{0\}\). Wir müssen zeigen, daß ein
	maximales Ideal \(\ideal n\) von \(B\) mit \(v \notin \ideal n\) existiert.
	Anwenden von~\prettyref{exer:global_extension_to_ac} auf die Ringerweiterung
	\(A \subset B[v^{-1}]\)
	liefert ein Element \(s \in A \setminus \{0\}\). Sei \(\ideal m\) ein
	maximales Ideal von \(A\) mit \(s \notin \ideal m\). Seien \(k \coloneqq
	A/\ideal m\) und \(L\) ein algebraischer Abschluß von \(k\). Die Projektion
	\(A \surjto k\) setzt sich zu einem Ringhomomorphismus \(\psi\colon B[v^{-1}]
	\to L\) fort. Zeige, daß \(\psi(v) \neq 0\) und daß \(B \cap ker \psi\)
	ein maximales Ideal von \(B\) ist.)
\end{exercise}

\begin{exercise}
	\label{exer:jacobson_ring}
	Zeige, daß folgende Aussagen über einen kommutativen Ring \(A\) äquivalent
	sind:
	\begin{enumerate}
	\item
		Jedes Primideal in \(A\) ist Schnitt maximaler Ideale in \(A\).
	\item
		Ist \(\phi\colon A \to B\) ein Homomorphismus kommutativer Ringe,
		so ist das Nilradikal in \(\phi(A)\) gleich dem Jacobsonschen
		Radikal.
	\item
		Jedes Primideal \(\ideal p\) in \(A\), welches nicht maximal ist, ist
		Schnitt aller Primideale, welche \(\ideal p\) echt enthalten.
	\end{enumerate}
	
	(Tip: Der schwierige Teil ist es, die erste Aussage aus der dritten zu
	folgern: Angenommen, die dritte Aussage sei wahr, allerdings gebe es ein
	Primideal \(\ideal p\), welches nicht Schnitt maximaler Ideale ist.
	In dem wir von \(A\) nach \(A/\ideal p\) übergehen, können wir annehmen,
	daß \(A\) ein Integritätsbereich mit Jacobsonschen Radikal \(\ideal j
	\neq (0)\) ist. Sei \(f \in \ideal j\) mit \(f \neq 0\). Dann ist \(A_f\)
	nicht der Nullring, besitzt also ein maximales Ideal \(\ideal m\). Für
	seine Kontraktion \(\ideal p \coloneqq A \cap \ideal m\) in \(A\) gilt dann,
	daß \(f \notin \ideal p\) und daß \(\ideal p\) maximal mit dieser
	Eigenschaft ist. Es ist \(\ideal p\) kein maximales Ideal in \(A\), ist aber
	auch nicht gleich dem Schnitt aller Primideale, welche \(\ideal p\) echt
	enthalten.)
	
	Ein kommutativer Ring \(A\), welcher diese drei äquivalenten Aussagen
	erfüllt heißt ein \emph{Jacobsonscher Ring}.
\end{exercise}

\begin{exercise}
	Sei \(A\) ein Jacobsonscher Ring (siehe~\prettyref{exer:jacobson_ring}).
	Sei \(B\) eine kommutative \(A\)-Algebra. Zeige:
	\begin{enumerate}
	\item
		Ist \(B\) ganz über \(A\), so ist \(B\) ein Jacobsonscher Ring.
	\item
		Ist \(B\) endlich erzeugt über \(A\), so ist \(B\) ebenfalls ein
		Jacobsonscher Ring.
		
		(Tip:~\prettyref{exer:trivial_jacobson_ideal}.)
	\end{enumerate}
\end{exercise}

\begin{exercise}
	Sei \(A\) ein kommutativer Ring. Zeige, daß die folgenden beiden
	Aussagen äquivalent sind:
	\begin{enumerate}
	\item
		Es ist \(A\) ein Jacobsonscher Ring.
	\item
		Jede endlich erzeugte kommutative \(A\)-Algebra \(B\), welche ein
		Körper ist, ist endlich über \(A\).
	\end{enumerate}
	
	(Tip: Um aus der ersten die zweite Aussage zu folgern: Reduziere auf den
	Fall, daß \(A\) ein Unterring von \(B\) ist. Nutze
	dann~\prettyref{exer:global_extension_to_ac}. Ist
	\(s \in A \setminus \{0\}\) wie dort, so existiert ein maximales Ideal
	\(\ideal m\) von \(A\) mit \(s \notin \ideal m\), und die
	Projektion \(A \surjto A/\ideal m \eqqcolon K\) setzt sich zu einem
	Ringhomomorphismus \(\psi\colon B \to L\) in einen algebraischen Abschluß
	\(L\) von \(K\) fort. Da \(B\) ein Körper ist, ist \(\psi\) injektiv
	und \(\phi(B)\) ist algebraisch über \(K\) und damit endlich algebraisch
	über \(K\).
	
	Um aus der zweiten die erste Aussage zu folgern: Benutze die dritte
	Charakterisierung aus~\prettyref{exer:jacobson_ring}. Sei \(\ideal p\)
	ein Primideal von \(A\), welches nicht maximal ist, und sei
	\(B \coloneqq A/\ideal p\). Sei \(f \in B \setminus \{0\}\). Dann ist
	\(B[f^{-1}]\) eine endlich erzeugte \(A\)-Algebra. Wäre \(B[f^{-1}]\) ein
	Körper, wäre es endlich über \(B\), also ganz über \(B\), also wäre
	\(B\) ein Körper nach~\prettyref{prop:fields_and_integral_extensions} im
	Widerspruch dazu, daß \(\ideal p\) kein maximales Ideal ist. Also ist
	\(B[f^{-1}]\) kein Körper, besitzt also ein nicht triviales Primideal
	\(\ideal q\), so daß \(\ideal p' \coloneqq B \cap \ideal q\) ein nicht
	triviales Primideal mit \(f \notin \ideal p'\) ist.)
\end{exercise}

\begin{exercise}
	Sei \(A \subset B\) eine Erweiterung lokaler Ringe. Wir sagen, \emph{\(B\)
	dominiere \(A\)}, falls das maximale Ideal \(\ideal m\) von \(A\) im
	maximalen Ideal \(\ideal n\) von \(B\) enthalten ist. (Dies ist äquivalent
	zu \(\ideal m = A \cap \ideal n\).)
	
	Sei \(K\) ein Körper. Sei \(\mathfrak S\) die Menge aller Unterringe von
	\(K\), welche lokal sind. Die Menge \(\mathfrak S\) wird durch die
	Dominanzbeziehung teilweise geordnet. Zeige, daß \(\mathfrak S\) maximale
	Elemente besitzt und daß ein Element \(A \in \mathfrak S\) genau dann
	maximal ist, wenn \(A\) ein Bewertungsring von \(K\) ist.
	
	(Tip:~\prettyref{thm:existence_of_valuation_rings}.)
\end{exercise}

\begin{exercise}
	Sei \(A\) ein Integritätsbereich mit Quotientenkörper \(K\). Zeige, daß
	die folgenden beiden Aussagen äquivalent sind:
	\begin{enumerate}
	\item
		Es ist \(A\) ein Bewertungsring für \(K\).
	\item
		Für je zwei Ideale \(\ideal a, \ideal b\) von \(A\) gilt
		\(\ideal a \subset \ideal b\) oder \(\ideal b \subset \ideal a\).
	\end{enumerate}
	
	Folgere dann: Ist \(A\) ein Bewertungsring und \(\ideal p\) ein Primideal
	von \(A\), so sind auch \(A_{\ideal p}\) und \(A/\ideal p\) Bewertungsringe
	(ihrer jeweiligen Quotientenkörper).
\end{exercise}

\begin{exercise}
	Sei \(A\) ein Bewertungsring mit Quotientenkörper \(K\). Zeige, daß jeder
	Unterring \(B\) von \(K\) mit \(A \subset B\) ein lokaler Ring von \(A\)
	ist, das heißt eine Lokalisierung von \(A\) an einem Primideal \(\ideal p\).
\end{exercise}

\begin{exercise}
	Sei \(A\) ein Bewertungsring mit Quotientenkörper \(K\). Die Gruppe
	\(A^\units\) der Einheiten von \(A\) bildet eine Untergruppe von
	\(K^\units\). Sei \(G \coloneqq \log K^\units/A^\units\) die Faktorgruppe,
	additiv geschreiben.
	
	Zeige, daß durch die Setzung \(\log [x]_{A^\units} \ge
	\log [y]_{A^\units} \iff xy^{-1} \in A\) für \(x, y \in K^\units\) eine
	Ordnung von \(G\) definiert wird, welche mit der Gruppenstruktur
	verträglich ist, das heißt \(\xi \ge \eta \implies \xi + \omega \ge
	\eta + \omega\) für alle \(\xi, \eta, \omega \in G\).
	
	Sei \(\nu\colon K \to G \cup \{\infty\}\) durch \(\nu(x) = \log [x]_{A^\units}\)
	für \(x \in K^\times\) und \(\nu(0) = \infty\) definiert. Zeige, daß
	\(\nu(x + y) \ge \min(\nu(x), \nu(y))\) für alle \(x, y \in K\).
\end{exercise}

\begin{exercise}
	\label{exer:valuation}
	Sei umgekehrt \(G\) eine vollständig geordnete abelsche Gruppe (additiv
	geschrieben).
	Sei \(K\) ein Körper. Eine \emph{Bewertung auf \(K\) mit Werten in \(G\)}
	ist eine Abbildung \(\nu\colon K \to G \cup \{\infty\}\) mit
	\(\nu(xy) = \nu(x) + \nu(y)\) und \(\nu(x + y) \ge \min(\nu(x), \nu(y))\)
	und \(\nu(x) = \infty \iff x = 0\)
	für alle \(x, y \in K\). Zeige, daß die Menge der Elemente \(x \in K\) mit
	\(\nu(x) \ge 0\) ein Bewertungsring von \(K\) ist.
	
	Dieser Ring ist der \emph{Bewertungsring von \(\nu\)} und die Untergruppe
	\(\nu(K^\units)\) von \(G\) ist die \emph{Bewertungsgruppe von \(\nu\)}.
	Im wesentlichen sind also die Konzepte "`Bewertungsring"' und
	"`Bewertung"' äquivalent.
\end{exercise}

\begin{exercise}
	Sei \(G\) eine vollständig geordnete abelsche Gruppe. Eine Untergruppe
	\(H\) von \(G\) heißt \emph{isoliert}, falls aus \(0 \le \beta \le \alpha\)
	mit \(\beta \in G, \alpha \in H\) auch \(\beta \in H\) folgt.
	
	Sei \(A\) ein Bewertungsring mit Bewertungsgruppe \(G\) von \(K\)
	(siehe~\prettyref{exer:valuation}). Sei \(\nu\colon K \to G \cup
	\{\infty\}\) die Bewertung.
	
	Sei \(\ideal p\) ein Primideal von \(A\).
	Zeige, daß \(\nu(A \setminus \ideal p)\) alle Elemente \(x\) mit \(x \ge 0\)
	einer isolierten Untergruppe \(H(\ideal p)\) durchläuft. Zeige weiter, daß
	so eine injektive Zuordnung von der Menge der Primideale von \(A\) in die
	Menge der isolierten Untergruppen von \(G\) definiert wird.
	
	Was sind die Bewertungsgruppen der Bewertungsringe \(A/\ideal p\) und
	\(A_{\ideal p}\) für ein Primideal \(\ideal p\)?
\end{exercise}

\begin{exercise}
	Sei \(G\) eine vollständig geordnete abgeschlossene Gruppe. Sei \(F\) ein
	beliebiger Körper. Mit \(A \coloneqq F[G]\) bezeichnen wir die
	\emph{Gruppenalgebra von \(G\) über \(F\)}: Es besitzt \(A\) als
	\(F\)-Vektorraum eine Basis \((x_\alpha)_{\alpha \in G}\) mit
	\(x_\alpha x_\beta =
	x_{\alpha \beta}\). Zeige, daß \(A\) ein Integritätsbereich ist.
	
	Ist \(u = a_1 x_{\alpha_1} + \dotsb + a_n x_{\alpha_n} \in A\) mit
	\(a_i \in F^\units\) und \(\alpha_1 < \dotsb < \alpha_n\), so definieren wir
	\(\nu_0(u) \coloneqq \alpha_1\). Zeige, daß die Abbildung
	\(\nu_0\colon A \setminus \{0\} \to G\) die Bedingungen
	\(\nu_0(xy) \nu_0(x) + \nu_0(y)\) und \(\nu_0(x + y) \ge \min(\nu(x), \nu(y))\)
	für \(x, y \in A \setminus \{0\}\) mit \(x + y \neq 0\) erfüllt.
	
	Sei \(K\) der Quotientenkörper \(A\). Zeigen Sie, daß \(\nu_0\) eindeutig zu
	einer Bewertung \(\nu\) auf \(K\) mit Bewertungsgruppe \(G\) fortgesetzt
	werden kann.
\end{exercise}

