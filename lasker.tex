\section{Primärzerlegung in noetherschen Ringen}

\subsection{Irreduzible Ideale}

\begin{frame}{Definition eines irreduziblen Ideals}
	Sei \(A\) ein kommutativer Ring.
	\begin{definition}<+->
		Ein Ideal \(\ideal a \neq (1)\) in \(A\) heißt \emph{irreduzibel}, falls
		für je zwei weitere Ideale \(\ideal b, \ideal c\) von \(A\) gilt:
		\(\ideal a = \ideal b \cap \ideal c \implies (\ideal a = \ideal b
		\lor \ideal a = \ideal c)\).
	\end{definition}
	\begin{lemma}<+->
		\label{lem:lasker1}
		Ist \(A\) noethersch, so ist jedes Ideal \(\ideal a\) ein endlicher Schnitt
		irreduzibler Ideale.
	\end{lemma}
	\begin{proof}<+->
		Angenommen, es gibt ein Ideal, welches sich nicht als endlicher Schnitt
		irreduzibler Ideale schreiben läßt. Dann gibt es auch ein solches
		maximales \(\ideal a\). Da \(\ideal a\) selbst nicht irreduzibel sein kann,
		existieren Ideale \(\ideal b, \ideal c\) mit \(\ideal a = \ideal b \cap \ideal c\),
		aber \(\ideal a \subsetneq \ideal b, \ideal c\). Damit sind \(\ideal b, \ideal c\)
		jeweils Schnitte endlich vieler irreduzibler Ideale, also auch \(\ideal a\).
		Widerspruch.
	\end{proof}
\end{frame}

\begin{frame}{Irreduzible Ideale in noetherschen Ringen}
	\begin{lemma}<+->
		\label{lem:lasker2}
		Sei \(\ideal a\) ein irreduzibles Ideal in einem noetherschen kommutativen
		Ring \(A\). Dann ist \(\ideal a\) ein Primärideal.
	\end{lemma}
	\begin{proof}<+->
		\begin{enumerate}[<+->]
		\item<.->
			Indem wir von \(A\) zu \(A/\ideal a\) übergehen, können wir uns auf den
			Fall beschränken, daß das Nullideal primär ist, wenn es irreduzibel ist.
		\item
			Sei also \(xy = 0\) mit \(y \neq 0\). Die Kette \(\ann(x) \subset \ann(x^2)
			\subset \dotsb\) ist stationär, also ist \(\ann(x^n) = \ann(x^{n + 1})\) für
			ein \(n\).
		\item
			Sei \(a \in (x^n) \cap (y)\). Dann ist \(a = b x^n\) für ein \(b \in A\) und
			\(a x = b x^{n + 1} = 0\). Insbesondere \(b \in \ann(x^{n + 1}) = \ann(x^{n})\),
			also \(a = b x^n = 0\). Folglich \((x^n) \cap (y) = (0)\).
		\item
			Ist \((0)\) irreduzibel, so folgt wegen \((y) \neq (0)\) damit \((x^n) = 0\),
			also \(x^n = 0\).
			\qedhere
		\end{enumerate}
	\end{proof}
\end{frame}

\subsection{Existenz der Primärzerlegung in noetherschen Ringen}

\begin{frame}{Existenz der Primärzerlegung in noetherschen Ringen}
	Sei \(A\) ein noetherscher kommutativer Ring.
	\begin{theorem}<+->
		In \(A\) ist jedes Ideal zerlegbar.
		\qed
	\end{theorem}
	\begin{proposition}<+->
		Jedes Ideal \(\ideal a\) in \(A\) enthält eine Potenz seines Wurzelideals.
	\end{proposition}
	\begin{proof}<+->
		\begin{enumerate}[<+->]
		\item<.->
			Seien \(x_1, \dotsc, x_k\) Erzeuger von \(\sqrt{\ideal a}\). Dann gilt
			\(x_i^{n_i} \in \ideal a\) für gewisse \(n_i\). Sei \(m \coloneqq
			\sum\limits_i (n_i - 1) + 1\).
		\item
			Es wird \(\sqrt{\ideal a}^m\) von Produkten \(x_1^{r_1} \dotsm x_k^{r_k}\) mit
			\(\sum\limits_i r_i = m\) erzeugt. Nach Definition von \(m\) gilt
			\(r_i \ge n_i\) für mindestens ein \(i\) und damit
			\(x_1^{r_1} \dotsm x_k^{r_k} \in \ideal a\).
			\qedhere
		\end{enumerate}
	\end{proof}
\end{frame}

\begin{frame}{Das Nilradikal in noetherschen Ringen ist nilpotent}
	Sei \(A\) ein noetherscher kommutativer Ring.
	\begin{corollary}<+->
		Das Nilradikal in \(A\) ist nilpotent.
	\end{corollary}
	\begin{proof}<+->
		Eine Potenz von \(\sqrt{(0)}\) ist in \((0)\) enthalten.
	\end{proof}
\end{frame}

\begin{frame}{Primäre Ideale zu maximalen Idealen in noetherschen Ringen}
	\begin{corollary}<+->
		Für ein maximales Ideal \(\ideal m\) und ein weiteres Ideal \(\ideal q\) eines
		noetherschen kommutativen Ringes \(A\) sind
		äquivalent:
		\begin{enumerate}[<+->]
		\item<.->
			Es ist \(\ideal q\) ein \(\ideal m\)-primäres Ideal.
		\item
			Es ist \(\sqrt{\ideal q} = \ideal m\).
		\item
			Es ist \(\ideal m^n \subset \ideal q \subset \ideal m\) für \(n \gg 0\).
		\end{enumerate}
	\end{corollary}
	\begin{proof}<+->
		Die Äquivalenz der ersten beiden Aussagen gilt in beliebigen kommutativen Ringen. Daß aus der
		zweiten die dritte folgt, ist die Aussage der Proposition. Die zweite folgt aus der dritten so:
		\(\ideal m = \sqrt{\ideal m^n} \subset \sqrt{\ideal q} \subset \sqrt{\ideal m} = \ideal m\).
	\end{proof}
\end{frame}

\begin{frame}{Assoziierte Primideale in noetherschen Ringen}
	\begin{proposition}<+->
		Sei \(\ideal a\) ein Ideal in einem noetherschen kommutativen Ring \(A\). Dann sind die
		zu \(\ideal a\) assoziierten Primideale genau die Primideale in \(A\) von der Form
		\((\ideal a : x)\) mit \(x \in A\).
	\end{proposition}
	\begin{proof}<+->
		Es reicht, die Aussage für \(A/\ideal a\) und das Nullideal zu beweisen, da die eigentliche
		Aussage dann durch Kontraktion nach \(A\) folgt. Wir zeigen also, daß in einem noetherschen
		Ring die zu \((0)\) assoziierten Primideale gerade die Primideale der Form \(\ann(x) = (0 : x)\)
		mit \(x \in A\) sind.
		\renewcommand{\qedsymbol}{}
	\end{proof}
\end{frame}

\begin{frame}{Fortsetzung des Beweises zu assoziierten Primidealen}
	\begin{proof}[Fortsetzung]<+->
		\begin{enumerate}[<+->]
		\item<.->
			Sei \(\bigcap\limits_{i = 1}^n \ideal q_i = (0)\) eine minimale Primärzerlegung. Sei \(\ideal p_i
			\coloneqq \sqrt{\ideal q_i}\). Sei \(\ideal a_i = \bigcap\limits_{j \neq i} \ideal q_j \neq (0)\).
			Wir hatten schon einmal \(\sqrt{\ann(x)} = \ideal p_i\) für alle \(x \in \ideal a_i \setminus \{0\}\)
			gezeigt, also \(\ann(x) \subset \ideal p_i\).%
			% XXX Das sollte vielleicht zu einem Hilfssatz werden
		\item
			Da \(\ideal p_i = \sqrt{\ideal q_i}\), ist \(\ideal p_i^m \subset \ideal q_i\) für \(m \gg 0\),
			also \(\ideal a_i \ideal p_i^m \subset \ideal a_i \cap \ideal p_i^m \subset \ideal a_i \cap \ideal q_i = (0)\).
			Sei \(m\) jetzt die kleinste natürliche Zahl mit \(\ideal a_i \ideal p_i^m = (0)\) und \(x \in \ideal a_i
			\ideal p_i^{m - 1} \setminus \{0\}\).
		\item
			Dann ist \(\ideal p_i x = 0\), also \(\ann(x) \supset \ideal p_i \), also \(\ann(x) = \ideal p_i\).
		\item
			Ist umgekehrt \(\ann(x)\) ein Primideal, so haben wir schon gezeigt, daß \(\ann(x) = \sqrt{\ann(x)}\) ein
			assoziiertes Primideal ist.
			\qedhere
		\end{enumerate}
	\end{proof}
\end{frame}

