\section{Reguläre lokale Ringe}

\subsection{Charakterisierung regulärer lokaler Ringe}

\begin{frame}{Reguläre lokale Ringe}
	\begin{theorem}<+->
		Sei \((A, \ideal m, F)\) ein noetherscher lokaler Ring der Dimension \(d\). Dann sind äquivalent:
		\begin{enumerate}[<+->]
		\item<.->
			\(\Graded_{\ideal m}(A, t)\) und \(F[X_1, \dotsc, X_d]\) sind als gewichtete \(F\)-Algebren isomorph.
		\item
			Für die Dimension des Zariskischen Kotangentialraumes von \(A\) gilt \(\dim_F \ideal m/\ideal m^2 = d\).
		\item
			Es existiert ein Parametersystem \((x_1, \dotsc, x_d)\) von \(A\) mit \(\ideal m = (x_1, \dotsc, x_d)\).
		\end{enumerate}
	\end{theorem}
	\begin{visibleenv}<+->
		Ein \emph{regulärer lokaler Ring \(A\)} ist ein noetherscher lokaler Ring, welcher die Bedingungen
		des Satzes erfüllt.
	\end{visibleenv}
\end{frame}

\begin{frame}{Beweis des Satzes über reguläre lokale Ringe}
	\begin{proof}<+->
		\begin{enumerate}[<+->]
		\item<.->
			Aus der ersten folgt sicherlich die zweite Aussage. Die dritte folgt aus der zweiten
			mit dem Nakayamaschen Lemma.
		\item
			Sei also \(\ideal m = (x_1, \dotsc, x_m)\). Dann ist \(\alpha\colon F[X_1, \dotsc, X_d] \to
			\Graded_{\ideal m}(t), X_i \mapsto \bar x_i t\) ein surjektiver Homomorphismus gewichteter
			Algebren.
		\item
			Der Homomorphismus ist aufgrund der Unabhängigkeitseigenschaft eines Parametersystems injektiv.
			\qedhere
		\end{enumerate}
	\end{proof}
\end{frame}

\begin{frame}{Reguläre lokale Ringe sind Integritätsbereiche}
	\begin{lemma}<+->
		Sei \(A\) ein kommutativer Ring. Sei \(\ideal a\) ein Ideal in \(A\) mit \(\bigcap\limits_n \ideal a^n = (0)\).
		Ist dann \(\Graded_{\ideal a}(A, t)\) ein Integritätsbereich, so auch \(A\).
	\end{lemma}
	\begin{proof}<+->
		Seien \(x, y \in A \setminus \{0\}\). Dann existieren \(r, s \in \set N_0\) mit \(x \in \ideal a^r \setminus \ideal a^{r + 1}\)
		und \(y \in \ideal a^s \setminus \ideal a^{s + 1}\). Dann sind \(\bar x t^r, \bar y t^s \neq 0 \in \Graded_{\ideal a}(t)\),
		also \(\bar x \cdot \bar y t^{r + s} \neq 0\), also \(x y \neq 0 \in A\).
	\end{proof}
	\begin{corollary}<+->
		Ein regulärer lokaler Ring ist ein Integritätsbereich.
	\end{corollary}
\end{frame}

\begin{frame}{Bemerkungen zu regulären lokalen Ringen}
	\begin{remark}<+->
		Ein regulärer lokaler Ring \((A, \ideal m, F)\) der Dimension \(1\) ist ein Integritätsbereich mit \(\dim_F \ideal m/\ideal m^2
		= 1\), also ein diskreter Bewertungsbereich.
		\\
		Umgekehrt ist ein diskreter Bewertungsbereich ein regulärer lokaler Ring der Dimension \(1\).
	\end{remark}
	\begin{remark}<+->
		Sei \((A, \ideal m)\) ein lokaler Ring. Ist dann \(\Graded_{\ideal m}(A, t)\) ein ganz abgeschlossener Integritätsbereich,
		so kann gezeigt werden, daß auch \(A\) ganz abgeschlossen ist. Jeder regulärer lokale Ring ist also ganz abgeschlossen.
		\\
		Es existieren aber ganz abgeschlossene lokale Integritätsbereiche mit Dimension größer als \(1\), welche nicht regulär sind.
	\end{remark}
\end{frame}

\subsection{Regularität als analytische Eigenschaft}

\begin{frame}{Regularität als analytische Eigenschaft}
	\begin{proposition}<+->
		Sei \((A, \ideal m)\) ein noetherscher lokaler Ring. Dann ist \(A\)
		genau dann regulär, wenn seine \(\ideal m\)-adische Vervollständigung
		\(\hat A\) ein regulärer noetherscher lokaler Ring ist.
	\end{proposition}
	\begin{proof}<+->
		\begin{enumerate}[<+->]
		\item<.->
			Wir haben schon gezeigt, daß \((\hat A, \hat{\ideal m})\) ein noetherscher 
			lokaler Ring ist, wenn \((A, \ideal m)\) ein noetherscher lokaler Ring
			ist.
		\item
			Da \(A\) noethersch ist, gilt außerdem
			\(\Graded_{\ideal m}(A, t) \cong \Graded_{\hat{\ideal m}}(\hat A, t)\). 
		\item
			Schließlich ist \(\dim A = \dim{\hat A}\).
			\qedhere
		\end{enumerate}
	\end{proof}
\end{frame}

\begin{frame}{Geometrische Interpretation}
	\begin{remark}<+->
		Sei \((A, \ideal m)\) ein lokaler Ring. Die \(\ideal m\)-adische
		Vervollständigung \((\hat A, \hat{\ideal m})\) heißt auch der
		\emph{analytische Halm von \(A\)}. 
		\\
		Wir haben damit gezeigt, daß der analytische Halm eines regulären
		lokalen Ringes \(A\) ein Integritätsbereich ist, was geometrisch so
		ausgedrückt wird, daß \(A\) nur einen analytischen Zweig habe. 
	\end{remark}
\end{frame}

\begin{frame}{Analytischer Halm regulärer lokaler Ringe im geometrischen Fall}
	\begin{example}<+->
		Sei \(K\) ein Körper und \((A, \ideal m, F)\) eine reguläre lokale
		\(K\)-Algebra, so daß \(K\) isomorph auf \(F\) abgebildet wird. 
		Sei \(d \coloneqq \dim A\).
		\\
		Dann ist \(\Graded_{\ideal m}(A, t) \cong K[X_1, \dotsc, X_d]\),
		woraus \(\hat A = \ps K{X_1, \dotsc, X_d}\) folgt.
		\\
		Damit hängt der analytische Halm in dieser Situation nur von der Dimension
		ab.
	\end{example}
\end{frame}

\begin{frame}{Regularität des Polynomrings}
	\begin{example}
		Sei \(K\) ein Körper. Sei \(\ideal m = (X_1 - x_1, \dotsc, X_n - x_n)\) ein
		maximales Ideal des Polynomringes \(A \coloneqq K[X_1, \dotsc, X_n]\).
		Dann ist \(A_{\ideal m}\) ein regulärer lokaler Ring der
		Dimension \(n\), denn
		\(\Graded_{\ideal m}(A, t)\) ist ein Polynomring in \(n\) Variablen.
	\end{example}
\end{frame}

