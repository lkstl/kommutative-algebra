\section{Transzendente Dimension}

\subsection{Transzendente Dimension}

\begin{frame}{Transzendente Dimension}
	Sei \(K\) ein Körper. Sei \(A\) ein endlich erzeugter Integritätsbereich über \(K\).
	\begin{definition}<+->
		Der Transzendenzgrad des Quotientenkörpers \(K(A)\) von \(A\) über \(K\) heißt die \emph{(transzendente)
		Dimension \(\trdim_K A\) von \(A\)}.
	\end{definition}
	\begin{remark}<+->
		Sind \(x_1, \dotsc, x_n\) Erzeuger von \(A\) über \(K\), so erzeugen die
		\(x_i\) auch den Quotientenkörper \(K(A)\) über \(A\). Damit muß
		\(\trdim_K A \leq n\) gelten. Die transzendente Dimension von \(A\) ist
		also endlich.
	\end{remark}
	\begin{visibleenv}<+->
		Im folgenden wollen wir zeigen, daß \(\trdim_K A = \dim A\).
	\end{visibleenv}
\end{frame}

\begin{frame}{Ein Lemma über Höhe und Tiefe in ganzen Erweiterungen}
	\begin{lemma}<+->
		Sei \(A \subset B\) eine ganze Erweiterung von
		Integritätsbereichen. Sei \(A\) weiter ganz abgeschlossen.
		Sei \(\ideal q\) ein Primideal in \(B\). Sei \(\ideal p \coloneqq A \cap \ideal q\).
		Dann gilt \(\height \ideal p = \height \ideal q\) und \(\depth \ideal p = \depth \ideal q\).
	\end{lemma}
	\begin{proof}<+->
		\begin{enumerate}[<+->]
		\item<.->
			Ist \(\ideal q' \subsetneq \ideal q''\) eine echte Inklusion von
			Primidealen in \(B\), so ist \(A \cap \ideal q' \subsetneq A
			\cap \ideal q''\) eine echte Inklusion von Primidealen in
			\(A\). Damit folgt
			\(\height \ideal p \ge \height \ideal q\), \(\depth \ideal p \ge \depth \ideal q\).
		\item
			Nach dem "`Going-Down"'-Satz kann jede absteigende Primidealkette in \(A\)
			zu einer Primidealkette in \(B\) hochgehoben werden. Damit folgt
			\(\height \ideal q \ge \height \ideal p\).
		\item
			Nach dem "`Going-Up"'-Satz kann jede aufsteigende Primidealkette in
			\(A\) zu einer Primidealkette in \(B\) hochgehoben werden. Damit
			folgt \(\depth \ideal q \ge \depth \ideal p\).
			\qedhere
		\end{enumerate}
	\end{proof}
\end{frame}

\begin{frame}{Dimension des Polynomrings}
	\begin{lemma}<+->
		Sei \(K\) ein Körper. Für jedes maximale Ideal \(\ideal m\) des
		Polynomringes \(A \coloneqq K[X_1, \dotsc, X_n]\) ist dann
		\(\dim A_{\ideal m} = n\).
	\end{lemma}
	\begin{proof}<+->
		\begin{enumerate}[<+->]
		\item<.->
			Sei \(L\) ein algebraischer Abschluß von \(K\).
			Dann ist \(B \coloneqq L[X_1, \dotsc, X_n]\) der ganze
			Abschluß von \(A\) in \(L\). Dann existiert ein maximales
			Ideal \(\ideal n\) von \(B\) mit \(\ideal m = A \cap \ideal n\).
		\item		
			Nach dem Hilbertschen Basissatz ist
			\(\ideal n = (X_1 - b_1, \dotsc, X_n - b_n)\) für gewisse
			\(b_i \in B\). Damit ist \(\dim B_{\ideal n} = n\) nach den
			schon angestellten Überlegungen.
		\item
			Daraus folgt nach dem letzten Hilfssatz, daß \(\dim A_{\ideal m}
			= \height \ideal m = \height \ideal n = \dim B_{\ideal n} = n\).
			\qedhere
		\end{enumerate}
	\end{proof}
\end{frame}

\begin{frame}{Transzendente Dimension und lokale Dimension}
	\begin{theorem}<+->
		Sei \(K\) ein Körper. Sei \(A\) ein endlich erzeugter
		Integritätsbereich über \(K\). Dann ist \(\dim A_{\ideal m} = \trdim_K A\) für
		alle maximalen Ideale \(\ideal m\) von \(A\).
	\end{theorem}
	\begin{proof}<+->
		\begin{enumerate}[<+->]
		\item<.->
			Nach der noetherschen Normalisierung existiert eine ganze
			Ringerweiterung der Form \(A' \coloneqq K[X_1, \dotsc, X_d] \subset A\).
			Damit ist \(\trdim_K A = \trdim_K A' = d\).
		\item
			Ist weiter \(\ideal m' \coloneqq A' \cap \ideal m\), so folgt
			\(\dim A_{\ideal m} = \dim A'_{\ideal m'} = d\).
			\qedhere
		\end{enumerate}
	\end{proof}
\end{frame}

\begin{frame}{Dimension und lokale Dimension}
	\begin{corollary}<+->
		Sei \(K\) ein Körper. Sei \(A\) ein endlich erzeugter Integritätsbereich
		über \(K\). Dann ist \(\dim A = \dim A_{\ideal m}\) für jedes maximale
		Ideal \(\ideal m\) von \(A\).
	\end{corollary}
	\begin{proof}<+->
		Es ist \(\dim A = \sup\limits_{\ideal m} \dim A_{\ideal m}\), wobei
		\(\ideal m\) alle maximalen Ideale von \(A\) durchläuft. Nach dem
		Satz haben aber alle \(A_{\ideal m}\) dieselbe Dimension, nämlich die
		transzendente.
	\end{proof}
\end{frame}

\begin{frame}{Dimension und Kodimension}
	\begin{theorem}<+->
		Sei \(K\) ein Körper. Sei \(A\) ein endlich erzeugter Integritätsbereich
		über \(K\). Dann gilt \(\height \ideal p + \dim A/\ideal p = \dim A\)
		für alle Primideale \(\ideal p\) von \(A\).
	\end{theorem}
	\begin{proof}<+->
		\begin{enumerate}[<+->]
		\item<.->
			Nach der noetherschen Normalisierung existiert ein endlicher,
			injektiver Homomorphismus \(A' \coloneqq K[X_1, \dotsc, X_n] \to A\).
			Sei \(\ideal m\) ein maximales Ideal, \(\ideal m' \coloneqq A' \cap \ideal m\). Dann ist \(A = \dim A_{\ideal m} =
			\dim A'_{\ideal m'} = n\). Sei \(\ideal p' = A' \cap \ideal p\). Da \(\height \ideal p = \height \ideal p'\)
			und \(\depth \ideal p = \depth \ideal p'\), reicht es, den Satz für \(A'\) und \(\ideal p'\) zu beweisen.
		\item
			Wieder aufgrund der noetherschen Normalisierung können wir davon ausgehen, daß \(\ideal p' = (X_{r + 1},
			\dotsc, X_n)\). Damit ist
			\(\dim A' \ge \height \ideal p' + \depth \ideal p' \ge (n - r) + r = n = \dim A'\).
			\qedhere
		\end{enumerate}
	\end{proof}
\end{frame}

