\section{Der assoziierte gewichtete Ring}

\subsection{Definition und grundlegende Eigenschaften des assoziierten gewichteten Ringes}

\begin{frame}{Definition des assoziierten gewichteten Ringes}
	\begin{definition}<+->
		Sei \(\ideal a\) ein Ideal in einem kommutativen Ring \(A\). Dann heißt der
		gewichtete kommutative Ring
		\[
			\Graded_{\ideal a}(t) = \Graded_{\ideal a}(A, t)
			\coloneqq \Rees_{\ideal a}(A, t)/t^{-1} \Rees_{\ideal a}(A, t)_+ 
			\cong \bigoplus\limits_{n = 0}^\infty \ideal a^n/\ideal a^{n + 1} t^n
		\]
		der \emph{assoziierte gewichtete Ring zur \(\ideal a\)-adischen Filtrierung von \(A\)}.
	\end{definition}
	\begin{visibleenv}<+->
		Ist \(x \in \ideal a^n\), so schreiben wir \(\bar x\) für die Restklasse modulo \(\ideal a^{n + 1}\). Damit können
		wir die Multiplikation auf \(\Graded_{\ideal a}(t)\) folgendermaßen beschreiben:
		\\
		Es ist \((\bar x t^m) \cdot (\bar y t^n) = \overline{xy} t^{m + n}\) für \(x \in \ideal a^m, y \in \ideal a^n\).
	\end{visibleenv}
\end{frame}

\begin{frame}{Assozierte gewichtete Moduln}
	Sei \(\ideal a\) ein Ideal in einem kommutativen Ring. 
	Sei \(M\) ein \(A\)-Modul zusammen mit einer \(\ideal a\)-Filtrierung \(M_\bullet\). 
	\begin{definition}<+->
		Der gewichtete \(A\)-Modul
		\[
			\Graded(M_\bullet, t) = \Rees(M_\bullet, t)/t^{-1} \Rees(M_\bullet, t)_+
			\cong \bigoplus\limits_{n = 0}^\infty M_n/M_{n + 1} t^n
		\]
		heißt der \emph{assoziierte gewichtete Modul zur Filtrierung \(M_\bullet\)}.
	\end{definition}
	\begin{visibleenv}<+->
		In kanonischer Weise ist \(\Graded(M_\bullet, t)\) sogar ein \(\Graded_{\ideal a}(t)\)-Modul. Schreiben wir
		wieder \(\bar\cdot\) für Restklasen, so ist die Multiplikation auf \(\Graded(M_\bullet, t)\) durch
		\((\bar a t^m) \cdot (\bar x t^n) = \overline{ax} t^{m + n}\) für \(a \in \ideal a^m, x \in M_n\) gegeben.
	\end{visibleenv}
	\begin{notation}<+->
		Ist \(M_\bullet\) die \(\ideal a\)-adische Filtrierung auf \(M\), so schreiben wir
		\(\Graded_{\ideal a}(M, t) = \Graded(M_\bullet, t)\).
	\end{notation}
\end{frame}

\begin{frame}{Morphismen zwischen assoziierten gewichteten Moduln}
	Sei \(\ideal a\) ein Ideal in einem kommutativen Ring \(A\).
	Seien \(M, N\) zwei \(A\)-Moduln jeweils zusammen mit einer \(\ideal a\)-Filtrierung \(M_\bullet\) bzw.\
	\(N_\bullet\).
	\\
	Sei \(\phi\colon M \to N\) ein filtrierter Homomorphismus von \(A\)-Moduln, das heißt \(\phi(M_n) \subset N_n\)
	für alle \(n\). Dann induziert \(\phi\) einen Homomorphismus
	\[
		\Graded(\phi)\colon \Graded(M_\bullet, t) \to \Graded(N_\bullet, t), \bar x t^n \mapsto \overline{\phi(x)} t^n,
	\]
	wobei wieder \(\bar x \in M_n/M_{n + 1}\) das Bild eines \(x \in M_n\) modulo \(M_{n + 1}\) ist.
\end{frame}

\begin{frame}{Endlichkeitseigenschaften des assoziierten gewichteten Ringes}
	\begin{proposition}<+->
		Sei \(A\) ein noetherscher kommutativer Ring. Für ein Ideal \(\ideal a\) in \(A\) gilt:
		\begin{enumerate}[<+->]
		\item<.->
			Der Ring \(\Graded_{\ideal a}(A, t)\) ist noethersch.
		\item
			Es sind \(\Graded_{\ideal a}(A, t)\) und \(\Graded_{\hat{\ideal a}_{\ideal a}}(\hat A_{\ideal a}, t)\)
			als gewichtete Ringe isomorph.
		\item
			Für jeden endlich erzeugten \(A\)-Modul \(M\) zusammen mit einer stabilen \(\ideal a\)-Filtrierung
			\(M_\bullet\) ist \(\Graded(M_\bullet, t)\) ein endlich erzeugter \(\Graded_{\ideal a}(A, t)\)-Modul.
		\end{enumerate}
	\end{proposition}
	\begin{proof}<+->
		\begin{enumerate}[<+->]
		\item<.->
			Da \(A\) noethersch ist, ist \(\ideal a\) durch endlich viele \(x_1, \dotsc, x_n \in A\) erzeugt.
			Ist \(\bar x_i\) das Bild von \(x_i\) in \(\ideal a/\ideal a^2\), so ist \(\Graded_{\ideal a}(t)\) als
			\(A\)-Algebra von \(\bar x_1 t, \dotsc, \bar x_n t\) erzeugt. Damit ist
			\(\Graded_{\ideal a}(t)\) noethersch.
		\item
			\(\Graded_{\ideal a}(t) \cong \bigoplus\limits_n \ideal a^n/\ideal a^{n + 1} t^n \cong \bigoplus\limits_n
			\hat{\ideal a}^n/\hat{\ideal a}^{n + 1} t^n \cong \Graded_{\hat{\ideal a}}(t)\).
			\renewcommand{\qedsymbol}{}
			\qedhere
		\end{enumerate}
	\end{proof}
\end{frame}

\begin{frame}{Fortsetzung des Beweises zu Endlichkeitseigenschaften}
	\begin{proof}[Beweis, daß \(\Graded(M_\bullet, t)\) endlich erzeugt ist]<+->
		\begin{enumerate}[<+->]
		\item<.->
			Sei jetzt \(M\) ein endlich erzeugter \(A\)-Modul und \(M_\bullet\) eine stabile \(\ideal a\)-Filtrierung.
			Dann existiert ein \(n_0 \in \set N_0\) mit \(M_{n_0 + r} = \ideal a^r M_{n_0}\) für alle \(r \ge 0\).
		\item
			Folglich ist \(\Graded(M_\bullet, t)\) als \(\Graded_{\ideal a}(t)\)-Modul durch
			\(\bigoplus\limits_{n = 0}^{n_0} M_n/M_{n + 1} t^n\) erzeugt. Die \(M_n/M_{n + 1}\) sind endlich erzeugte
			\(A\)-Moduln, da \(A\) noethersch ist und \(M_n\) Untermodul des endlich erzeugten Moduls \(M\) ist.
		\item
			Da \(\ideal a \subset \ann M_n/M_{n + 1}\), ist \(M_n/M_{n + 1}\) auch ein endlich erzeugter
			\(A/\ideal a\)-Modul.
			Damit ist \(\bigoplus_{n = 0}^{n_0} M_n/M_{n + 1} t^n\) ein endlich erzeugter \(A/\ideal a\)-Modul.
		\item
			Es folgt, daß \(\Graded(M_\bullet, t)\) ein endlich erzeugter \(\Graded_{\ideal a}(t)\)-Modul ist.
			\qedhere
		\end{enumerate}
	\end{proof}
\end{frame}

\subsection{Endlichkeitseigenschaften der Vervollständigung}

\begin{frame}{Injektivität und Surjektivität von vervollständigten Homomorphismen}
	\begin{lemma}<+->
		Seien \(A\) und \(B\) zwei abelsche Gruppen (d.h.\ \(\set Z\)-Moduln) jeweils zusammen mit einer
		Filtrierung \(A_\bullet\) bzw.\ \(B_\bullet\), welche jeweils eine Umgebungsbasis um \(0\) einer
		Topologie auf \(A\) bzw.\ \(B\) bilden.
		\\
		Ist dann \(\phi\colon A \to B\) ein filtrierter Homomorphismus,
		so gilt:
		\begin{enumerate}[<+->]
		\item<.->
			Ist \(\Graded(\phi)\colon \Graded(A_\bullet, t) \to \Graded(B_\bullet, t)\) injektiv,
			so ist auch \(\hat\phi\colon \hat A \to \hat B\) injektiv.
		\item
			Ist \(\Graded(\phi)\colon \Graded(A_\bullet, t) \to \Graded(B_\bullet, t)\) surjektiv,
			so ist auch \(\hat\phi\colon \hat A \to \hat B\) surjektiv.
		\end{enumerate}
	\end{lemma}
\end{frame}

\begin{frame}[Beweis zur Injektivität und Surjektivität von vervollständigten Homomorphismen]
	\begin{proof}<+->
		\begin{enumerate}[<+->]
		\item<.->
			Die Reihen im folgenden Diagramm sind exakt:
			\[
				\begin{CD}
					0 @>>> A_n/A_{n + 1} @>>> A/A_{n + 1} @>>> A/A_n @>>> 0 \\
					& & @V{\psi_n}VV @V{\phi_{n + 1}}VV @V{\phi_n}VV \\
					0 @>>> B_n/B_{m + 1} @>>> B/B_{n + 1} @>>> B/B_n @>>> 0.
				\end{CD}
			\]
			Nach dem Schlangenlemma existiert damit eine exakte Sequenz
			\(0 \to \ker \psi_n \to \ker \phi_{n + 1} \to \ker \phi_n \to \coker \psi_n \to \coker \phi_{n + 1} \to \coker \phi_n \to 0	
			\).
		\item
			Sind die \(\psi_n\) injektiv bzw.\ surjektiv, folgt durch Induktion nach \(n\), daß die \(\phi_n\) injektiv bzw.\ surjektiv
			sind. Im letzteren Fall ist außerdem \((\ker \phi_n)_n\) ein surjektives System.
		\item
			Im ersten Fall ist \(\hat \phi\) injektiv, da der inverse Limes linksexakt ist.
		\item
			Im zweiten Falle ist \(\hat \phi\) surjektiv, da in diesem Falle \({\varprojlim\limits_n}^1 (\ker \phi_n) = 0\).
			\qedhere
		\end{enumerate}
	\end{proof}
\end{frame}

\begin{frame}{Moduln mit endlich erzeugtem assoziierten gewichteten Modul}
	\begin{proposition}<+->
		\label{prop:weighted_mod_is_ft}
		Sei \(\ideal a\) ein Ideal in einem kommutativen Ring \(A\). Sei \(M\) ein \(A\)-Modul zusammen mit
		einer \(\ideal a\)-Filtrierung \(M_\bullet\). Sei \(A\) vollständig bezüglich der \(\ideal a\)-adischen Topologie,
		und sei \(M\) in seiner Filtrationstopologie hausdorffsch, also \(\bigcap\limits_n M_n = 0\). Sei schließlich
		\(\Graded(M_\bullet, t)\) als \(\Graded_{\ideal a}(A, t)\)-Modul endlich erzeugt. Dann ist \(M\) ein endlich
		erzeugter \(A\)-Modul.
	\end{proposition}
\end{frame}

\begin{frame}{Beweis der Proposition}
	\begin{proof}<+->
		\begin{enumerate}[<+->]
		\item<.->
			Seien \(x_1, \dotsc, x_r\) mit \(x_i \in M_{n(i)}\) für \(n(i) \in \set N_0\), so daß die Bilder
			\(\bar x_i t^{n(i)} \in M_{n(i)}/M_{n(i) + 1}\) den \(\Graded_{\ideal a}(t)\)-Modul \(\Graded(M_\bullet)\) erzeugen.
		\item
			Für jedes \(i\) sei \(F^i\) der \(A\)-Modul \(A\) zusammen mit der stabilen \(\ideal a\)-Filtrierung \(F^i_\bullet\)
			mit	\(F^i_k = \ideal a^{k + n(i)}\). Seien \(F \coloneqq\bigoplus\limits_{i = 1}^r F^i, F_\bullet
			\coloneqq \bigoplus\limits_{i = 1}^r F^i_\bullet\). Dann ist
			\(\phi\colon F \to M, (a_1, \dotsc, a_r) \mapsto a_1 x_1 + \dotsb a_r x_r\) ein Homomorphismus filtrierter Gruppen.
		\item
			Der induzierte Homomorphismus \(\Graded(\phi)\colon \Graded(F_\bullet) \to \Graded(M_\bullet)\) ist nach
			Konstruktion surjektiv. Damit ist der Homomorphismus \(\hat\phi\colon \hat F \to \hat M\) zwischen den
			Vervollständigungen surjektiv.
		\item
			Da \(F \cong A^r\) als topologische \(A\)-Moduln und \(A\) vollständig ist, ist auch \(F \cong \hat F\).
			Damit ist \(F \to \hat F \to \hat M\) surjektiv. Da \(M \to \hat M\) aufgrund der Hausdorffeigenschaft von
			\(M\) injektiv ist, muß damit \(\phi\colon F \to M\) surjektiv sein. Folglich ist \(M\)
			als \(A\)-Modul von den \(x_i\) erzeugt.
			\qedhere
		\end{enumerate}
	\end{proof}
\end{frame}

\begin{frame}{Moduln mit noetherschen assoziierten gewichteten Moduln}
	\begin{corollary}<+->
		Sei \(\ideal a\) ein Ideal in einem kommutativen Ring \(A\). Sei \(M\) ein \(A\)-Modul zusammen mit
		einer \(\ideal a\)-Filtrierung \(M_\bullet\). Sei \(A\) vollständig bezüglich der \(\ideal a\)-adischen Topologie,
		und sei \(M\) in seiner Filtrationstopologie hausdorffsch, also \(\bigcap\limits_n M_n = 0\). Sei schließlich
		\(\Graded(M_\bullet, t)\) ein noetherscher \(\Graded_{A, \ideal a}(t)\)-Modul. Dann ist \(M\) ein noetherscher
		\(A\)-Modul.
	\end{corollary}
	\begin{proof}<+->
		\begin{enumerate}[<+->]
		\item<.->
			Es ist zu zeigen, daß jeder Untermodul \(M'\) von \(M\) endlich erzeugt ist. Es \(M'_\bullet \coloneqq
			M' \cap M_\bullet\) eine \(\ideal a\)-Filtration auf \(M'\), und die Einbettung \(M' \to M\) induziert
			eine Einbettung \(\Graded(M'_\bullet, t) \to \Graded(M_\bullet, t)\).
		\item
			Da \(\Graded(M_\bullet, t)\) noethersch ist, ist \(\Graded(M'_\bullet, t)\) endlich erzeugt.
		\item
			Wegen \(\bigcap\limits_n M'_n \subset \bigcap\limits_n M_n = (0)\) ist \(M'\) schließlich hausdorffsch.
			Nach der Proposition ist \(M'\) damit endlich erzeugt.
			\qedhere
		\end{enumerate}
	\end{proof}
\end{frame}

\begin{frame}{Endlichkeitseigenschaft der Vervollständigung}
	\begin{theorem}<+->
		\label{thm:compl_is_noeth}
		Sei \(A\) ein noetherscher kommutativer Ring. Für jedes Ideal \(\ideal a\) von \(A\) ist
		die Vervollständigung \(\hat A_{\ideal a}\) noethersch.
	\end{theorem}
	\begin{proof}<+->
		Es ist \(\Graded_{\ideal a}(t) \cong \Graded_{\hat {\ideal a}}(t)\), und diese Ringe sind noethersch.
		Damit können wir die letzte Folgerung auf den vollständigen Ring \(\hat A\) und den \(\hat A\)-Modul
		\(\hat A\) mit der \(\hat{\ideal a}\)-adischen Filtrierung anwenden (welche hausdorffsch ist) und erhalten,
		daß \(\hat A\) ein noetherscher \(\hat A\)-Modul ist, also ein noetherscher Ring.
	\end{proof}
\end{frame}

\begin{frame}{Beispiel des Potenzreihenrings}
	\begin{corollary}<+->
		Für jeden noetherschen kommutativen Ring \(A\) ist der Potenzreihenring \(\ps A{X_1, \dotsc, X_n}\) in \(n\)
		Variablen noethersch.
	\end{corollary}
	\begin{proof}<+->
		Nach dem Hilbertschen Basissatz ist \(A[X_1, \dotsc, X_n]\) noethersch. Es ist \(\ps A{X_1, \dotsc, X_n}\) die
		\((X_1, \dotsc, X_n)\)-adische Vervollständigung von \(A[X_1, \dotsc, X_n]\).
	\end{proof}
	\begin{example}<+->
		Für jeden Körper \(K\) ist \(\ps K{X_1, \dotsc, X_n}\) noethersch.
	\end{example}
\end{frame}

