\section{Exakte Sequenzen}

\subsection{Definition und erste Eigenschaften}

\begin{frame}{Definition}
	\begin{definition}<+->
		Sei \(A\) ein Ring.
		Eine Sequenz
		\[
			\dotsc \to M^{i - 1} \xrightarrow{\phi^{i - 1}} M^i \xrightarrow{\phi^i} M^{i + 1}
			\to \dotsc
		\]
		von \(A\)-Moduln und \(A\)-Modulhomomorphismen heißt \emph{exakt bei \(M^i\)}, falls
		\(\im \phi^{i - 1} = \ker \phi^i\).
		\\
		Die Sequenz heißt \emph{exakt}, falls sie exakt bei jedem \(M^i\) ist.
	\end{definition}
\end{frame}

\begin{frame}{Injektivität und Surjektivität und exakte Sequenzen}
	Sei \(A\) ein Ring.
	\begin{example}<+->
		Eine Sequenz der Form \(0 \to M' \xrightarrow\phi M\) von \(A\)-Moduln ist genau dann exakt,
		wenn \(\phi\) injektiv ist.
	\end{example}	
	\begin{example}<+->
		Eine Sequenz der Form \(M \xrightarrow{\psi} M'' \to 0\) von \(A\)-Moduln ist genau dann exakt,
		wenn \(\psi\) surjektiv ist.
	\end{example}
\end{frame}

\begin{frame}{Kurze exakte Seqenzen}
	Sei \(A\) ein Ring.
	\begin{definition}<+->
		Eine \emph{kurze exakte Seqenz von \(A\)-Moduln} ist eine exakte Sequenz der Form
		\(0 \to M' \to M \to M'' \to 0\).
	\end{definition}
	\begin{example}<+->
		Eine Seqenz der Form \(0 \to M' \xrightarrow\phi M \xrightarrow\psi M'' \to 0\) ist genau dann
		exakt, wenn \(\phi\) injektiv ist, \(\psi\) surjektiv ist und \(\psi\) einen Isomorphismus
		\(\coker \phi = M/\im \phi \cong M''\) induziert.
	\end{example}
	\begin{example}<+->
		Jede lange exakte Sequenz \(\dotsc \to M^{i - 1} \xrightarrow{\phi^{i - 1}} M^i
		\xrightarrow{\phi^i} M^{i + 1}	\to \dotsc\) zerfällt in kurze exakte Sequenzen:
		\\
		Ist \(N^i = \im \phi^{i - 1} = \ker \phi^i\), haben wir kurze exakte Sequenzen
		\(0 \to N^i \to M^i \to N^{i + 1} \to 0\) für alle \(i\).
	\end{example}
\end{frame}

\begin{frame}{Rechtsexakte Sequenzen}
	\begin{proposition}<+->
		Sei \(A\) ein kommutativer Ring. Eine Sequenz
		\(E\colon M' \xrightarrow\phi M \xrightarrow\psi M'' \to 0\) von \(A\)-Moduln ist
		genau dann exakt, wenn für alle \(A\)-Moduln \(N\) auch
		\(\Hom(E, N)\colon 0 \to \Hom(M'', N) \xrightarrow{\psi^*} \Hom(M, N) \xrightarrow{\phi^*} \Hom(M', N)\)
		exakt ist.
	\end{proposition}
	\begin{proof}<+->
		\begin{enumerate}[<+->]
		\item<.->
			Wir zeigen eine Richtung: Sei die Hom-Sequenz exakt für alle \(N\). Wir wählen \(N = \coker \psi\).
			Ist \(\pi\colon M'' \to N\) die kanonische Projektion, so ist
			\(\pi \circ \psi = 0\). Da \(\psi^*\) injektiv ist, folgt \(\pi = 0\), also
			\(N = 0\). Damit ist \(\psi\) surjektiv.
		\item
			Für \(N = M''\) ist \(\psi \circ \phi = \phi^* \psi^*(\id_{M''}) = 0\),
			also \(\im \phi \subset \ker \psi\).
		\item
			Wir wählen \(N = \coker \phi\). Ist \(\pi\colon M \to N\) die kanonische Projektion,
			so ist \(\pi \circ \phi = 0\), also \(\pi \in \ker \phi^*\). Damit existiert ein
			\(\xi\colon M'' \to N\) mit \(\pi = \xi \circ \psi\), also \(\im \phi = \ker \pi \supset
			\ker \psi\).
			\qedhere
		\end{enumerate}
	\end{proof}
\end{frame}

\begin{frame}{Linksexakte Sequenzen}
	\begin{proposition}<+->
		Sei \(A\) ein kommutativer Ring. Eine Sequenz
		\(F\colon 0 \to N' \xrightarrow\phi N \xrightarrow\psi N''\) von \(A\)-Moduln ist
		genau dann exakt, wenn für alle \(A\)-Moduln \(M\) die Sequenz
		\(\Hom(M, F)\colon 0 \to \Hom(M, N') \xrightarrow{\phi_*} \Hom(M, N) \xrightarrow{\psi_*} \Hom(M, N'')\)
		exakt ist.
		\qed
	\end{proposition}
\end{frame}

\subsection{Das Schlangenlemma}

\begin{frame}{Das Schlangenlemmas}
	\begin{proposition}[Schlangenlemma]<+->
		Sei \(A\) ein Ring. Ist
		\[
			\begin{CD}
				0 @>>> M' @>{\alpha}>> M @>{\beta}>> M'' @>>> 0 \\
				& & @V{\phi'}VV @V{\phi}VV @V{\phi''}VV \\
				0 @>>> N' @>{\alpha'}>> N @>{\beta'}>> N'' @>>> 0
			\end{CD}
		\]
		ein kommutatives Diagramm von \(A\)-Moduln mit exakten Zeilen, so
		existiert eine kanonische exakte Sequenz der Form
		\(0 \to \ker \phi' \xrightarrow{\alpha_*} \ker \phi \xrightarrow{\beta_*} \ker \phi''
			\xrightarrow{\delta}
			\coker \phi' \xrightarrow{\alpha'_*} \coker \phi \xrightarrow{\beta'_*}
			\coker \phi'' \to 0\).
		\qed
	\end{proposition}
	\begin{visibleenv}<+->
		Der \emph{(Ko-)Randhomomorphismus \(\delta\)} ist folgendermaßen definiert: Sei \(x'' \in \ker 
		\phi''\). Dann ist \(x'' = \beta(x)\) für ein \(x \in M\).
		\\
		Da \(\beta'(\phi(x)) = \phi''(\beta(x)) = 0\), existiert ein \(y' \in N'\) mit \(\alpha'(y') =
		\phi(x)\).
		\\
		Schließlich ist \(\delta(x'')\) das Bild von \(y'\) in \(\coker \phi'\).
	\end{visibleenv}
\end{frame}

\begin{frame}{Die lange exakte Kohomologiesequenz}
	\begin{remark}
		Das Schlangenlemma ist ein Spezialfall der langen exakten Kohomologiesequenz der
		homologischen Algebra.
	\end{remark}
\end{frame}

\subsection{Additive Funktionen}

\begin{frame}{Definition additiver Funktionen}
	\begin{definition}<+->
		Sei \(A\) ein Ring. Sei \(\mathfrak C\) eine Klasse von \(A\)-Moduln. Eine Abbildung
		\(\lambda\colon \mathfrak C \to G\) in eine abelsche Gruppe heißt \emph{additive Funktion},
		falls für alle kurzen exakten Sequenzen
		\(0 \to C' \to C \to C'' \to 0\) von Moduln aus \(\mathfrak C\) gilt, daß \(\lambda(C) =
		\lambda(C') + \lambda(C'')\).
	\end{definition}
	\begin{example}<+->
		Seien \(K\) ein Körper und \(\mathfrak C\) die Klasse der endlich-dimensionalen \(K\)-Vektorräume.
		Dann ist \(\dim\colon \mathfrak C \to \set Z\) eine additive Funktion.
	\end{example}
\end{frame}

\begin{frame}{Additive Funktionen und beschränkte Sequenzen}
	\begin{proposition}<+->
		Sei \(A\) ein Ring. Sei \(\mathfrak C\) eine Klasse von \(A\)-Moduln und \(\lambda\colon \mathfrak C
		\to G\) eine additive Funktion. Sei
		\(0 \to M^0 \xrightarrow{\phi^0} M^1 \xrightarrow{\phi^1} \dotsb \to M^n \to 0\) eine exakte Sequenz von Moduln in 
		\(\mathfrak C\),
		so daß auch die Kerne der \(\phi^i\) zu \(\mathfrak C\) gehören.
		\\
		Dann gilt \(\sum\limits_{i = 0}^n (-1)^i \lambda(M^i) = 0\).
	\end{proposition}
	\begin{proof}<+->
		\begin{enumerate}[<+->]
		\item<.->
			Die exakte Sequenz zerfällt in kurze exakte Sequenzen der Form
			\(0 \to N^i \to M^i \to N^{i + 1} \to 0\) (mit \(N^0 = N^{n + 1} = 0\)),
			wobei \(N^i \in \mathfrak C\).
		\item
			Daher gilt \(\lambda(M^i) = \lambda(N^i) + \lambda(N^{i + 1})\). Addieren wir diese Gleichungen
			alternierend, hebt sich die rechte Seite weg.
			\qedhere
		\end{enumerate}
	\end{proof}
\end{frame}

