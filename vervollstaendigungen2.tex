\section{Vervollständigungen II}

\subsection{Exaktheitseigenschaften inverser Limiten}

\begin{frame}{Surjektive Systeme}
	\begin{definition}<+->
		Ein inverses System \(\dotsb \to A_1 \to A_0\) von Gruppen heißt \emph{surjektives System}, falls
		die Gruppenhomomorphismen \(A_n \to A_{n - 1}\) surjektiv sind.
	\end{definition}
	\begin{example}<+->
		Sei \(G\) eine topologische Gruppe mit einer Umgebungsbasis \(G_0 \supset G_1 \supset \dotsb\) von
		\(0\) aus Normalteilern. Dann ist das inverse System \(\dotsb \to G/G_n \to \dotsb \to G/G_1 \to G/G_0\)
		ein surjektives.
	\end{example}
\end{frame}

\begin{frame}{Homomorphismen inverser Systeme}
	\begin{definition}<+->
		Seien \(A_\bullet\colon \dotsb \xrightarrow{\alpha_1} A_1 \xrightarrow{\alpha_0} A_0\) und
		\(B_\bullet\colon \dotsb \xrightarrow{\beta_1} B_1 \xrightarrow{\beta_0} B_0\) inverse Systeme
		topologischer Gruppen. Eine Familie \((\phi_n\colon A_n \to B_n)_n\) stetiger Gruppenhomomorphismen
		ist ein \emph{Homomorphismus inverser Systeme}, wenn \(\beta_n \circ \phi_n = \phi_{n - 1} \circ \alpha_n\colon A_n \to
		B_{n - 1}\) für alle \(n \ge 1\).
	\end{definition}
	\begin{example}<+->
		Sind \(\phi_\bullet\colon A_\bullet \to B_\bullet\)
		und \(\psi_\bullet\colon B_\bullet \to C_\bullet\) zwei Homomorphismen inverser Systeme, so ist ihre
		\emph{Verknüpfung} \(\psi_\bullet \circ \phi_\bullet \coloneqq (\psi_n \circ \phi_n)\colon A_\bullet \to
		C_\bullet\) wieder ein Homomorphismus inverser Systeme.
	\end{example}
	\begin{example}<+->
		Ist \(A_\bullet\) ein inverses System, so ist \(\id_{A_\bullet} \coloneqq (\id_{A_n})\colon
		A_\bullet \to A_\bullet\) ein Homomorphismus inverser Systeme.
	\end{example}
\end{frame}

\begin{frame}{Homomorphismen inverser Systeme und der inverse Limes}
	Sei \(\phi_\bullet\colon A_\bullet \to B_\bullet\) ein Homomorphismus inverser Systeme. Dann definiert
	\[
		\varprojlim\limits_n \phi_n\colon \varprojlim\limits_n A_n \to \varprojlim\limits_n B_n,
		(\xi_n)_n \mapsto (\phi_n(\xi_n))_n
	\]
	einen stetigen Gruppenhomomorphismus.
	\begin{example}<+->
		Ist \(\psi_\bullet\colon B_\bullet \to C_\bullet\) ein weiterer Homomorphismus inverser Systeme,
		so ist
		\((\varprojlim\limits_n \psi_n) \circ (\varprojlim\limits_n \phi_n) = \varprojlim\limits_n (\psi_n \circ \phi_n)\colon
		\varprojlim\limits_n A_n \to \varprojlim\limits_n C_n\).
	\end{example}
	\begin{example}<+->
		Es ist \(\varprojlim\limits_n \id_{A_n} = \id_{\varprojlim\limits A_n}\).
	\end{example}
\end{frame}

\begin{frame}{Exakte Sequenzen inverser Systeme}
	\begin{definition}<+->
		Eine Sequenz \(\dotsb \to A_\bullet^{i - 1} \xrightarrow{\phi_\bullet^{i - 1}} A_\bullet^i \xrightarrow{\phi_\bullet^i}
		A_\bullet^{i + 1} \to \dotsb\) inverser Systeme abelscher Gruppen \(A_\bullet^i\)
		heißt \emph{exakt bei \(A_\bullet^i\)},
		falls die induzierten Sequenzen \(A_n^{i - 1} \xrightarrow{\phi_n^{i - 1}} A_n^i \xrightarrow{\phi_n^i} A_n^{i + 1} \to \dotsb\)
		für alle \(n \in \set N_0\) exakt sind.
	\end{definition}
	\begin{visibleenv}<+->
		Den Begriff der kurzen exakten Sequenz inverser Systeme abelscher Gruppen definieren wir auf die offensichtliche Art
		und Weise.
	\end{visibleenv}
	\begin{remark}<+->
		Ist \(\dotsb \to A_\bullet^{i - 1} \xrightarrow{\phi_\bullet^{i - 1}} A_\bullet^i \xrightarrow{\phi_\bullet^i}
		A_\bullet^{i + 1} \to \dotsb\) eine exakte Sequenz inverser Systeme abelscher Gruppen, so ist die
		induzierte Sequenz
		\(\dotsb \to \varprojlim\limits_n A_n^{i - 1} \to \varprojlim\limits_n A_n^i \to \varprojlim\limits_n A_n^{i + 1} \to \dotsb\)
		abelscher Gruppen im allgemeinen nicht mehr exakt.
	\end{remark}
\end{frame}

\begin{frame}{Linksexaktheit des inversen Limes}
	\begin{proposition}<+->
		Sei \(0 \to A_\bullet \to B_\bullet \to C_\bullet \to 0\) eine exakte Sequenz inverser Systeme abelscher Gruppen. Dann
		ist die induzierte Sequenz \(0 \to \varprojlim\limits_n A_n \to \varprojlim\limits_n B_n \to \varprojlim\limits_n C_n\)
		exakt.
	\end{proposition}
\end{frame}

\begin{frame}{Beweis der Linksexaktheit}
	\begin{proof}<+->
		\begin{enumerate}[<+->]
		\item<.->
			Sei \(A_\bullet\colon \dotsb \xrightarrow{\alpha_2} A_1 \xrightarrow{\alpha_1} A_0\).
			Sei \(A \coloneqq \prod\limits_n A_n\). Setze \(d^A\colon A \to A, (a_n)_n \mapsto
			(a_n - \alpha_{n + 1}(a_{n + 1}))_n\). Dann ist \(\ker d^A = \varprojlim\limits_n A_n\).
		\item
			Definiere \(B, C\) und \(d^B, d^C\) analog. Dann ist
			\[
				\begin{CD}
					0 @>>> A @>>> B @>>> C @>>> 0 \\
					& & @V{d^A}VV @V{d^B}VV @V{d^C}VV \\
					0 @>>> A @>>> B @>>> C @>>> 0
				\end{CD}
			\]
			ein kommutatives Diagram mit exakten Reihen.
		\item
			Nach dem Schlangenlemma ist \(0 \to \ker d^A \to \ker d^B \to \ker d^C \to {\varprojlim\limits_n}^1 A_n\)
			mit \({\varprojlim\limits_n}^1 A_n \coloneqq \coker d^A\) exakt.
			\qedhere
		\end{enumerate}
	\end{proof}
\end{frame}

\begin{frame}{Exaktheit bei surjektiven Systemen}
	\begin{proposition}<+->
		Sei \(0 \to A_\bullet \to B_\bullet \to C_\bullet \to 0\) eine exakte Sequenz inverser Systeme abelscher Gruppen. 
		Sei weiter \(A_\bullet\) ein surjektives System.
		Dann
		ist die induzierte Sequenz \(0 \to \varprojlim\limits_n A_n \to \varprojlim\limits_n B_n \to \varprojlim\limits_n C_n \to 0\)
		exakt.
	\end{proposition}
	\begin{proof}<+->
		Sei \(A_\bullet\colon \dotsb \xrightarrow{\alpha_2} A_1 \xrightarrow{\alpha_1} A_0\).
		Es ist zu zeigen, daß \({\varprojlim\limits_n}^1 A_n = 0\), daß also \(d^A\) surjektiv ist. Dies folgt aus der Tatsache,
		daß sich die Gleichungen \(\eta_n - \alpha_{n + 1}(\eta_{n + 1}) = \xi_n\) induktiv für \(\eta_n \in A_n\) lösen lassen, wenn
		die \(\xi_n \in A\) gegeben sind.
	\end{proof}
\end{frame}

\subsection{Vollständige topologische Gruppen}

\begin{frame}{Exaktheit der Vervollständigung}
	\begin{corollary}<+->
		Sei \(0 \to G' \xrightarrow{\iota} G \xrightarrow{\pi} G'' \to 0\) eine exakte Sequenz
		abelscher Gruppen. Definiere eine Folge \(G_0 \supset G_1 \supset \dotsb\) von
		Untergruppen von \(G\) eine Topologie auf \(G\). Versehen wir \(G'\) und \(G''\)
		mit den durch die Folgen \(\iota^{-1} G_0 \supset \iota^{-1} G_1 \supset \dotsb\) bzw.\
		\(\pi(G_0) \supset \pi(G_1) \supset \dotsb\) definierten Topologien, so ist
		die induzierte Sequenz
		\(0 \to \hat G' \to \hat G \to \hat G'' \to 0\) exakt.
	\end{corollary}
	\begin{proof}<+->
		Die Sequenzen \(0 \to G'/\iota^{-1} (G_n) \to G/G_n \to G/\pi(G_n) \to 0\) sind exakt
		und bilden eine exakte Sequenz surjektiver Systeme. Damit ist ihr inverser Limes
		\(0 \to \hat G' \to \hat G \to \hat G'' \to 0\) exakt.
	\end{proof}
\end{frame}

\begin{frame}{Folgerung aus der Exaktheit}
	\begin{corollary}<+->
		Sei \(G\) eine abelsche topologische Gruppe. Sei \(G_0 \supset G_1 \supset \dotsb\) eine
		Umgebungsbasis von \(0\) aus Untergruppen. Für jedes \(n\) induziert der kanonische Homomorphismus
		\(G \to \hat G\) dann einen Isomorphismus \(G/G_n \to \hat G/\hat G_n\).
	\end{corollary}
	\begin{proof}<+->
		Es ist \(0 \to G_n \to G \to G/G_n \to 0\) eine exakte Sequenz. Versehen wir \(G_n\) mit
		der Teilraumtopologie und \(G/G_n\) mit der diskreten Topologie,
		können wir die vorherige Proposition anwenden und erhalten die exakte Sequenz
		\(0 \to \hat{G_n} \to \hat G \to \widehat{G/G_n} = G/G_n \to 0\).
	\end{proof}
\end{frame}

\begin{frame}{Vollständige topologische Gruppen}
	\begin{definition}<+->
		Eine topologische Gruppe heißt \emph{vollständig}, wenn der kanonische Homomorphismus
		\(G \to \hat G\) ein Isomorphismus ist.
	\end{definition}
	\begin{remark}<+->
		Eine vollständige topologische Gruppe ist also insbesondere hausdorffsch.
	\end{remark}
	\begin{proposition}<+->
		Sei \(G\) eine abelsche topologische Gruppe. Sei \(G_0 \supset G_1 \supset \dotsb\) eine
		Umgebungsbasis von \(0\) aus Untergruppen. Dann ist die Vervollständigung \(\hat G\) vollständig.
	\end{proposition}
	\begin{proof}<+->
		Es ist \(\hat G \cong \varprojlim\limits_n G/G_n \cong \varprojlim\limits_n \hat G/\hat G_n
		\cong \hat{\hat G}\).
	\end{proof}
\end{frame}

\subsection{Topologische Ringe und Moduln}

\begin{frame}{Topologische Ringe}
	\begin{definition}<+->
		Ein \emph{topologischer Ring} ist ein Ring \(A\), dessen additive Gruppe eine topologische Gruppe ist, so
		daß die Multiplikation \(A \times A \to A, (x, y) \mapsto x y\) eine stetige Abbildung ist.
	\end{definition}
	\begin{example}<+->
		Sei \(\ideal a\) ein Ideal eines Ringes \(A\). Dann gibt es genau eine Topologie auf
		der additiven Gruppe von \(A\), so daß \((1) \supset \ideal a \supset \ideal a^2 \supset \dotsb\) zu einer
		Umgebungsbasis von~\(0\) wird, die \emph{\(\ideal a\)-adische Topologie}.
		\\
		Da die \(\ideal a^n\) Ideale sind, läßt sich zeigen, daß \(A\) damit zu einem topologischen Ring wird.
	\end{example}
	\begin{remark}<+->
		Die Vervollständigung \(\hat A\) eines topologischen Ringes \(A\) als topologische Gruppe ist wieder in kanonischer Weise
		ein Ring. Außerdem ist die kanonische Abbildung \(A \to \hat A\) ein stetiger Ringhomomorphismus.
	\end{remark}
\end{frame}

\begin{frame}{Vervollständigung topologischer Ringe}
	Sei \(\ideal a\) ein Ideal eines Ringes \(A\). Wir versehen \(A\) mit der \(\ideal a\)-adischen Topologie.
	\begin{example}<+->
		Es ist \(A\) genau dann hausdorffsch, wenn \(\bigcap\limits_n \ideal a^n = (0)\).
	\end{example}
	\begin{visibleenv}<+->
		Mit \(\hat A = \hat A_{\ideal a}\) bezeichnen wir die Vervollständigung von \(A\).
	\end{visibleenv}
	\begin{definition}<.->
		Der Ring \(\hat A_{\ideal a}\) heißt die \emph{\(\ideal a\)-adische Vervollständigung von \(A\)}.
	\end{definition}
\end{frame}

\begin{frame}{Topologische Moduln}
	Sei \(A\) ein topologischer Ring. 
	\begin{definition}<+->
		Ein \emph{topologischer \(A\)-Modul \(M\)} ist ein \(A\)-Modul \(M\), dessen
		additive Gruppe eine topologische Gruppe ist, so daß die Multiplikation \(A \times M \to M, (a, x) \mapsto a x\)
		eine stetige Abbildung ist.
	\end{definition}
	\begin{remark}<+->
		Die Vervollständigung \(\hat M\) eines topologischen \(A\)-Moduls \(M\) als topologische Gruppe ist in kanonischer Weise
		ein \(\hat A\)-Modul. Außerdem ist die kanonische Abbildung \(M \to \hat M\) ein stetiger Homomorphismus
		\(M \to \hat M^{A}\) von \(A\)-Moduln.
	\end{remark}
\end{frame}

\begin{frame}{Vervollständigung topologischer Moduln}
	Sei \(\ideal a\) ein Ideal eines Ringes \(A\). Sei \(M\) ein \(A\)-Modul. 
	\begin{example}<+->
		Sei \(\ideal a\) ein Ideal eines Ringes \(A\). Sei \(M\) ein \(A\)-Modul. Dann gibt es genau eine Topologie
		auf der additiven Gruppe von \(M\), so daß \(M \supset \ideal a M \supset \ideal a^2 M \supset \dotsb\) zu einer
		Umgebungsbasis von \(0\) wird, die \emph{\(\ideal a\)-adische Topologie}.
		\\
		In der Tat wird \(M\) damit zu einem topologischen \(A\)-Modul, wenn \(M\) mit der \(\ideal a\)-adischen Topologie
		versehen wird.
	\end{example}
	\begin{visibleenv}<+->
		Wir versehen \(A\) und \(M\) mit der
		\(\ideal a\)-adischen Topologie.
		\\
		Mit \(\hat M = \hat M_{\ideal a}\) bezeichnen wir die Vervollständigung von \(M\).
	\end{visibleenv}
	\begin{definition}<+->
		Der \(\hat A_{\ideal a}\)-Modul \(\hat M_{\ideal a}\) heißt die \emph{\(\ideal a\)-adische Vervollständigung von \(M\)}.
	\end{definition}
\end{frame}

\begin{frame}{Vervollständigung von Homomorphismen}
	Sei \(\ideal a\) ein Ideal eines Ringes \(A\). Sei \(\phi\colon M \to N\) ein Homomorphismus von \(A\)-Moduln.
	\\
	Dann gilt \(\phi(\ideal a^n M) \subset \ideal a^n \phi(M) \subset \ideal a^n N\), es ist
	\(\phi\) damit stetig bezüglich der \(\ideal a\)-adischen Topologien auf \(M\) und \(N\).
	\\
	Damit definiert \(\phi\) eine Abbildung \(\hat\phi = \hat\phi_{\ideal a}\colon \hat M_{\ideal a} \to \hat N_{\ideal a}\).
	\begin{proposition}<+->
		Die Abbildung \(\hat\phi_{\ideal a}\colon \hat M_{\ideal a} \to \hat N_{\ideal a}\) ist ein stetiger
		Homomorphismus topologischer Ringe.
		\qed
	\end{proposition}
\end{frame}

\begin{frame}{Beispiele Vervollständigungen topologischer Ringe}
	\begin{example}<+->
		Sei \(K\) ein Körper. Vervollständigen wir \(K[x]\) bezüglich der \((x)\)-adischen Topologie, so erhalten
		wir \(\widehat{K[x]}_{(x)} = \ps K x\) den Potenzreihenring über \(K\) als Vervollständigung.
	\end{example}
	\begin{example}<+->
		Sei \(p\) eine Primzahl. Die Vervollständigung \(\set Z_p \coloneqq \hat{\set Z}_{(p)}\) heißt der
		\emph{Ring der \(p\)-adischen Zahlen}. Elemente in \(\set Z\) können wir als Reihen
		\(\sum_{n = 0}^\infty a_n p^n\) mit \(0 \le a_n < p\) darstellen.
		\\
		Es gilt \(\lim\limits_n p^n = 0\) in diesem Ring.
	\end{example}
\end{frame}

