\section{Lokale Eigenschaften}

\subsection{Trivialität von Moduln}

\begin{frame}{Lokale Eigenschaften}
	Wir nennen eine Eigenschaft kommutativer Ringe (bzw.~Moduln über einem kommutativen Ring) \emph{lokal},
	falls folgendes gilt:
	
	Ein kommutativer Ring \(A\) (bzw.~ein Modul \(M\) über einem kommutativen Ring) hat die Eigenschaft genau dann, wenn
	alle seine Halme \(A_{\ideal p}\) (bzw.~\(M_{\ideal p}\)) an allen Primidealen \(\ideal p\) die Eigenschaft hat.
\end{frame}

\begin{frame}{Trivialität von Moduln}
	\begin{proposition}<+->
		Sei \(A\) ein kommutativer Ring. Sei \(M\) ein \(A\)-Modul. Dann sind folgende Aussagen äquivalent:
		\begin{enumerate}[<+->]
		\item<.->
			\(M = 0\).
		\item
			\(M_{\ideal p} = 0\) für alle Primideale \(\ideal p\) von \(A\).
		\item
			\(M_{\ideal m} = 0\) für alle maximalen Ideale \(\ideal m\) von \(A\).
		\end{enumerate}
	\end{proposition}
	\begin{proof}<+->
		\begin{enumerate}[<+->]
		\item<.->
			Aus der ersten folgt sicherlich die zweite Aussage und aus der zweiten die dritte. 
		\item
			Sei \(M_{\ideal m} = 0\) für alle maximalen Ideale \(\ideal m\) von \(A\). Sei \(x \in M\).
			Angenommen \(x \neq 0\).
			Dann ist das Ideal \(\ann(x)\) in einem maximalen Ideal \(\ideal m\) von
			\(A\) enthalten.
		\item
			Es ist \(\frac x 1 = 0 \in M_{\ideal m}\), also \(t x = 0\) für ein \(t \in A \setminus \ideal m\).
			Damit ist \(t \notin \ann(x)\), Widerspruch.
		\qedhere
		\end{enumerate}
	\end{proof}
\end{frame}

\subsection{Injektivität und Surjektivität}

\begin{frame}{Injektivität ist eine lokale Eigenschaft}
	\begin{proposition}<+->
		\label{prop:inj_is_local}
		Sei \(A\) ein kommutativer Ring. Sei \(\phi\colon M \to N\) ein
		Homomorphismus von \(A\)-Moduln. Dann
		sind folgende Aussagen äquivalent:
		\begin{enumerate}[<+->]
		\item<.->
			\(\phi\colon M \to N\) ist injektiv.
		\item
			\(\phi_{\ideal p}\colon M_{\ideal p} \to N_{\ideal p}\) ist injektiv für alle Primideale \(\ideal p\) von \(A\).
		\item
			\(\phi_{\ideal m}\colon M_{\ideal m} \to N_{\ideal m}\) ist injektiv für alle maximalen Ideale \(\ideal m\) von \(A\).
		\end{enumerate}
	\end{proposition}
	\begin{proof}<+->
		\begin{enumerate}[<+->]
		\item<.->
			Lokalisierung erhält Injektivität.
		\item
			Jedes maximale Ideal ist ein Primideal.
		\item
			Sei \(\phi_{\ideal m}\) für alle maximalen Ideale \(\ideal m\) injektiv. Sei \(M' = \ker \phi\).
			Dann ist \(0 \to M' \to M \to N\) exakt, also auch \(0 \to M'_{\ideal m} \to M_{\ideal m} \to N_{\ideal m}\).
			Damit ist \(M'_{\ideal m} = \ker \phi_{\ideal m} = 0\). Es folgt nach der letzten Aussage, daß \(M' = 0\).
			Also ist \(\phi\) injektiv.
			\qedhere
		\end{enumerate}
	\end{proof}
\end{frame}

\begin{frame}{Surjektivität ist eine lokale Eigenschaft}
	\begin{proposition}<+->
		Sei \(A\) ein kommutativer Ring. Sei \(\phi\colon M \to N\) ein Homomorphismus von \(A\)-Moduln. Dann
		sind folgende Aussagen äquivalent:
		\begin{enumerate}[<+->]
		\item<.->
			\(\phi\colon M \to N\) ist surjektiv.
		\item
			\(\phi_{\ideal p}\colon M_{\ideal p} \to N_{\ideal p}\) ist surjektiv für alle Primideale \(\ideal p\) von \(A\).
		\item
			\(\phi_{\ideal m}\colon M_{\ideal m} \to N_{\ideal m}\) ist surjektiv für alle maximalen Ideale \(\ideal m\) von \(A\).
			\qed
		\end{enumerate}
	\end{proposition}
\end{frame}

\subsection{Flachheit}

\begin{frame}{Flachheit ist eine lokale Eigenschaft}
	\begin{proposition}<+->
		Sei \(A\) ein kommutativer Ring. Sei \(M\) ein \(A\)-Modul. Dann sind folgende Aussagen äquivalent:
		\begin{enumerate}[<+->]
		\item<.->
			\(M\) ist ein flacher \(A\)-Modul.
		\item
			\(M_{\ideal p}\) ist ein flacher \(A_{\ideal p}\)-Modul für alle Primideale \(\ideal p\) von \(A\).
		\item
			\(M_{\ideal m}\) ist ein flacher \(A_{\ideal m}\)-Modul für alle maximalen Ideale \(\ideal m\) von \(A\).
		\end{enumerate}
	\end{proposition}
	\begin{proof}<+->
		\begin{enumerate}[<+->]
		\item<.->
			\(M_{\ideal p} = A_{\ideal p} \otimes_A M\) und
			Skalarerweiterung erhält Flachheit.
		\item
			Jedes maximale Ideal ist ein Primideal.
		\item
			Sei \(M_{\ideal m}\) flach über \(A_{\ideal m}\) für alle maximalen Ideale \(\ideal m\). Sei \(N \to P\) ein
			injektiver Homomorphismus von \(A\)-Moduln. Es folgt, daß \(N_{\ideal m} \to P_{\ideal m}\) injektiv ist.
			Nach Voraussetzung ist dann \((N \otimes_A M)_{\ideal m} = N_{\ideal m} \otimes_{A_{\ideal m}} M_{\ideal m} \to
			P_{\ideal m} \otimes_{A_{\ideal m}} M_{\ideal m} = (P \otimes_A M)_{\ideal m}\) injektiv. 
			Da \(\ideal m\) beliebig ist, folgt, daß \(N \otimes_A M \to P \otimes_A M\) injektiv ist.
			\qedhere
		\end{enumerate}
	\end{proof}
\end{frame}

