\subsection{Primärzerlegung}

\begin{exercise}
	Sei \(\ideal a\) ein zerlegbares Ideal in einem kommutativen Ring
	\(A\) mit \(\ideal a = \sqrt{\ideal a}\). Zeige, daß \(\ideal a\) keine
	assoziierten eingebetteten Primideale besitzt.
\end{exercise}

\begin{exercise}
	Zeige, daß in dem Polynomring \(\set Z[t]\) das Ideal \(\ideal m = (2, t)\)
	maximal ist. Zeige weiter, daß das Ideal \(\ideal q = (4, t)\) ein
	\(\ideal m\)-primäres Ideal ist, aber keine Potenz von \(\ideal m\).
\end{exercise}

\begin{exercise}
	Sei \(K\) ein Körper. Seien \(\ideal p_1 \coloneqq (x, y),
	\ideal p_2 \coloneqq (x, z),
	\ideal m \coloneqq (x, y, z)\) drei Ideale im Polynomring \(K[x, y, z]\).
	Zeige, daß \(\ideal p_1, \ideal p_2\) Primideale sind und daß \(\ideal m\)
	ein maximales Ideal ist.
	
	Sei \(\ideal a = \ideal p_1 \ideal p_2\). Zeige, daß \(\ideal a = \ideal p_1
	\cap \ideal p_2 \cap \ideal m^2\) eine minimale Primärzerlegung von
	\(\ideal a\) ist. Welche Komponenten sind isoliert und welche eingebettet?
\end{exercise}

\begin{exercise}
	\label{exer:primary_in_poly}
	Sei \(A\) ein kommutativer Ring. Zeige:
	\begin{enumerate}
	\item
		Ist \(\ideal a\) ein Ideal in \(A\), so ist \(\ideal a[x]\)
		(vergleiche~\prettyref{exer:poly_over_mod})
		die Erweiterung von \(\ideal a\) nach \(A[x]\).
	\item
		Seien \(\ideal p\) ein Primideal in \(A\) und \(\ideal q\) ein
		\(\ideal p\)-primäres Ideal. Dann ist \(\ideal q[x]\) ein
		\(\ideal p[x]\)-primäres Ideal.
		
		(Tip: Nach~\prettyref{exer:poly_over_prime} ist \(\ideal p[x]\) ein
		Primideal. Nutze~\prettyref{exer:polys}.)
	\item
		Ist \(\ideal a = \bigcap\limits_{i = 1}^n \ideal q_i\) eine minimale
		Primärzerlegung eines Ideals \(\ideal a\) in \(A\), so ist \(\ideal a[x]
		= \bigcap\limits_{i = 1}^n \ideal q_i[x]\) eine minimale Primärzerlegung
		von \(\ideal a[x]\).
	\item
		Ist \(\ideal p\) ein zu einem zerlegbaren Ideal \(\ideal a\)
		assoziiertes isoliertes Primideal, so ist \(\ideal p[x]\) ein zum Ideal
		\(\ideal a[x]\) assoziiertes isoliertes Primideal.
	\end{enumerate}
\end{exercise}

\begin{exercise}
	Sei \(K\) ein Körper. Zeige, daß die Ideale \(\ideal p_i \coloneqq
	(x_1, \dotsc, x_i)\) von \(K[x_1,\ldots,x_n]\) für \(1 \leq i \leq n\) alle Primideale sind und
	daß ihre Potenzen alle Primärideale sind.
	
	(Tip:~\prettyref{exer:primary_in_poly}.)
\end{exercise}

\begin{exercise}
	\label{exer:assoc_primes_to_zero}
	Sei \(A\) ein kommutativer Ring. Sei \(D(A)\) die Menge der Primideale
	\(\ideal p\) von \(A\), für die ein \(a \in A\) existiert, so daß
	\(\ideal p\) ein minimales Element in der Menge Primideale ist, die
	\((0 : a)\) umfassen. Zeige:
	\begin{enumerate}
	\item
		Ein Element \(x \in A\) ist genau dann ein Nullteiler,
		wenn \(x \in \ideal p\) für ein \(\ideal p \in D(A)\).
	\item
		Sei \(S \subset A\) multiplikativ abgeschlossen. Dann ist
		\(D(S^{-1} A) = \{S^{-1} \ideal p \mid \ideal p \cap S = \emptyset,
			\ideal p \in D(A)\}\).
	\item
		Ist das Nullideal zerlegbar, so ist \(D(A)\) die Menge der assoziierten
		Primideale von \(0\).
	\end{enumerate}
\end{exercise}

\begin{exercise}
	\label{exer:zero_germ}
	Sei \(A\) ein kommutativer Ring. Für jedes Primideal \(\ideal p\) in \(A\)
	sei \(S_{\ideal p}(0)\) der Kern des Strukturmorphismus' \(A \to A_\ideal p\).
	Zeige:
	\begin{enumerate}
	\item
		\(S_{\ideal p}(0) \subset \ideal p\).
	\item
		Es ist \(\sqrt{S_{\ideal p}(0)} = \ideal p\) genau dann, wenn
		\(\ideal p\) ein minimales Primideal von \(A\) ist.
	\item
		Ist \(\ideal p'\) ein Primideal in \(A\) mit \(\ideal p' \subset
		\ideal p\), so folgt \(S_{\ideal p}(0) \subset S_{\ideal p'}(0)\).
	\item
		Sei \(D(A)\) wie in~\prettyref{exer:assoc_primes_to_zero} definiert.
		Dann ist \(\bigcap\limits_{\ideal p \in D(A)} S_{\ideal p}(0) = (0)\).
	\end{enumerate}
\end{exercise}

\begin{exercise}
	\label{exer:everywhere_zero_germ}
	Sei \(\ideal p\) ein Primideal in einem kommutativen Ring \(A\).
	Sei \(S_{\ideal p}(0)\) wie in~\prettyref{exer:zero_germ} definiert. Zeige:
	\begin{enumerate}
	\item
		Ist \(\ideal p\) ein minimales Primideal, so ist
		\(S_{\ideal p}(0)\) das kleinste \(\ideal p\)-primäre Ideal.
	\item
		Sei \(\ideal a\) der Schnitt aller \(S_{\ideal p}(0)\), wobei
		\(\ideal p\) über alle minimalen Primideale von \(A\) läuft. Dann
		ist \(\ideal a\) im Nilradikal von \(A\) enthalten.
	\item
		Das Nullideal von \(A\) sei zerlegbar. Dann ist \(\ideal a = 0\) genau
		dann, falls jedes zu \((0)\) assoziierte Primideal \(\ideal a\) isoliert
		ist.
	\end{enumerate}
\end{exercise}

\begin{exercise}
	\label{exer:saturation}
	Sei \(S\) eine multiplikativ abgeschlossene Teilmenge eines kommutativen
	Ringes \(A\). Mit \(S(\ideal a)\) bezeichnen wir wie üblich die Sättigung
	eines Ideales \(\ideal a\) nach \(S\).
	Zeige:
	\begin{enumerate}
	\item
		Für je zwei Ideale \(\ideal a, \ideal b\) von \(A\) gilt
		\(S(\ideal a) \cap S(\ideal b) = S(\ideal a \cap \ideal b)\).
	\item
		Für ein Ideal \(\ideal a\) von \(A\) gilt \(S(\sqrt{\ideal a})
		= \sqrt{S(\ideal a)}\).
	\item
		Für ein Ideal \(\ideal a\) von \(A\) gilt \(S(\ideal a) = (1)\) genau
		dann, wenn \(\ideal a \cap S \neq \emptyset\).
	\item
		Seien \(S_1, S_2 \subset A\) multiplikativ abgeschlossen. Für ein
		Ideal \(\ideal a\) gilt dann \(S_1(S_2(\ideal a))
		= (S_1 S_2)(\ideal a)\).
	\end{enumerate}
	
	Sei \(\ideal a\) zerlegbar. Zeige, daß die Menge aller \(S(\ideal a)\),
	wobei \(S\) alle multiplikativ abgeschlossenen Teilmengen von \(A\)
	durchläuft, endlich ist.
\end{exercise}

\begin{exercise}
	\label{exer:symbolic_power}
	Sei \(\ideal p\) ein Primideal eines kommutativen Ringes \(A\). Sei
	\(n \in \set N_0\). Die \emph{\(n\)-te symbolische Potenz von \(\ideal p\)}
	ist die Sättigung \(\ideal p^{(n)} \coloneqq S_{\ideal p}(\ideal p^n)\),
	wobei \(S_{\ideal p} \coloneqq A \setminus \ideal p\).
	Zeige:
	\begin{enumerate}
	\item
		Es ist \(\ideal p^{(n)}\) ein \(\ideal p\)-primäres Ideal.
	\item
		Ist \(\ideal p^n\) zerlegbar, so ist \(\ideal p^{(n)}\) seine
		\(\ideal p\)-primäre Komponente.
	\item
		Ist \(\ideal p^{(m)} \ideal p^{(n)}\) zerlegbar, so ist
		\(\ideal p^{(m + n)}\) seine \(\ideal p\)-primäre Komponente.
	\item
		Es ist \(\ideal p^{(n)} = \ideal p^n\) genau dann, wenn \(\ideal p^n\)
		ein \(\ideal p\)-primäres Ideal ist.
	\end{enumerate}
\end{exercise}

\begin{exercise}
	Sei \(\ideal a\) ein zerlegbares Ideal in einem kommutativen Ring \(A\).
	Sei \(\ideal p\) ein maximales Element unter allen Idealen der Form
	\((\ideal a : x)\) mit \(x \in A\) und \(x \notin \ideal a\). Zeige, daß
	\(\ideal p\) ein zu \(\ideal a\) assoziiertes Primideal ist.
\end{exercise}

\begin{exercise}
	\label{exer:component_to_isolated_set}
	Sei \(\ideal a\) ein zerlegbares Ideal in einem kommutativen Ring \(A\).
	Sei \(\mathfrak S\) eine isolierte Menge von zu \(\ideal a\)
	assoziierten Primidealen. Sei \(\ideal q_{\mathfrak S}\) der Schnitt aller
	\(\ideal p\)-primären Komponenten von \(\ideal a\) mit \(\ideal p \in
	\mathfrak S\). Sei \(f \in A\), so daß für alle zu \(\ideal a\) assoziierten
	Primideale gilt, daß \(f \in \ideal p \iff \ideal p \notin \mathfrak S\).
	Wir setzen \(S_f \coloneqq \{f^n \mid n \in \set N_0\}\). Zeige, daß
	\(\ideal q_{\mathfrak S} = S_f(\ideal a) = (\ideal a \colon f^n)\)
	für alle \(n \gg 0\).
\end{exercise}

\begin{exercise}
	Sei \(A\) ein kommutativer Ring, in dem jedes Ideal zerlegbar ist. Zeige,
	daß \(S^{-1} A\) dieselbe Eigenschaft hat, wenn \(S \subset A\)
	multiplikativ abgeschlossen ist.
\end{exercise}

\begin{exercise}
	\label{exer:prop_l1}
	Sei \(A\) ein kommutativer Ring mit der folgenden Eigenschaft (L1):
	Zu jedem Ideal \(\ideal a \neq (1)\) und zu jedem Primideal \(\ideal p\)
	in \(A\) existiert ein \(x \notin \ideal p\) mit \(S_{\ideal p}(\ideal a)
	= (\ideal a : x)\), wobei \(S_{\ideal p} \coloneqq A \setminus \ideal p\).
	
	Zeige, daß dann jedes Ideal in \(A\) der Schnitt (möglicherweise unendlich
	vieler) primärer Ideale ist.
	
	(Tip: Sei \(\ideal p_1\) ein minimales Element der Menge aller Primideale,
	welche \(\ideal a\) umfassen. Nach~\prettyref{exer:everywhere_zero_germ}
	ist \(\ideal q_1 = S_{\ideal p_1}\). Weiter ist \(\ideal q_1 = (\ideal a :
	x)\) für ein \(x \notin \ideal p_1\). Folgere, daß \(\ideal a = \ideal q_1
	\cap (\ideal a + (x))\).
	
	Sei sodann \(\ideal a_1\) ein maximales Element unter allen Idealen
	\(\ideal b \supset \ideal a\) mit \(\ideal q_1 \cap \ideal b = \ideal a\)
	und \(x \in \ideal a_1\), so daß \(\ideal a_1 \not\subset \ideal p_1\).
	Wiederhole die Konstruktion diesmal mit \(\ideal a_1\) und so weiter.
	
	Im \(n\)-ten Schritt haben wir \(\ideal a = \ideal q_1 \cap \dotsb \cap
	\ideal q_n \cap \ideal a_n\), wobei die \(\ideal q_i\) Primärideale sind
	und \(\ideal a_n\) maximal unter allen Idealen \(\ideal b\) mit
	\(\ideal b \supset \ideal a_{n - 1} = \ideal a_n \cap \ideal q_n\), so
	daß \(\ideal a = \ideal q_1 \cap \dotsb \cap q_n \cap \ideal b\) und
	\(\ideal a_n \not\subset \ideal p_n\).
	
	Sollte in irgendeinem Schnitt \(\ideal a_n = (1)\) gelten, hört die
	Konstruktion auf und \(\ideal a\) ist folglich ein endlicher Schnitt
	primärer Ideale. Im anderen Fall fahre mit transfiniter Induktion fort, und
	zwar unter Beachtung der Tatsache, daß jedes \(\ideal a_n\) das Ideal
	\(\ideal a_{n - 1}\) echt enthält.)
\end{exercise}

\begin{exercise}
	Betrachte folgende Eigenschaft (L2) kommutativer Ringe \(A\): Ist
	\(\ideal a\) ein Ideal von \(A\) und \(S_1 \supset S_2 \supset \dotsb\)
	eine absteigende Kette multiplikativ abgeschlossener Teilmengen von \(A\),
	so existiert ein \(n \in \set N_0\) mit
	\(S_n(\ideal a) = S_{n + 1}(\ideal a) = \dotsb\).
	
	Zeige, daß folgende Aussagen für einen kommutativen Ring \(A\)
	äquivalent sind:
	\begin{enumerate}
	\item
		Jedes Ideal in \(A\) ist zerlegbar.
	\item
		Der Ring \(A\) erfüllt die Eigenschaften (L1)
		(siehe~\prettyref{exer:prop_l1}) und (L2).
	\end{enumerate}
	
	(Tip: Um aus der ersten die zweite Aussage zu folgern,
	benutze~\prettyref{exer:saturation}
	und~\prettyref{exer:component_to_isolated_set}.
	
	Um aus der zweiten die erste Aussage zu folgern, gehe folgendermaßen vor:
	Gilt \(S_n = S_{\ideal p_1} \cap \dotsb \cap S_{\ideal p_n}\) in der
	Notation von~\prettyref{exer:prop_l1}, so ist \(S_n \cap \ideal a_n \neq
	\emptyset\), also \(S_n(\ideal a_n) = (1)\), also
	\(S_n(\ideal a) = \ideal q_1 \cap \dotsb \cap \ideal q_n\).
	Folgere mit (L2), daß die Kontruktion nach einer endlichen Anzahl von
	Schritten aufhören muß.
\end{exercise}

\begin{exercise}
	Sei \(\ideal p\) ein Primideal eines kommutativen Ringes \(A\). Zeige, daß
	jedes \(\ideal p\)-primäre Ideal das Ideal \(S_{\ideal p}(0) = \ker
	(A \to A_{\ideal p})\) umfaßt.
	
	Erfülle \(A\) die folgende Bedingung: Für jedes Primideal \(\ideal p\) sei der Schnitt
	aller \(\ideal p\)-primären Ideale gerade \(S_{\ideal p}(0)\). (Wir
	werden später sehen, daß alle sogenannten noetherschen Ringe diese Bedingung
	erfüllen.) % FIXME Reference missing.
	Seien weiter \(\ideal p_1, \dotsc, \ideal p_n\) paarweise verschiedene
	Primideale von \(A\), welche nicht minimal sind. Zeige, daß dann ein Ideal
	\(\ideal a\) existiert, dessen assoziierte Primideale gerade \(\ideal p_1,
	\dotsc, \ideal p_n\) sind.
	
	(Tip: Beweis per Induktion über \(n\). Der Fall \(n = 1\) ist trivial, denn
	wir können \(\ideal a = \ideal p_1\) setzen. Sei also \(n > 1\), und sei
	\(\ideal p_n\) maximales Element der Menge \(\ideal p_1, \dotsc,
	\ideal p_n\). Nach Induktionsvoraussetzung existiert ein Ideal \(\ideal b\)
	und eine Primärzerlegung \(\ideal b = \ideal q_1 \cap \dotsb
	\ideal q_{n - 1}\), wobei \(\ideal q_i\) ein \(\ideal p_i\)-primäres Ideal
	ist.
	
	Angenommen, \(\ideal b \subset S_{\ideal p_n}(0)\). Sei dann \(\ideal p\)
	ein minimales Primideal von \(A\) mit \(\ideal p \subset \ideal p_n\).
	Dann ist \(S_{\ideal p_n}(0) \subset S_{\ideal p}(0)\), also
	\(\ideal b \subset S_{\ideal p}(0)\). Ziehen der Wurzel liefert
	nach~\prettyref{exer:zero_germ}, daß \(\ideal p_1 \cap \dotsb \cap \ideal
	p_{n - 1} \subset \ideal p\), also \(\ideal p_i \subset \ideal p\) für ein
	\(i < n\), also \(\ideal p_i = \ideal p\), da \(\ideal p\) minimal ist.
	Widerspruch, da kein \(\ideal p_i\) minimal ist.
	
	Also ist \(\ideal b \not\subset S_{\ideal p_n}(0)\). Damit existiert ein
	\(\ideal p_n\)-primäres Ideal \(\ideal q_n\) mit \(\ideal b
	\not\subset \ideal q_n\). Zeige, daß \(\ideal a = \ideal q_1 \cap \dotsb
	\cap \ideal q_n\) die gewünschten Eigenschaften hat.)
\end{exercise}

\begin{exercise}
	Sei \(A\) ein kommutativer Ring. Seien \(M\) ein \(A\)-Modul und
	\(N \subset M\) ein Untermodul. Die \emph{Wurzel von \(N\) in \(M\)} ist
	\[
		\sqrt{N}_M \coloneqq \{x \in A \mid \exists q \in \set N_0\colon
		x^q M \subset N\}.
	\]
	Zeige, daß \(\sqrt{N}_M = \sqrt{(N : M)} = \sqrt{\ann(M/N)}\). Insbesondere
	ist \(\sqrt{N}_M\) ein Ideal von \(A\).
	
	Formuliere und beweise entsprechende Aussagen für \(\sqrt{}_M\) wie
	in~\prettyref{prop:radical}.
\end{exercise}

\begin{exercise}
	Sei \(A\) ein kommutativer Ring. Sei \(M\) ein \(A\)-Modul. Jedes Element
	\(x \in A\) definiert einen Endomorphismus \(\phi_x\colon M \to M,
	m \mapsto x m\) von \(M\). Das Element \(x\) heißt
	\emph{regulär in \(M\)}, falls \(\phi_x\) injektiv ist, und andernfalls
	\emph{Nullteiler in \(M\)}. Weiter heißt \(x\) \emph{nilpotent in \(M\)},
	falls \(\phi_x\) nilpotent ist, falls also \(\phi_x^n = 0\) für ein
	\(n \in \set N_0\).
	
	Ein Untermodul \(Q\) von \(M\) heißt \emph{primär}, falls die Nullteiler
	in \(M/Q\) genau die nilpotenten Elemente in \(M/Q\) sind.
	
	Zeige, daß für einen primären Untermodul \(Q\) von \(M\) das Ideal
	\((Q : M)\) ein primäres Ideal ist und damit \(\ideal p \coloneqq
	\sqrt{Q}_M\) ein Primideal. Wir sagen in diesem Falle, daß
	\(Q\) ein \emph{\(\ideal p\)-primärer Untermodul von \(M\)} ist.
	
	Formuliere und beweise entsprechende Aussagen für \(Q\) wie
	in~\prettyref{lem:primary1} und~\prettyref{lem:primary2}.
\end{exercise}

\begin{exercise}
	\label{exer:primary_decomp_for_mod}
	Sei \(A\) ein kommutativer Ring. Seien \(M\) ein \(A\)-Modul und
	\(N\) ein Untermodul in \(M\). Eine \emph{Primärzerlegung von \(N\) in
	\(M\)} ist eine Darstellung \(N = Q_1 \cap \dotsb \cap Q_n\) von \(N\)
	als Schnitt primärer Untermoduln in \(M\). Sie heißt \emph{minimal},
	falls alle \(\ideal p_i \coloneqq \sqrt{Q_i}_M\) paarweise verschieden sind
	und falls \(Q_i \not\supset \bigcap\limits_{j \neq i} Q_j\) für alle \(i\).
	
	Beweise die Entsprechung des ersten
	Eindeutigkeitssatzes~\ref{thm:first_uniqueness}, nämlich daß die
	Primideale \(\ideal p_i\) nur von \(N\) und \(M\) abhängen. Diese heißen
	die \emph{zu \(N\) in \(M\) assoziierten Primideale}. Zeige weiter, daß
	diese auch die Primideale sind, welche zu \(0\) in \(M/N\) assoziiert sind.
\end{exercise}

\begin{exercise}
	Formuliere und beweise entsprechende Aussagen für die Primärzerlegung von
	Moduln wie
	in~\prettyref{prop:isolated_prime},
	\prettyref{prop:union_of_assoc_primes},
	\prettyref{prop:primaries_in_localisation},
	\prettyref{cor:correspondence_for_primaries},
	\prettyref{prop:sat_decomp},
	\prettyref{thm:second_uniqueness}
	und~\prettyref{cor:second_uniqueness}.
	
	(Tip: Ohne Einschränkung kann davon ausgegangen werden, daß \(N = 0\).)
\end{exercise}

