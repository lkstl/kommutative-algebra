\section{Grundlagen aus der Einführung in die Algebra}

%\subsection{Gruppen}

\subsection{Körper}

\begin{definition}
	Die \emph{Charakteristik \(\chr K\)} eines Körpers \(K\) ist die
	kleinste positive Zahl \(p \in \set N\) mit \(p \cdot 1 = 0 \in K\) oder
	\(0\), wenn kein solches \(p\) existiert.
\end{definition}

\begin{example}
	Der Körper \(\set Q\) der rationalen Zahlen ist ein Körper der
	Charakteristik \(0\).
\end{example}

\begin{proposition}
	Die Charakteristik eines Körpers ist entweder \(0\) oder eine Primzahl.
\end{proposition}

\subsection{Algebraische Erweiterungen}

\begin{definition}
	Ein Körper \(K\) heißt \emph{algebraisch abgeschlossen}, falls
	jedes normierte Polynom \(f \in K[x]\) vollständig in Linearfaktoren zerfällt,
	falls also \(x_1, \dotsc, x_n \in K\) mit \(f = (x - x_1) \dotsm (x - x_n)\) existieren.
\end{definition}

Diese Definition ist gleichbedeutend damit, daß jedes Polynom über \(K\) mindestens eine Nullstelle besitzt.

\begin{example}<+->
	Der Körper \(\set C\) der komplexen Zahlen ist algebraisch abgeschlossen.
\end{example}

\begin{definition}
	Sei \(K \subset L\) eine Körpererweiterung.
	\begin{enumerate}
	\item
		Ein Element \(x \in L\) heißt \emph{algebraisch über \(K\)}, falls es
		einer Gleichung der Form \(x^n + a_1 x^{n + 1} + \dotsb + a_n\) mit \(a_i \in K\)
		genügt.
	\item
		Ein Element \(x \in L\) heißt \emph{transzendent über \(K\)}, falls es
		nicht algebraisch ist.
	\item
		Die Körpererweiterung \(K \subset L\) heißt \emph{algebraisch}, falls jedes
		Element von \(L\) algebraisch über \(K\) ist.
	\end{enumerate}
\end{definition}

\begin{definition}
	Eine Körpererweiterung \(K \subset L\) heißt \emph{endlich}, falls
	\(L\) als \(K\)-Vektorraum endlich-dimensional ist. Die Zahl
	\([L : K] \coloneqq \dim_K L\) heißt der \emph{Grad der Körpererweiterung}.
\end{definition}

\begin{example}
	Jede endliche Körpererweiterung ist algebraisch.
\end{example}

\begin{example}
	Sei \(K \subset L\) eine Körpererweiterung. Für ein \(x \in L\) bezeichne
	\(K(x)\) den kleinsten Zwischenkörper von \(K \subset L\), welcher \(x\) enthält.
	Dann ist \(K \subset K(x)\) genau dann eine endliche Körpererweiterung, wenn
	\(x\) algebraisch über \(K\) ist.
\end{example}

\begin{definition}
	Sei \(K \subset L\) eine Körpererweiterung. Ist \(x \in L\) algebraisch
	über \(K\), so heißt das normierte Polynom \(m \in K[x]\) minimalen Grades
	mit \(m(x) = 0 \in L\) das \emph{Minimalpolynom von \(x\) über \(K\)}.
\end{definition}

\begin{definition}
	Sei \(K \subset L\) eine Körpererweiterung. Zerfällt ein Polynom \(f \in K[x]\)
	über \(L\) vollständig in Linearfaktoren, so heißt \(L\) ein \emph{Zerfällungskörper von \(K\)}.
\end{definition}

\begin{theorem}
	Sei \(K\) ein Körper. Dann existiert eine
	algebraische Körpererweiterung \(K \subset L\), so
	daß \(L\) ein algebraisch abgeschlossener Körper ist. Insbesondere ist
	\(L\) ein Zerfällungskörper für jedes Polynom über \(K\).
\end{theorem}

Der Körper \(L\) heißt ein \emph{algebraischer Abschluß von \(K\)}.

\begin{theorem}
	Sei \(K \subset K'\) eine algebraische Körpererweiterung. Sei \(\phi\colon K \to L\)
	ein Körperhomomorphismus in einen algebraisch abgeschlossenen Körper \(L\).
	\begin{enumerate}
	\item
		Der Homomorphismus \(\phi\) läßt sich zu einem Körperhomomorphismus \(K' \to L\) fortsetzen.
	\item
		Die Anzahl \([K' : K]_\sep\) der möglichen Fortsetzungen hängt nur von der Körpererweiterung
		\(K \subset K'\) ab und heißt ihr \emph{Separabilitätsgrad}.
	\item
		Es gilt \([K' : K]_\sep \le [K' : K]\). Insbesondere ist der Separabilitsgrad einer endlichen
		Körpererweiterung endlich.
	\end{enumerate}
\end{theorem}

\begin{definition}
	Eine endliche Körpererweiterung \(K \subset K'\) heißt \emph{separabel}, falls
	\([K' : K]_\sep = [K' : K]\).
\end{definition}

\begin{proposition}
	Ist \(K\) ein Körper der Charakteristik Null, so ist jede endliche Körpererweiterung \(K \subset K'\)
	separabel.
\end{proposition}

\subsection{Galoissche Theorie}

\begin{definition}
	Sei \(K \subset L\) eine endliche Körpererweiterung. Ist \(x \in L\), so bezeichnen wir
	mit \(\phi_x\colon L \to L, y \mapsto x y\) den durch Multiplikation mit \(x\) induzierten
	Endomorphismus des \(K\)-Vektorraumes \(L\).
	
	Seine Spur \(\tr_K \phi_x\) heißt die \emph{Spur \(\tr_{L/K}(x)\) von \(x\) in der Körpererweiterung
	\(K \subset L\)}.
\end{definition}

\begin{proposition}
	Sei \(K \subset L\) eine endliche Körpererweiterung.
	\begin{enumerate}
	\item
		Die Spur \(\tr_{L/K}\colon L \to K\) ist eine \(K\)-lineare Abbildung.
	\item
		Sei \(x \in L\) mit Minimalpolynom \(x^n + a_1 x^{n - 1} + \dotsb + a_n\) über \(K\).
		Dann ist \(\tr_{L/K} (x) = - [L : K(x)] \cdot a_1\), insbesondere also ein Vielfaches eines
		Koeffizienten des Minimalpolynomes.
	\end{enumerate}
\end{proposition}

\begin{theorem}
	Sei \(K \subset L\) eine separable endliche Körpererweiterung. Dann ist die Bilinearform
	\[
		L \times L \to K, (x, y) \mapsto \tr_{L/K} (xy)
	\]
	auf \(L\) über \(K\) nicht ausgeartet.
\end{theorem}

\begin{corollary}
	Ist \(K \subset L\) eine separable endliche Körpererweiterung, und ist
	\((x_1, \dotsc, x_n)\) eine Basis von \(L\) über \(K\), so existiert genau eine Basis
	\((y_1, \dotsc, y_n)\) von \(L\) über \(K\) mit \(\tr_{L/K} (x_i y_j) = \kron_{ij}\).
\end{corollary}

\subsection{Transzendente Erweiterungen}

\begin{definition}
	Sei \(K \subset L\) eine Körpererweiterung. Eine Familie \((x_i)_{i \in I}\)
	von	Elementen in \(L\) heißt \emph{algebraisch unabhängig}, wenn 
	für je endlich
	viele Elemente \(x_{i_1}, \dotsc, x_{i_n}\) der Folge und \(f \in K[t_1, \dotsc, t_n]\)
	mit \(f(x_{i_1}, \dotsc, x_{i_n}) = 0\) schon \(f = 0\) folgt.
\end{definition}

\begin{remark}
	Sei \(K \subset L\) eine Körpererweiterung. Eine Familie \((x_i)_{i \in I}\)
	von Elementen in \(L\) ist also genau dann algebraisch unabhängig, wenn
	der Ringhomomorphismus \(K[(t_i)] \to L, t_i \mapsto x_i\) injektiv ist.
\end{remark}

\begin{definition}
	Sei \(K \subset L\) eine Körpererweiterung. Eine \emph{Transzendenzbasis
	von \(L\) über \(K\)} ist eine maximale Familie algebraisch unabhängiger
	Elemente.
\end{definition}

\begin{remark}
	Sei \(K \subset L\) eine Körpererweiterung. Dann ist \((x_i)_{i \in I}\)
	genau dann eine Transzendenzbasis von \(L\) über \(K\), wenn
	\(K((t_i)) \to L, t_i \mapsto x_i\) eine wohldefinierte algebraische
	Körpererweiterung ist.
\end{remark}

\begin{proposition}
	Sei \(K \subset L\) eine Körpererweiterung. Dann besitzt \(L\) eine
	Transzendenzbasis über \(K\) und je zwei Transzendenzbasen haben dieselbe
	Mächtigkeit.
\end{proposition}

\begin{definition}
	Sei \(K \subset L\) eine Körpererweiterung. Der
	\emph{Transzendenzgrad \(\trdeg_K L\) von
	\(L\) über \(K\)} ist die Mächtigkeit einer Transzendenzbasis von \(L\)
	über \(K\).
\end{definition}

\begin{example}
	Eine Körpererweiterung \(K \subset L\) ist genau dann algebraisch, wenn
	der Transzendenzgrad von \(L\) über \(K\) gleich \(0\) ist.
\end{example}

