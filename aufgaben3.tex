\subsection{Lokalisierungen von Ringen und Moduln}

\begin{exercise}
	Sei \(S\) eine multiplikativ abgeschlossene Teilmenge eines kommutativen Ringes \(A\).
	Sei \(M\) ein endlich erzeugter \(A\)-Modul. Zeige, daß \(S^{-1} M = 0\) genau dann, wenn
	ein \(s \in S\) mit \(s M = 0\) existiert.
\end{exercise}

\begin{exercise}
	Sei \(\ideal a\) ein Ideal in einem kommutativen Ring \(A\). Sei \(S = 1 + \ideal a\). Zeige, daß
	\(S^{-1} \ideal a\) im Jacobsonschen Radikal von \(S^{-1} A\) enthalten ist.
\end{exercise}

\begin{exercise}
	Sei \(A\) ein kommutativer Ring. Seien \(S, T \subset A\) zwei multiplikativ abgeschlossene
	Teilmengen. Sei \(S^{-1} T\) das Bild von \(T\) in \(S^{-1} A\). Zeige, daß die Ringe
	\((ST)^{-1} A\) und \((S^{-1} T)^{-1} S^{-1} A\) isomorph sind.
\end{exercise}

\begin{exercise}
	Sei \(\phi\colon A \to B\) ein Homomorphismus kommutativer Ringe. Sei \(S \subset A\) multiplikativ
	abgeschlossen. Zeige, daß \(S^{-1} (B^A)\) und \(((\phi(S))^{-1} B)^{S^{-1} A}\) isomorph als
	\(S^{-1} A\)-Algebren sind.
\end{exercise}

\begin{exercise}
	Sei \(A\) ein kommutativer Ring. Für jedes Primideal \(\ideal p\) von \(A\) besitze \(A_{\ideal p}\) kein
	nilpotentes Element außer \(0\). Zeige, daß \(A\) außer \(0\) kein nilpotentes Element besitzt.
	
	Folgt aus der Tatsache, daß alle Halme \(A_{\ideal p}\) Integritätsbereiche sind, die Tatsache, daß \(A\)
	ein Integritätsbereich ist?
\end{exercise}

\begin{exercise}
	\label{exer:max_mult_closed}
	Sei \(A\) ein kommutativer Ring mit \(A \neq 0\). Sei \(\mathfrak S\) die Menge der multiplikativ abgeschlossenen
	Teilmengen \(S\) von \(A\) mit \(0 \notin S\). Zeige, daß \(\mathfrak S\) bezüglich der Inklusion maximale Elemente
	besitzt und daß \(S \in \mathfrak S\) genau dann maximal ist, falls \(A \setminus S\) ein minimales Primideal von \(A\)
	ist.
\end{exercise}

\begin{exercise}
	\label{exer:mult_sat}
	Wir nennen eine multiplikativ abgeschlossene Teilmenge \(S\) eines kommutativen Ringes \(A\) \emph{gesättigt},
	falls aus \(xy \in S\) schon \(x \in S\) und \(y \in S\) folgt. Zeige:
	\begin{enumerate}
	\item
		Es ist \(S\) genau dann gesättigt, wenn \(A \setminus S\) eine Vereinigung von Primidealen ist.
	\item
		Es gibt eine eindeutige, kleinste gesättigte multiplikativ abgeschlossene Teilmenge \(\bar S \subset A\) mit
		\(S \subset \bar S\), nämlich das Komplement der Vereinigung aller Primideale von \(A\), welche \(S\) nicht schneiden.
		
		Es heißt \(\bar S\) die \emph{Sättigung von \(S\)}.
	\end{enumerate}
	
	Berechne die Sättigung einer multiplikativ abgeschlossenen Teilmenge der Form \(1 + \ideal a\), wobei
	\(\ideal a\) ein Ideal von \(A\) ist.
\end{exercise}

\begin{exercise}
	Sei \(A\) ein kommutativer Ring. Seien \(S, T \subset A\) zwei multiplikativ abgeschlossene Teilmengen von \(A\) mit
	\(S \subset T\). Sei \(\phi\colon S^{-1} A \to T^{-1} A, \frac a s \mapsto \frac a s\). Zeige, daß die folgenden Aussagen
	äquivalent sind:
	\begin{enumerate}
	\item	
		Der Ringhomomorphismus \(\phi\) ist bijektiv.
	\item
		Für alle \(t \in T\) ist \(\frac t 1\) eine Einheit in \(S^{-1} A\).
	\item
		Für alle \(t \in T\) existiert ein \(x \in A\) mit \(x t \in S\).
	\item
		Es ist \(T\) in der Sättigung \(\bar S\) von \(S\) enthalten (\prettyref{exer:mult_sat}).
	\item
		Für jedes Primideal \(\ideal p\) von \(A\) mit \(\ideal p \cap T \neq \emptyset\) gilt auch
		\(\ideal p \cap S \neq \emptyset\).
	\end{enumerate}
\end{exercise}

\begin{exercise}
	Sei \(A\) ein kommutativer Ring. Die Menge \(S_0\) der regulären Elemente von \(A\) ist eine
	gesättigte, multiplikativ abgeschlossene Teilmenge. Damit ist die Menge \(D\) der Nullteiler von \(A\)
	eine Vereinigung von Primidealen nach~\prettyref{exer:ideals_of_zero_divs}. Zeige, daß jedes minimale Primideal
	von \(A\) in \(D\) enthalten ist.
	
	(Tip:~\prettyref{exer:max_mult_closed}.)
	
	Der Ring \(S_0^{-1} A\) heißt der \emph{vollständige Quotientenring von \(A\)}. Zeige:
	\begin{enumerate}
	\item
		Die Teilmenge \(S_0\) ist die größte multiplikativ abgeschlossene Teilmenge \(S\) von \(A\), 
		für die \(A \to S^{-1} A\) injektiv ist.
	\item
		Jedes Element in \(S_0^{-1} A\) ist entweder ein Nullteiler oder eine Einheit.
	\item	
		Ein kommutativer Ring \(A\), in dem jede Nichteinheit ein Nullteiler ist, ist gleich
		seinem vollständigen Quotientenring, das heißt \(A \to S^{-1}_0 A\) ist ein Isomorphismus.
	\end{enumerate}
\end{exercise}

\begin{exercise}
	\label{exer:trivial_stalks_in_closed_subset}
	Sei \(\ideal a\) ein Ideal in einem kommutativen Ring \(A\). Sei \(M\) ein \(A\)-Modul. Für die Halme an
	allen maximalen Idealen \(\ideal m\) von \(A\) mit \(\ideal m \supset \ideal a\) gelte \(M_{\ideal m} = 0\). Zeige,
	daß \(M = \ideal a M\).
	
	(Tip: Gehe auf den \(A/\ideal a\)-Modul \(M/\ideal a M\) über und nutze die Lokalität der Trivialität eines Moduls.)
\end{exercise}

\begin{exercise}
	Sei \(A\) ein kommutativer Ring. Sei \(F \coloneqq A^n\) als \(A\)-Modul. Zeige, daß jede Menge von \(n\) Erzeugern von
	\(F\) als \(A\)-Modul schon eine Basis von \(F\) als \(A\)-Modul ist.
	
	(Tip: Seien \(x_1, \dotsc, x_n\) Erzeuger von \(F\). Sei \((e_1, \dotsc, e_n)\) die kanonische Basis von \(F\). Definiere
	\(\phi\colon F \to F\) durch \(\phi(e_i) = x_i\). Dann ist \(\phi\) surjektiv. Es ist zu zeigen, daß \(\phi\) ein Isomorphismus
	ist. Da Injektivität eine lokale Eigenschaft ist, können wir annehmen, daß \(A\) ein lokaler Ring ist. Sei etwa \(k = A/\ideal m\)
	der Restklassenkörper von \(A\). Sei weiter \(N = \ker \phi\). Da \(F\) ein flacher \(A\)-Modul ist, führt die exakte
	Sequenz \(0 \to N \to F \xrightarrow{\phi} F \to 0\) zu einer exakten Sequenz \(0 \to k \otimes N \to k \otimes F \xrightarrow
	{\id_k \otimes \phi} k \otimes F \to 0\). Es ist \(k \otimes F = k^n\) ein \(n\)-dimensionaler \(k\)-Vektorraum. Aus der Surjektivität von
	\(\id_k \otimes \phi\) folgt damit auch die Injektivität, also \(k \otimes N = 0\).
	Nach~\prettyref{exer:fg_ker_map_to_free} ist \(N\) weiter endlich erzeugt. Nach dem Nakayamaschen Lemma ist damit \(N = 0\). Damit 
	ist \(\phi\) ein Isomorphismus.)
	
	Folgere, daß jede Erzeugermenge von \(F\) aus mindestens \(n\) Elementen bestehen muß.
\end{exercise}

