\subsection{Vervollständigungen}

\begin{exercise}
	\label{exer:compl_not_right_exact}
	Sei \(p\) eine Primzahl. Sei \(\alpha_n\colon \set Z/(p) \injto \set Z/(p^n), [x]_{(p)} \mapsto [p^{n - 1} x]_{(p^n)}\).
	Seien \(A \coloneqq \bigoplus\limits_{n = 1}^\infty \set Z/(p)\) und \(B \coloneqq \bigoplus\limits_{n = 1}^\infty \set Z/(p^n)\).
	Sei \(\alpha\colon A \injto B, (\xi_1, \xi_2, \dotsc) \mapsto (\alpha_1(\xi_1), \alpha_2(\xi_2), \dotsc)\).
	\\
	Zeige, daß die \(p\)-adische Vervollständigung von \(A\) wieder \(A\) ist.
	Zeige weiter, daß die Vervollständigung von \(A\) bezüglich der von der \(p\)-adischen Topologie auf \(B\) induzierten Topologie
	durch \(\prod\limits_{n = 1}^\infty \set Z/(p)\) gegeben ist.
	
	Folgere, daß die \(p\)-adische Vervollständigung kein rechtsexakter Funktor auf der Kategorie aller \(\set Z\)-Moduln ist.
\end{exercise}

\begin{exercise}
	In den Bezeichnungen von \prettyref{exer:compl_not_right_exact} sei \(A_n \coloneqq \alpha^{-1} (B p^n)\). Betrachte die
	exakten Sequenzen
	\[
		0 \to A_n \to A \to A/A_n \to 0.
	\]
	Zeige, daß \(\varprojlim\) kein rechtsexakter Funktor ist, und berechne \({\varprojlim\limits_n}^1 A_n\).
\end{exercise}

\begin{exercise}
	Sei \(\ideal a\) ein Ideal in einem noetherschen kommutativen Ring. Sei \(M\) ein endlich erzeugter \(A\)-Modul.
	Zeige mit \prettyref{thm:krull} und \prettyref{exer:trivial_stalks_in_closed_subset}, daß
	\[
		\bigcap\limits_{n = 1}^\infty \ideal a^n M = \bigcap\limits_{\ideal m \supset \ideal a} \ker(M \to M_{\ideal m}),
	\]
	wobei \(\ideal m\) über alle maximalen Ideale läuft, welche \(\ideal a\) enthalten.
\end{exercise}

\begin{exercise}
	Sei \(\ideal a\) ein Ideal in einem noetherschen kommutativen Ring \(A\). Für jedes \(x \in A\) bezeichnen wir mit
	\(\hat x\) das Bild unter dem kanonischen Homomorphismus \(A \to \hat A = \hat A_{\ideal a}\). Zeige, daß \(\hat x\) ein
	reguläres Element in \(\hat A\) ist, wenn \(x\) ein reguläres Element in \(A\) ist.
	
	Folgt daraus, daß \(\hat A\) ein Integritätsbereich ist, wenn \(A\) ein Integritätsbereich ist?
	
	(Tip: Nutze die Exaktheit der Vervollständigung der Sequenz \(0 \to A \xrightarrow{x \cdot} A\) aus.)
\end{exercise}

\begin{exercise}
	Seien \(\ideal a, \ideal b\) zwei Ideale in einem noetherschen kommutativen Ring \(A\). Zeige, daß für einen endlich erzeugten
	\(A\)-Modul ein Isomorphismus
	\[
		\widehat{(\hat M_{\ideal a})}_{\hat{\ideal b}_{\ideal a}} \cong \hat M_{\ideal a + \ideal b}
	\]
	existiert.
	
	(Tip: Betrachte die \(\ideal a\)-adische Vervollständigung der kurzen exakten Sequenz
	\[
		0 \to \ideal b^m M \to M \to M/\ideal b^m M \to 0
	\]
	und verwende \prettyref{prop:compl_as_scalar_ext}. Dann benutze die Isomorphismen
	\[
		\varprojlim\limits_m (\varprojlim\limits_n M/(\ideal a^n M + \ideal b^m M)) \cong
		\varprojlim\limits_n M/(\ideal a^n M + \ideal b^n M)
	\]
	und die Inklusionen \((\ideal a + \ideal b)^{2n} \subset \ideal a^n + \ideal b^n
	\subset (\ideal a + \ideal b)^n\).
\end{exercise}

\begin{exercise}
	Sei \(\ideal a\) ein Ideal in einem Ring \(A\). Zeige, daß \(\ideal a\) genau dann
	im Jacobsonschen Radikal enthalten ist, wenn jedes maximale Ideal von \(A\) abgeschlossen
	bezüglich der \(\ideal a\)-adischen Topologie ist.
\end{exercise}

\begin{exercise}
	\label{exer:hensel}
	Sei \((A, \ideal m, F)\) ein lokaler Ring, welcher \(\ideal m\)-adisch vollständig ist. Für ein Polynom
	\(f \in A[x]\) bezeichne \(\bar f \in F[x]\) die Reduktion modulo \(\ideal m\).
	
	Zeige das "`Henselsche Lemma"':
	Ist \(f \in A[x]\) ein normiertes Polynom mit \(\bar f = \tilde g \tilde h\) für teilerfremde normierte Polynome \(\tilde g,
	\tilde h \in F[x]\), so existieren normierte Polynome \(g, h \in A[x]\) mit \(f = g h\) und \(\bar g = \tilde g\) und
	\(\bar h = \tilde h\).
	
	(Tip: Nimm an, daß induktiv Polynome \(g_k, h_k \in A[x]\) mit \(g_k h_k - f \in A[x] \ideal m^k\) konstruiert worden sind.
	Dann folgere aus der Teilerfremdheit von \(\tilde g, \tilde h\), daß für jede Zahl \(1 \leq p \leq n\) Polynome
	\(\tilde a_p, \tilde b_p \in F[x]\) mit \(x^p = \tilde a_p \bar g_k + \tilde b_p \bar h_k\) existieren.
	Schließlich folgere aus der Vollständigkeit von \(A\), daß die Folgen \(g_k, h_k\) gegen die gewünschten Polynome \(g, h \in A[x]\)
	konvergieren.)
\end{exercise}

\begin{exercise}
	\begin{enumerate}
	\item
		Sei \((A, \ideal m, F)\) ein lokaler Ring, welcher \(\ideal m\)-adisch vollständig ist. Sei \(f \in A[x]\).
		Sei weiter \(\tilde a \in F\) eine einfache Nullstelle der Reduktion \(\bar f \in F[x]\) von \(f\) modulo \(\ideal m\).
		Zeige, daß \(f\) eine einfache Nullstelle \(a \in A\) mit \(\bar a = \tilde a\) besitzt.
	\item
		Zeige, daß \(2\) eine Quadratwurzel in den \(7\)-adischen ganzen Zahlen \(\set Z_7\) besitzt.
	\item
		Seien \(K\) ein Körper und \(f \in K[x, y]\). Besitze \(f(0, y) \in K[y]\) eine einfache Nullstelle in \(a_0 \in K\).
		Zeige, daß eine formale Potenzreihe \(g = \sum\limits_{n = 0}^\infty a_n x^n \in \ps K x\) mit \(f(x, g(x)) = 0\) existiert.
	\end{enumerate}
\end{exercise}

\begin{exercise}
	Zeige, daß die Umkehrung von \prettyref{thm:compl_is_noeth} falsch ist, selbst unter der Annahme, daß \(A\) lokal ist und
	\(\hat A\) ein endlich erzeugter \(A\)-Modul.
	
	(Tip: Nimm als \(A\) die Lokalisierung des Ringes aller \(\Cont^\infty\)-Funktionen auf \(\set R\) nach dem maximalen Ideal
	der bei \(0\) verschwindenden Funktionen. Benutze den Borelschen Satz, daß jede Potenzreihe über \(\set R\) als Taylorreihe einer
	\(\Cont^\infty\)-Funktion auf \(\set R\) realisiert werden kann.
\end{exercise}

