\section{Operationen mit Idealen}

\subsection{Summe, Schnitt und Produkt von Idealen}

\begin{frame}{Summe und Schnitt von Idealen}
    Sei \(A\) ein Ring.
    \begin{definition}<+->
        Sei \((\ideal a_i)_{i \in I}\) eine Familie von Idealen von \(A\). Dann ist
        ihre \emph{Summe \(\sum\limits_{i \in I} \ideal a_i\)} das kleinste Ideal von
        \(A\), welches die Ideale \(\ideal a_i\) umfaßt.
    \end{definition}
    \begin{visibleenv}<+->
        Es ist \(\sum\limits_{i \in I} \ideal a_i = \left\{\sum\limits_{k = 1}^n x_k
        \mid x_k \in \ideal a_{i(k)}, i_k \in I\right\}\).
    \end{visibleenv}
    \begin{remark}<+->
        Da der Schnitt einer Familie von Idealen wieder ein Ideal ist, bilden die
        Ideale von \(A\) damit einen vollständigen Verband bezüglich der
        Inklusionsordnung.
    \end{remark}
    \begin{remark}<+->
        Die Vereinigung zweier Ideale von \(A\) ist im allgemeinen kein Ideal.
    \end{remark}
\end{frame}

\begin{frame}{Produkt von Idealen}
    Seien \(A\) ein Ring und \(\ideal a, \ideal b\) zwei Ideale von \(A\).
    \begin{definition}<+->
        \begin{enumerate}[<+->]
        \item<.->
            Das \emph{Produkt \(\ideal a \ideal b\) der Ideale \(\ideal a\)
            und \(\ideal b\)} ist das Ideal welches von allen Elementen der Form
            \(x y\) mit \(x \in \ideal a\) und \(y \in \ideal b\) erzeugt wird.
        \item
            Sei \(n \in \set N_0\). Die \emph{\(n\)-te Potenz \(\ideal a^n\) des
            Ideals \(\ideal a\)} ist
            \(\underbrace{\ideal a \dotsm \ideal a}_{n}\). Hierbei setzen wir
            \(\ideal a^0 = (1)\).
        \end{enumerate}
    \end{definition}
    \begin{example}<+->
    	Es ist \(\ideal a \ideal b \subset \ideal a \cap \ideal b\).
    \end{example}
    \begin{proposition}<+->
        Seien \(m, n \in \set N_0\). Dann ist
        \(\ideal a^m \ideal a^n = \ideal a^{m + n}\).
        \qed
    \end{proposition}
\end{frame}

\begin{frame}{Beispiele zu Summen, Schnitten und Produkten von Idealen}
    \begin{example}<+->
        Seien \(m, n \in \set Z\).
        \begin{enumerate}[<+->]
        \item<.->
            Das Ideal \((m) + (n) = (m, n)\) ist das Ideal, welches
            von einem größten gemeinsamen Teiler von \(m\) und \(n\) erzeugt wird.
        \item
            Das Ideal \((m) \cap (n)\) ist das Ideal, welches von einem kleinsten   
            gemeinsamen Vielfachen von \(m\) und \(n\) erzeugt wird.
        \item
            Das Ideal \((m) \cdot (n)\) ist das Ideal, welches vom Produkt \(mn\)
            erzeugt wird.
        \end{enumerate}
    \end{example}
    \begin{example}<+->
        Sei \(K\) ein Körper. Sei \(\ideal m\) das von \(x_1, \dotsc, x_s\) in
        \(K[x_1, \dotsc, x_s]\) erzeugte Ideal. Dann ist \(\ideal m^n\) das Ideal
        aller Polynome ohne Monome mit Grad kleiner als \(n\).
    \end{example}
\end{frame}

\begin{frame}{Rechenregeln mit Idealen}
    Sei \(A\) ein Ring und seien \(\ideal a, \ideal b, \ideal c\) drei Ideale
    von \(A\).
    \begin{proposition}<+->
        \begin{enumerate}[<+->]
        \item<.->
            Summe, Schnitt und Produkt von Idealen sind
            jeweils assoziative Operationen. Summe und Schnitt sind
            außerdem kommutativ. Das Produkt ist kommutativ, wenn der Ring
            kommutativ ist.
        \item
            Es gilt das Distributivgesetz: \(\ideal a (\ideal b + \ideal c)
            = \ideal a \ideal b + \ideal a \ideal c\).
        \item
            Es gilt das Modularitätsgesetz: Ist \(\ideal a \supset \ideal b\)
            oder \(\ideal a \supset \ideal c\), folgt
            \(\ideal a \cap (\ideal b + \ideal c)
            = \ideal a \cap \ideal b + \ideal a \cap \ideal c\).
            \qed
        \end{enumerate}
    \end{proposition}
\end{frame}

\begin{frame}{Koprime Ideale}
    Seien \(\ideal a, \ideal b\) zwei Ideale in einem kommutativen Ring \(A\).
    \begin{definition}<+->
        Die Ideale \(\ideal a\) und \(\ideal b\) heißen \emph{koprim}, wenn
        \(\ideal a + \ideal b = (1)\).
    \end{definition}
    \begin{visibleenv}<+->
        Die Ideale \(\ideal a\) und \(\ideal b\) sind also genau dann koprim,
        wenn ein \(x \in \ideal a\) und ein \(y \in \ideal b\) mit \(x + y = 1\)
        existieren.
    \end{visibleenv}
    \begin{lemma}<+->
        Seien \(\ideal a\) und \(\ideal b\) koprim. Dann gilt \(\ideal a \cap
        \ideal b = \ideal a \ideal b\).
    \end{lemma}
    \begin{proof}<+->
        Es ist
        \(\ideal a \cap \ideal b = (\ideal a + \ideal b)(\ideal a
        \cap \ideal b) = \ideal a (\ideal a \cap \ideal b) + \ideal b (\ideal a
        \cap \ideal b) \subset \ideal a \ideal b \subset \ideal a \cap \ideal b\).
    \end{proof}
    \begin{example}<+->
        Im Ring der ganzen Zahlen sind \((m)\) und \((n)\) genau dann koprim,
        wenn \(m\) und \(n\) teilerfremd sind.
    \end{example}
\end{frame}

\begin{frame}{Der Schnitt paarweise koprimer Ideale}
    \begin{proposition}<+->
        Seien \(\ideal a_1, \dotsc, \ideal a_n\) paarweise koprime Ideale eines
        kommutativen Ringes \(A\). Dann gilt \(\prod\limits_{i = 1}^n \ideal a_i
        = \bigcap\limits_{i = 1}^n \ideal a_i\).
    \end{proposition}
    \begin{proof}<+->
        \begin{enumerate}[<+->]
        \item<.->
            Der Fall \(n = 0\) ist trivial.
        \item
            Sei schon bewiesen, daß \(\ideal b \coloneqq
            \prod\limits_{i = 1}^{n - 1} \ideal a_i
            = \bigcap\limits_{i = 1}^{n - 1} \ideal a_i\). Da \(\ideal a_i\)
            und \(\ideal a_n\) für \(i < n\) koprim sind, existieren \(x_i \in
            \ideal a_i\) und \(y_i \in \ideal a_n\) mit \(x_i + y_i = 1\).
            Damit ist \(\prod\limits_{i = 1}^{n - 1} x_i
            = \prod\limits_{i = 1}^{n - 1} (1 - y_i) = 1\) modulo
            \(\ideal a_n\). Es folgt, daß \(\ideal b\) und \(\ideal a_n\)
            koprim sind, also ist
            \(\prod\limits_{i = 1}^n \ideal a_i = \ideal b \ideal a_n = \ideal b
            \cap \ideal a_n = \bigcap\limits_{i = 1}^n \ideal a_i\).
            \qedhere
        \end{enumerate}
    \end{proof}
\end{frame}

\subsection{Direkte Produkte}

\begin{frame}{Definiton des direkten Produktes von Ringen}
    Sei \((A_i)_{i \in I}\) eine Familie von Ringen. Auf der Menge
    \(A \coloneqq \prod\limits_{i \in I} A_i\) der Folgen
    \(x \coloneqq (x_i)_{i \in I}\)
    mit \(x_i \in A_i\) definieren wir eine Addition und Multiplikation durch
    gliedweise Addition und Multiplikation. Dann wird \(A\) mit der Null
    \((0)_{i \in I}\) und der Eins \((1)_{i \in I}\) zu einem Ring.
    \begin{definition}<+->
        Der Ring \(\prod\limits_{i \in I} A_i\) ist das \emph{direkte Produkt
        über die Familie \((A_i)_{i \in I}\)}.
    \end{definition}
    \begin{proposition}<+->
        Die Projektionen \(\pi_i\colon A \to A_i, x \mapsto x_i\) sind
        Ringhomomorphismen.
        \qed
    \end{proposition}
    \begin{visibleenv}<+->
        Genauer ist die Ringstruktur auf \(A\) gerade so gewählt, daß die \(\pi_i\)
        Ringhomomorphismen werden.
    \end{visibleenv}
\end{frame}

\begin{frame}{Der Chinesische Restsatz}
    \begin{proposition}<+->
        Seien \(\ideal a_1, \dotsc, \ideal a_n\) Ideale in einem kommutativen
        Ring \(A\). Dann ist \(\phi\colon A \to \prod\limits_{i = 1}^n
        A/\ideal a_i, x \mapsto (x + \ideal a_1, \dotsc, x + \ideal a_n)\)
        genau dann surjektiv, wenn die Ideale \(\ideal a_i\) paarweise
        koprim sind.
    \end{proposition}
    \begin{proof}<+->
        \begin{enumerate}[<+->]
        \item<.->
            Sei zunächst \(\phi\) surjektiv. Wir zeigen, daß etwa \(\ideal a_1\)
            und \(\ideal a_2\) koprim sind: Es existiert ein \(x \in A\)
            mit \(\phi(x) = (1, 0, \dotsc, 0)\). Es folgt, daß \(1 = (1 - x) + x
            \in \ideal a_1 + \ideal a_2\).
        \item
            Seien umgekehrt die \(\ideal a_i\) paarweise koprim. Wir
            zeigen, daß ein \(x \in A\) mit \(\phi(x) = (1, 0, \dotsc, 0)\)
            existiert. Da \(\ideal a_1\) und \(\ideal a_i\) für \(i > 1\) koprim
            sind, existieren \(u_i \in \ideal a_1\) und \(v_i \in \ideal a_i\)
            mit \(u_i + v_i = 1\). Setze \(x \coloneqq \prod\limits_{i = 2}^n
            v_i = \prod\limits_{i = 2}^n (1 - u_i)\). Dann ist \(x = 0\) modulo
            \(\ideal a_i\) für \(i > 1\) und \(x = 1\) modulo \(\ideal a_1\).
            \qedhere
        \end{enumerate}
    \end{proof}
\end{frame}

\begin{frame}{Injektivität beim Chinesischen Restsatz}
    Seien \(\ideal a_1, \dotsc, \ideal a_n\) Ideale in einem kommutativen Ring
    \(A\). Dann ist der Kern von \(\phi\colon A \to \prod\limits_{i = 1}^n
    A/\ideal a_i, x \mapsto (x + \ideal a_1, \dotsc, x + \ideal a_n)\)
    offensichtlich durch \(\bigcap\limits_{i = 1}^n \ideal a_i\) gegeben.
    \\
    Insbesondere ist \(\phi\) genau dann injektiv, wenn
    \(\bigcap\limits_{i = 1}^n \ideal a_i = (0)\).
\end{frame}

\subsection{Ideale in Primidealen}

\begin{frame}{Ideale in Vereinigungen von Primidealen}
    \begin{proposition}<+->
    	\label{prop:ideal_in_union_of_primes}
        Seien \(\ideal p_1, \dotsc, \ideal p_n\) Primideale in einem kommutativen Ring
        \(A\). Ist dann \(\ideal a\) ein Ideal von \(A\) mit \(\ideal a
        \subset \bigcup\limits_{i = 1}^n \ideal p_i\), so ist \(\ideal a \subset
        \ideal p_i\) für ein \(i\).
    \end{proposition}
    \begin{proof}<+->
        \begin{enumerate}[<+->]
        \item<.->
            Wir zeigen \((\forall i\colon \ideal a \not\subset \ideal p_i)
            \implies \ideal a \not\subset \bigcup\limits_{i = 1}^n \ideal p_i\).
        \item
            Der Fall \(n = 0\) ist trivial.
        \item
            Sei \(\ideal a \not\subset \ideal p_i\) für alle \(i\). Sei
            schon bewiesen, daß daraus
            \(\ideal a \not\subset \bigcup\limits_{i = 1, i \neq j}^n \ideal p_i\)
            für alle \(j\) folgt. Damit existieren
            \(x_j \in \ideal a\) mit \(x_j \notin \ideal p_i\) für \(i \neq j\).
        \item
            Ist dann \(x_j \notin \ideal p_j\) für ein \(j\) sind wir fertig.
            Ansonsten ist \(x_j \in \ideal p_j\) für alle \(j\). Damit ist
            \(y \coloneqq
            \sum\limits_{j = 1}^n x_1 \dotsm \widehat{x_j} \dotsm x_n
            \in \ideal a\), aber \(y \notin \ideal p_i\) für alle \(i\).
            \qedhere
        \end{enumerate}
    \end{proof}
\end{frame}

\begin{frame}{Schnitte von Idealen in Primidealen}
    \begin{proposition}<+->
        Seien \(\ideal a_1, \dotsc, \ideal a_n\) Ideale in einem kommutativen
        Ring \(A\) und \(\ideal p\) ein Primideal mit \(\ideal p \supset
        \bigcap\limits_{i = 1}^n \ideal a_i\). Dann ist \(\ideal p \supset
        \ideal a_i\) für ein \(i\).
        
        Ist \(\ideal p = \bigcap\limits_{i = 1}^n \ideal a_i\), folgt
        \(\ideal p = \ideal a_i\) für ein \(i\).
    \end{proposition}
    \begin{proof}<+->
        \begin{enumerate}[<+->]
        \item<.->
            Sei \(\ideal p \not\supset \ideal a_i\) für alle \(i\). Dann
            existieren \(x_i \in \ideal a_i\) mit \(x_i \notin \ideal p\).
            Dann ist \(x \coloneqq \prod\limits_{i = 1}^n x_i \subset
            \bigcap\limits_{i = 1}^n \ideal a_i\), aber \(x \notin \ideal p\),
            da \(\ideal p\) prim ist. Damit ist \(\ideal p \not\supset
            \bigcap\limits_{i = 1}^n \ideal a_i\).
        \item
            Ist \(\ideal p = \bigcap\limits_{i = 1}^n \ideal a_i\), ist
            \(\ideal p \subset \ideal a_i\) für alle \(i\) und damit \(\ideal p
            = \ideal a_i\) für ein \(i\).
            \qedhere
        \end{enumerate}
    \end{proof}
\end{frame}

\subsection{Der Idealquotient}

\begin{frame}{Definition des Idealquotienten und des Annihilators}
	Seien \(\ideal a, \ideal b\) zwei Ideale eines kommutativen Ringes \(A\).
	Dann ist \((\ideal a : \ideal b) := \{x \in A \mid x \ideal b \subset \ideal a\}\)
	ein Ideal von \(A\).
	\begin{definition}<+->
		Das Ideal \((\ideal a : \ideal b)\) ist der \emph{Idealquotient von
		\(\ideal a\) nach \(\ideal b\)}.
	\end{definition}
	\begin{notation}<+->
		Ist \(\ideal a\) ein Hauptideal \((x)\), schreiben wir \((x : \ideal b)\)
		anstelle von \(((x) : \ideal b)\). Ist \(\ideal b\) ein Hauptideal \((y)\),
		schreiben wir analog \((\ideal b : y)\) für \((\ideal b : (y))\).
	\end{notation}	
	\begin{definition}<+->
		Das Ideal \((0 : \ideal b)\) ist der \emph{Annulator \(\ann \ideal b\) von
		\(\ideal b\)}.
	\end{definition}
	\begin{visibleenv}<+->
		Es ist also \(\ann \ideal b = \{x \in A \mid x \ideal b = 0\}\).
	\end{visibleenv}
	\\
	\begin{visibleenv}<+->
		Die Menge der Nullteiler von \(A\) ist durch
		\(\bigcup\limits_{x \in A \setminus \{0\}} \ann(x)\)
		gegeben.
	\end{visibleenv}
\end{frame}

\begin{frame}{Der Idealquotient für Ideale im Ring der ganzen Zahlen}
	\begin{example}<+->
		Seien \((m), (n)\) zwei Ideale im Ring der ganzen Zahlen. Seien die
		Primfaktorzerlegungen von \(m\) und \(n\) durch \(m = \prod\limits_p p^{e_p}\)
		und \(n = \prod\limits_p p^{f_p}\) gegeben. Dann ist \((m : n) = (q)\) mit
		\(q = \prod\limits_p p^{\max(e_p - f_p, 0)}\).
		\\
		Es folgt, daß \(q = m/(m, n)\), wobei \((m, n)\) hier für einen größten
		gemeinsamen Teiler von \(m\) und \(n\) steht.
	\end{example}
\end{frame}

\begin{frame}{Rechenregeln für den Idealquotienten}
	\begin{proposition}<+->
		Seien \(\ideal a, \ideal b, \ideal c\) drei Ideale eines kommutativen
		Ringes \(A\).
		\begin{enumerate}[<+->]
		\item<.->
			\(\ideal a \subset (\ideal a : \ideal b)\).
		\item
			\((\ideal a : \ideal b) \ideal b \subset \ideal a\).
		\item
			\(((\ideal a : \ideal b) : \ideal c) = (\ideal a : \ideal b \ideal c)\).
		\item
			Sei \((\ideal a_i)_{i \in I}\) eine Familie von Idealen in \(A\).
			Dann ist \((\bigcap\limits_{i \in I} \ideal a_i \colon \ideal b)
			= \bigcap\limits_{i \in I} (\ideal a_i \colon \ideal b)\).
		\item
			Sei \((\ideal b_i)_{i \in I}\) eine Familie von Idealen in \(A\).
			Dann ist
			\((\ideal a \colon \sum\limits_{i \in I} \ideal b_i)
			= \bigcap\limits_{i \in I} (\ideal a : \ideal b_i)\).
			\qed
		\end{enumerate}
	\end{proposition}
\end{frame}

\subsection{Das Wurzelideal}

\begin{frame}{Definition des Wurzelideals zu einem Ideal}
    Sei \(\ideal a\) ein Ideal eines kommutativen Ringes. Sei
    \(\sqrt\ideal a \coloneqq \{x \in A \mid x^n \in \ideal a\ 
     \text{für ein \(n \in \set N_0\)}\}\).
    \begin{proposition}<+->
        Sei \(\pi\colon A \to A/\ideal a\) der kanonische Homomorphismus und
        \(\ideal n\) das Nilradikal von \(A/\ideal a\). Dann
        ist \(\sqrt \ideal a = \pi^{-1}(\ideal n)\).
        \qed
    \end{proposition}
    \begin{visibleenv}<+->
        Insbesondere ist \(\sqrt\ideal a\) damit ein Ideal.
    \end{visibleenv}
    \begin{definition}<+->
        Das Ideal \(\sqrt\ideal a\) ist das \emph{Wurzelideal zu \(\ideal a\)}.
    \end{definition}
    \begin{example}<+->
        Das Nilradikal von \(A\) ist \(\sqrt{(0)}\).
    \end{example}
\end{frame}

\begin{frame}{Rechenregeln für das Wurzelideal}
    Seien \(\ideal a, \ideal b\) Ideale und \(\ideal p\) ein Primideal eines
    kommutativen Ringes \(A\).
    \begin{proposition}<+->
    	\label{prop:radical}
        \begin{enumerate}[<+->]
        \item<.->
            \(\sqrt{\ideal a} \supset \ideal a\).
        \item
            \(\sqrt{\sqrt{\ideal a}} = \sqrt{\ideal a}\).
        \item
            \(\sqrt{\ideal a \ideal b} = \sqrt{\ideal a \cap \ideal b}
            = \sqrt{\ideal a} \cap \sqrt{\ideal b}\).
        \item
            \(\sqrt{\ideal a} = (1) \iff \ideal a = (1)\).
        \item
            \(\sqrt{\ideal a + \ideal b} = \sqrt{\sqrt{\ideal a}
            + \sqrt{\ideal b}}\).
        \item
            \(\sqrt{\ideal p^n} = \ideal p\) für \(n > 0\).
            \qed
        \end{enumerate}
    \end{proposition}
    \begin{definition}<+->
        Das Ideal \(\ideal a\) heißt \emph{Wurzelideal}, falls \(\sqrt{\ideal a}
        = \ideal a\).
    \end{definition}
\end{frame}

\begin{frame}{Das Wurzelideal als Schnitt von Primidealen}
    \begin{proposition}<+->
        Das Wurzelideal von \(\ideal a\) ist der Schnitt \(\bigcap\limits_{\ideal p \supset \ideal a} \ideal p\) aller Primideale
        \(\ideal p\), welche \(\ideal a\) enthalten.
    \end{proposition}
    \begin{proof}<+->
        Sei \(\pi\colon A \to A/\ideal a\) der kanonische Homomorphismus und
        \(\ideal n\) das Nilradikal von \(A/\ideal a\).
        Dann ist
        \(\bigcap\limits_{\ideal p \supset \ideal a} \ideal p
        = \pi^{-1}(\bigcap\limits_{\bar{\ideal p}} \bar{\ideal p})
        = \pi^{-1}(\ideal n) = \sqrt{\ideal a}\),
	    wobei \(\ideal p\) die Primideale von \(A\) und \(\bar{\ideal p}\) die
	    Primideale von \(A/\ideal a\) durchläuft.
    \end{proof}
\end{frame}

\begin{frame}{Die Menge der Nullteiler als Vereinigung von Wurzelidealen zu
Annulatoren}
    Sei \(A\) ein kommutativer Ring. Analog zum Wurzelideal eines Ideals von
    \(A\) können wir auch die Wurzelmenge \(\sqrt S\) zu einer Teilmenge \(S\)
    von \(A\) definieren. Ist \((S_i)_{i \in I}\) eine Familie von Teilmengen,
    gilt \(\sqrt{\bigcup\limits_{i \in I} S_i} = \bigcup\limits_{i \in I}
    \sqrt{S_i}\).
    \begin{proposition}<+->
        Die Menge \(D\) der Nullteiler von \(A\) ist durch
        \(\bigcup\limits_{x \in A \setminus \{0\}} \sqrt{\ann (x)}\)
        gegeben.
    \end{proposition}
    \begin{proof}<+->
        \(D = \sqrt{D} = \sqrt{\bigcup\limits_{x \in A \setminus \{0\}}
        \ann (x)} = \bigcup\limits_{x \in A \setminus \{0\}} \sqrt{\ann (x)}\).
    \end{proof}
\end{frame}

\begin{frame}{Koprime Wurzelideale}
    \begin{example}<+->
        Sei \((m) \neq (0)\) ein Ideal im Ring der ganzen Zahlen und seien
        \(p_1, \dotsc, p_r\) die verschiedenen Primteiler von \((m)\). Dann ist
        \(\sqrt{(m)} = (p_1 \dotsm p_r) = \bigcap\limits_{i = 1}^r (p_i)\).
    \end{example}
    \begin{proposition}<+->
        Seien \(\ideal a, \ideal b\) Ideale in einem Ring, so daß
        \(\sqrt\ideal a\) und \(\sqrt \ideal b\) koprim sind. Dann sind auch
        \(\ideal a\) und \(\ideal b\) koprim.
    \end{proposition}
    \begin{proof}<+->
        Aus \(\sqrt{\ideal a + \ideal b} = \sqrt{\sqrt{\ideal a}
        + \sqrt{\ideal b}} = \sqrt{(1)} = (1)\) folgt \(\ideal a + \ideal b
        = (1)\).
    \end{proof}
\end{frame}

