% subequations zählt zähler hoch!

% FIXME: Echte Variablen in Polynomringen groß schreiben. Dazu Notation hinzufügen.

\mode<all>{\title{Kommutative Algebra}}
\mode<presentation>{\author[Marc Nieper-Wißkirchen]{Prof.~Dr.~Marc Nieper-Wißkirchen}}
\mode<article>{\author{Marc A.~Nieper-Wißkirchen%
    \thanks{marc.nieper-wisskirchen@math.uni-augsburg.de}}}

\begin{document}

\maketitle

\tableofcontents

\frame{\titlepage}

\part<article>{Ringe und Ideale}

\lecture{Ringe und Ideale}{Ringe und Ideale}
\mode<all>\section{Ringe und Ringhomomorphismen}

\subsection{Ringe}

\begin{frame}{Definition eines Ringes}
    \begin{definition}
        Ein \emph{Ring \(A\)} ist eine Menge mit zwei ausgezeichneten Elementen,
        \(0\) und \(1\), und zwei binären Operationen \(+\) und \(\cdot\), so daß
        \pause
        \begin{enumerate}[<+->]
        \item
            \((A, +, 0)\) eine abelsche Gruppe ist,
        \item
            \((A, \cdot, 1)\) ein Monoid ist und
        \item
            die Multiplikation distributiv über die Addition ist, also
            \(x (y + z) = x y + x z\) und \((y + z) x = y x + z x\) für
            alle \(x, y, z \in A\).
        \end{enumerate}
        \visible<+->{%
            Der Ring heißt \emph{kommutativ}, falls \((A, \cdot, 1)\) ein
            kommutatives Monoid ist.}
    \end{definition}
    \begin{visibleenv}<+->
        Wir nennen \((A, +, 0)\) die \emph{abelsche Gruppe von \(A\)} und
        \((A, \cdot, 1)\) das \emph{multiplikative Monoid von \(A\)}.
    \end{visibleenv}
\end{frame}

\begin{frame}{Multiplikation mit Null}
    Sei \(A\) ein Ring.
    \begin{proposition}<+->
        Sei \(x \in A\). Dann ist \(0 \cdot x = 0\).
    \end{proposition}
    \begin{proof}<+->
        \(0 \cdot x = 0 \cdot x + 0 \cdot x - 0 \cdot x
        = (0 + 0) \cdot x - 0 \cdot x = 0 \cdot x - 0 \cdot x = 0.\)
    \end{proof}
    \begin{corollary}<+->
        Sei \(x \in A\). Dann ist \((-1) \cdot x = -x\).
    \end{corollary}
    \begin{proof}<+->
        \(x + (-1) \cdot x = 1 \cdot x + (-1) \cdot x
        = (1 - 1) \cdot x = 0 \cdot x = 0\).
    \end{proof}
\end{frame}

\begin{frame}{Explizite Beschreibung eines Ringes}
    Eine Menge \(A\) mit zwei ausgezeichneten Elementen \(0\) und \(1\) und
    zwei binären Operationen \(+\) und \(\cdot\) ist also genau dann ein Ring,
    falls für alle \(x, y, z \in A\) gilt:
    \begin{align*}
        x + (y + z) & = (x + y) + z ; \\
        y + x & = x + y; \\
        0 + x & = x; \\
        \exists (-x) \in A\colon -x + x & = 0; \\
        x (y z) & = (x y) z; \\
        1 \cdot x & = x; & x \cdot 1 & = x; \\
        x (y + z) & = x y + x z; & (y + z) x & = y x + z x; \\
        0 \cdot x & = 0; & x \cdot 0 & = 0.
    \end{align*}
\end{frame}

\begin{frame}{Beispiele von Ringen}   
    \begin{example}<+->
        Die ganzen Zahlen \(\set Z\) bilden einen kommutativen Ring. Dieser Ring
        ist grundlegend für die algebraische Zahlentheorie.
    \end{example}
    \begin{example}<+->
        Der Polynomring \(K[x_1, \dotsc, x_n]\) in \(n\) Variablen über einem
        Körper \(K\) ist ein kommutativer Ring. Diese Ringe sind grundlegend für
        die algebraische Geometrie.
    \end{example}
    \begin{visibleenv}<+->
        Bei der Definition eines Ringes ist \(0 = 1\) nicht ausgeschlossen worden:
        \begin{example}<+(-1)->
            Gilt \(0 = 1\) in einem Ring \(A\), folgt \(x = 1 \cdot x
            = 0 \cdot x = 0\) für alle \(x \in A\).
            Damit ist \(A = \{0\}\). Ein solcher Ring heißt der
            \emph{Nullring} und wird mit \(0\) bezeichnet.
        \end{example}
    \end{visibleenv}
\end{frame}

\subsection{Unterringe}

\begin{frame}{Definition eines Unterrings}
    \begin{definition}<+->
        Eine Teilmenge \(B\) eines Ringes \(A\) heißt \emph{Unterring von
        \(A\)}, falls \(B\) eine Untergruppe der abelschen Gruppe von \(A\)
        und ein Untermonoid des multiplikativen Monoides von \(A\) ist.
    \end{definition}
    Wir betrachten \(B\) auf offensichtliche Weise wieder als Ring.
    \\
    \begin{visibleenv}<+->
        Eine Teilmenge \(B\) von \(A\) ist also genau dann ein Unterring,
        falls für alle \(x, y \in B\) gilt:
        \begin{align*}
            x + y & \in B; \\
            -x & \in B; \\
            0 & \in B; \\
            x y & \in B; \\
            1 & \in B.
        \end{align*}
    \end{visibleenv}
\end{frame}

\begin{frame}{Beispiele für Unterringe}
    \begin{example}<+->
        Die ganzen Zahlen \(\set Z\) bilden einen Unterring der rationalen Zahlen
        \(\set Q\).
    \end{example}
    \begin{example}<+->
        Wir können einen Körper \(K\) als Unterring des Ringes \(K[x]\) der Polynome
        in einer Variablen über \(K\) auffassen.
    \end{example}
    \begin{example}<+->
        Der (bezüglich der Inklusionsordnung) größte Unterring eines Ringes \(A\)
        ist der Ring \(A\) selbst.
    \end{example}
\end{frame}

\begin{frame}{Der von einer Teilmenge erzeugte Unterring}
    \begin{proposition}<+->
        Ist \((B_i)_{i \in I}\) eine Familie von Unterringen eines
        Ringes \(A\),
        so ist der Schnitt \(\bigcap\limits_{i \in I} B_i\) ein Unterring von \(A\).
    \end{proposition}
    \begin{proof}<+->
        Da die \(B_i\) jeweils \(0\) und \(1\) enthalten und jeweils abgeschlossen
        unter Addition und Multiplikation sind, folgt, daß auch der Schnitt \(0\)
        und \(1\) enthält und abgeschlossen unter Addition und Multiplikation ist.
    \end{proof}
    \begin{construction}<+->
        Sei \(S\) eine Teilmenge eines Ringes \(A\). Dann ist der Schnitt
        aller Unterringe von \(A\), welche \(S\) enthalten, der kleinste Unterring
        von \(A\), welcher \(S\) enthält.
        \qed
    \end{construction}        
\end{frame}

\subsection{Ringhomomorphismen}

\begin{frame}{Definition eines Ringhomomorphismus}
    \begin{definition}<+->
        Ein \emph{Ringhomomorphismus \(\phi\)} ist eine Abbildung
        \(\phi\colon A \to B\) zwischen zwei Ringen, welche einen Homomorphismus
        zwischen den abelschen Gruppen und einen Homomorphismus zwischen den
        multiplikativen Monoiden von \(A\) und \(B\) induziert.
    \end{definition}
    \begin{visibleenv}<+->
        Eine Abbildung \(\phi\colon A \to B\) ist also genau dann ein
        Ringhomomorphismus, falls für alle \(x, y \in A\) gilt:
        \begin{align*}
        \phi(x + y) & = \phi(x) + \phi(y); \tag{$*$}
        \\
        \phi(0)     & = 0 && \text{(folgt schon aus ($*$))}; \\
        \phi(x y)   & = \phi(x) \phi(y); \\
        \phi(1)     & = 1.
        \end{align*}
    \end{visibleenv}
\end{frame}

\begin{frame}{Beispiele von Ringhomomorphismen}
    \begin{example}<+->
        Die Identität \(\id_A\) eines Ringes ist ein Ringhomomorphismus.
    \end{example}
    \begin{example}<+->
        Die Inklusion \(B \injto A\) eines Unterrings ist ein Ringhomomorphismus.
    \end{example}
    \begin{proposition}<+->
        Seien \(\phi\colon A \to B\) und \(\psi\colon B \to C\)
        Ringhomomorphismen. Dann ist \(\psi \circ \phi\colon A \to C\) ein
        Ringhomomorphismus.
    \end{proposition}
    \begin{proof}<+->
        Die Verknüpfung zweier Gruppen- bzw. Monoidhomomorphismen ist wieder
        ein Gruppen- bzw.~Monoidhomomorphismus.
    \end{proof}
\end{frame}

\begin{frame}{Abbildungen als Ringhomomorphismen}
    \begin{visibleenv}<+->
        Seien \(A\) und \(B\) zwei Ringe und \(\phi\colon A \to B\) eine Abbildung.
    \end{visibleenv}
    \begin{proposition}<+->
        Sei \(\pi\colon C \surjto A\) ein surjektiver Ringhomomorphismus. Ist dann
        \(\phi \circ \pi\colon C \to B\) ein Ringhomomorphismus, so auch \(\phi\colon A \to B\).
    \end{proposition}
    \begin{proof}<+->
        Seien \(a, a' \in A\). Dann existieren \(c, c' \in C\) mit \(\pi(c) = a\)
        und \(\pi(c') = a'\).
        Es folgt \(\phi(a a') = \phi(\pi(c) \pi(c')) = \phi(\pi(c c')) = \phi(\pi(c))
        \phi(\pi(c')) = \phi(a) \phi(a')\). Analog erhält \(\phi\) auch Addition,
        \(0\) und \(1\).
    \end{proof}
    \begin{proposition}<+->
        Sei \(\iota\colon B \injto C\) ein injektiver Ringhomomorphismus. Ist dann
        \(\iota \circ \phi\colon A \to C\) ein Ringhomomorphismus, so auch \(\phi\colon A \to B\).
        \qed
    \end{proposition}
\end{frame}

\begin{frame}{Ringisomorphismen}
    \begin{visibleenv}<+->
        Sei \(\phi\colon A \to B\) ein Ringhomomorphismus.
    \end{visibleenv}    
    \begin{definition}<.->
        Der Homomorphismus \(\phi\) heißt \emph{Isomorphismus}, falls
        ein Ringhomomorphismus \(\check\phi\colon B \to A\) existiert, so daß
        \(\check \phi \circ \phi = \id_A\) und \(\phi \circ \check \phi = \id_B\).
    \end{definition}
    \begin{proposition}<+->
        Ist \(\phi\colon A \to B\) eine bijektive Abbildung, so ist \(\phi\) schon ein 
        Isomorphismus.
    \end{proposition}
    \begin{proof}<+->
        Da \(\id_B = \phi \circ \phi^{-1}\) ein Ringhomomorphismus ist und
        \(\phi\) ein injektiver Ringhomomorphismus ist, ist auch \(\phi^{-1}\)
        ein Ringhomomorphismus.
    \end{proof}
\end{frame}

\subsection{Bezeichnungen}

\begin{frame}{Bezeichnungen von Ringen, Ringelementen und Ringhomomorphismen}
    \begin{convention}<+->
        Ringe bezeichnen wir mit großen lateinischen Buchstaben
        \(A, B, C, \dotsc\), Ringelemente mit kleinen lateinischen Buchstaben
        \(a, b, c, f, g, h, x, y, z, \dotsc\).
    \end{convention}
    \begin{visibleenv}<+->
        Betrachten wir einen Ring als Rechenbereich, nennen wir seine Elemente
        auch \emph{Zahlen}. Betrachten wir einen Ring als Algebra von Funktionen auf
        einem Raum, heißen die Ringelemente entsprechend \emph{Funktionen}.
    \end{visibleenv}
    \begin{convention}<+->
        Ringhomomorphismen bezeichnen wir mit kleinen griechischen
        Buchstaben \(\phi, \psi, \dotsc\).
    \end{convention}
\end{frame}


\mode<all>\section{Ideale und Quotientenringe}

\subsection{Ideale}

\begin{frame}{Definition eines Ideals}
    \begin{definition}<+->
        Eine Teilmenge \(\ideal a\) eines Ringes \(A\) heißt
        \emph{(beidseitiges) Ideal von \(A\)}, falls
        \begin{enumerate}[<+->]
        \item<.->
            \(\ideal a\) eine Untergruppe der abelschen Gruppe von \(A\) ist und
        \item
            \(\ideal a\) abgeschlossen unter der Multiplikation mit
            Elementen aus \(A\) ist, also \(x a \in \ideal a\) und
            \(a x \in \ideal a\) für alle \(a \in \ideal a\) und \(x \in A\).
        \end{enumerate}
    \end{definition}
    \begin{remark}<+->
        Im allgemeinen ist ein Ideal kein Unterring und ein Unterring kein Ideal.
    \end{remark}
    \begin{visibleenv}<+->
        Eine Teilmenge \(\ideal a\) von \(A\) ist genau dann ein Ideal,
        falls für alle \(a, a' \in \ideal a\) und \(x \in A\) gilt:
        \begin{align*}
            a + a' & \in \ideal a; &
            0 & \in \ideal a; \\
            x a & \in \ideal a; &
            a x & \in \ideal a.
        \end{align*}
    \end{visibleenv}
\end{frame}

\begin{frame}{Das von einer Teilmenge erzeugte Ideal}
    \begin{proposition}<+->
        Ist \((\ideal a_i)_{i \in I}\) eine Familie von Idealen eines
        Ringes \(A\),
        so ist der Schnitt \(\bigcap\limits_{i \in I} \ideal a_i\) ein Ideal
        von \(A\).
        \qed
    \end{proposition}
    \begin{definition}<+->
        Sei \((f_i)_{i \in I}\) eine Familie von Elementen eines Ringes \(A\).
        Dann heißt der Schnitt \((f_i \mid i \in I)\) aller Ideale
        von \(A\), welche jedes \(f_i\) enthalten, das \emph{von
        der Familie \((f_i)_{i \in I}\) erzeugte Ideal in \(A\)}.
    \end{definition}
    \begin{proposition}<+->
        Das von einer Familie von Elementen in einem Ring \(A\) erzeugte Ideal
        ist das kleinste Ideal in \(A\), welches jedes Element der Familie
        enthält.
        \qed
    \end{proposition}
\end{frame}

\begin{frame}{Beispiele für Ideale}
    \begin{example}<+->
        Sei \((f_i)_{i \in I}\) eine Familie von Elementen in einem kommutativen
        Ring \(A\). Dann ist
        \((f_i \mid i \in I)
        = \left\{\sum\limits_{k = 1}^n a_k f_{i_k} \mid a_k \in A,
        i_k \in I\right\}\).
    \end{example}
    \begin{example}<+->
        Sei \(n \in \set Z\) eine ganze Zahl. Dann ist das Ideal \((n)\) die Menge
        der durch \(n\) teilbaren Zahlen.
    \end{example}
    \begin{example}<+->
        Das \emph{Nullideal \((0)\)} ist das kleinste Ideal in einem Ring, denn
        \((0) = \{0\}\).
    \end{example}
    \begin{example}<+->
        Das \emph{Einsideal \((1)\)} ist das größte Ideal in einem Ring \(A\), denn
        \((1) = A\).
    \end{example}
\end{frame}

\subsection{Bild und Kern}

\begin{frame}{Bilder und Urbilder unter Ringhomomorphismen}
    Sei \(\phi\colon A \to B\) ein Ringhomomorphismus.
    \begin{proposition}<+->
        \begin{enumerate}[<+->]
        \item<.->
            Für jedes Ideal \(\ideal b\) von \(B\) ist \(\ideal a \coloneqq
            \phi^{-1}(\ideal b)\) ein Ideal von \(A\).
        \item
            Für jeden Unterring \(A'\) von \(A\) ist \(\phi(A')\) ein Unterring von
            \(B\).        
        \end{enumerate}
    \end{proposition}
    \begin{proof}<+->
        \begin{enumerate}[<+->]
        \item<.->
            Als Urbild einer Untergruppe unter einem Gruppenhomomorphismus ist
            \(\ideal a\) eine Untergruppe der abelschen Gruppe von
            \(A\).
        \item
            Für \(a \in \ideal a\) und \(x \in A\) ist \(\phi(x a) = \phi(x)
            \phi(a) \in \ideal b\), also \(x a \in \ideal a\). Analog ist
            \(a x \in \ideal a\).
        \item
            Bilder von Untergruppen bzw.~-monoiden unter
            Gruppen- bzw.~Monoidhomomorphismen sind Untergruppen
            bzw.~-monoide.
            \qedhere
        \end{enumerate}
    \end{proof}
\end{frame}

\begin{frame}{Mengentheoretisches Bild und Kern eines Ringhomomorphismus}
    \begin{visibleenv}<+->
        Sei \(\phi\colon A \to B\) ein Ringhomomorphismus.
    \end{visibleenv}
    \begin{example}<.->
        Das mengentheoretische Bild \(\phi(A)\) von \(\phi\) ist ein Unterring von
        \(B\).
    \end{example}
    \begin{example}<+->
        Das Urbild \(\phi^{-1}((0)) = \{x \in A \mid \phi(x) = 0\}\) des Nullideals
        von \(B\) ist ein Ideal von \(A\).
    \end{example}    
    \begin{definition}<+->    
        Das Ideal \(\phi^{-1}((0))\) heißt der \emph{Kern \(\ker \phi\) von \(\phi\)}.
    \end{definition}   
    \begin{proposition}<+->
        Der Homomorphismus \(\phi\) ist genau dann injektiv, wenn
        \(\ker \phi = (0)\).
        \qed
    \end{proposition}
\end{frame}

\subsection{Quotientenringe}

\begin{frame}{Quotientenringe}
    \begin{proposition}<+->
        Sei \(\ideal a\) ein Ideal eines Ringes \(A\). Dann gibt es genau eine
        Ringstruktur auf der Menge \(A/\ideal a\) der
        \(\ideal a\)\nobreakdash-Nebenklassen,
        so daß die kanonische Abbildung \(\pi\colon A \surjto
        A/\ideal a, x \mapsto [x] := x + \ideal a\) ein Ringhomomorphismus ist.
    \end{proposition}
    \begin{proof}<+->
        \begin{enumerate}[<+->]
        \item<.->
            Auf \(A/\ideal a\) existiert genau eine Struktur einer abelschen
            Gruppe, so daß \(\pi\) ein Gruppenhomomorphismus abelscher Gruppen
            wird.
        \item
            Da \(\pi((x + a) (y + a')) = \pi(x y + x a' + a y + a a')
            = \pi(x y)\) für \(x, y \in A\) und
            \(a, a' \in \ideal a\), gibt es auf \(A/\ideal a\) genau eine
            Multiplikation, welche von \(\pi\) respektiert wird,
            nämlich \((x + \ideal a) (y + \ideal a) = xy + \ideal a\).
        \item
            Schließlich muß die Eins in \(A/\ideal a\) das Bild von \(1 \in A\)
            unter \(\pi\) sein.
        \item
            Da \(\pi\) surjektiv ist, folgen die Ringaxiome für \(A/\ideal a\)
            aus denen für \(A\).
            \qedhere
        \end{enumerate} 
    \end{proof}
\end{frame}

\begin{frame}{Kern des kanonischen Homomorphismus}
	Sei \(A\) ein Ring.
	\begin{example}<+->
		Seien \(\ideal a\) ein Ideal und \(\pi\colon
		A \surjto A/\ideal a\) der kanonische Homomorphismus. Dann ist
		\(\ker \pi = \ideal a\).
	\end{example}
	\begin{example}<+->
		Ist \(x \in A\) ein Element, so ist \(x = 0\) genau dann, wenn der
		kanonische Homomorphismus \(A \surjto A/(x)\) injektiv ist.
	\end{example}
\end{frame}

\begin{frame}{Ideale in Quotientenringen}
    \begin{visibleenv}<+->
        Seien \(\ideal a\) ein Ideal in einem Ring \(A\) und
        \(\pi\colon A \surjto A/\ideal a\) der kanonische Homomorphismus.
    \end{visibleenv}
    \begin{definition}<.->
        Der Ring \(A/\ideal a\) heißt der \emph{Quotientenring von \(A\) nach
        \(\ideal a\)}.
    \end{definition}
    \begin{proposition}<+->
        Durch \(\ideal x = \pi^{-1}(\bar{\ideal x})\) wird eine bijektive
        ordnungserhaltende Korrespondenz zwischen den
        Idealen \(\ideal x\) von \(A\) mit \(\ideal x \supset \ideal a\) und den
        Idealen \(\bar{\ideal x}\) von \(A/\ideal a\) gegeben.
    \end{proposition}
    \begin{proof}<+->
        \begin{enumerate}[<+->]
        \item<.->
            Ist \(\ideal y\) ein Ideal von \(A\), so ist
            \(\bar{\ideal y} \coloneqq \pi(\ideal y)\) ein Ideal von
            \(A/\ideal a\), da \(\pi\) surjektiv ist.
        \item
            Es ist \(\pi^{-1}(\bar {\ideal y}) = \ideal y + \ideal a\).
        \qedhere
        \end{enumerate}
    \end{proof}
\end{frame}

\begin{frame}{Der Homomorphiesatz für Ringe}
    \begin{proposition}<+->[Homomorphiesatz für Ringe]
        Sei \(\phi\colon A \to B\) ein Homomorphismus von Ringen. Dann gibt es
        einen Ringisomorphismus
        \(\underline\phi\colon A/\ker \phi \to \phi(A),\ [x] \mapsto \phi(x)\).
    \end{proposition}
    \begin{proof}<+->
        \begin{enumerate}
        \item<.->
            Nach dem Homomorphiesatz für abelsche Gruppen wird
            durch \([x] \mapsto \phi(x)\) ein Gruppenisomorphismus definiert.
        \item
            Da die kanonische Abbildung \(\pi\colon A \to A/\ker \phi\) surjektiv ist
            und \(\pi\) und \(\underline\phi \circ \pi = \phi\) Ringhomomorphismen sind,
            ist auch \(\underline\phi\) ein Ringhomomorphismus.
            \qedhere
        \end{enumerate}
    \end{proof}
    \begin{remark}<+->
        Ideale sind also genau diejenigen Teilmengen von Ringen, welche
        Kerne von surjektiven Homomorphismen sind.
    \end{remark}
\end{frame}

\subsection{Bezeichnungen}

\begin{frame}{Bezeichnungen von Idealen und Konventionen für Quotientenringe}
    \begin{convention}<+->
        Ideale bezeichnen wir mit kleinen deutschen Buchstaben \(\ideal a, \ideal b,
        \ideal c, \dots\).
    \end{convention}
    \begin{visibleenv}<+->
        Sei \(\ideal a\) ein Ideal in einem Ring \(A\). Sei \(x \in A\).\\
        Die Nebenklasse \([x] \in A/\ideal a\) von \(x\) bezeichnen wir häufig
        wieder mit \(x\),\\
        eventuell mit dem Zusatz "`in \(A/\ideal a\)"' oder
        "`modulo \(\ideal a\)"'.\\
        Anstelle von \([x] = 0\) sagen wir etwa, daß "`\(x = 0\) in
        \(A/\ideal a\)"'\\
        oder "`\(x = 0\) modulo \(\ideal a\)"'.
    \end{visibleenv}
\end{frame}



\lecture{Primideale und maximale Ideale}{Primideale und maximale Ideale}
\mode<all>\setcounter{section}{2}
\mode<all>\section{Nullteiler, nilpotente Elemente und Einheiten}

\subsection{Integritätsbereiche}

\begin{frame}{Reguläre Elemente und Nullteiler}
    \begin{visibleenv}<+->
        Sei \(x\) ein Element eines kommutativen Ringes \(A\).
    \end{visibleenv}
    \begin{definition}<.->
        \begin{enumerate}[<+->]
        \item<.->
            Das Element \(x\) heißt \emph{regulär}, falls für alle \(y \in A\)
            aus \(x y = 0\) schon \(y = 0\) folgt.
        \item
            Ein Element \(x\) ist ein \emph{Nullteiler}, wenn es nicht regulär ist.
        \item
            Der Ring \(A\) heißt ein \emph{Integritätsbereich}, falls \{0\} der
            einzige Nullteiler in \(A\) ist.
        \end{enumerate}
    \end{definition}
    \begin{visibleenv}<+->
        Das Element \(x\) ist also genau dann ein Nullteiler, falls ein
        \(y \in A\) mit \(y \neq 0\), aber \(x y = 0\) existiert.
    \end{visibleenv}
    \\
    \begin{visibleenv}<+->
        Der Ring \(A\) ist weiter genau dann ein Integritätsbereich,\\
        wenn \(0 \neq 1\)
        in \(A\)\\ und aus \(x y = 0\) für \(x, y \in A\) schon \(x = 0\) oder
        \(y = 0\) folgt.
    \end{visibleenv}
\end{frame}

\begin{frame}{Unterringe von Integritätsbereichen}
	\begin{proposition}<+->
		Sei \(\phi\colon A \to B\) ein injektiver Homomorphismus kommutativer Ringe.
		Ist dann \(B\) ein Integritätsbereich, so auch \(A\).
	\end{proposition}
	\begin{proof}<+->
		Sei also \(B\) ein Integritätsbereich.
		Da \(0 \neq 1\) in \(B\), muß wegen \(\phi(0) = 0\) und \(\phi(1) = 1\)
		auch \(0 \neq 1\) in \(A\) gelten.
		Seien weiter \(f, g \in A\) mit \(f g = 0\) gegeben. Es folgt
		\(\phi(fg) = 0\) in \(B\). Damit ist \(\phi(f) = 0\) oder \(\phi(g) = 0\).
		Da \(\phi\) injektiv ist, folgt dann \(f = 0\) oder \(g = 0\).
	\end{proof}
\end{frame}

\begin{frame}{Beispiele zu Integritätsbereichen}
    \begin{example}<+->
        Der Nullring ist kein Integritätsbereich, denn die Null ist im Nullring 
        regulär.  
    \end{example}
    \begin{example}<+->
        Der Ring \(\set Z\) der ganzen Zahlen ist ein Integritätsbereich.
    \end{example}
    \begin{visibleenv}<+->
        Sei \(K\) ein Körper.
    \end{visibleenv}
    \begin{example}<.->
        Der Polynomring \(K[x_1, \dotsc, x_n]\) in \(n\) Variablen über einem Körper
        \(K\) ist ein Integritätsbereich.
    \end{example}
    \begin{example}<+->
        Der Ring \(A \coloneqq K[x, y]/(x y)\) ist kein Integritätsbereich, da zum
        Beispiel \(x\) und \(y\) Nullteiler in \(A\) sind.
    \end{example}
\end{frame}

\begin{frame}{Hauptidealbereiche}
    \begin{definition}<+->
        \begin{enumerate}[<+->]
        \item<.->
            Ein Ideal \(\ideal a\) in einem kommutativen Ring \(A\) heißt
            \emph{Hauptideal}, falls \(\ideal a = (a)\) für ein \(a \in A\).
        \item
            Ein Integritätsbereich \(A\) heißt \emph{Hauptidealbereich}, falls 
            jedes Ideal in \(A\) ein Hauptideal ist.
        \end{enumerate}
    \end{definition}
    \begin{visibleenv}<+->
        Sei \(K\) ein Körper.
    \end{visibleenv}
    \begin{example}<+->
        Die Ringe \(\set Z\) und \(K[x]\) sind Hauptidealbereiche.
    \end{example}
    \begin{remark}<+->
        Für \(n \ge 2\) ist \(K[x_1, \dotsc, x_n]\) kein
        Hauptidealbereich. Das Ideal \((x_1, \dotsc, x_n)\)
        läßt sich nicht von weniger als \(n\) Elementen erzeugen.
    \end{remark}
\end{frame}

\subsection{Nilpotente Elemente}

\begin{frame}{Nilpotente Elemente}
    \begin{definition}<+->
        Ein Element \(x \in A\) eines kommutativen Ringes \(A\) heißt
        \emph{nilpotent}, falls ein \(n \in \set N_0\)
        mit \(x^n = 0\) existiert.
    \end{definition}
    \begin{example}<+->
        In jedem kommutativen Ring ist \(0\) ein nilpotentes Element.
    \end{example}
    \begin{proposition}<+->
        Ist \(A\) ein Integritätsbereich, so ist \(0\) das einzige nilpotente Element
        von \(A\).
    \end{proposition}
    \begin{proof}<+->
        Sei \(x \in A\) nilpotent und \(n \in \set N_0\) die kleinste natürliche Zahl
        mit \(x^n = 0\). Dann ist \(n \ge 1\). Aus \(x \cdot x^{n - 1} = 0\)
        und \(x^{n - 1} \neq 0\)
        folgt dann \(x = 0\).
    \end{proof}
\end{frame}

\begin{frame}{Beispiele für nilpotente Elemente}
    \begin{example}<+->
        Sei \(K\) ein Körper. Im Ring \(K[x]/(x^2)\) ist \(x\) ein nilpotentes
        Element.
    \end{example}
    \begin{example}<+->
        Sei \(\phi\colon A \to B\) ein Homomorphismus von Ringen. Sei \(a\) ein
        nilpotentes Element von \(A\). Dann ist \(\phi(a)\) ein nilpotentes Element
        von \(B\).
    \end{example}
\end{frame}

\subsection{Einheiten}

\begin{frame}{Einheiten}
    \begin{visibleenv}<+->
        Sei \(x \in A\) ein Element eines Ringes.
    \end{visibleenv}
    \begin{definition}<+->
        \begin{enumerate}[<+->]
        \item<.->
            Das Element \(x\) heißt \emph{Einheit in \(A\)}, falls \(x\) 
            Einheit des multiplikativen Monoides von \(A\) ist.
        \item<+->
            Die Untergruppe \(A^\units\) im multiplikativen Monoid von \(A\), die
            von den Einheiten in \(A\) gebildet wird, heißt die
            \emph{Einheitengruppe von \(A\)}.
        \item
            Der Ring \(A\) heißt \emph{Schiefkörper}, falls \{0\} die einzige
            Nichteinheit in \(A\) ist. Ein kommutativer Schiefkörper heißt
            \emph{Körper}.
        \end{enumerate}       
    \end{definition}
    \begin{visibleenv}<+->
        Das Element \(x\) ist also genau dann eine Einheit, falls ein \(y \in A\) mit
        \(x y = 1 = y x\) existiert, nämlich die \emph{Inverse} \(y = x^{-1}\)
        von \(x\).
    \end{visibleenv}
    \begin{example}<+->
        Der Nullring ist kein Körper, denn die Null ist im Nullring invertierbar.
    \end{example}
\end{frame}

\begin{frame}{Beispiel für Einheiten und Einheitengruppen}
    \begin{example}<+->
        Die Einheitengruppe des Ringes \(\set Z\) der ganzen Zahlen ist
        \(\set Z^\units = \{1, -1\}\). 
        Insbesondere bilden die ganzen Zahlen keinen Körper.
    \end{example}
    \begin{example}<+->
        Die Einheitengruppe des Ringes \(\set Q\) der rationalen Zahlen ist
        \(\set Q^\units = \set Q \setminus \{0\}\). Damit ist \(\set Q\) ein
        Körper.
    \end{example}
    \begin{example}<+->
        Sei \(a \in A\) ein Element eines kommutativen Ringes. In \(A[x]/(xa - 1)\)
        ist \(a\) eine Einheit mit Inverse \(x\).
    \end{example}
    \begin{example}<+->
        Seien \(a, b \in A\) Elemente eines kommutativen Ringes, so daß \(ab\)
        eine Einheit ist.
        Dann ist auch \(a\) eine Einheit mit \(a^{-1} = b (ab)^{-1}\). 
    \end{example}
\end{frame}

\begin{frame}{Beispiele und Propositionen über Einheiten}
    \begin{visibleenv}<+->
        Sei \(x \in A\) ein Element eines kommutativen Ringes \(A\).
    \end{visibleenv}
    \begin{proposition}<+->
        Das Element \(x\) ist genau dann eine Einheit, falls
        \((x) = (1)\).
    \end{proposition}
    \begin{proof}<+->
        Es ist \(1 = x^{-1} \cdot x \in (x)\) und damit \((1) \subset (x)\), also \((1) = (x)\), falls \(x\) Einheit ist.
        \\
        Die umgekehrte Implikation ist ebenso klar.
    \end{proof}
    \begin{proposition}<+->
        Ist \(x\) eine Einheit, ist \(x\) auch regulär.
    \end{proposition}
    \begin{proof}<+->
        Sei \(x \cdot y = 0\) für ein \(y \in A\). Multiplizieren mit \(x^{-1}\) von
        links liefert \(y = 0\).
    \end{proof}
\end{frame}

\subsection{Charakterisierung von Körpern}

\begin{frame}{Ein Lemma über Ideale in Körpern}
    \begin{lemma}<+->
        Sei \(A\) ein Körper. Dann besitzt \(A\) genau zwei Ideale (nämlich
        \((0)\) und \((1)\)).
    \end{lemma}
    \begin{proof}<+->
        Sei \(A\) ein Körper. Da \(0\) keine Einheit ist, ist \((0) \neq (1)\).
        Sei jetzt \(\ideal a \neq (0)\) ein Ideal in \(A\). Dann existiert ein
        \(x \in \ideal a\) mit \(x \neq 0\), also \(x \in A^\units\). Es folgt
        \((1) = (x) \subset \ideal a\), also \(\ideal a = (1)\).
    \end{proof}
\end{frame}

\begin{frame}{Ein Lemma über Ringe mit genau zwei Idealen}
    \begin{lemma}<+->
        Sei \(\phi\colon A \to B\) ein Ringhomomorphismus kommutativer Ringe.
        Besitze \(A\) genau zwei
        Ideale. Dann ist \(\phi\) genau dann injektiv, wenn \(B\) nicht der Nullring
        ist.
    \end{lemma}
    \begin{proof}<+->
        \begin{enumerate}[<+->]
        \item<.->
            Sei \(\ker \phi = (0)\). Dann ist \(\phi\) injektiv. Weiter ist
            \(1 = \phi(1) \neq 0 \in B\), da \(1 \notin (0) \subset A\), also ist
            \(B\) nicht der Nullring.
        \item
            Sei \(\ker \phi = (1)\). Dann ist \(\phi\) nicht injektiv, da
            \((1) \neq (0) \subset A\). Weiter ist \(1 = \phi(1) = 0 \in B\), also ist
            \(B\) der Nullring.
            \qedhere
        \end{enumerate}
    \end{proof}
\end{frame}

\begin{frame}{Ein Lemma über die Charakterisierung von Körpern}
    \begin{lemma}<+->
        Sei \(A\) ein kommutativer Ring, so daß jeder Ringhomomorphismus \(\phi\colon A \to B\)
        in einen weiteren kommutativen Ring \(B\) genau dann injektiv ist, wenn \(B\) nicht der
        Nullring ist. Dann ist \(A\) ein Körper.
    \end{lemma}
    \begin{proof}<+->
        Sei \(x \in A\). Dann gilt:
        \(x \in A^\units \iff A/(x) = 0 \iff \ker(A \surjto A/(x)) \neq 0
        \iff x \neq 0\).
    \end{proof}
\end{frame}

\begin{frame}{Eine Proposition über die Charakterisierung von Körpern}
    \begin{visibleenv}<+->
        Zusammengefaßt haben wir also gezeigt:
    \end{visibleenv}
    \begin{proposition}<.->
        Sei \(A\) ein kommutativer Ring. Dann sind die folgenden Aussagen äquivalent:
        \begin{enumerate}
        \item
            \(A\) ist ein Körper.
        \item
            \(A\) besitzt genau zwei Ideale (nämlich \((0)\) und \((1)\)).
        \item
            Ein Ringhomomorphismus \(A \to B\) in einen kommutativen Ring \(B\) ist genau dann
            injektiv, wenn \(B\) nicht der Nullring ist.
            \qedhere
        \end{enumerate}
    \end{proposition}
\end{frame}


\mode<all>\section{Primideale und maximale Ideale}

\subsection{Primideale}

\begin{frame}{Definition von Prim- und maximalen Idealen}
    Seien \(\ideal p, \ideal m\) zwei Ideale eines kommutativen Ringes \(A\).
    \begin{definition}<+->
        \begin{enumerate}[<+->]
        \item<.->
            Das Ideal \(\ideal p\) von \(A\) heißt \emph{prim},
            falls \(1 \notin \ideal p\) und falls
            aus \(a b \in \ideal p\) für \(a, b \in A\) schon \(a \in \ideal p\)
            oder \(b \in \ideal p\) folgt.
        \item
            Das Ideal \(\ideal m\) von \(A\) heißt \emph{maximal},
            falls für jedes Ideal \(\ideal a\) von \(A\)
            mit \(\ideal a \supset \ideal m\) \alert{entweder} \(\ideal a = \ideal m\)
            \alert{oder} \(\ideal a = (1)\) gilt.
        \end{enumerate}
    \end{definition}
    \begin{proposition}<+->
        Das Ideal \(\ideal p\) ist genau dann prim, wenn \(A/\ideal p\) ein 
        Integritätsbereich ist.
        \qed
    \end{proposition}
    \begin{example}<+->
        Das Nullideal ist genau dann prim, wenn \(A\) ein Integritätsbereich ist.
    \end{example}
\end{frame}

\begin{frame}{Beispiele von Primidealen}
    \begin{example}<+->
        Jedes Ideal in \(\set Z\) ist von der Form \((m)\) mit \(m \ge 0\). Das 
        Ideal \((m)\) ist genau dann prim, falls \(m = 0\) oder \(m\) eine Primzahl
        ist. Im Falle, daß \(p\) eine Primzahl ist, ist \((p)\) ein maximales Ideal.
    \end{example}
    \begin{example}<+->
        Sei \(f \in A \coloneqq K[x_1, \dotsc, x_n]\) ein irreduzibles Polynom über
        dem Körper \(K\). Da \(A\) ein faktorieller Ring ist, ist \((f)\) ein 
        Primideal von \(A\).
        
        Das Ideal \((x_1, \dotsc, x_n)\) ist ein maximales Ideal in \(A\).
    \end{example}
    \begin{proposition}<+->
        Sei \(\ideal a\) ein Ideal eines kommutativen Ringes \(A\) und
        \(\pi\colon A \to A/\ideal a\) der kanonische Homomorphismus.
        Durch \(\ideal p = \pi^{-1}(\bar{\ideal p})\) wird eine bijektive 
        ordnungserhaltende Korrespondenz zwischen den Primidealen \(\ideal p\)
        von \(A\) mit \(\ideal p \supset \ideal a\) und den Primidealen
        \(\bar{\ideal p}\) von \(A/\ideal a\) gegeben.
        \qed
    \end{proposition}
\end{frame}

\begin{frame}{Primideale in Hauptidealbereichen}
    \begin{proposition}<+->
        Sei \(\ideal p\) ein Primideal in einem Hauptidealbereich \(A\). Ist
        \(\ideal p \neq (0)\), so ist \(\ideal p\) ein maximales Ideal.
    \end{proposition}
    \begin{proof}<+->
        \begin{enumerate}[<+->]
        \item<.->
            Sei \(\ideal p = (p)\) mit \(p \neq 0\). Sei \((q) \supset (p)\). Wir
            müssen zeigen, daß \(q\) eine Einheit ist oder daß \((q) = (p)\).
        \item
            Da \((q) \supset (p)\), existiert ein \(x \in A\) mit \(p = x q\).
        \item
            Aus \(x q \in (p)\) folgt \(x \in (p)\) oder \(q \in (p)\).
        \item
            Sei \(x \in (p)\). Dann existiert ein \(y \in A\) mit \(x = y p\), also
            \(p = y q p\). Da \(p \neq 0\), folgt \(y q = 1\). Also ist \(q\) eine
            Einheit.
        \item
            Sei \(q \in (p)\). Dann ist \((q) \subset (p)\) und damit \((q) = (p)\).
            \qedhere
        \end{enumerate}
    \end{proof}
\end{frame}

\subsection{Maximale Ideale}

\begin{frame}{Maximale Ideale}
    Sei \(A\) ein kommutativer Ring.
    \begin{proposition}<+->
        Sei \(\ideal a\) ein Ideal von \(A\) und \(\pi\colon A \to A/\ideal a\) der
        kanonische Homomorphismus.
        Durch \(\ideal m = \pi^{-1}(\bar{\ideal m})\) wird eine bijektive 
        ordnungserhaltende Korrespondenz zwischen den maximalen Idealen \(\ideal m\)
        von \(A\) mit \(\ideal m \supset \ideal a\) und den maximalen Idealen
        \(\bar{\ideal m}\) von \(A/\ideal a\) gegeben.
        \qed
    \end{proposition}
    \begin{proposition}<+->
        Ein Ideal \(\ideal m\) von \(A\) ist genau dann maximal, wenn
        \(A/\ideal m\) ein Körper ist.
        \qed
    \end{proposition}
    \begin{corollary}<+->
        Jedes maximale Ideal von \(A\) ist prim.
        \qed
    \end{corollary}
\end{frame}

\begin{frame}{Existenz maximaler Ideale}
    \begin{theorem}<+->
        Ein Ring \(A\) besitzt genau dann ein maximales Ideal, wenn \(A\) nicht der
        Nullring ist.
    \end{theorem}
    \begin{proof}<+->
        \begin{enumerate}[<+->]
        \item<.->
            Ein maximales Ideal von \(A\) ist gerade ein maximales Element des
            Systems \(\mathfrak A\) aller Ideale \(\ideal a\) von \(A\) mit
            \(\ideal a \neq (1)\) bezüglich der Inklusionsordnung.
        \item
            Jede Kette \(\mathfrak C\) in \(\mathfrak A\) besitzt eine obere Schranke
            in \(\mathfrak A\), nämlich \(\bigcup \mathfrak C\).
        \item
            Nach dem Zornschen Lemma besitzt \(\mathfrak A\) damit genau dann ein
            maximales Element, falls \(\mathfrak A\) nicht leer ist.
        \item
            Das System \(\mathfrak A\) ist genau nicht leer, wenn \(A\) nicht der
            Nullring ist, denn dann ist \((0) \in \mathfrak A\).
            \qedhere
        \end{enumerate}
    \end{proof}
\end{frame}

\begin{frame}{Folgerungen aus der Existenz maximaler Ideale}
    Sei \(A\) ein kommutativer Ring.
    \begin{corollary}<+->
        Sei \(\ideal a\) ein Ideal von \(A\). Dann existiert genau dann ein
        maximales Ideal \(\ideal m\) von \(A\) mit \(\ideal a \subset \ideal m\),
        wenn \(\ideal a \neq (1)\).
    \end{corollary}
    \begin{proof}<+->
        Der Ring \(A/\ideal a\) besitzt genau dann ein maximales Ideal, wenn
        \(\ideal a \neq (1)\). Ein solches maximales Ideal entspricht einem maximalen
        Ideal \(\ideal m\) von \(A\) mit \(\ideal m \supset \ideal a\).
    \end{proof}
    \begin{corollary}<+->
        Ein Element \(x\) von \(A\) liegt genau dann in einem maximalen Ideal von
        \(A\), wenn \(x\) keine Einheit ist.
        \qed
    \end{corollary}
\end{frame}

\subsection{Lokale Ringe}

\begin{frame}{Lokale Ringe}
    \begin{definition}<+->
        Ein \emph{lokaler Ring \(A\)} ist ein kommutativer Ring mit genau einem
        maximalen Ideal \(\ideal m\). Der Körper \(A/\ideal m\) ist der
        \emph{Restklassenkörper von \(A\)}.
    \end{definition}
    \begin{example}<+->
        Körper sind lokale Ringe.
    \end{example}
    \begin{notation}<+->
        Ist \(A\) ein lokaler Ring mit maximalem Ideal \(\ideal m\) und
        Restklassenkörper \(F\), so schreiben wir häufig \((A, \ideal m)\) oder
        \((A, \ideal m, F)\) für \(A\).
    \end{notation}
    \begin{visibleenv}<+->
        Die folgende Definition verallgemeinert den Begriff des lokalen Ringes.
    \end{visibleenv}
    \begin{definition}<.->
        Ein \emph{halblokaler Ring \(A\)} ist ein kommutativer Ring mit nur
        endlich vielen maximalen Idealen.
    \end{definition}
\end{frame}

\begin{frame}{Ringe, in denen die Nichteinheiten ein Ideal bilden}
    \begin{proposition}<+->
        Sei \(\ideal m\) ein Ideal eines kommutativen Ringes \(A\), so daß
        \(A \setminus \ideal m = A^\units\). Dann ist \((A, \ideal m)\) ein
        lokaler Ring.
    \end{proposition}
    \begin{proof}<+->
        Sei \(\ideal a \neq (1)\) ein Ideal von \(A\). Wegen \(1 \neq \ideal a\)
        enthält \(\ideal a\) keine Einheiten. Also ist
        \(\ideal a \subset A \setminus A^\units = \ideal m\). Folglich
        ist \(\ideal m\) ein maximales Ideal und das einzige.
    \end{proof}
\end{frame}

\begin{frame}{Ein Kriterium für einen lokalen Ring}
    \begin{proposition}<+->
        Sei \(\ideal m\) ein maximales Ideal eines kommutativen Ringes \(A\),
        so daß jedes Element von \(1 + \ideal m\) eine Einheit in \(A\) ist.
        Dann ist \(A\) ein lokaler Ring.
    \end{proposition}
    \begin{proof}<+->
        \begin{enumerate}[<+->]
        \item<.->
            Wir zeigen, daß \(A \setminus \ideal m = A^\units\). Sei dazu
            \(x \in A \setminus \ideal m\).
        \item
            Da \(\ideal m\) ein maximales Ideal ist, ist das von \(\ideal m\)
            und \(x\) aufgespannte Ideal das Einsideal.
        \item
            Damit existieren ein \(y \in A\) und ein \(t \in \ideal m\) mit
            \(x y + t = 1\), also \(x y = 1 - t \in 1 + \ideal m\).
        \item
            Nach Voraussetzung ist \(x y\) damit eine Einheit. Folglich ist
            auch \(x \in A^\units\).
            \qedhere
        \end{enumerate}
    \end{proof}
\end{frame}

\subsection{Bezeichnungen}

\begin{frame}{Bezeichnungen für Primideale}
    \begin{convention}<+->
        Primideale bezeichnen wir mit kleinen deutschen Buchstaben
        \(\ideal p, \ideal q, \dotsc\).
    \end{convention}
    \begin{convention}<+->
        Maximale Ideale bezeichnen wir mit kleinen deutschen Buchstaben
        \(\ideal m, \ideal n, \dotsc\).
    \end{convention}    
\end{frame}



\lecture{Operationen mit Idealen}{Operationen mit Idealen}
\mode<all>\setcounter{section}{4}
\mode<all>\section{Das Nil- und das Jacobsonsche Radikal}

\subsection{Das Nilradikal}

\begin{frame}{Definition des Nilradikals}
    \begin{proposition}<+->
        Sei \(A\) ein kommutativer Ring. Die Menge \(\ideal n\) der nilpotenten
        Elemente von \(A\) ist ein Ideal, das \alert{Nilradikal von \(A\)}. Der Ring
        \(A/\ideal n\) hat außer \(0\) keine nilpotenten Elemente.
    \end{proposition}
    \begin{proof}<+->
        \begin{enumerate}[<+->]
        \item<.->
            Es ist \(0 \in \ideal n\). Ist \(x \in \ideal n\) und \(a \in A\), so
            ist auch \(a x \in \ideal n\), denn \((a x)^n = a^n x^n = 0\) für
            \(n \gg 0\).
        \item   
            Seien \(x, y \in \ideal n\), etwa \(x^m = 0\) und \(y^n = 0\) für
            \(n, m \ge 0\). Nach dem in jedem kommutativen Ring gültigen Binomialsatz
            ist dann
            \((x + y)^{m + n - 1} = \sum\limits_{r + s = m + n - 1} \binom{m + n - 1} r
            x^r y^s = 0\). Also folgt \(x + y \in \ideal n\).
        \item
            Sei \(x \in A\). Sei \(x\) nilpotent in \(A/\ideal n\), das heißt
            \(x^n \in \ideal n\) für \(n \gg 0\). Nach Definition von \(\ideal n\)
            ist \(x^{kn} = (x^n)^k = 0\) für \(k \gg 0\). Folglich ist auch
            \(x \in \ideal n\). Damit ist \(x = 0\) in \(A/\ideal n\).
            \qedhere
        \end{enumerate}
    \end{proof}
\end{frame}

\begin{frame}{Der Schnitt aller Primideale}
    \begin{proposition}<+->
        Das Nilradikal \(\ideal n\) eines kommutativen Ringes \(A\) ist der Schnitt
        \(\bigcap \ideal p\) aller seiner Primideale.
    \end{proposition}
    \begin{proof}<+->
        \begin{enumerate}[<+->]
        \item<.->
            Sei \(f \in \ideal n\), also \(f^n = 0\) für \(n \gg 0\). Ist \(\ideal p\)
            ein Primideal, so folgt aus \(f^n = 0 \in \ideal p\) schon \(f \in \ideal p\),
            also \(f \in \bigcap \ideal p\).
        \item
            Sei umgekehrt \(f \notin \ideal n\). Sei \(\mathfrak A\) die Menge der
            Ideale \(\ideal a\) von \(A\) mit \(f^n \notin \ideal a\) für alle
            \(n \ge 0\). Da \((0) \in \mathfrak A\), besitzt \(\mathfrak A\) nach dem 
            Zornschen Lemma ein maximales Element \(\ideal p \neq (1)\). Insbesondere
            \(f \notin \ideal p\).
        \item
            Es reicht dann zu zeigen, daß \(\ideal p\) prim ist. Dazu seien \(x, y
            \in A \setminus \ideal p\) gegeben. Nach Maximalität von \(\ideal p\)
            ist \(f^m\) im von \(\ideal p\) und \(x\) aufgespannten Ideal
            \(\ideal p + (x)\) für ein
            \(m \ge 0\). Analog ist \(f^n \in \ideal p + (y)\) für ein \(n \ge 0\).
            Es folgt \(f^{m + n} \in \ideal p + (xy)\). Damit ist \(xy \notin \ideal p\).
            \qedhere
        \end{enumerate}
    \end{proof}
\end{frame}

\subsection{Das Jacobsonsche Radikal}

\begin{frame}{Das Jacobsonsche Radikal}
    Sei \(A\) ein kommutativer Ring.
    \begin{definition}<+->
        Das \emph{Jacobsonsche Radikal von \(A\)} ist der Schnitt aller maximalen 
        Ideale von \(A\).
    \end{definition}
    \begin{proposition}
        Ein Element \(x\) von \(A\) ist genau dann im Jacobsonschen Radikal
        \(\ideal j\), wenn \(1 - xy\) für alle \(y \in A\) eine Einheit ist.
    \end{proposition}
    \begin{proof}<+->
        \begin{enumerate}[<+->]
        \item<.->
            Sei \(1 - xy\) keine Einheit. Dann ist \(1 - xy \in \ideal m\) für ein
            maximales Ideal \(\ideal m\). Da \(1 \notin \ideal m\), folgt
            \(xy \notin \ideal m\), also \(x \notin \ideal j\).
        \item
            Sei umgekehrt \(x \notin \ideal j\), also \(x \notin \ideal m\) für ein
            maximales Ideal \(\ideal m\). Damit erzeugen \(x\) und \(\ideal m\) das
            Einsideal, also existiert ein \(y \in A\), so daß \(1 - xy \in \ideal m\).
            Damit ist \(1 - xy\) keine Einheit.
            \qedhere
        \end{enumerate}
    \end{proof}
\end{frame}

\begin{frame}{Interpretation des Jacobsonschen Radikals}
    \begin{visibleenv}<+->
        Das Jacobsonsche Radikal besteht also aus allen Elementen eines kommutativen
        Ringes, welche in einem gewissen Sinne "`nahe bei \(0\) sind"'.
    \end{visibleenv}
\end{frame}


\mode<all>\section{Operationen mit Idealen}

\subsection{Summe, Schnitt und Produkt von Idealen}

\begin{frame}{Summe und Schnitt von Idealen}
    Sei \(A\) ein Ring.
    \begin{definition}<+->
        Sei \((\ideal a_i)_{i \in I}\) eine Familie von Idealen von \(A\). Dann ist
        ihre \emph{Summe \(\sum\limits_{i \in I} \ideal a_i\)} das kleinste Ideal von
        \(A\), welches die Ideale \(\ideal a_i\) umfaßt.
    \end{definition}
    \begin{visibleenv}<+->
        Es ist \(\sum\limits_{i \in I} \ideal a_i = \left\{\sum\limits_{k = 1}^n x_k
        \mid x_k \in \ideal a_{i(k)}, i_k \in I\right\}\).
    \end{visibleenv}
    \begin{remark}<+->
        Da der Schnitt einer Familie von Idealen wieder ein Ideal ist, bilden die
        Ideale von \(A\) damit einen vollständigen Verband bezüglich der
        Inklusionsordnung.
    \end{remark}
    \begin{remark}<+->
        Die Vereinigung zweier Ideale von \(A\) ist im allgemeinen kein Ideal.
    \end{remark}
\end{frame}

\begin{frame}{Produkt von Idealen}
    Seien \(A\) ein Ring und \(\ideal a, \ideal b\) zwei Ideale von \(A\).
    \begin{definition}<+->
        \begin{enumerate}[<+->]
        \item<.->
            Das \emph{Produkt \(\ideal a \ideal b\) der Ideale \(\ideal a\)
            und \(\ideal b\)} ist das Ideal welches von allen Elementen der Form
            \(x y\) mit \(x \in \ideal a\) und \(y \in \ideal b\) erzeugt wird.
        \item
            Sei \(n \in \set N_0\). Die \emph{\(n\)-te Potenz \(\ideal a^n\) des
            Ideals \(\ideal a\)} ist
            \(\underbrace{\ideal a \dotsm \ideal a}_{n}\). Hierbei setzen wir
            \(\ideal a^0 = (1)\).
        \end{enumerate}
    \end{definition}
    \begin{example}<+->
    	Es ist \(\ideal a \ideal b \subset \ideal a \cap \ideal b\).
    \end{example}
    \begin{proposition}<+->
        Seien \(m, n \in \set N_0\). Dann ist
        \(\ideal a^m \ideal a^n = \ideal a^{m + n}\).
        \qed
    \end{proposition}
\end{frame}

\begin{frame}{Beispiele zu Summen, Schnitten und Produkten von Idealen}
    \begin{example}<+->
        Seien \(m, n \in \set Z\).
        \begin{enumerate}[<+->]
        \item<.->
            Das Ideal \((m) + (n) = (m, n)\) ist das Ideal, welches
            von einem größten gemeinsamen Teiler von \(m\) und \(n\) erzeugt wird.
        \item
            Das Ideal \((m) \cap (n)\) ist das Ideal, welches von einem kleinsten   
            gemeinsamen Vielfachen von \(m\) und \(n\) erzeugt wird.
        \item
            Das Ideal \((m) \cdot (n)\) ist das Ideal, welches vom Produkt \(mn\)
            erzeugt wird.
        \end{enumerate}
    \end{example}
    \begin{example}<+->
        Sei \(K\) ein Körper. Sei \(\ideal m\) das von \(x_1, \dotsc, x_s\) in
        \(K[x_1, \dotsc, x_s]\) erzeugte Ideal. Dann ist \(\ideal m^n\) das Ideal
        aller Polynome ohne Monome mit Grad kleiner als \(n\).
    \end{example}
\end{frame}

\begin{frame}{Rechenregeln mit Idealen}
    Sei \(A\) ein Ring und seien \(\ideal a, \ideal b, \ideal c\) drei Ideale
    von \(A\).
    \begin{proposition}<+->
        \begin{enumerate}[<+->]
        \item<.->
            Summe, Schnitt und Produkt von Idealen sind
            jeweils assoziative Operationen. Summe und Schnitt sind
            außerdem kommutativ. Das Produkt ist kommutativ, wenn der Ring
            kommutativ ist.
        \item
            Es gilt das Distributivgesetz: \(\ideal a (\ideal b + \ideal c)
            = \ideal a \ideal b + \ideal a \ideal c\).
        \item
            Es gilt das Modularitätsgesetz: Ist \(\ideal a \supset \ideal b\)
            oder \(\ideal a \supset \ideal c\), folgt
            \(\ideal a \cap (\ideal b + \ideal c)
            = \ideal a \cap \ideal b + \ideal a \cap \ideal c\).
            \qed
        \end{enumerate}
    \end{proposition}
\end{frame}

\begin{frame}{Koprime Ideale}
    Seien \(\ideal a, \ideal b\) zwei Ideale in einem kommutativen Ring \(A\).
    \begin{definition}<+->
        Die Ideale \(\ideal a\) und \(\ideal b\) heißen \emph{koprim}, wenn
        \(\ideal a + \ideal b = (1)\).
    \end{definition}
    \begin{visibleenv}<+->
        Die Ideale \(\ideal a\) und \(\ideal b\) sind also genau dann koprim,
        wenn ein \(x \in \ideal a\) und ein \(y \in \ideal b\) mit \(x + y = 1\)
        existieren.
    \end{visibleenv}
    \begin{lemma}<+->
        Seien \(\ideal a\) und \(\ideal b\) koprim. Dann gilt \(\ideal a \cap
        \ideal b = \ideal a \ideal b\).
    \end{lemma}
    \begin{proof}<+->
        Es ist
        \(\ideal a \cap \ideal b = (\ideal a + \ideal b)(\ideal a
        \cap \ideal b) = \ideal a (\ideal a \cap \ideal b) + \ideal b (\ideal a
        \cap \ideal b) \subset \ideal a \ideal b \subset \ideal a \cap \ideal b\).
    \end{proof}
    \begin{example}<+->
        Im Ring der ganzen Zahlen sind \((m)\) und \((n)\) genau dann koprim,
        wenn \(m\) und \(n\) teilerfremd sind.
    \end{example}
\end{frame}

\begin{frame}{Der Schnitt paarweise koprimer Ideale}
    \begin{proposition}<+->
        Seien \(\ideal a_1, \dotsc, \ideal a_n\) paarweise koprime Ideale eines
        kommutativen Ringes \(A\). Dann gilt \(\prod\limits_{i = 1}^n \ideal a_i
        = \bigcap\limits_{i = 1}^n \ideal a_i\).
    \end{proposition}
    \begin{proof}<+->
        \begin{enumerate}[<+->]
        \item<.->
            Der Fall \(n = 0\) ist trivial.
        \item
            Sei schon bewiesen, daß \(\ideal b \coloneqq
            \prod\limits_{i = 1}^{n - 1} \ideal a_i
            = \bigcap\limits_{i = 1}^{n - 1} \ideal a_i\). Da \(\ideal a_i\)
            und \(\ideal a_n\) für \(i < n\) koprim sind, existieren \(x_i \in
            \ideal a_i\) und \(y_i \in \ideal a_n\) mit \(x_i + y_i = 1\).
            Damit ist \(\prod\limits_{i = 1}^{n - 1} x_i
            = \prod\limits_{i = 1}^{n - 1} (1 - y_i) = 1\) modulo
            \(\ideal a_n\). Es folgt, daß \(\ideal b\) und \(\ideal a_n\)
            koprim sind, also ist
            \(\prod\limits_{i = 1}^n \ideal a_i = \ideal b \ideal a_n = \ideal b
            \cap \ideal a_n = \bigcap\limits_{i = 1}^n \ideal a_i\).
            \qedhere
        \end{enumerate}
    \end{proof}
\end{frame}

\subsection{Direkte Produkte}

\begin{frame}{Definiton des direkten Produktes von Ringen}
    Sei \((A_i)_{i \in I}\) eine Familie von Ringen. Auf der Menge
    \(A \coloneqq \prod\limits_{i \in I} A_i\) der Folgen
    \(x \coloneqq (x_i)_{i \in I}\)
    mit \(x_i \in A_i\) definieren wir eine Addition und Multiplikation durch
    gliedweise Addition und Multiplikation. Dann wird \(A\) mit der Null
    \((0)_{i \in I}\) und der Eins \((1)_{i \in I}\) zu einem Ring.
    \begin{definition}<+->
        Der Ring \(\prod\limits_{i \in I} A_i\) ist das \emph{direkte Produkt
        über die Familie \((A_i)_{i \in I}\)}.
    \end{definition}
    \begin{proposition}<+->
        Die Projektionen \(\pi_i\colon A \to A_i, x \mapsto x_i\) sind
        Ringhomomorphismen.
        \qed
    \end{proposition}
    \begin{visibleenv}<+->
        Genauer ist die Ringstruktur auf \(A\) gerade so gewählt, daß die \(\pi_i\)
        Ringhomomorphismen werden.
    \end{visibleenv}
\end{frame}

\begin{frame}{Der Chinesische Restsatz}
    \begin{proposition}<+->
        Seien \(\ideal a_1, \dotsc, \ideal a_n\) Ideale in einem kommutativen
        Ring \(A\). Dann ist \(\phi\colon A \to \prod\limits_{i = 1}^n
        A/\ideal a_i, x \mapsto (x + \ideal a_1, \dotsc, x + \ideal a_n)\)
        genau dann surjektiv, wenn die Ideale \(\ideal a_i\) paarweise
        teilerfremd sind.
    \end{proposition}
    \begin{proof}<+->
        \begin{enumerate}[<+->]
        \item<.->
            Sei zunächst \(\phi\) surjektiv. Wir zeigen, daß etwa \(\ideal a_1\)
            und \(\ideal a_2\) koprim sind: Es existiert ein \(x \in A\)
            mit \(\phi(x) = (1, 0, \dotsc, 0)\). Es folgt, daß \(1 = (1 - x) + x
            \in \ideal a_1 + \ideal a_2\).
        \item
            Seien umgekehrt die \(\ideal a_i\) paarweise teilerfremd. Wir
            zeigen, daß ein \(x \in A\) mit \(\phi(x) = (1, 0, \dotsc, 0)\)
            existiert. Da \(\ideal a_1\) und \(\ideal a_i\) für \(i > 1\) koprim
            sind, existieren \(u_i \in \ideal a_1\) und \(v_i \in \ideal a_i\)
            mit \(u_i + v_i = 1\). Setze \(x \coloneqq \prod\limits_{i = 2}^n
            v_i = \prod\limits_{i = 2}^n (1 - u_i)\). Dann ist \(x = 0\) modulo
            \(\ideal a_i\) für \(i > 1\) und \(x = 1\) modulo \(\ideal a_1\).
            \qedhere
        \end{enumerate}
    \end{proof}
\end{frame}

\begin{frame}{Injektivität beim Chinesischen Restsatz}
    Seien \(\ideal a_1, \dotsc, \ideal a_n\) Ideale in einem kommutativen Ring
    \(A\). Dann ist der Kern von \(\phi\colon A \to \prod\limits_{i = 1}^n
    A/\ideal a_i, x \mapsto (x + \ideal a_1, \dotsc, x + \ideal a_n)\)
    offensichtlich durch \(\bigcap\limits_{i = 1}^n \ideal a_i\) gegeben.
    \\
    Insbesondere ist \(\phi\) genau dann injektiv, wenn
    \(\bigcap\limits_{i = 1}^n \ideal a_i = (0)\).
\end{frame}

\subsection{Ideale in Primidealen}

\begin{frame}{Ideale in Vereinigungen von Primidealen}
    \begin{proposition}<+->
    	\label{prop:ideal_in_union_of_primes}
        Seien \(\ideal p_1, \dotsc, \ideal p_n\) Primideale in einem kommutativen Ring
        \(A\). Ist dann \(\ideal a\) ein Ideal von \(A\) mit \(\ideal a
        \subset \bigcup\limits_{i = 1}^n \ideal p_i\), so ist \(\ideal a \subset
        \ideal p_i\) für ein \(i\).
    \end{proposition}
    \begin{proof}<+->
        \begin{enumerate}[<+->]
        \item<.->
            Wir zeigen \((\forall i\colon \ideal a \not\subset \ideal p_i)
            \implies \ideal a \not\subset \bigcup\limits_{i = 1}^n \ideal p_i\).
        \item
            Der Fall \(n = 0\) ist trivial.
        \item
            Sei \(\ideal a \not\subset \ideal p_i\) für alle \(i\). Sei
            schon bewiesen, daß daraus
            \(\ideal a \not\subset \bigcup\limits_{i = 1, i \neq j}^n \ideal p_i\)
            für alle \(j\) folgt. Damit existieren
            \(x_j \in \ideal a\) mit \(x_j \notin \ideal p_i\) für \(i \neq j\).
        \item
            Ist dann \(x_j \notin \ideal p_j\) für ein \(j\) sind wir fertig.
            Ansonsten ist \(x_j \in \ideal p_j\) für alle \(j\). Damit ist
            \(y \coloneqq
            \sum\limits_{j = 1}^n x_1 \dotsm \widehat{x_j} \dotsm x_n
            \in \ideal a\), aber \(y \notin \ideal p_i\) für alle \(i\).
            \qedhere
        \end{enumerate}
    \end{proof}
\end{frame}

\begin{frame}{Schnitte von Idealen in Primidealen}
    \begin{proposition}<+->
        Seien \(\ideal a_1, \dotsc, \ideal a_n\) Ideale in einem kommutativen
        Ring \(A\) und \(\ideal p\) ein Primideal mit \(\ideal p \supset
        \bigcap\limits_{i = 1}^n \ideal a_i\). Dann ist \(\ideal p \supset
        \ideal a_i\) für ein \(i\).
        
        Ist \(\ideal p = \bigcap\limits_{i = 1}^n \ideal a_i\), folgt
        \(\ideal p = \ideal a_i\) für ein \(i\).
    \end{proposition}
    \begin{proof}<+->
        \begin{enumerate}[<+->]
        \item<.->
            Sei \(\ideal p \not\supset \ideal a_i\) für alle \(i\). Dann
            existieren \(x_i \in \ideal a_i\) mit \(x_i \notin \ideal p\).
            Dann ist \(x \coloneqq \prod\limits_{i = 1}^n x_i \subset
            \bigcap\limits_{i = 1}^n \ideal a_i\), aber \(x \notin \ideal p\),
            da \(\ideal p\) prim ist. Damit ist \(\ideal p \not\supset
            \bigcap\limits_{i = 1}^n \ideal a_i\).
        \item
            Ist \(\ideal p = \bigcap\limits_{i = 1}^n \ideal a_i\), ist
            \(\ideal p \subset \ideal a_i\) für alle \(i\) und damit \(\ideal p
            = \ideal a_i\) für ein \(i\).
            \qedhere
        \end{enumerate}
    \end{proof}
\end{frame}

\subsection{Der Idealquotient}

\begin{frame}{Definition des Idealquotienten und des Annihilators}
	Seien \(\ideal a, \ideal b\) zwei Ideale eines kommutativen Ringes \(A\).
	Dann ist \((\ideal a : \ideal b) := \{x \in A \mid x \ideal b \subset \ideal a\}\)
	ein Ideal von \(A\).
	\begin{definition}<+->
		Das Ideal \((\ideal a : \ideal b)\) ist der \emph{Idealquotient von
		\(\ideal a\) nach \(\ideal b\)}.
	\end{definition}
	\begin{notation}<+->
		Ist \(\ideal a\) ein Hauptideal \((x)\), schreiben wir \((x : \ideal b)\)
		anstelle von \(((x) : \ideal b)\). Ist \(\ideal b\) ein Hauptideal \((y)\),
		schreiben wir analog \((\ideal b : y)\) für \((\ideal b : (y))\).
	\end{notation}	
	\begin{definition}<+->
		Das Ideal \((0 : \ideal b)\) ist der \emph{Annulator \(\ann \ideal b\) von
		\(\ideal b\)}.
	\end{definition}
	\begin{visibleenv}<+->
		Es ist also \(\ann \ideal b = \{x \in A \mid x \ideal b = 0\}\).
	\end{visibleenv}
	\\
	\begin{visibleenv}<+->
		Die Menge der Nullteiler von \(A\) ist durch
		\(\bigcup\limits_{x \in A \setminus \{0\}} \ann(x)\)
		gegeben.
	\end{visibleenv}
\end{frame}

\begin{frame}{Der Idealquotient für Ideale im Ring der ganzen Zahlen}
	\begin{example}<+->
		Seien \((m), (n)\) zwei Ideale im Ring der ganzen Zahlen. Seien die
		Primfaktorzerlegungen von \(m\) und \(n\) durch \(m = \prod\limits_p p^{e_p}\)
		und \(n = \prod\limits_p p^{f_p}\) gegeben. Dann ist \((m : n) = (q)\) mit
		\(q = \prod\limits_p p^{\max(e_p - f_p, 0)}\).
		\\
		Es folgt, daß \(q = m/(m, n)\), wobei \((m, n)\) hier für einen größten
		gemeinsamen Teiler von \(m\) und \(n\) steht.
	\end{example}
\end{frame}

\begin{frame}{Rechenregeln für den Idealquotienten}
	\begin{proposition}<+->
		Seien \(\ideal a, \ideal b, \ideal c\) drei Ideale eines kommutativen
		Ringes \(A\).
		\begin{enumerate}[<+->]
		\item<.->
			\(\ideal a \subset (\ideal a : \ideal b)\).
		\item
			\((\ideal a : \ideal b) \ideal b \subset \ideal a\).
		\item
			\(((\ideal a : \ideal b) : \ideal c) = (\ideal a : \ideal b \ideal c)\).
		\item
			Sei \((\ideal a_i)_{i \in I}\) eine Familie von Idealen in \(A\).
			Dann ist \((\bigcap\limits_{i \in I} \ideal a_i \colon \ideal b)
			= \bigcap\limits_{i \in I} (\ideal a_i \colon \ideal b)\).
		\item
			Sei \((\ideal b_i)_{i \in I}\) eine Familie von Idealen in \(A\).
			Dann ist
			\((\ideal a \colon \sum\limits_{i \in I} \ideal b_i)
			= \bigcap\limits_{i \in I} (\ideal a : \ideal b_i)\).
			\qed
		\end{enumerate}
	\end{proposition}
\end{frame}

\subsection{Das Wurzelideal}

\begin{frame}{Definition des Wurzelideals zu einem Ideal}
    Sei \(\ideal a\) ein Ideal eines kommutativen Ringes. Sei
    \(\sqrt\ideal a \coloneqq \{x \in A \mid x^n \in \ideal a\ 
     \text{für ein \(n \in \set N_0\)}\}\).
    \begin{proposition}<+->
        Sei \(\pi\colon A \to A/\ideal a\) der kanonische Homomorphismus und
        \(\ideal n\) das Nilradikal von \(A/\ideal a\). Dann
        ist \(\sqrt \ideal a = \pi^{-1}(\ideal n)\).
        \qed
    \end{proposition}
    \begin{visibleenv}<+->
        Insbesondere ist \(\sqrt\ideal a\) damit ein Ideal.
    \end{visibleenv}
    \begin{definition}<+->
        Das Ideal \(\sqrt\ideal a\) ist das \emph{Wurzelideal zu \(\ideal a\)}.
    \end{definition}
    \begin{example}<+->
        Das Nilradikal von \(A\) ist \(\sqrt{(0)}\).
    \end{example}
\end{frame}

\begin{frame}{Rechenregeln für das Wurzelideal}
    Seien \(\ideal a, \ideal b\) Ideale und \(\ideal p\) ein Primideal eines
    kommutativen Ringes \(A\).
    \begin{proposition}<+->
    	\label{prop:radical}
        \begin{enumerate}[<+->]
        \item<.->
            \(\sqrt{\ideal a} \supset \ideal a\).
        \item
            \(\sqrt{\sqrt{\ideal a}} = \sqrt{\ideal a}\).
        \item
            \(\sqrt{\ideal a \ideal b} = \sqrt{\ideal a \cap \ideal b}
            = \sqrt{\ideal a} \cap \sqrt{\ideal b}\).
        \item
            \(\sqrt{\ideal a} = (1) \iff \ideal a = (1)\).
        \item
            \(\sqrt{\ideal a + \ideal b} = \sqrt{\sqrt{\ideal a}
            + \sqrt{\ideal b}}\).
        \item
            \(\sqrt{\ideal p^n} = \ideal p\) für \(n > 0\).
            \qed
        \end{enumerate}
    \end{proposition}
    \begin{definition}<+->
        Das Ideal \(\ideal a\) heißt \emph{Wurzelideal}, falls \(\sqrt{\ideal a}
        = \ideal a\).
    \end{definition}
\end{frame}

\begin{frame}{Das Wurzelideal als Schnitt von Primidealen}
    \begin{proposition}<+->
        Das Wurzelideal von \(\ideal a\) ist der Schnitt \(\bigcap\limits_{\ideal p \supset \ideal a} \ideal p\) aller Primideale
        \(\ideal p\), welche \(\ideal a\) enthalten.
    \end{proposition}
    \begin{proof}<+->
        Sei \(\pi\colon A \to A/\ideal a\) der kanonische Homomorphismus und
        \(\ideal n\) das Nilradikal von \(A/\ideal a\).
        Dann ist
        \(\bigcap\limits_{\ideal p \supset \ideal a} \ideal p
        = \pi^{-1}(\bigcap\limits_{\bar{\ideal p}} \bar{\ideal p})
        = \pi^{-1}(\ideal n) = \sqrt{\ideal a}\),
	    wobei \(\ideal p\) die Primideale von \(A\) und \(\bar{\ideal p}\) die
	    Primideale von \(A/\ideal a\) durchläuft.
    \end{proof}
\end{frame}

\begin{frame}{Die Menge der Nullteiler als Vereinigung von Wurzelidealen zu
Annulatoren}
    Sei \(A\) ein kommutativer Ring. Analog zum Wurzelideal eines Ideals von
    \(A\) können wir auch die Wurzelmenge \(\sqrt S\) zu einer Teilmenge \(S\)
    von \(A\) definieren. Ist \((S_i)_{i \in I}\) eine Familie von Teilmengen,
    gilt \(\sqrt{\bigcup\limits_{i \in I} S_i} = \bigcup\limits_{i \in I}
    \sqrt{S_i}\).
    \begin{proposition}<+->
        Die Menge \(D\) der Nullteiler von \(A\) ist durch
        \(\bigcup\limits_{x \in A \setminus \{0\}} \sqrt{\ann (x)}\)
        gegeben.
    \end{proposition}
    \begin{proof}<+->
        \(D = \sqrt{D} = \sqrt{\bigcup\limits_{x \in A \setminus \{0\}}
        \ann (x)} = \bigcup\limits_{x \in A \setminus \{0\}} \sqrt{\ann (x)}\).
    \end{proof}
\end{frame}

\begin{frame}{Koprime Wurzelideale}
    \begin{example}<+->
        Sei \((m) \neq (0)\) ein Ideal im Ring der ganzen Zahlen und seien
        \(p_1, \dotsc, p_r\) die verschiedenen Primteiler von \((m)\). Dann ist
        \(\sqrt{(m)} = (p_1 \dotsm p_r) = \bigcap\limits_{i = 1}^r (p_i)\).
    \end{example}
    \begin{proposition}<+->
        Seien \(\ideal a, \ideal b\) Ideale in einem Ring, so daß
        \(\sqrt\ideal a\) und \(\sqrt \ideal b\) koprim sind. Dann sind auch
        \(\ideal a\) und \(\ideal b\) koprim.
    \end{proposition}
    \begin{proof}<+->
        Aus \(\sqrt{\ideal a + \ideal b} = \sqrt{\sqrt{\ideal a}
        + \sqrt{\ideal b}} = \sqrt{(1)} = (1)\) folgt \(\ideal a + \ideal b
        = (1)\).
    \end{proof}
\end{frame}



\lecture{Moduln}{Moduln}
\mode<all>\setcounter{section}{6}
\mode<all>\section{Erweiterungen und Kontraktionen von Idealen}

\subsection{Kontraktionen}

\begin{frame}{Definition der Kontraktion eines Ideals}
	Sei \(\phi\colon A \to B\) ein Homomorphismus kommutativer Ringe.
	Sei \(\ideal b\) ein Ideal von \(B\).
	\\
	Wir haben gesehen, daß das Urbild \(\phi^{-1}(\ideal b)\)
	ein Ideal von
	\(A\) ist.
	\begin{definition}<+->
		Das Ideal \(A \cap \ideal b := \phi^{-1}(\ideal b)\) von \(A\) heißt die
		\emph{Kontraktion von \(\ideal b\) (bezüglich \(\phi\))}.
	\end{definition}
	\begin{remark}<+->
		Ist \(\phi\) die Inklusion eines Unterringes \(A\) in \(B\), ist die 
		Kontraktion von \(\ideal b\) in der Tat der mengentheoretische Schnitt
		von \(A\) mit \(\ideal b\).
	\end{remark}
	\begin{visibleenv}<+->
		Es ist \(A \cap \ideal b\) der Kern des Homomorphismus
		\(A \to B/\ideal b, a \mapsto [\phi(a)]\).
		\\
		Nach dem Homomorphiesatz existiert damit ein injektiver Homomorphismus
		\(A/(A \cap \ideal b) \to B/\ideal b\).
	\end{visibleenv}
\end{frame}

\begin{frame}{Kontraktionen von Primidealen}
	\begin{proposition}<+->
		Ist \(\ideal q\) ein Primideal von \(B\), so ist \(A \cap \ideal q\) ein
		Primideal von \(A\).
	\end{proposition}
	\begin{proof}<+->
		Da ein injektiver Ringhomorphismus \(A/(A \cap \ideal q) \injto B/\ideal q\) existiert,
		muß \(A/(A \cap \ideal q)\) mit \(B/\ideal q\) auch ein Integritätsbereich sein.
	\end{proof}
	\begin{example}<+->
		Die Kontraktion eines maximalen Ideals ist im allgemeinen nicht mehr maximal.
		Sei etwa \(\phi\colon \set Z \injto \set Q\) die Inklusion der ganzen in die
		rationalen Zahlen. Dann ist die Kontraktion des maximalen Ideals \((0)\) in
		\(\set Q\) das Primideal \((0)\) in \(\set Z\), welches aber nicht maximal 
		ist. 
	\end{example}
\end{frame}

\subsection{Erweiterungen}

\begin{frame}{Definition der Erweiterung eines Ideals}
	Sei \(\phi\colon A \to B\) ein Homomorphismus kommutativer Ringe.
	Sei \(\ideal a\) ein Ideal von \(A\).
	\\
	Im allgemeinen ist das Bild \(\phi(\ideal a)\) kein Ideal in \(B\), aber wir 
	können das von \(\phi(\ideal a)\) erzeugte Ideal \((\phi(\ideal a))\) in
	\(B\) betrachten.
	\begin{definition}<+->
		Das Ideal \(B \ideal a := (\phi(\ideal a))\) von \(B\) heißt die
		\emph{Erweiterung von \(\ideal a\) (bezüglich \(\phi\))}.
	\end{definition}
	\begin{remark}<+->
		Ist \(\phi\) die Inklusion eines Unterringes \(A\) in \(B\), ist die 
		Erweiterung von \(\ideal a\) in der Tat die Menge der
		\(B\)\nobreakdash-Linearkombinationen von Elementen in \(\ideal a\).
	\end{remark}
	\begin{example}<+->
		Die Erweiterung eines Primideals ist im allgemeinen nicht mehr prim.
		Sei etwa \(\phi\colon \set Z \injto \set Q\) die Inklusion der ganzen in die
		rationalen Zahlen. Ist dann \(\ideal a \neq (0)\) ein nicht triviales
		Ideal von \(\set Z\), ist \(\set Q \ideal a = (1)\).
	\end{example}
\end{frame}

\begin{frame}{Ein Beispiel aus der algebraischen Zahlentheorie}
	\begin{example}<+->
		Sei die kanonische Injektion \(\set Z \to \set Z[\iu]\) gegeben, wobei \(\iu^2 = -1\).
		Der Ring \(\set Z[\iu]\) ist wie \(\set Z\) ein euklidischer Ring
		und damit ebenso ein Hauptidealbereich. Die Erweiterung eines
		Primideals \((p)\) in \(\set Z\) ist dann wie folgt gegeben:
		\begin{enumerate}[<+->]
		\item
			Ist \(p = 2\), ist \(\set Z[\iu](p) = (1 + \iu)(1 + \iu)\), das Quadrat eines
			Primideals in \(\set Z\).
		\item
			Ist \(p = 1\) modulo \(4\), ist \(\set Z[\iu](p)\) das Produkt zweier
			verschiedener Primideale, also etwa \(\set Z[\iu](5) = (2 + \iu)(2 - \iu)\).
			\\
			Diese nicht triviale Tatsache ist effektiv der Fermatsche
			Zwei-Quadrate-Satz, der besagt, daß eine Primzahl \(p\) mit \(p = 1\)
			modulo \(4\) als Summe zweier Quadratzahlen dargestellt werden 
			kann. (Also etwa \(5 = 2^2 + 1^2\).)
		\item
			Ist \(p = 3\) modulo \(4\), ist \(\set Z[\iu](p)\) ein Primideal.
		\end{enumerate}
	\end{example}
\end{frame}

\subsection{Operationen mit Erweiterungen und Kontraktionen}

\begin{frame}{Erweiterungen von Kontraktionen und Kontraktionen von Erweiterungen}
	\begin{proposition}<+->
		Sei \(\phi\colon A \to B\) ein Homomorphismus kommutativer Ringe.
		Sei \(\ideal a\) ein
		Ideal von \(A\) und \(\ideal b\) ein Ideal von \(B\). Dann gilt:
		\begin{enumerate}[<+->]
		\item<.->
			\(\ideal a \subset A \cap (B \ideal a)\) und
			\(\ideal b \supset B (A \cap \ideal b)\).
		\item
			\(A \cap \ideal b = A \cap (B (A \cap \ideal b))\) und
			\(B \ideal a = B (A \cap (B \ideal a))\).
		\end{enumerate}
	\end{proposition}
	\begin{proof}<+->
		\begin{enumerate}[<+->]
		\item<.->
			\(A \cap (B \ideal a) \supset \phi^{-1}(\phi(\ideal a)) \supset
			\ideal a\).
		\item
			\(\phi(\phi^{-1}(\ideal b)) \subset \ideal b\). Damit auch
			\((\phi(\phi^{-1}(\ideal b))) \subset (\ideal b) = \ideal b\).
		\item
			Aus \(B (A \cap \ideal b) \subset \ideal b\) 
			folgt \(A \cap (B (A \cap \ideal b)) \subset
			A \cap \ideal b \subset A \cap (B (A \cap \ideal b))\). 
		\item
			Aus \(\ideal a \subset A \cap (B \ideal a)\)
			folgt \(B \ideal a \subset B (A \cap (B \ideal a)) \subset
			B \ideal a\).
			\qedhere
		\end{enumerate}
	\end{proof}
\end{frame}

\begin{frame}{Erweiterte und kontrahierte Ideale}
	\begin{proposition}<+->
		Sei \(\phi\colon A \to B\) ein Homomorphismus kommutativer Ringe.
		Durch \(\ideal a = A \cap \ideal b\) und \(\ideal b = B \ideal a\) wird
		eine bijektive ordnungserhaltende Korrespondenz zwischen den kontrahierten
		Idealen \(\ideal a\) von \(A\) und den erweiterten Idealen \(\ideal b\)
		von \(B\) gegeben.
	\end{proposition}
	\begin{proof}<+->
		\begin{enumerate}[<+->]
		\item<.->
			Ist \(\ideal a\) ein kontrahiertes Ideal von \(A\), also etwa
			\(\ideal a = A \cap \ideal b\), so ist \(\ideal a = 
			A \cap \ideal b = A \cap (B (A \cap \ideal b)) = A \cap
			(B \ideal a)\), also die Kontraktion
			eines erweiterten Ideals von \(B\).
		\item
			Ist \(\ideal b\) ein erweitertes Ideal von \(B\), also etwa
			\(\ideal b = B \ideal a\), so ist \(\ideal b = B \ideal a
			= B (A \cap (B \ideal a)) = B (A \cap \ideal b)\),
			also die Erweiterung eines kontrahierten
			Ideals von \(A\).
			\qedhere
		\end{enumerate}
	\end{proof}
\end{frame}

\begin{frame}{Rechenregeln für Erweiterungen}
	\begin{proposition}<+->
		Sei \(\phi\colon A \to B\) ein Homomorphismus kommutativer Ringe.
		Seien \(\ideal a, \ideal a_1,
		\ideal a_2\) Ideale von \(A\). Dann gilt:
		\begin{enumerate}[<+->]
		\item<.->
			\(B(\ideal a_1 + \ideal a_2) = B \ideal a_1 + B \ideal a_2\).
		\item
		    \(B(\ideal a_1 \cap \ideal a_2) \subset B \ideal a_1 \cap
		    B \ideal a_2\).
		\item
		    \(B(\ideal a_1 \ideal a_2) = (B \ideal a_1) (B \ideal a_2)\).
		\item
		    \(B(\ideal a_1 : \ideal a_2) \subset (B \ideal a_1 : B \ideal a_2)\).
		\item
		    \(B \sqrt{\ideal a} \subset \sqrt{B \ideal a}\).
		    \qed
		\end{enumerate}
	\end{proposition}
\end{frame}

\begin{frame}{Rechenregeln für Kontraktionen}
	\begin{proposition}<+->
		Sei \(\phi\colon A \to B\) ein Ringhomomorphismus. Seien \(\ideal b,
		\ideal b_1, \ideal b_2\) Ideale von \(B\). Dann gilt:
		\begin{enumerate}[<+->]
		\item<.->
		    \(A \cap (\ideal b_1 + \ideal b_2) \supset (A \cap \ideal b_1)
		    + (A \cap \ideal b_2)\).
		\item
		    \(A \cap (\ideal b_1 \cap \ideal b_2) = (A \cap \ideal b_1) \cap
		    (A \cap \ideal b_2)\).
		\item   
		    \(A \cap (\ideal b_1 \ideal b_2) \supset (A \cap \ideal b_1)
		    (A \cap \ideal b_2)\).
		\item
		    \(A \cap (\ideal b_1 : \ideal b_2) \subset (A \cap \ideal b_1 :
		    A \cap \ideal b_2)\).
		\item
		    \(A \cap \sqrt{\ideal b} = \sqrt{A \cap \ideal b}\).
		    \qed
		\end{enumerate}
	\end{proposition}
\end{frame}


\part<article>{Moduln}
\mode<all>\section{Moduln und Modulhomomorphismen}

\subsection{Moduln}

\begin{frame}{Definition eines Moduls}
	Sei \(A\) ein Ring.
	\begin{definition}
		Ein \emph{\(A\)-(Links-)Modul \(M\)} ist eine abelsche Gruppe \((M, +, 0)\) 
		zusammen mit einer Abbildung \(\cdot\colon A \times M \to M\), so daß
		\begin{enumerate}[<+->]
		\item
			die Multiplikation eine Operation des multiplikativen Monoides von
			\(A\) auf der Menge \(M\) ist, also \((ab) x = a (b x)\) und
			\(1 \cdot x = x\) für alle \(a, b \in A\) und \(x \in M\) gilt, und
		\item
			die Multiplikation distributiv über die Addition ist, also
			\(a (x + y) = a x + a y\) und \((a + b) x = a x + b x\) für alle
			\(a, b \in A\) und \(x, y \in M\).
		\end{enumerate}
	\end{definition}
	\begin{remark}<+->
		Alternativ läßt sich ein \(A\)-Modul als eine abelsche Gruppe zusammen mit
		einem Ringhomomorphismus \(A \to \End_{\set Z}(M), a \mapsto (m \mapsto a m)\)
		definieren, wobei \(\End_{\set Z}(M)\) für den Endomorphismenring der
		abelschen Gruppe \(M\) steht.
	\end{remark}
\end{frame}

\begin{frame}{Multiplikation mit Null}
	Seien \(A\) ein Ring und \(M\) ein \(A\)-Modul.
    \begin{proposition}<+->
        Sei \(x \in M\). Dann ist \(0 \cdot x = 0\).
    \end{proposition}
    \begin{proof}<+->
        \(0 \cdot x = 0 \cdot x + 0 \cdot x - 0 \cdot x
        = (0 + 0) \cdot x - 0 \cdot x = 0 \cdot x - 0 \cdot x = 0.\)
    \end{proof}
    \begin{corollary}<+->
        Sei \(x \in M\). Dann ist \((-1) \cdot x = -x\).
    \end{corollary}
    \begin{proof}<+->
        \(x + (-1) \cdot x = 1 \cdot x + (-1) \cdot x
        = (1 - 1) \cdot x = 0 \cdot x = 0\).
    \end{proof}
\end{frame}

\begin{frame}{Beispiele von Moduln}
	\begin{example}<+->
		Jedes Ideal \(\ideal a\) eines Ringes \(A\) wird durch Einschränkung der
		Multiplikation zu einem \(A\)-Modul. Insbesondere ist \(A\) selbst in
		kanonischer Weise ein \(A\)-Modul.
		\\
		Betrachten wir das Nullideal als \(A\)-Modul schreiben wir in der Regel \(0\) statt \((0)\).		
	\end{example}
	\begin{example}<+->
		Ein Modul über einem (Schief-)Körper \(K\) ist dasselbe wie ein
		\(K\)-(Links-)Vektorraum.
	\end{example}
	\begin{example}<+->
		Ein \(\set Z\)-Modul \(M\) ist dasselbe wie eine abelsche Gruppe
		(\(n x = (\underbrace{1 + \dotsb + 1}_n) x = \underbrace{x + \dotsb + x}_n,
		n \in \set N_0, x \in M\)).
	\end{example}
\end{frame}

\begin{frame}{Weitere Beispiele von Moduln}
	Sei \(K\) ein Körper.
	\begin{example}<+->
		Ein \(K[x]\)-Modul \(V\) ist dasselbe wie ein
		\(K\)-Vektorraum \(V\) zusammen mit einem Endomorphismus \(V \to V,
		v \mapsto x v\).
	\end{example}
	\begin{example}<+->
		Sei \(G\) eine endliche Gruppe und \(A \coloneqq K[G] =
		\left\{\sum\limits_{g \in G} a_g \cdot g \mid g \in G, a_g \in K\right\}\) die Gruppenalgebra von
		\(G\) über \(K\). Ein \(A\)-Modul \(V\) ist dasselbe wie eine \(K\)-Vektorraum \(V\)
		zusammen mit einer \(K\)-linearen Darstellung \(G \to \End_K(V)\) von \(G\)
		auf \(V\).
	\end{example}
\end{frame}

\subsection{Modulhomomorphismen}

\begin{frame}{Definition eines Modulhomomorphismus}
	Sei \(A\) ein Ring.
	\begin{definition}<+->
		Ein \emph{\(A\)-Modulhomomorphismus \(\phi\)} ist eine Abbildung \(\phi\colon
		M \to N\) zwischen zwei \(A\)-Moduln, welche einen Homomorphismus zwischen
		den abelschen Gruppen von \(M\) und \(N\) induziert, welcher mit der Operation
		des multiplikativen Monoids von \(A\) verträglich ist, das heißt
		\(\phi(a x) = a \phi(x)\) für alle \(a \in A\) und \(x \in M\).
	\end{definition}
	\begin{visibleenv}<+->
		Eine Abbildung \(\phi\colon M \to N\) ist also genau dann ein 
		Modulhomomorphismus, falls für alle \(a \in A\) und \(x, y \in M\) gilt:
		\begin{align*}
			\phi(x + y) & = \phi(x) + \phi(y); \tag{$*$} \\
			\phi(0) & = 0 && \text{(folgt schon aus ($*$))}; \\
			\phi(a x) & = a \phi(x),
		\end{align*}
		wenn die Abbildung also \(A\)-linear ist.
	\end{visibleenv}
\end{frame}

\begin{frame}{Der Homomorphismenmodul}
	Sei \(A\) ein kommutativer Ring. Seien \(M, N\) zwei \(A\)-Moduln.
	Auf der Menge \(\Hom_A(M, N)\) der Modulhomomorphismen \(M \to N\) definieren wir
	eine Addition durch \((\phi + \psi)(x) = \phi(x) + \psi(x)\). Zusammen mit der
	Nullabbildung als Null wird \(\Hom_A(M, N)\) damit zu einer abelschen Gruppe,
	durch die Setzung \((a \phi)(x) = a \phi(x)\) sogar zu einem \(A\)-Modul.
	\begin{definition}<+->
		Der \(A\)-Modul \(\Hom_A(M, N)\) ist der \emph{Modul der Homomorphismen von
		\(M\) nach \(N\)}.
	\end{definition}
	\begin{visibleenv}<+->
		Ergibt sich der Ring \(A\) aus dem Kontext, schreiben wir in der Regel
		\(\Hom(M, N) = \Hom_A(M, N)\).
	\end{visibleenv}
	\begin{example}<+->
		Seien \(\phi\colon M' \to M\) und \(\psi\colon N \to N'\) zwei Homomorphismen
		von \(A\)-Moduln. Dann sind
		\(\phi^*\colon \Hom_A(M, N) \to \Hom_A(M', N), \chi \mapsto \chi \circ \phi\)
		und
		\(\psi_*\colon \Hom(M, N) \to \Hom(M, N'), \chi \mapsto \psi \circ \chi\)
		Homomorphismen von \(A\)-Moduln.
	\end{example}
\end{frame}

\begin{frame}{Modulisomorphismen}
	Sei \(A\) ein Ring. Sei \(\phi\colon M \to N\) ein Homomorphismus
	von \(A\)-Moduln.
	\begin{definition}<+->
		Der Homomorphismus \(\phi\) heißt \emph{Isomorphismus}, falls ein
		Modulhomomorphismus \(\check \phi\colon N \to M\) existiert, so daß
		\(\check \phi \circ \phi = \id_M\) und \(\phi \circ \check \phi = \id_N\).
	\end{definition}
	\begin{proposition}<+->
		Ist \(\phi\) bijektiv, so ist \(\phi\) schon ein Isomorphismus.
		\qed
	\end{proposition}
	\begin{example}<+->
		Sei \(A\) kommutativ. Es ist
		\(\Hom(A, M) \to M, \phi \mapsto \phi(1)\)
		ein Isomorphismus von \(A\)-Moduln, denn ein \(A\)-Modulhomomorphismus
		\(\phi\colon A \to M\) ist schon eindeutig durch \(\phi(1)\) bestimmt, was
		ein beliebiges Element aus \(M\) sein kann.
	\end{example}
\end{frame}

\subsection{Bezeichnungen}

\begin{frame}{Bezeichnungen von Moduln, Modulelementen und Modulhomomorphismen}
	\begin{convention}<+->
		Moduln bezeichnen wir mit großen lateinischen Buchstaben \(L, M, N, \dotsc\),
		Modulelemente mit kleinen lateinischen Buchstaben \(m, n, x, y, z, \dotsc\).
	\end{convention}
	\begin{visibleenv}<+->
		Betrachten wir einen Ring als Algebra von Funktionen über einem Raum,
		heißen die Elemente eines Moduls über diesem Ring auch \emph{Schnitte}.
	\end{visibleenv}
	\begin{convention}<+->
		Modulhomomorphismen bezeichnen wir mit kleinen griechischen Buchstaben
		\(\phi, \psi, \dotsc\).
	\end{convention}
\end{frame}


\mode<all>\section{Untermoduln und Quotientenmoduln}

\subsection{Untermoduln, Quotientenmoduln, Kerne und Kokerne}

\begin{frame}{Definition eines Untermoduls}
	Sei \(A\) ein Ring.
	\begin{definition}
		Eine Teilmenge \(M'\) eines \(A\)-Moduls \(M\) heißt \emph{Untermodul von
		\(M\)}, falls
		\begin{enumerate}[<+->]
			\item \(M'\) eine Untergruppe der abelschen Gruppe von \(M\) ist und
			\item \(M'\) abgeschlossen unter Multiplikation mit Elementen aus \(A\) ist,
			also \(a x \in M'\) für alle \(a \in A\) und \(x \in M'\).
		\end{enumerate}
	\end{definition}
	\begin{visibleenv}<+->
		Eine Teilmenge \(M'\) von \(M\) ist genau dann ein Untermodul, falls für alle
		\(x, x' \in M'\) und \(a \in A\) gilt:
		\begin{align*}
			x + x' & \in M'; & 0 & \in M'; \\
			a x & \in M'.
		\end{align*}
	\end{visibleenv}
	\begin{remark}<+->
		Durch Einschränkung der Multiplikation wird jeder Untermodul eines \(A\)-Moduls
		selbst zu einem \(A\)-Modul.
	\end{remark}
\end{frame}

\begin{frame}{Beispiele für Untermoduln}
	Sei \(A\) ein Ring.
	\begin{example}<+->
		Jeder \(A\)-Modul \(M\) besitzt die beiden trivialen Untermoduln \(\{0\}\) und
		\(M\).
	\end{example}
	\begin{example}<+->
		Sei \(A\) kommutativ. Eine Teilmenge \(\ideal a\) von \(A\) ist genau dann
		ein Ideal von \(A\), wenn \(\ideal a\) ein Untermodul von \(A\) ist,
		wobei wir \(A\) in kanonischer Weise als Modul über sich selbst auffassen.
	\end{example}
\end{frame}

\begin{frame}{Quotientenmoduln}
	Sei \(A\) ein Ring. Sei \(M'\) ein Untermodul eines \(A\)-Moduls \(M\).
	\begin{proposition}<+->
		Es gibt genau eine
		Modulstruktur auf der Menge \(M/M'\) der \(M'\)-Nebenklassen, so daß die
		kanonische Abbildung \(\pi\colon M \surjto M/M', x \mapsto [x] \coloneqq
		x + M'\) ein Homomorphismus von \(A\)-Moduln wird.
		\qed
	\end{proposition}
	\begin{definition}<+->
		Der \(A\)-Modul \(M/M'\) heißt der \emph{Quotientenmodul von \(M\) nach
		\(M'\)}.
	\end{definition}
	\begin{proposition}<+->
		Sei \(\pi\colon M \to M/M'\) die kanonische Abbildung.
		Durch \(N = \pi^{-1}(\bar N)\) wird eine bijektive, ordnungserhaltende
		Korrespondenz zwischen den Untermoduln \(N\) von \(M\) mit \(N \supset M'\)
		und den Untermoduln \(\bar N\) von \(M/M'\) gegeben.
		\qed
	\end{proposition}
\end{frame}

\begin{frame}{Kern, Bild und Kokern}
	\begin{visibleenv}<+->
		Sei \(A\) ein Ring. Sei \(\phi\colon M \to N\) ein Homomorphismus von \(A\)-
		Moduln. Es sind
		\[\ker \phi \coloneqq \{x \in M \mid \phi(x) = 0\} \subset
		M\] und \[\im \phi \coloneqq \{\phi(x) \mid x \in M\} \subset N\]
		Untermoduln von \(M\) beziehungsweise \(N\).
	\end{visibleenv}
	\begin{definition}<+->
		\begin{enumerate}[<+->]
		\item<.->
			Der Untermodul \(\ker \phi\) von \(M\) heißt der \emph{Kern von \(\phi\)}.
		\item
			Der Untermodul \(\im \phi\) von \(N\) heißt das
			\emph{Bild von \(\phi\)}.
		\item
			Der Quotientenmodul \(\coker \phi \coloneqq N/\im \phi\) heißt der
			\emph{Kokern von \(\phi\)}.
		\end{enumerate}
	\end{definition}
	\begin{proposition}<+->
		Es ist \(\coker \phi = 0\) genau dann, wenn \(\phi\) surjektiv ist.
		\qed
	\end{proposition}
\end{frame}

\begin{frame}{Der Homomorphiesatz für Moduln}
	Sei \(A\) ein Ring. Sei \(\phi\colon M \to N\) ein Homomorphismus von
	\(A\)-Moduln. 
	\begin{proposition}[Homomorphiesatz für Moduln]<+->
		Ist \(M'\) ein Untermodul von \(M\)
		mit \(M' \subset \ker \phi\), so gibt es genau
		einen Modulhomomorphismus \(\underline \phi\colon M/M' \to N,
		[x] \mapsto \phi(x)\) mit \(\ker \underline{\phi} = \ker \phi/M'\).
	\end{proposition}
	\begin{proof}<+->
		\begin{enumerate}[<+->]
		\item<.->
			Die Existenz von \(\underline \phi\) wird wie beim Homomorphiesatz für
			Ringe bewiesen.
		\item
			Es ist \([x] \in \ker{\underline\phi} \iff 
			\underline\phi([x]) = 0 \iff
			\phi(x) = 0 \iff
			x \in \ker\phi \iff [x] \in \ker\phi/M'\).
			\qedhere
		\end{enumerate}
	\end{proof}
	\begin{example}<+->
		Es existiert ein kanonischer Isomorphismus \(M/\ker \phi \to \im \phi\) von
		\(A\)-Moduln.
	\end{example}
\end{frame}

\subsection{Bezeichnungen}

\begin{frame}{Bezeichnungen für Unter- und Quotientenmoduln}
	\begin{convention}<+->
		Untermoduln von Moduln \(M, N, L, \dotsc\) bezeichnen wir häufig mit Strichen
		\(M', N', L', \dotsc\).
	\end{convention}
	\begin{convention}<+->
		Quotientenmoduln von Moduln \(M, N, L, \dotsc\) bezeichnen wir häufig mit
		Doppelstrichen \(M'', N'', L'', \dotsc\).
	\end{convention}
\end{frame}



\lecture{Endlich erzeugte Moduln und exakte Sequenzen}{Endlich erzeugte Moduln und exakte Sequenzen}
\mode<all>\setcounter{section}{9}
\mode<all>\section{Operationen auf Untermoduln}

\subsection{Summe und Schnitt}

\begin{frame}{Definition der Summe von Untermoduln}
	\begin{visibleenv}<+->
		Sei \(A\) ein Ring. Sei \(M\) ein \(A\)-Modul.
		Sei \((M_i)_{i \in I}\) eine Familie von Untermoduln von \(M\).
		\\
		Mit \(\sum\limits_{i \in I} M_i \subset M\) bezeichnen wir die Teilmenge aller
		endlichen
		Summen der Form \(\sum\limits_{i \in I} x_i\) mit \(x_i \in M_i\). Hierbei heißt
		endlich, daß \(x_i = 0\) für fast alle (das heißt, alle bis auf endlich viele)
		\(i \in I\).
		\\
		Es ist \(\sum\limits_{i \in I} M_i\) ein Untermodul von \(M\).
	\end{visibleenv}
	\begin{definition}<+->
		Der Untermodul \(\sum\limits_{i \in I} M_i\) von \(M\) heißt die
		\emph{Summe der Untermoduln \(M_i\)}.
	\end{definition}
	\begin{remark}<+->
		Die Summe \(\sum\limits_{i \in I} M_i\) ist der kleinste Untermodul von \(M\),
		welcher alle \(M_i\) umfaßt.
	\end{remark}
\end{frame}

\begin{frame}{Der Schnitt einer Familie von Untermoduln}
	Sei \(A\) ein Ring. Sei \(M\) ein \(A\)-Modul.
	\begin{proposition}<+->
		Sei \((M_i)_{i \in I}\) eine Familie von
		Untermoduln von \(M\). Dann ist der Schnitt \(\bigcap\limits_{i \in I} M_i \subset M\)
		ein Untermodul von \(M\).
	\end{proposition}
	\begin{remark}<+->
		Damit bilden die Untermoduln von \(M\) einen vollständigen Verband bezüglich der
		Inklusionsordnung.
	\end{remark}
\end{frame}

\subsection{Die Isomorphiesätze}

\begin{frame}{Der erste Isomorphiesatz}
	\begin{proposition}[Erster Isomorphiesatz]<+->
		Sei \(A\) ein Ring. Sei \(M\) ein \(A\)-Modul, und seien \(M_1, M_2 \subset M\)
		zwei Untermoduln. Dann existiert ein kanonischer Isomorphismus
		\((M_1 + M_2)/M_1 \isoto M_2/(M_1 \cap M_2)\) von \(A\)-Moduln.
	\end{proposition}
	\begin{proof}<+->
		\begin{enumerate}[<+->]
		\item<.->
			Sei \(\theta\colon M_2 \to (M_1 + M_2)/M_1, x \mapsto x + M_1\). Dann ist \(\theta\)
			ein surjektiver Homomorphismus von \(A\)-Moduln.
		\item
			Der Kern von \(\theta\) ist \(M_1 \cap M_2\). Damit folgt die Aussage aus dem Homomorphiesatz.
			\qedhere
		\end{enumerate}
	\end{proof}
\end{frame}

\begin{frame}{Der zweite Isomorphiesatz}
	\begin{proposition}[Zweiter Isomorphiesatz]<+->
		Sei \(A\) ein Ring. Sei \(L\) ein \(A\)-Modul, und seien \(N \subset M \subset L\)
		Untermoduln. Dann existiert ein kanonischer Isomorphismus
		\((L/N)/(M/N) \isoto L/M\) von \(A\)-Moduln.
	\end{proposition}
	\begin{proof}<+->
		\begin{enumerate}[<+->]
		\item<.->
			Sei \(\theta\colon L/N \to L/M, x + N \mapsto x + M\). Dann ist \(\theta\) ein wohldefinierter,
			surjektiver Homomorphismus von \(A\)-Moduln.
		\item
			Der Kern von \(\theta\) ist \(M/N\). Damit folgt die Aussage aus dem Homomorphiesatz.
			\qedhere
		\end{enumerate}
	\end{proof}
\end{frame}

\subsection{Operationen mit Moduln}

\begin{frame}{Produkt eines Ideals mit einem Modul}
	Sei \(A\) ein kommutativer Ring. Seien \(M\) ein \(A\)-Modul und \(\ideal a\) ein Ideal in \(A\).
	\\
	Mit \(\ideal a M\) bezeichnen wir die Teilmenge aller endlichen Summen der Form \(\sum\limits_i
	a_i x_i\) mit \(a_i \in \ideal a\) und \(x_i \in M\).
	\\
	Es ist \(\ideal a M\) ein Untermodul von \(M\).
	\begin{definition}<+->
		Der Untermodul \(\ideal a M\) von \(M\) heißt das \emph{Produkt von \(\ideal a\) und \(M\)}.
	\end{definition}
	\begin{notation}<+->
		Ist \(\ideal a\) ein Hauptideal \((a)\), schreiben wir \(a M\) anstelle von \((a) M\).
	\end{notation}
	\begin{visibleenv}<+->
		Der Untermodul \(a M\) enthält genau die Elemente der Form \(a x\) von \(M\) mit \(x \in M\).
	\end{visibleenv}
	\begin{remark}<+->
		Auf diese Weise läßt sich ein Produkt von Moduln im allgemeinen nicht definieren.
	\end{remark}
\end{frame}

\begin{frame}{Quotient zweier Untermoduln}
	Sei \(A\) ein kommutativer Ring. Seien \(N, P\) zwei Untermoduln eines \(A\)-Moduls \(M\).
	\\
	Dann ist \((N : P) \coloneqq \{a \in A \mid a P \subset N\}\) ein Ideal von \(A\).
	\begin{definition}<+->
		Das Ideal \((N : P)\) heißt der \emph{Quotient von \(N\) nach \(P\)}.
	\end{definition}
	\begin{remark}<+->
		Betrachten wir zwei Ideale von \(A\) als Untermoduln von \(A\), stimmt deren Idealquotient mit dem
		Quotient als Moduln überein.
	\end{remark}
	\begin{definition}<+->
		Das Ideal \((0:M)\) ist der \emph{Annulator \(\ann M\)} von \(M\).
	\end{definition}
	\begin{visibleenv}<+->
		Es ist also \(\ann M = \{a \in A \mid a M = 0\}\).
	\end{visibleenv}
\end{frame}

\begin{frame}{Der Annulator eines Moduls}
	\begin{visibleenv}<+->
		Sei \(A\) ein kommutativer Ring. Sei \(M\) ein \(A\)-Modul.
		\\
		Ist \(\ideal a\) ein Ideal von \(A\) mit \(\ideal a \subset \ann M\), können wir \(M\) als
		einen
		\(A/\ideal a\)-Modul \(M^{\ideal a}\) betrachten:
		\\
		Als abelsche Gruppen stimmen \(M\) und \(M^{\ideal a}\) überein.
		\\
		Die Multiplikation auf \(M^{\ideal a}\) ist durch \((a + \ideal a) \cdot x = a x\) mit \(a \in A\)
		und \(x \in M\) definiert. Diese ist wohldefiniert, da \(a x = 0\)
		für \(a \in \ideal a \subset \ann M\).
	\end{visibleenv}
	\begin{visibleenv}<+->
		Anstelle von \(M^{\ideal a}\) sagen wir häufig auch "`\(M\) als \(A/\ideal a\)-Modul"'.
	\end{visibleenv}
	\begin{definition}<+->
		Der \(A\)-Modul \(M\) heißt \emph{treu}, falls \(\ann M = 0\).
	\end{definition}
	\begin{visibleenv}<+->
		Es ist \(M\) also genau dann treu, falls aus \(\forall x \in M\colon ax = 0\) mit \(a \in A\) schon
		\(a = 0\) folgt.
	\end{visibleenv}
	\begin{example}<+->
		Es ist \(M\) treu als \((A/\ann M)\)-Modul.
	\end{example}
\end{frame}

\begin{frame}{Rechenregeln für den Annulator}
	\begin{proposition}<+->
		Sei \(A\) ein kommutativer Ring. Seien \(N, P\) zwei Untermoduln eines \(A\)-Moduls \(M\).
		Dann gilt:
		\begin{enumerate}[<+->]
		\item<.->
			\(\ann(N + P) = \ann N \cap \ann P\).
		\item
			\((N : P) = \ann((N + P)/N)\).
			\qed
		\end{enumerate}
	\end{proposition}
\end{frame}

\subsection{Endlich erzeugte Moduln}

\begin{frame}{Endlich erzeugte Moduln}
	\begin{visibleenv}<+->
		Sei \(A\) ein Ring. Seien \(M\) ein \(A\)-Modul und \(x \in M\).
		\\
		Dann ist \(A x \coloneqq \{a x \mid a \in A\}\) ein Untermodul von \(M\), die Teilmenge der
		Vielfachen von \(x\).
	\end{visibleenv}
	\begin{definition}<+->
		Gilt \(M = \sum\limits_{i \in I} A x_i\) für eine Familie \((x_i)_{i \in I}\) von Elementen
		in \(M\), so bilden die \(x_i\) eine \emph{Familie von Erzeugern von \(M\)}.
	\end{definition}
	\begin{visibleenv}<+->
		In diesem Falle läßt sich also jedes Element von \(M\) als (nicht notwendigerweise eindeutige)
		endliche Linearkombination der \(x_i\) darstellen.
	\end{visibleenv}
	\begin{definition}<+->
		Der \(A\)-Modul \(M\) heißt \emph{endlich erzeugt}, falls er eine endliche Familie von
		Erzeugern besitzt.
	\end{definition}
\end{frame}


\mode<all>\section{Direkte Summen und Produkte}

\subsection{Definition von direkter Summe und Produkt}

\begin{frame}{Definition des direkten Produktes}
	\begin{visibleenv}<+->
		Sei \(A\) ein Ring. Sei \((M_i)_{i \in I}\) eine Familie von \(A\)-Moduln. Auf der
		Menge \(M \coloneqq \prod\limits_{i \in I} M_i\) der Folgen
		\(x \coloneqq (x_i)_{i \in I}\) mit \(x_i \in M_i\) definieren wir eine Addition
		und eine Multiplikation mit Ringelementen gliedweise.
		\\
		Dann wird \(M\) mit der Null \((0)_{i \in I}\) zu einem \(A\)-Modul.
	\end{visibleenv}
	\begin{definition}<+->
		Der \(A\)-Modul \(\prod\limits_{i \in I} M_i\) ist das \emph{direkte Produkt über die
		Familie \((M_i)_{i \in I}\)}.
	\end{definition}
	\begin{proposition}<+->
		Die Projektionen \(\pi_i\colon M \to M_i, x \mapsto x_i\) sind Homomorphismen von
		\(A\)-Moduln.
		\qed
	\end{proposition}
	\begin{visibleenv}<+->
		Die \(A\)-Modulstruktur auf \(M\) ist gerade so gewählt, daß die \(\pi_i\)
		Homomorphismen von \(A\)-Moduln werden.
	\end{visibleenv}
\end{frame}

\begin{frame}{Definition der direkten Summe}
	\begin{visibleenv}<+->
		Sei \(A\) ein Ring. Sei \((M_i)_{i \in I}\) eine Familie von \(A\)-Moduln. Sei
		\(M \coloneqq \bigoplus\limits_{i \in I} M_i\) die Teilmenge derjenigen Familien
		\(x = (x_i)_{i \in I}
		\in \prod\limits_{i \in I} M_i\), für die für fast alle \(i \in I\) gilt \(x_i = 0\).
		\\
		Es ist \(\bigoplus\limits_{i \in I} M_i\) ein Untermodul des direkten Produktes
		\(\prod\limits_{i \in I} M_i\).
	\end{visibleenv}
	\begin{definition}<+->
		Der \(A\)-Modul \(\bigoplus\limits_{i \in I} M_i\) heißt die \emph{direkte Summe über
		die Familie \((M_i)_{i \in I}\)}.
	\end{definition}
	\begin{remark}<+->
		Ist die Indexmenge \(I\) endlich, stimmen direktes Produkt und direkte Summe überein.
	\end{remark}
	\begin{proposition}<+->
		Die Inklusionen \(\iota_i\colon M_i \to M, m \mapsto x\) mit \(x_j = m\) für \(j = i\) und
		\(x_j = 0\) sonst sind ein Homomorphismus von \(A\)-Moduln.
	\end{proposition}
\end{frame}

\subsection{Direkte Summenzerlegungen von Ringen}

\begin{frame}{Zerlegung eines Ringes in eine direkte Summe endlich vieler Ideale}
	\begin{example}<+->
		Sei \(A = \prod\limits_{i = 1}^n A_i\) ein endliches direktes Produkt kommutativer Ringe. Sei
		\(\ideal a_i \subset A\) die Teilmenge aller \((a_j)\) mit \(a_j = 0\) für \(j \neq i\) und
		\(a_i \in A_i\) beliebig.
		\\
		Dann ist \(\ideal a_i\) ein Ideal in \(A\). Weiter besitzt \(A\) als \(A\)-Modul die direkte
		Summenzerlegung \(A = \ideal a_1 \oplus \dotsc \oplus \ideal a_n\).
	\end{example}
	\begin{example}<+->
		Sei \(A = \ideal a_1 \oplus \dotsc \oplus \ideal a_n\) eine Zerlegung eines kommutativen Ringes \(A\)
		als \(A\)-Modul in eine direkte Summe von Idealen.
		\\
		Dann existiert ein kanonischer Isomorphismus \(A \cong \prod\limits_{i = 1}^n A/\ideal b_i\)
		mit \(\ideal b_i \coloneqq \bigoplus\limits_{j \neq i} \ideal a_j\). 
		\\
		Ist \(e_i \in A\) das Einselement des Ringes \(A/\ideal b_i \cong \ideal a_i\), so ist
		\(\ideal a_i = (e_i)\).
	\end{example}
\end{frame}


\mode<all>\section{Endlich erzeugte Moduln}

\subsection{Freie Moduln}

\begin{frame}{Definition freier Moduln}
	Sei \(A\) ein Ring.
	\begin{definition}<+->
		Ein \emph{freier \(A\)-Modul \(M\)} ist
		ein zu einem \(A\)-Modul der Form \(\bigoplus\limits_{i \in I} M_i\) isomorpher
		\(A\)-Modul, wobei \(M_i \cong A\) als \(A\)-Modul.
	\end{definition}
	\begin{visibleenv}<+->
		Häufig wird für \(M\) auch die Notation \(A^{(I)}\) verwendet.
	\end{visibleenv}
	\begin{example}<+->
		Ein endlich erzeugter freier Modul ist damit ein freier Modul isomorph zu
		\(A^n = \underbrace{A \oplus \dotsb \oplus A}_n\).
		\\
		(Es ist \(A^0 = 0\) der Nullmodul.)
	\end{example}
\end{frame}

\begin{frame}{Endlich erzeugte Moduln als Quotienten freier Moduln}
	\begin{proposition}<+->
		Sei \(A\) ein Ring. Sei \(M\) ein \(A\)-Modul. Dann ist \(M\)
		genau dann ein endlich erzeugter \(A\)-Modul, wenn \(M\) ein
		Quotient eines \(A\)-Moduls der Form \(A^n\) für ein \(n \in \set N_0\)
		ist.
	\end{proposition}
	\begin{proof}<+->
	\begin{enumerate}[<+->]
	\item<.->
		Sei \(M\) endlich erzeugt, etwa mit Erzeugern \(x_1, \dotsc, x_n\). Dann ist
		\(\phi\colon A^n \to M, (a^1, \dotsc, a^n) \mapsto a^1 x_1 + \dotsb + a^n x_n\) ein
		surjektiver Homomorphismus von \(A\)-Moduln. Nach dem Homomorphiesatz ist also
		\(M \cong A^n/\ker \phi\).
	\item
		Sei \(M\) ein Quotient von \(A^n\). Dann existiert ein surjektiver Homomorphismus
		\(\phi\colon A^n \to M\) von \(A\)-Moduln. Ist dann \(e_i \coloneqq (0, \dotsc, 1, \dotsc, 0) \in A^n\)
		mit der Eins an Position \(i\), so erzeugen \(e_1, \dotsc, e_n\) den \(A\)-Modul \(A^n\).
		Also erzeugen die \(\phi(e_i)\) den \(A\)-Modul \(M\).
		\qedhere
	\end{enumerate}
	\end{proof}
\end{frame}

\subsection{Das Nakayamasche Lemma}

\begin{frame}{Ganzheit eines Endomorphismus}
	\begin{proposition}<+->
		Sei \(A\) ein kommutativer Ring.
		Seien \(M\) ein endlich erzeugter \(A\)-Modul und \(\ideal a\) ein Ideal von \(A\).
		Sei \(\phi\colon M \to M\) ein Endomorphismus von \(M\) mit \(\im \phi \subset \ideal a M\).
		\\
		Dann erfüllt \(\phi\) eine Gleichung der Form
		\(\phi^n + a_1 \phi^{n - 1} + \dotsb + a_n = 0\) mit \(a_i \in \ideal a\).
	\end{proposition}
	\begin{proof}<+->
	\begin{enumerate}[<+->]
	\item<.->
		Erzeugen \(x_1, \dotsc, x_n\) den \(A\)-Modul \(M\). Da \(\phi(x_i) \in \ideal a M\),
		existieren \(a^i_j \in \ideal a\) mit \(\phi(x_j) = \sum\limits_i a^i_j x_i\), also
		\(\sum\limits_i (\kron^i_j \phi - a^i_j) x_i = 0\) für alle \(j\).
	\item
		Multiplizieren wir die linke Seite mit der Adjunkten der Matrix \((\kron^i_j \phi - a^i_j)\)
		erhalten wir, daß \(\det (\kron^i_j \phi - a^i_j)\) alle Erzeuger \(x_i\) auslöscht und damit
		der Nullmorphismus ist. Ausmultiplizieren der Determinanten liefert eine Gleichung der
		gesuchten Form.
		\qedhere
	\end{enumerate}
	\end{proof}
\end{frame}

\begin{frame}{Nakayamasches Lemma}
	Sei \(A\) ein kommutativer Ring. Sei \(M\) ein endlich erzeugter \(A\)-Modul.
	\begin{corollary}<+->
		Sei \(\ideal a\) ein Ideal von \(A\) mit \(\ideal a M = M\). Dann existiert ein \(x \in A\) mit
		\(x = 1\) modulo \(\ideal a\) und \(x M = 0\).
	\end{corollary}
	\begin{proof}<+->
		Anwenden der Proposition auf \(\id_M\colon M \to M\) liefert einen Ausdruck der Form
		\(x \coloneqq 1 + a_1 + \dotsb + a_n\) mit \(a_i \in \ideal a\) und \(x M = 0\).
	\end{proof}
	\begin{proposition}[Nakayamasches Lemma]<+->
		Sei \(\ideal a\) ein Ideal von \(A\), welches im Jacobsonschen Radikal \(\ideal j\) von \(A\)
		enthalten ist.
		Dann folgt aus \(\ideal a M = M\) schon \(M = 0\).
	\end{proposition}
	\begin{proof}<+->
		Nach der Folgerung existiert ein \(x = 1\) modulo \(\ideal a\)
		mit \(x M = 0\). Da \(1 - x \in \ideal j\),
		ist \(x \in A^\units\), also \(M = x^{-1} x M = 0\).
	\end{proof}
\end{frame}

\begin{frame}{Folgerung aus dem Nakayamaschen Lemma}
	Sei \(A\) ein kommutativer Ring. Sei \(M\) ein endlich erzeugter \(A\)-Modul.
	\begin{corollary}<+->
		Seien \(N \subset M\) ein Untermodul und \(\ideal a\) ein Ideal von \(A\), welches im
		Jacobsonschen Radikal von \(A\) enthalten ist. Dann folgt aus \(M = \ideal a M + N\) schon
		\(M = N\).
	\end{corollary}
	\begin{proof}<+->
		Wegen \(\ideal a (M/N) = (\ideal a M + N)/N\) reicht es, das Nakayamasche Lemma auf \(M/N\) 
		anzuwenden.
	\end{proof}
\end{frame}

\begin{frame}{Die spezielle Faser}
	\begin{visibleenv}<+->
		Sei \((A, \ideal m, F)\) ein lokaler Ring.
		\\
		Sei \(M\) ein endlich erzeugter \(A\)-Modul. Dann ist \(\ideal m \subset \ann (M/\ideal m M)\), also
		ist \(M/\ideal m M\) in natürlicher Weise ein (endlich-dimensionaler) \(F = A/\ideal m\)-Vektorraum.
		\\
		Wir nennen den \(k\)-Vektorraum \(M(\ideal m) \coloneqq M/\ideal m M\) auch die \emph{spezielle Faser
		von \(M\)}.
		\\
		Das Bild eines Elementes \(x \in M\) in \(M(\ideal m)\) heißt auch der \emph{Wert des Schnittes
		\(x\) in der speziellen Faser}.
	\end{visibleenv}
	\begin{proposition}<+->
		Seien \(x_1, \dotsc, x_n\) Schnitte von \(M\), deren Werte in \(M(\ideal m)\) eine Basis bilden.
		Dann erzeugen \(x_1, \dotsc, x_n\) den \(A\)-Modul \(M\).
	\end{proposition}
	\begin{proof}<+->
		Sei \(N\) der von den \(x_i\) erzeugte Untermodul von \(M\). Nach Voraussetzung ist die
		Komposition \(N \to M \to M/\ideal m M\) surjektiv, also \(N + \ideal m M = M\).
		Nach der letzten Folgerung ist daher \(M = N\).
	\end{proof}
\end{frame}

\mode<all>\section{Exakte Sequenzen}

\subsection{Definition und erste Eigenschaften}

\begin{frame}{Definition}
	\begin{definition}<+->
		Sei \(A\) ein Ring.
		Eine Sequenz
		\[
			\dotsc \to M^{i - 1} \xrightarrow{\phi^{i - 1}} M^i \xrightarrow{\phi^i} M^{i + 1}
			\to \dotsc
		\]
		von \(A\)-Moduln und \(A\)-Modulhomomorphismen heißt \emph{exakt bei \(M^i\)}, falls
		\(\im \phi^{i - 1} = \ker \phi^i\).
		\\
		Die Sequenz heißt \emph{exakt}, falls sie exakt bei jedem \(M^i\) ist.
	\end{definition}
\end{frame}

\begin{frame}{Injektivität und Surjektivität und exakte Sequenzen}
	Sei \(A\) ein Ring.
	\begin{example}<+->
		Eine Sequenz der Form \(0 \to M' \xrightarrow\phi M\) von \(A\)-Moduln ist genau dann exakt,
		wenn \(\phi\) injektiv ist.
	\end{example}	
	\begin{example}<+->
		Eine Sequenz der Form \(M \xrightarrow{\psi} M'' \to 0\) von \(A\)-Moduln ist genau dann exakt,
		wenn \(\psi\) surjektiv ist.
	\end{example}
\end{frame}

\begin{frame}{Kurze exakte Seqenzen}
	Sei \(A\) ein Ring.
	\begin{definition}<+->
		Eine \emph{kurze exakte Seqenz von \(A\)-Moduln} ist eine exakte Sequenz der Form
		\(0 \to M' \to M \to M'' \to 0\).
	\end{definition}
	\begin{example}<+->
		Eine Seqenz der Form \(0 \to M' \xrightarrow\phi M \xrightarrow\psi M'' \to 0\) ist genau dann
		exakt, wenn \(\phi\) injektiv ist, \(\psi\) surjektiv ist und \(\psi\) einen Isomorphismus
		\(\coker \phi = M/\im \phi \cong M''\) induziert.
	\end{example}
	\begin{example}<+->
		Jede lange exakte Sequenz \(\dotsc \to M^{i - 1} \xrightarrow{\phi^{i - 1}} M^i
		\xrightarrow{\phi^i} M^{i + 1}	\to \dotsc\) zerfällt in kurze exakte Sequenzen:
		\\
		Ist \(N^i = \im \phi^{i - 1} = \ker \phi^i\), haben wir kurze exakte Sequenzen
		\(0 \to N^i \to M^i \to N^{i + 1} \to 0\) für alle \(i\).
	\end{example}
\end{frame}

\begin{frame}{Rechtsexakte Sequenzen}
	\begin{proposition}<+->
		Sei \(A\) ein kommutativer Ring. Eine Sequenz
		\(E\colon M' \xrightarrow\phi M \xrightarrow\psi M'' \to 0\) von \(A\)-Moduln ist
		genau dann exakt, wenn für alle \(A\)-Moduln \(N\) auch
		\(\Hom(E, N)\colon 0 \to \Hom(M'', N) \xrightarrow{\psi^*} \Hom(M, N) \xrightarrow{\phi^*} \Hom(M', N)\)
		exakt ist.
	\end{proposition}
	\begin{proof}<+->
		\begin{enumerate}[<+->]
		\item<.->
			Wir zeigen eine Richtung: Sei die Hom-Sequenz exakt für alle \(N\). Wir wählen \(N = \coker \psi\).
			Ist \(\pi\colon M'' \to N\) die kanonische Projektion, so ist
			\(\pi \circ \psi = 0\). Da \(\psi^*\) injektiv ist, folgt \(\pi = 0\), also
			\(N = 0\). Damit ist \(\psi\) surjektiv.
		\item
			Für \(N = M''\) ist \(\psi \circ \phi = \phi^* \psi^*(\id_{M''}) = 0\),
			also \(\im \phi \subset \ker \psi\).
		\item
			Wir wählen \(N = \coker \phi\). Ist \(\pi\colon M \to N\) die kanonische Projektion,
			so ist \(\pi \circ \phi = 0\), also \(\pi \in \ker \phi^*\). Damit existiert ein
			\(\xi\colon M'' \to N\) mit \(\pi = \xi \circ \psi\), also \(\im \phi = \ker \pi \supset
			\ker \psi\).
			\qedhere
		\end{enumerate}
	\end{proof}
\end{frame}

\begin{frame}{Linksexakte Sequenzen}
	\begin{proposition}<+->
		Sei \(A\) ein kommutativer Ring. Eine Sequenz
		\(F\colon 0 \to N' \xrightarrow\phi N \xrightarrow\psi N''\) von \(A\)-Moduln ist
		genau dann exakt, wenn für alle \(A\)-Moduln \(M\) die Sequenz
		\(\Hom(M, F)\colon 0 \to \Hom(M, N') \xrightarrow{\phi_*} \Hom(M, N) \xrightarrow{\psi_*} \Hom(M, N'')\)
		exakt ist.
		\qed
	\end{proposition}
\end{frame}

\subsection{Das Schlangenlemma}

\begin{frame}{Das Schlangenlemmas}
	\begin{proposition}[Schlangenlemma]<+->
		Sei \(A\) ein Ring. Ist
		\[
			\begin{CD}
				0 @>>> M' @>{\alpha}>> M @>{\beta}>> M'' @>>> 0 \\
				& & @V{\phi'}VV @V{\phi}VV @V{\phi''}VV \\
				0 @>>> N' @>{\alpha'}>> N @>{\beta'}>> N'' @>>> 0
			\end{CD}
		\]
		ein kommutatives Diagramm von \(A\)-Moduln mit exakten Zeilen, so
		existiert eine kanonische exakte Sequenz der Form
		\(0 \to \ker \phi' \xrightarrow{\alpha_*} \ker \phi \xrightarrow{\beta_*} \ker \phi''
			\xrightarrow{\delta}
			\coker \phi' \xrightarrow{\alpha'_*} \coker \phi \xrightarrow{\beta'_*}
			\coker \phi'' \to 0\).
		\qed
	\end{proposition}
	\begin{visibleenv}<+->
		Der \emph{(Ko-)Randhomomorphismus \(\delta\)} ist folgendermaßen definiert: Sei \(x'' \in \ker 
		\phi''\). Dann ist \(x'' = \beta(x)\) für ein \(x \in M\).
		\\
		Da \(\beta'(\phi(x)) = \phi''(\beta(x)) = 0\), existiert ein \(y' \in N'\) mit \(\alpha'(y') =
		\phi(x)\).
		\\
		Schließlich ist \(\delta(x'')\) das Bild von \(y'\) in \(\coker \phi'\).
	\end{visibleenv}
\end{frame}

\begin{frame}{Die lange exakte Kohomologiesequenz}
	\begin{remark}
		Das Schlangenlemma ist ein Spezialfall der langen exakten Kohomologiesequenz der
		homologischen Algebra.
	\end{remark}
\end{frame}

\subsection{Additive Funktionen}

\begin{frame}{Definition additiver Funktionen}
	\begin{definition}<+->
		Sei \(A\) ein Ring. Sei \(\mathfrak C\) eine Klasse von \(A\)-Moduln. Eine Abbildung
		\(\lambda\colon \mathfrak C \to G\) in eine abelsche Gruppe heißt \emph{additive Funktion},
		falls für alle kurzen exakten Sequenzen
		\(0 \to C' \to C \to C'' \to 0\) von Moduln aus \(\mathfrak C\) gilt, daß \(\lambda(C) =
		\lambda(C') + \lambda(C'')\).
	\end{definition}
	\begin{example}<+->
		Seien \(K\) ein Körper und \(\mathfrak C\) die Klasse der endlich-dimensionalen \(K\)-Vektorräume.
		Dann ist \(\dim\colon \mathfrak C \to \set Z\) eine additive Funktion.
	\end{example}
\end{frame}

\begin{frame}{Additive Funktionen und beschränkte Sequenzen}
	\begin{proposition}<+->
		Sei \(A\) ein Ring. Sei \(\mathfrak C\) eine Klasse von \(A\)-Moduln und \(\lambda\colon \mathfrak C
		\to G\) eine additive Funktion. Sei
		\(0 \to M^0 \xrightarrow{\phi^0} M^1 \xrightarrow{\phi^1} \dotsb \to M^n \to 0\) eine exakte Sequenz von Moduln in 
		\(\mathfrak C\),
		so daß auch die Kerne der \(\phi^i\) zu \(\mathfrak C\) gehören.
		\\
		Dann gilt \(\sum\limits_{i = 0}^n (-1)^i \lambda(M^i) = 0\).
	\end{proposition}
	\begin{proof}<+->
		\begin{enumerate}[<+->]
		\item<.->
			Die exakte Sequenz zerfällt in kurze exakte Sequenzen der Form
			\(0 \to N^i \to M^i \to N^{i + 1} \to 0\) (mit \(N^0 = N^{n + 1} = 0\)),
			wobei \(N^i \in \mathfrak C\).
		\item
			Daher gilt \(\lambda(M^i) = \lambda(N^i) + \lambda(N^{i + 1})\). Addieren wir diese Gleichungen
			alternierend, hebt sich die rechte Seite weg.
			\qedhere
		\end{enumerate}
	\end{proof}
\end{frame}



\lecture{Tensorprodukte von Moduln}{Tensorprodukte von Moduln}
\mode<all>\setcounter{section}{13}
\mode<all>\section{Tensorprodukte von Moduln}

\subsection{Bilineare Abbildungen und das Tensorprodukt}

\begin{frame}{Bilineare Abbildungen}
	Sei \(A\) ein kommutativer Ring.
	\begin{definition}<+->
		Seien \(M, N, P\) drei \(A\)-Moduln. Eine Abbildung \(\beta\colon M \times N \to P\) heißt
		\emph{\(A\)-bilinear}, falls für alle \(x \in M\) die Abbildung \(y \mapsto \beta(x, y)\) und für
		alle \(y \in N\) die Abbildung \(x \mapsto \beta(x, y)\) Homomorphismen von \(A\)-Moduln sind.
	\end{definition}
	\begin{example}<+->
		Fassen wir \(A\) als \(A\)-Modul auf, ist die Multiplikationsabbildung \(A \times A \to A,
		(a, a') \mapsto a a'\) eine \(A\)-bilineare Abbildung.
	\end{example}
\end{frame}

\begin{frame}{Das Tensorprodukt}
	Sei \(A\) ein kommutativer Ring.
	\begin{proposition}<+->
		Seien \(M, N\) zwei \(A\)-Moduln. 
		\begin{enumerate}[<+->]
		\item<.->
			Es existiert ein \(A\)-Modul \(T\) zusammen mit einer
			\(A\)-bilinearen Abbildung \(\gamma\colon M \times N \to T\) mit der folgenden
			(universellen) Eigenschaft:
			\\	
			Für jeden weiteren \(A\)-Modul \(P\) zusammen mit einer \(A\)-bilinearen Abbildung
			\(\beta\colon M \times N \to P\) existiert genau eine \(A\)-lineare Abbildung
			\(\underline\beta\colon 
			T \to P\) mit \(\beta = \underline\beta \circ \gamma\).
		\item	
			Sind \((T, \gamma), (T', \gamma')\) zwei solcher Paare mit dieser Eigenschaft, so
			existiert genau ein Isomorphismus \(\phi\colon T \to T'\) mit \(\phi\circ \gamma = \gamma'\). 
		\end{enumerate}
	\end{proposition}
	\begin{visibleenv}<+->
		Es faktorisiert also jede \(A\)-bilineare Abbildung auf \(M \times N\) über \(T\).
	\end{visibleenv}
\end{frame}

\begin{frame}{Eindeutigkeit des Tensorproduktes}
	\begin{proof}[Eindeutigkeit]<+->
	\begin{enumerate}[<+->]
		\item<.->
			Aufgrund der universellen Eigenschaft für \((T, \gamma)\) existiert eine \(A\)-lineare Abbildung
			\(\phi\colon T \to T'\) mit \(\gamma' = \phi \circ \gamma\).
		\item
			Vertauschen der Rollen von \(T\) und \(T'\) liefert weiter eine \(A\)-lineare Abbildung
			\(\phi'\colon T' \to T\) mit \(\gamma = \phi' \circ \gamma'\).
		\item
			Es folgt, daß \(\gamma = (\phi' \circ \phi) \circ \gamma\). Nach der Eindeutigkeitsaussage der 
			universellen Eigenschaft für \(T\) muß daher \(\phi' \circ \phi = \id_T\) gelten.
		\item
			Analog folgt \(\phi \circ \phi' = \id_{T'}\).
			\renewcommand\qedsymbol{}
			\qedhere
		\end{enumerate}
	\end{proof}
\end{frame}

\begin{frame}{Konstruktion des Tensorproduktes}
	\begin{proof}[Konstruktion]<+->
		\begin{enumerate}[<+->]
		\item<.->
			Sei \(C\) der freie \(A\)-Modul \(A^{(M \times N)}\). Die Elemente von \(C\) können wir uns als
			formale Linearkombinationen \(\sum\limits_{i = 1}^n a^i (x_i, y_i)\) mit \(a_i \in A, x_i \in M,
			y_i \in N\) vorstellen.
		\item
			Sei \(D\) der von allen Elementen der Form
			\begin{align*}
				& (x + x', y) - (x, y) - (x', y),  \\
				& (x, y + y') - (x, y) - (x, y'),  \\
				& (ax, y) - a(x, y), \\
				& (x, ay) - a(x, y)
			\end{align*}
			mit \(x, x' \in M\), \(y, y' \in N\) und \(a \in A\) erzeugte Untermodul von \(C\).
		\item
			Wir setzen \(T \coloneqq C/D\). Wir schreiben \(x \otimes y \in T\) für das Bild des Elementes 
			\((x, y) \in C\) in \(T\).
			\renewcommand\qedsymbol{}
			\qedhere
		\end{enumerate}
	\end{proof}
\end{frame}

\begin{frame}{Universelle Eigenschaft des Tensorproduktes}
	\begin{proof}[Beweis der universellen Eigenschaft]<+->
		\begin{enumerate}[<+->]
		\item<.->
			Damit ist \(T\) erzeugt von Elementen der Form \(x \otimes y\),
			wobei folgende Gleichungen gelten:
			\begin{align*}
				& (x + x') \otimes y = x \otimes y + x' \otimes y, &
				& x \otimes (y + y') = x \otimes y + x \otimes y', \\
				& (ax) \otimes y = a (x \otimes y), &
				& x \otimes (ay) = a (x \otimes y).
			\end{align*}
			\\
			Also ist \(\gamma\colon M \times N
			\to T, (x, y) \mapsto x \otimes y\) eine \(A\)-bilineare Abbildung.
		\item
			Jede Abbildung \(\beta\colon M \times N \to P\) induziert eine \(A\)-lineare Abbildung
			\(\beta'\colon C \to P, (x, y) \mapsto \beta(x, y)\).
			\\
			Ist \(\beta\) eine
			\(A\)-bilineare Abbildung verschwindet \(\beta'\) auf den Erzeugern von \(D\), induziert also
			nach dem Homomorphiesatz eine \(A\)-lineare Abbildung \(\underline\beta\colon T \to P,
			x \otimes y \mapsto \beta(x, y)\).
		\item
			Da \(T\) von Elementen der Form \(x \otimes y\) erzeugt wird, ist \(\underline\beta\) die einzige
			Abbildung mit \(\beta = \underline\beta \circ \gamma\).
			\qedhere
		\end{enumerate}
	\end{proof}
\end{frame}

\begin{frame}{Notation für das Tensorprodukt}
	Sei \(A\) ein kommutativer Ring. Seien \(M\) und \(N\) zwei \(A\)-Moduln.
	\\
	Der in der letzten Proposition konstruierte \(A\)-Modul \(T\) heißt das \emph{Tensorprodukt von
	\(M\) und \(N\)}.
	\begin{notation}<+->
		Wir schreiben
		\(M \otimes_A N\) für das Tensorprodukt von \(M\) mit \(N\).
	\end{notation}
	\begin{visibleenv}<+->
		Im Falle, daß der Ring \(A\) aus dem Kontext hervorgeht, schreiben wir auch \(M \otimes N\) anstelle
		von \(M \otimes_A N\).
	\end{visibleenv}
	\begin{notation}<+->
		Ist \(\beta\colon M \times N \to P\) eine bilineare Abbildung in einen weiteren \(A\)-Modul, so
		schreiben wir \(M \otimes N \to P, x \otimes y \mapsto \beta(x, y)\) für diejenige \(A\)-lineare
		Abbildung \(\underline\beta\) mit \(\underline\beta(x \otimes y) = \beta(x, y)\).
	\end{notation}
\end{frame}
	
\begin{frame}{Das Tensorprodukt endlich erzeugter Modul}
	\begin{remark}<+->
		Das Tensorprodukt \(M \otimes N\) ist als \(A\)-Modul durch die \emph{Produkte} \(x \otimes y\) mit
		\(x \in M\) und  \(y \in N\) erzeugt.
		\\
		Sind \((x_i)_{i \in I}\) und \((y_j)_{j \in J}\) Familien von Erzeugern von \(M\) beziehungsweise
		\(N\), so erzeugen die Elemente \(x_i \otimes y_j\) das Tensorprodukt \(M \otimes N\).
		\\
		Insbesondere ist \(M \otimes N\) ein endlich erzeugter \(A\)-Modul, falls \(M\) und \(N\) endlich
		erzeugte \(A\)-Moduln sind.
	\end{remark}
\end{frame}

\begin{frame}{Beispiele von Tensorprodukten}
	\begin{example}<+->
		Wir betrachten \(\set Z\) und \(\set Z/(2)\) als \(\set Z\)-Moduln. Sei \(x \in \set Z/(2)\) das
		nicht verschwindende Element. Sei \(2 \set Z \subset \set Z\) der Untermodul der geraden ganzen
		Zahlen.
		\\
		Im Tensorprodukt \(\set Z \otimes_{\set Z} \set Z/(2)\) verschwindet das Element \(2 \otimes x\), denn
		\(2 \otimes x = 1 \otimes (2x) = 1 \otimes 0 = 0\).
		\\
		Auf der anderen Seite verschwindet das Element \(2 \otimes x\) nicht im Tensorprodukt
		\(2 \set Z \otimes_{\set Z} \set Z/(2)\).
	\end{example}
	\begin{visibleenv}<+->
		Damit ist die Notation \(x \otimes y\) mehrdeutig, solange das Tensorprodukt, in dem das Element enthalten
		ist, nicht festgelegt worden ist.
	\end{visibleenv}
\end{frame}

\begin{frame}{Verschwindende Tensorprodukte}
	\begin{corollary}<+->
		Sei \(A\) ein kommutativer Ring. Seien \(M, N\) zwei \(A\)-Moduln und \(x_i \in M\) und
		\(y_i \in N\) mit \(\sum\limits_{i} x_i \otimes y_i = 0\) in \(M \otimes N\). Dann existieren endlich
		erzeugte Untermoduln \(M_0 \subset M\) und \(N_0 \subset N\), so daß
		\(\sum\limits_{i} x_i \otimes y_i = 0\) in \(M_0 \otimes N_0\).
	\end{corollary}
	\begin{proof}<+->
		\begin{enumerate}[<+->]
		\item<.->
			Wir benutzen dieselbe Notation wie im Beweis der Existenz des Tensorproduktes, insbesondere also
			\(M \otimes N = T = C/D\).
		\item
			Da \(\sum\limits_{i} x_i \otimes y_i = 0\) in \(M \otimes N\), folgt
			\(\sum\limits_{i} (x_i, y_i) \in D\), ist also eine endliche Summe von Erzeugern \(d_j\).
		\item
			Sei \(M_0\) der Untermodul von \(M\), welcher von allen \(x_i\) und allen Elementen von \(M\),
			welche als
			erste Komponenten in den \(d_j\) auftauchen, erzeugt wird.
			Der Untermodul \(N_0\) von \(N\) wird auf analoge Weise definiert.
			\qedhere
		\end{enumerate}
	\end{proof}
\end{frame}

\begin{frame}{Bemerkung zur Konstruktion des Tensorproduktes}
	\begin{remark}<+->
		Ab jetzt spielt die genaue Konstruktion des Tensorproduktes für uns keine Rolle mehr. Wesentlich
		ist seine universelle Eigenschaft.
	\end{remark}
\end{frame}

\subsection{Multilineare Abbildungen und mehrfache Tensorprodukte}

\begin{frame}{Multlineare Abbildungen}
	In Verallgemeinerung des Begriffes der bilinearen Abbildung definieren wir:
	\begin{definition}<+->
		Sei \(A\) ein kommutativer Ring. Sei \(M_1, \dotsc, M_r, P\) eine Folge von \(A\)-Moduln.
		\\
		Eine Abbildung \(\mu\colon M_1 \times \dotsb \times M_r \to P\) heißt \emph{\(A\)-multilinear}, falls
		sie linear in jedem Argument ist.
	\end{definition}
\end{frame}

\begin{frame}{Mehrfache Tensorprodukte}
	\begin{proposition}<+->
		Sei \(A\) ein kommutativer Ring. Sei \(M_1, \dotsc, M_r, P\) eine Folge von \(A\)-Moduln.
		\begin{enumerate}[<+->]
		\item<.->
			Es existiert ein \(A\)-Modul \(T\) zusammen mit einer \(A\)-multilinearen Abbildung
			\(\gamma\colon M_1 \times \dotsb \times M_r \to T\) mit der folgenden (universellen) Eigenschaft:
			\\
			Für jeden weiteren \(A\)-Modul \(P\) zusammen mit einer \(A\)-multilinearen Abbildungen
			\(\mu\colon M_1 \times \dotsb \times M_r \to P\) existiert genau eine \(A\)-lineare Abbildung
			\(\underline\mu\colon T \to P\) mit \(\mu = \underline\mu \circ \gamma\).
		\item
			Sind \((T, \gamma), (T', \gamma')\) zwei solcher Paare mit dieser Eigenschaft, so
			existiert genau ein Isomorphismus \(\phi\colon T \to T'\) mit \(\phi \circ \gamma = \gamma'\).
			\qed
		\end{enumerate}
	\end{proposition}
	\begin{visibleenv}<+->
		Wir schreiben \(M_1 \otimes \dotsb \otimes M_r\) für \(T\).
	\end{visibleenv}
\end{frame}

\subsection{Kanonische Isomorphismen zwischen Tensorprodukten}

\begin{frame}{Kanonische Isomorphismen zwischen Tensorprodukten}
	\begin{proposition}<+->
		Sei \(A\) ein kommutativer Ring. Seien \(M, N, P\) drei \(A\)-Moduln. Dann existieren Isomorphismen
		\begin{enumerate}[<+->]
		\item<.->
			\(M \otimes N \isoto N \otimes M, x \otimes y \mapsto y \otimes x\),
		\item
			\((M \otimes N) \otimes P \isoto M \otimes (N \otimes P) \isoto  M \otimes N \otimes P,
			(x \otimes y) \otimes z \mapsto x \otimes (y \otimes z) \mapsto x \otimes y \otimes z\),
		\item
			\((M \oplus N) \otimes P \isoto (M \otimes P) \oplus (N \otimes P),
			(x, y) \otimes z \mapsto (x \otimes z, y \otimes z)\),
		\item
			\(A \otimes M \isoto M, a \otimes x \mapsto ax\).
		\end{enumerate}
	\end{proposition}
\end{frame}

\begin{frame}{Beweis der kanonischen Isomorphismen}
	\begin{proof}<+->
		\begin{enumerate}[<+->]
		\item<.->
			In allen Fällen ist nachzuweisen, daß die so definierten Abbildungen wohldefiniert sind und 
			Umkehrungen	besitzen.
			\\
			Wir beweisen dies exemplarisch am Beispiel \((M \otimes N) \otimes P
			\isoto M \otimes N \otimes P, (x \otimes y) \otimes z \mapsto x \otimes y \otimes z\).
		\item
			Sei zunächst \(z \in P\). Die Abbildung \(M \times N \to M \otimes N \otimes P, (x, y) \mapsto x \otimes y
			\otimes z\) ist bilinear in \(x, y\) und induziert damit einen Homomorphismus
			\(M \otimes N \to M \otimes N \otimes P, x \otimes y \mapsto x \otimes
			y \otimes z\).
		\item
			Die Abbildung \((M \otimes N) \times P \to M \otimes N \otimes P, (t, z) \mapsto t \otimes z\)
			ist bilinear in \(t, z\) und induziert damit einen Homomorphismus \((M \otimes N) \otimes P
			\to M \otimes N \otimes P, (x \otimes y) \otimes z \mapsto x \otimes y \otimes z\).
		\item
			Die Wohldefiniertheit von
			\(M \otimes N \otimes P \to (M \otimes N) \otimes P, x \otimes y \otimes z \mapsto
			(x \otimes y) \otimes z\) wird analog gezeigt. Dies ist die Umkehrung.
			\qedhere
		\end{enumerate}
	\end{proof}
\end{frame}

\begin{frame}{Verträglichkeit der Tensorprodukte über verschiedene Ringe}
	Seien \(A, B\) zwei kommutative Ringe.
	\begin{definition}<+->
		Ein \((A, B)\)-Bimodul \(N\) ist eine abelsche Gruppe \(N\), welche zugleich ein \(A\)-Modul
		und ein \(B\)-Modul ist, so daß die beiden Stukturen kompatibel sind, nämlich
		\(a (b x) = b (a x)\) für alle \(a \in A\), \(b \in B\) und \(x \in N\).
	\end{definition}
	\begin{proposition}<+->
		Sei \(M\) ein \(A\)-Modul, \(P\) ein \(B\)-Modul und \(N\) ein \((A, B)\)-Bimodul. Dann ist
		\(M \otimes_A N\) durch Multiplikation im zweiten Faktor in natürlicher Weise ein \(B\)-Modul und
		\(N \otimes_B P\) durch Multiplikation im ersten Faktor in natürlicher Weise ein \(A\)-Modul.
		\\
		Schließlich ist \((M \otimes_A N) \otimes_B P \isoto M \otimes_A (N \otimes_B P),
		(x \otimes y) \otimes z \mapsto x \otimes (y \otimes z)\) ein Isomorphismus abelscher Gruppen.
		\qed
	\end{proposition}
\end{frame}

\subsection{Funktorialität des Tensorproduktes}

\begin{frame}{Das Tensorprodukt zweier Abbildungen}
	\begin{example}<+->
		Sei \(A\) ein kommutativer Ring. Seien \(\phi\colon M \to M'\) und \(\psi\colon N \to N'\)
		Homomorphismen von \(A\)-Moduln.
		\\
		Es ist \(M \times N \to M' \otimes N', (x, y) \mapsto \phi(x) \otimes \psi(y)\) eine \(A\)-bilineare
		Abbildung und induziert daher einen Homomorphismus
		\[
			\phi \otimes \psi\colon M \otimes N \to M' \otimes N', x \otimes y \mapsto \phi(x) \otimes \psi(y)
		\]
		von \(A\)-Moduln.
	\end{example}
\end{frame}

\begin{frame}{Verträglichkeit des Tensorproduktes mit Verknüpfungen}
	\begin{proposition}<+->
		Sei \(A\) ein kommutativer Ring. Seien \(M \xrightarrow\phi M' \xrightarrow{\phi'} M''\) und
		\(N \xrightarrow{\psi} N' \xrightarrow{\psi'} N''\) Homomorphismen von \(A\)-Moduln. Dann ist
		\[
			(\phi' \circ \phi) \otimes (\psi' \circ \psi) = (\phi' \otimes \psi') \circ
			(\phi \otimes \psi)\colon M \otimes N \to M'' \otimes N''.
		\]
	\end{proposition}
	\begin{proof}<+->
		Die Homomorphismen \((\phi' \circ \phi) \otimes (\psi' \circ \psi)\) und \((\phi' \otimes \psi')
		\circ (\phi \otimes \psi)\) stimmen auf allen Elementen der Form \(x \otimes y \in M \otimes N\)
		überein. Da diese Elemente \(M \otimes N\) erzeugen, folgt die Gleichheit der Homomorphismen.
	\end{proof}
\end{frame}


\mode<all>\section{Skalareinschränkungen und -erweiterungen}

\subsection{Skalareinschränkung}

\begin{frame}{Definition der Skalareinschränkung}
	Sei \(\phi\colon A \to B\) ein Homomorphismus kommutativer Ringe. Sei \(N\) ein \(B\)-Modul.
	\\
	Wir definieren einen \(A\)-Modul \(N^A\) wie folgt:
	\\
	Die abelsche Gruppe von \(N^A\) ist die abelsche Gruppe von \(N\). Die Multiplikation mit
	Elementen aus \(A\) wird durch \(a y \coloneqq \phi(a) y\) mit \(a \in A\) und \(y \in N\)
	definiert.
	\begin{definition}<+->
		Der \(A\)-Modul \(N^A\) heißt die \emph{Skalareinschränkung von \(N\) (vermöge \(\phi\)) auf \(A\)}.
	\end{definition}
	\begin{example}<+->
		Da wir \(B\) als Modul über sich selbst auffassen können, erhalten wir insbesondere den
		\(A\)-Modul \(B^A\).
	\end{example}
\end{frame}

\begin{frame}{Endlichkeit von Skalareinschränkungen}
	\begin{proposition}<+->
		Sei \(\phi\colon A \to B\) ein Homomorphismus kommutativer Ringe. Sei \(N\) ein \(B\)-Modul.
		Ist \(B^A\) ein endlich erzeugter \(A\)-Modul und \(N\) ein endlich erzeugter \(B\)-Modul,
		so ist \(N^A\) ein endlich erzeugter \(A\)-Modul.
	\end{proposition}
	\begin{proof}<+->
		Sei \(N\) als \(B\)-Modul von den Elementen \(y_1, \dotsc, y_n \in N\) erzeugt. Sei weiter
		\(B^A\) als \(A\)-Modul von den Elementen \(b_1, \dotsc, b_m \in B\) erzeugt. Dann erzeugen die
		Produkte \(b_1 y_1, \dotsc, b_m y_n \in N\) den \(A\)-Modul \(N^A\).
	\end{proof}
\end{frame}

\subsection{Skalarerweiterung}

\begin{frame}{Definition der Skalarerweiterung}
	Sei \(\phi\colon A \to B\) ein Homomorphismus kommutativer Ringe. Sei \(M\) ein \(A\)-Modul.
	\\
	Wir definieren einen \(B\)-Modul \(M_B\) wie folgt:
	\\
	Die abelsche Gruppe von \(M_B\) ist die abelsche Gruppe von \(B^A \otimes_A M\). Die
	Multiplikation mit Elementen aus \(B\) wird durch \(b (b' \otimes x) \coloneqq (b b') \otimes x\)
	mit \(b, b' \in B\) und \(x \in M\) definiert.
	\begin{definition}<+->
		Der \(B\)-Modul \(M_B\) heißt die \emph{Skalarerweiterung von \(M\) (vermöge \(\phi\)) auf \(B\)}.
	\end{definition}
	\begin{example}<+->
		Sei \(N\) ein \(B\)-Modul. Dann ist \(N^A \otimes_A M = N \otimes_A M\), indem wir \(N\) als
		\((A, B)\)-Bimodul auffassen. Insbesondere ist \(N^A \otimes_A M\) (durch Multiplikation von links) in
		kanonischer Weise ein \(B\)-Modul.
		\\	
		In dieser Situation existiert ein kanonischer Isomorphismus \(N \otimes_B M_B \cong
		N^A \otimes_A M\) von \(B\)-Moduln.
	\end{example}
\end{frame}

\begin{frame}{Endlichkeit von Skalarerweiterungen}
	\begin{proposition}<+->
		Ist \(M\) als \(A\)-Modul endlich erzeugt, so ist \(M_B\) als \(B\)-Modul endlich erzeugt.
	\end{proposition}
	\begin{proof}<+->
		Sei \(M\) als \(A\)-Modul von \(x_1, \dotsc, x_m \in M\) erzeugt. Dann wird \(M_B\) als \(B\)-Modul
		durch \(1 \otimes x_1, \dotsc, 1 \otimes x_m \in M_B\) erzeugt.
	\end{proof}
\end{frame}


\mode<all>\section{Exaktheitseigenschaften des Tensorproduktes}

\subsection{Tensorprodukte und Homomorphismenmoduln}

\begin{frame}{Tensorprodukte und Homomorphismenmoduln}
	Sei \(A\) ein kommutativer Ring. Seien \(M, N, P\) drei \(A\)-Moduln.
	\\
	Sei \(\phi\colon M \otimes N \to P\) eine \(A\)-lineare Abbildung. Diese
	definiert eine \(A\)-lineare Abbildung
	\((\phi N)\colon M \to \Hom(N, P), x \mapsto (y \mapsto \phi(x \otimes y))\).
	\begin{proposition}<+->
		Die Abbildung
		\[
			\Hom(M \otimes N, P) \to \Hom(M, \Hom(N, P)), \phi \mapsto (\phi N)
		\]
		ist ein Isomorphismus von \(A\)-Moduln.
	\end{proposition}
	\begin{proof}<+->
		\begin{enumerate}[<+->]
		\item<.->
			Sei \(\psi\colon M \to \Hom(N, P)\) eine \(A\)-lineare Abbildung.
			\\
			Diese definiert eine \(A\)-lineare Abbildung
			\((N\psi)\colon M \otimes N \to P, x \otimes y \mapsto \psi(x)(y)\).
		\item
			Die Abbildung \(\psi \mapsto (N\psi)\) ist die Umkehrung der
			Abbildung \(\phi \mapsto (\phi N)\).
		\qedhere
		\end{enumerate}
	\end{proof}
\end{frame}

\subsection{Rechtsexaktheit des Tensorproduktes}

\begin{frame}{Rechtsexaktheit des Tensorproduktes}
	\begin{proposition}<+->
		Sei \(A\) ein kommutativer Ring. Sei \(E\colon M' \xrightarrow{\phi} M \xrightarrow{\psi} M'' \to 0\)
		eine exakte Sequenz von \(A\)-Moduln und \(N\) ein weiterer \(A\)-Modul. Dann ist auch die Sequenz
		\(E \otimes N\colon M' \otimes N \xrightarrow{\phi \otimes \id_N} M \otimes N \xrightarrow{\psi \otimes \id_N} M'' \otimes N
		\to 0\) exakt.
	\end{proposition}
	\begin{proof}<+->
		\begin{enumerate}[<+->]
		\item<.->
			Sei \(P\) ein beliebiger \(A\)-Modul. Aus der Exaktheit von \(E\) folgt die Exaktheit
			der Sequenz \(\Hom(E, \Hom(N, P))\).
		\item
			Diese Sequenz ist isomorph zur Sequenz \(\Hom(E \otimes N, P)\), welche damit auch exakt ist.
		\item
			Da \(P\) beliebig ist, folgt die Exaktheit von \(E \otimes N\).
			\qedhere
		\end{enumerate}
	\end{proof}
\end{frame}

\begin{frame}{Rechtsexaktheit adjungierter Funktoren}
	\begin{remark}<+->
		Sei \(A\) ein kommutativer Ring. Sei \(N\) ein \(A\)-Modul. Definieren wir
		\(T(M) \coloneqq M \otimes N\) und \(U(P) \coloneqq \Hom(N, P)\) für \(A\)-Moduln
		\(M\) und \(P\), so haben wir die Existenz eines natürlichen Isomorphismus'
		\[
			\Hom(T(M), P) = \Hom(M, U(P))
		\]
		gezeigt.
		\\
		In der Sprache der Kategorientheorie ist \(T\) damit das Linksadjungierte zu \(U\) und
		das Rechtsadjungierte zu \(T\).
		\\
		Der Beweis der letzten Proposition zeigt allgemeiner, daß jeder Funktor, welcher
		ein linksadjungierter ist, rechtsexakt ist.
		\\
		Entsprechend ist ein Funktor, welcher ein linksadjungierter ist, ein linksexakter.
	\end{remark}
\end{frame}

\begin{frame}{Flache Moduln}
	\begin{remark}<+->
		Sei \(A\) ein kommutativer Ring.
		Sei \(M' \to M \to M''\) eine exakte Sequenz von
		\(A\)-Moduln. Im allgemeinen ist dann das Tensorprodukt \(M' \otimes N \to M \otimes N
		\to M'' \otimes N\) mit einem beliebigen \(A\)-Moduln \(N\) nicht mehr exakt.
	\end{remark}
	\begin{example}<+->
		Sei die exakte Sequenz \(0 \to \set Z \xrightarrow\phi \set Z\) mit \(\phi(x) = 2 x\)
		von \(\set Z\)-Moduln gegeben.
		\\
		Sei weiter der \(\set Z\)-Modul \(N \coloneqq \set Z/(2)\) gegeben.
		Das Tensorprodukt \(0 \to \set Z \otimes N \xrightarrow{\phi \otimes \id_N} \set Z \otimes N\)
		der Sequenz mit \(N\) ist nicht exakt, denn für alle \(x \otimes y \in \set Z \otimes N\) ist
		\((\phi \otimes \id_N)(x \otimes y) = 2x \otimes y = x \otimes 2y = x \otimes 0 = 0\), also
		\(\phi \otimes \id_N = 0\). Allerdings ist \(\set Z \otimes N\) nicht der Nullmodul.
	\end{example}
\end{frame}

\subsection{Flachheit}

\begin{frame}{Flache Moduln}
	Sei \(A\) ein kommutativer Ring. Sei \(N\) ein \(A\)-Modul.
	\begin{definition}<+->
		Der \(A\)-Modul \(N\) heißt \emph{flach}, falls für jede exakte Sequenz \(E\) von \(A\)-Moduln
		auch die tensorierte Sequenz \(E \otimes N\) exakt ist.
	\end{definition}
	\begin{lemma}<+->
		Sei \(\phi \otimes \id_N\colon M' \otimes N \to M \otimes N\) für jede injektive lineare Abbildung
		\(\phi\colon M' \to M\) endlich erzeugter \(A\)-Moduln injektiv. Dann ist auch
		\(\phi \otimes \id_N\) für jede injektive lineare Abbildung \(\phi\colon M' \to M\) beliebiger
		\(A\)-Moduln injektiv.
	\end{lemma}
\end{frame}

\begin{frame}{Beweis des Hilfssatzes}
	\begin{proof}<+->
		\begin{enumerate}[<+->]
		\item<.->
			Sei also \(\phi\colon M' \to M\) eine injektive lineare Abbildung zwischen \(A\)-Moduln. Sei
			\(u = \sum x_i' \otimes y_i \in \ker (\phi \otimes \id_N)\), also \(\sum \phi(x_i') \otimes y_i = 0
			\in M \otimes N\).
		\item
			Sei \(M_0' \subset M'\) der durch die \(x_i'\) erzeugte Untermodul und sei \(u_0 = \sum x_i' \otimes y_i 
			\in M_0' \otimes N\).
		\item
			Es existiert ein endlich erzeugter Untermodul \(M_0 \subset M\) mit \(\phi(M_0') \subset M_0\), so daß
			\(\sum \phi(x_i') \otimes y_i = 0 \in M_0 \otimes N\). Damit ist also \((\phi_0 \otimes \id_N)(u_0) = 0\),
			wobei \(\phi_0 \coloneqq \phi|_{M_0'}\colon M_0' \to M_0\).
		\item 
			Da \(M_0', M_0\) endlich erzeugt sind, folgt damit nach Voraussetzung, daß \(u_0 = 0\), also \(u = 0\).
			\qedhere
		\end{enumerate}
	\end{proof}
\end{frame}

\begin{frame}{Charakterisierungen von Flachheit}
	\begin{proposition}<+->
		Sei \(A\) ein kommutativer Ring. Sei \(N\) ein \(A\)-Modul. Dann sind äquivalent:
		\begin{enumerate}[<+->]
		\item<.->
			Der \(A\)-Modul \(N\) ist flach.
		\item
			Für jede kurze exakte Sequenz \(E\colon 0 \to M' \to M \to M'' \to 0\) von \(A\)-Moduln ist die tensorierte
			Sequenz \(E \otimes N\) exakt.
		\item
			Für jede injektive \(A\)-lineare Abbildung \(\phi\colon M' \to M\) zwischen \(A\)-Moduln ist die
			Abbildung \(\phi \otimes \id_N\colon M' \otimes N \to M \otimes N\) injektiv.
		\item
			Für jede injektive \(A\)-lineare Abbildung \(\phi\colon M' \to M\) zwischen endlich erzeugten \(A\)-Moduln
			ist die Abbildung \(\phi\otimes\id_N\colon M' \otimes N \to M \otimes N\) injektiv.
		\end{enumerate}
	\end{proposition}
\end{frame}

\begin{frame}{Beweis zur Charakterisierung von Flachheit}
	\begin{proof}<+->
		\begin{enumerate}[<+->]
		\item<.->
			Die Äquivalenz der ersten beiden Aussagen folgt aus der Tatsache, daß jede lange exakte Sequenz in kurze
			exakte Sequenzen zerfällt werden kann.
		\item
			Die Äquivalenz der zweiten und dritten Aussage folgt aus der Rechtsexaktheit des Tensorproduktes.
		\item
			Die Äquivalenz der letzten beiden Aussagen folgt aus dem letzten Hilfssatz.
			\qedhere
		\end{enumerate}
	\end{proof}
\end{frame}

\begin{frame}{Flachheit von Skalarerweiterungen}
	\begin{proposition}<+->
		Sei \(\phi\colon A \to B\) ein Homomorphismus kommutativer Ringe. Ist \(M\) ein flacher \(A\)-Modul, so ist
		\(M_B\) ein flacher \(B\)-Modul.
	\end{proposition}
	\begin{proof}<+->
		Dies folgt aus den kanonischen Isomorphismen \(N \otimes_B M_B \cong N^A \otimes_A M\) für \(B\)-Moduln \(N\).
	\end{proof}
\end{frame}



\lecture{Algebren}{Algebren}
\mode<all>\setcounter{section}{16}
\mode<all>\section{Algebren}

\subsection{Definition von Algebren}

\begin{frame}{Definition einer kommutativen Algebra}
	Sei \(A\) ein kommutativer Ring.
	\begin{definition}<+->
		Eine \emph{kommutative \(A\)-Algebra \(B\)} ist ein kommutativer Ring \(B\) zusammen mit einem
		Ringhomomorphismus \(\phi\colon A \to B\), dem \emph{Strukturmorphismus der Algebra \(B\)}.
	\end{definition}
	\begin{remark}<+->
		Ist \(B\) eine \(A\)-Algebra, so können wir insbesondere die Skalareinschränkung
		\(B^A\) definieren. Damit ist eine Multiplikation mit Elementen aus \(A\) auf der
		\(B\) zugrundeliegenden abelschen Gruppe durch \(a b \coloneqq \phi(a) b \in B\) mit \(a \in A\)
		und \(b \in B\) definiert.
		\\
		Die so definierte Struktur ist kompatibel mit der multiplikativen Struktur auf \(B\).
	\end{remark}
\end{frame}

\begin{frame}{Beispiele von Algebren}
	\begin{example}<+->
		Sei \(K\) ein Körper und \(B \neq 0\) eine kommutative \(K\)-Algebra. Da der Strukturmorphismus in diesem
		Falle injektiv ist, können wir \(K\) kanonisch mit seinem Bild in \(B\) identifizieren.
		\\
		Damit ist eine kommutative Algebra über einem Körper nichts anderes als ein kommutativer Ring, welcher \(K\)
		als Unterring enthält.
	\end{example}
	\begin{example}<+->
		Sei \(B\) ein beliebiger kommutativer Ring. Da genau ein Ringhomomorphismus \(\set Z \to B\) existiert,
		nämlich \(n \mapsto n \cdot 1_B\), wobei \(1_B\) die Eins in \(B\) bezeichnet, wird jeder kommutativer
		Ring auf genau eine Weise zu einer \(\set Z\)-Algebra.
	\end{example}
\end{frame}

\begin{frame}{Algebrenhomomorphismen}
	Sei \(A\) ein kommutativer Ring. Seien \(B, C\) zwei kommutative \(A\)-Algebren. 
	\begin{definition}<+->
		Ein \emph{Homomorphismus
		\(\chi\colon B \to C\) von \(A\)-Algebren} ist ein Ringhomomorphismus \(\chi\colon B \to C\), welcher
		einen Homomorphismus \(\chi\colon B^A \to C^A\) von \(A\)-Moduln induziert.
	\end{definition}
	\begin{visibleenv}<+->
		Ein Ringhomomorphismus \(\chi\colon B \to C\) ist also genau dann ein Homomorphismus von \(A\)-Algebren,
		falls \(\chi(a b) = a \chi(b)\) für alle \(a \in A\) und \(b \in B\).
	\end{visibleenv}
	\begin{remark}<+->
		Seien \(\phi\colon A \to B\) und \(\psi\colon A \to C\) die beiden Strukturhomomorphismen. Ein
		Ringhomomorphismus \(\chi\colon B \to C\) ist genau dann ein Homomorphismus von \(A\)-Algebren,
		falls \(\chi \circ \phi = \psi\).	
	\end{remark}
\end{frame}

\subsection{Endliche Algebren und Algebren endlichen Typs}

\begin{frame}{Endliche Algebren}
	\begin{definition}<+->
		Sei \(A\) ein kommutativer Ring. Eine kommutative \(A\)-Algebra \(B\) heißt eine \emph{endliche \(A\)-Algebra},
		falls \(B^A\) als \(A\)-Modul endlich erzeugt ist.
	\end{definition}
	\begin{visibleenv}<+->
		Es ist \(B\) also genau dann eine endliche \(A\)-Algebra, falls endlich viele Elemente \(b_1, \dotsc, b_n \in B\)
		existieren, so daß jedes andere Element von \(B\) als eine \(A\)-Linearkombination der \(b_i\) geschrieben werden
		kann.
	\end{visibleenv}
\end{frame}

\begin{frame}{Algebren endlichen Typs}
	\begin{definition}<+->
		Sei \(A\) ein kommutativer Ring. Eine kommutative \(A\)-Algebra \(B\) heißt eine \emph{\(A\)-Algebra endlichen Typs},
		falls endlich viele Elemente \(b_1, \dotsc, b_n \in B\) existieren, so daß jedes Element von \(B\) als Polynom
		in den \(b_i\) mit Koeffizienten aus \(A\) geschrieben werden kann.
	\end{definition}
	\begin{visibleenv}<+->
		Es ist \(B\) also genau dann eine \(A\)-Algebra endlichen Typs, falls ein surjektiver \(A\)-Algebrenhomomorphismus von
		einem Polynomring \(A[x_1, \dotsc, x_n]\) auf \(B\) existiert.
	\end{visibleenv}
	\begin{example}<+->
		Ein kommutativer Ring \(B\) heißt \emph{endlich erzeugt}, falls er eine \(\set Z\)-Algebra endlichen Typs ist.
		\\
		Dies ist gleichbedeutend damit, daß endlich viele Elemente \(b_1, \dotsc, b_n\) von \(B\) existieren, so daß jedes
		Element von \(B\) als Polynom in den \(b_i\) mit ganzzahligen Koeffizienten geschrieben werden kann.
	\end{example}
\end{frame}


\mode<all>\section{Tensorprodukte von Algebren}

\subsection{Definition des Tensorproduktes zweier Algebren}

\begin{frame}{Ein Produkt auf dem Tensorprodukt zweier Algebren}
	Sei \(A\) ein kommutativer Ring, und seien \(B\) und \(C\) zwei kommutative \(A\)-Algebren.
	\\
	Da \(B\) und \(C\) insbesondere \(A\)-Moduln sind, können wir den \(A\)-Modul \(D \coloneqq B^A \otimes_A C^A\)
	betrachten.
	\\
	Die Abbildung \(B \times C \times B \times C \mapsto D, (b, c, b', c') \mapsto bb' \otimes cc'\)
	ist \(A\)-linear in jedem Faktor, so daß sie einen Homomorphismus
	\(B \otimes C \otimes B \otimes C \to D, b \otimes c \otimes b' \otimes c' \mapsto bb' \otimes cc'\)
	von \(A\)-Moduln induziert.
	\\
	Durch Klammersetzung erhalten wir also einen Homomorphismus \(D \otimes D \to D\) von \(A\)-Moduln,
	welcher wiederum zu einer \(A\)-bilinearen Abbildung
	\[
		\mu\colon D \times D \to D, (b \otimes c, b' \otimes c') \mapsto bb' \otimes cc'
	\]
	korrespondiert.
\end{frame}

\begin{frame}{Das Tensorprodukt zweier Algebren als Algebra}
	Sei \(A\) ein kommutativer Ring und seien \(\phi\colon A \to B\) und \(\psi\colon A \to C\) zwei
	kommutative \(A\)-Algebren.
	\\
	Wir definieren eine kommutative \(A\)-Algebra \(B \otimes_A C\) wie folgt: Als abelsche Gruppe sei
	\(B \otimes_A C\) die abelsche Gruppe des \(A\)-Moduls \(D = B^A \otimes_A C^A\).
	\\
	Die Multiplikation ist \(B \otimes_A C\) wird durch die eben definierte Abbildung
	\[
		\mu\colon (B \otimes_A C) \times (B \otimes_A C) \to B \otimes_A C, (b \otimes c, b' \otimes c') \mapsto bb' \otimes cc'
	\]
	gegeben.
	\\
	Die Eins ist durch \(1_B \otimes 1_C \in B \otimes_A C\) gegeben.
	\\
	Schließlich ist der Strukturhomomorphismus der \(A\)-Algebra durch \(A \to B \otimes_A C, a \mapsto \phi(a) \otimes
	1 = 1 \otimes \psi(a)\) gegeben.
	\begin{definition}<+->
		Die kommutative \(A\)-Algebra \(B \otimes_A C\) heißt das \emph{Tensorprodukt der kommutativen \(A\)-Algebren \(B\) und \(C\)}.
	\end{definition}
\end{frame}

\begin{frame}{Bemerkung zur Modulstruktur des Tensorproduktes von Algebren}
	\begin{remark}<+->
		Sei \(A\) ein kommutativer Ring. Seien \(B, C\) zwei kommutative \(A\)-Algebren. Als \(A\)-Algebra trägt \(B \otimes_A C\)
		insbesondere die Struktur eines \(A\)-Moduls, nämlich \((B \otimes_A C)^A\). Auf der anderen Seite ist die \(B \otimes_A C\)
		zugrundeliegende abelsche Gruppe in natürlicher Weise ein \(A\)-Modul, nämlich \(B^A \otimes_A C^A\). Beide \(A\)-Modulstrukturen
		stimmen überein, das heißt die identische Abbildung \(\id\colon B^A \otimes_A C^A \to (B \otimes_A C)^A\) ist ein
		Isomorphismus von \(A\)-Moduln.
	\end{remark}
\end{frame}


\mode<all>\section{Gerichtete Limiten}

\subsection{Definition des gerichteten Limes}

\begin{frame}{Gerichtete Mengen}
	\begin{definition}<+->
		Eine \emph{gerichtete Menge} ist eine nicht leere teilweise geordnete Menge
		\(I = (I, \le)\), so daß für jedes Paar von Elementen \(i, j \in I\) ein
		\(k \in I\) mit \(i \le k\) und \(j \le k\) existiert.
	\end{definition}
	\begin{visibleenv}<+->
		Eine teilweise geordnete Menge ist also genau gerichtet, wenn jede endliche
		Teilmenge eine obere Schranke besitzt.
	\end{visibleenv}
\end{frame}

\begin{frame}{Gerichtete Systeme von Moduln}
	Sei \(A\) ein Ring. Sei \(I\) eine gerichtete Menge. Sei weiter \((M_i)_{i \in I}\)
	eine Familie von \(A\)-Moduln. Schließlich sei für jedes Paar \(i, j \in I\) mit
	\(i \le j\) ein Homomorphismus \(\mu^i_j\colon M_i \to M_j\) von \(A\)-Moduln
	gegeben.
	\begin{definition}<+->
		Das Datum \(M_\bullet = (M_i, \mu^i_j)\) heißt ein \emph{gerichtetes System
		von \(A\)-Moduln über \(I\)}, falls folgende Axiome erfüllt sind:
		\begin{enumerate}[<+->]
		\item<.->
			Für alle \(i \in I\) ist \(\mu^i_i = \id_{M_i}\colon M_i \to M_i\).
		\item
			Für alle \(i \le j \le k\) ist \(\mu^i_k = \mu^j_k \circ \mu^i_j\colon M_i \to M_k\).
		\end{enumerate}
	\end{definition}
\end{frame}

\begin{frame}{Der gerichtete Limes}
	Seien \(A\) ein Ring und \(I\) eine gerichtete Menge. Sei \(M_\bullet
	= (M_i, \mu^i_j)\) ein gerichtetes System von \(A\)-Moduln über \(I\).
	\\
	Sei \(C \coloneqq \bigoplus\limits_{i \in I} M_i\). Wir identifizieren die
	\(M_i\) mit ihren kanonischen Bildern in \(C\).
	Sei \(D\) der Untermodul von \(C\), welcher von allen Elementen der Form
	\(x_i - \mu^i_j(x_i)\) mit \(i \le j\) und \(x_i \in M_i\) erzeugt wird.
	\\
	Sei \(\mu\colon C \surjto M \coloneqq C/D\) die kanonische Projektion.
	\\
	Sei schließlich \(\mu^i \coloneqq \mu|M_i\colon M_i \to M\).
	\begin{definition}<+->
		Der Modul \(M\) heißt der \emph{gerichtete Limes von \(M_\bullet\)}. Die
		kanonischen \(A\)-linearen Abbildungen \(\mu^i\colon M_i \to M\) heißen
		die \emph{Strukturhomomorphismen von \(M\)}.
	\end{definition}
	\begin{notation}<+->
		Wir schreiben \(\varinjlim\limits_{i \in I} M_i\) für den
		gerichteten Limes des gerichteten Systems \(M_\bullet\).
	\end{notation}
\end{frame}

\subsection{Universelle Eigenschaft des gerichteten Limes}

\begin{frame}{Elemente im gerichteten Limes}
	Seien \(A\) ein Ring und \(I\) eine gerichtete Menge. Sei
	\(M_\bullet = (M_i, \mu^i_j)\) ein gerichtetes System von \(A\)-Moduln über
	\(I\).
	\\
	Seien die \(\mu^i\colon M_i \to M \coloneqq \varinjlim\limits_{i \in I} M_i\) die
	Strukturhomomorphismen.
	\begin{proposition}<+->
		Sei \(x \in M\).
		Dann existiert ein \(i \in I\) und ein \(x_i \in M_i\)
		mit \(\mu^i(x_i) = x\).
		\qed
	\end{proposition}
	\begin{proposition}<+->
		Seien \(i \in I\) und \(x_i \in M_i\) mit \(\mu^i(x_i) = 0 \in M\).
		Dann existiert ein \(j \ge i\) mit \(\mu^i_j(x_i) = 0 \in M_j\).
	\end{proposition}
\end{frame}

\begin{frame}{Universelle Eigenschaft des gerichteten Limes}
	\mode<presentation>{Seien \(A\) ein Ring und \(I\) eine gerichtete Menge. Sei
	\(M_\bullet = (M_i, \mu^i_j)\) ein gerichtetes System von \(A\)-Moduln über
	\(I\).
	\\
	Seien die \(\mu^i\colon M_i \to M \coloneqq \varinjlim\limits_{i \in I} M_i\) die
	Strukturhomomorphismen.}
	\begin{proposition}<+->
		Sei \(N\) ein \(A\)-Modul. Sei weiter für alle \(i \in I\) eine \(A\)-lineare
		Abbildung \(\alpha^i\colon M_i \to N\) gegeben, so daß \(\alpha^i = \alpha^j \circ \mu^i_j\)
		für alle Paare \(i \le j\). Dann existiert genau eine \(A\)-lineare
		Abbildung \(\alpha\colon M \to N\) mit \(\alpha^i = \alpha \circ \mu^i\)
		für alle \(i \in I\).
		\qed
	\end{proposition}
\end{frame}

\begin{frame}{Moduln als gerichtete Limiten endlich erzeugter}
	\begin{example}<+->
		Sei \(A\) ein Ring. 
		Sei \(M\) ein \(A\)-Modul.
		Sei \((M_i)_{i \in I}\) eine Familie von Untermoduln von \(M\).
		\\
		Für je zwei Elemente \(i, j \in I\) existieren ein \(k \in I\) mit
		\(M_i + M_j \subset M_k\). Durch die Setzung \(i \le j \iff
		M_i \subset M_j\) wird \(I\) zu einer gerichteten Menge.
		\\
		Weiter sei im Falle \(i \le j\) die Abbildung \(\mu^i_j\colon M_i \to M_j\)
		die Inklusionsabbildung.
		\\
		Dann ist
		\[
			\varinjlim\limits_{i \in I} M_i \cong \sum\limits_{i \in I} M_i
			= \bigcup\limits_{i \in I} M_i.
		\]
	\end{example}
	\begin{visibleenv}<+->
		Damit ist insbesondere jeder \(A\)-Modul der gerichtete Limes seiner
		endlich erzeugten Untermoduln.
	\end{visibleenv}
\end{frame}

\subsection{Exakte Sequenzen gerichteter Systeme}

\begin{frame}{Homomorphismen gerichteteter Systeme}
	Seien \(A\) ein Ring und \(I\) eine gerichtete Menge. Seien \(M_\bullet
	= (M_i, \mu^i_j)\) und \(N_\bullet = (N_i, \nu^i_j)\) zwei gerichtete Systeme
	von \(A\)-Moduln über \(I\).
	\\
	Mit \(\mu^i\colon M^i \to M \coloneqq \varinjlim\limits_{i \in I} M_i\)
	und \(\nu^i\colon N^i \to N \coloneqq \varinjlim\limits_{i \in I} N_i\)
	bezeichnen wir die Strukturhomomorphismen der gerichteten Limiten.
	\\
	Sei \(\phi_\bullet = (\phi_i)\) eine Familie von \(A\)-linearen Abbildungen
	\(\phi_i\colon M_i \to N_i\).
	\begin{definition}<+->
		Die Familie \(\phi_\bullet\) heißt ein \emph{Homomorphismus
		\(\phi_\bullet\colon M_\bullet \to N_\bullet\) gerichteter Systeme}, falls
		\(\phi_j \circ \mu_j^i = \nu_j^i \circ \phi_i\colon M_i \to N_j\) für
		alle \(i \le j\).
	\end{definition}
\end{frame}

\begin{frame}{Der gerichtete Limes als Funktor}
	Seien \(A\) ein Ring und \(I\) eine gerichtete Menge. Seien \(M_\bullet
	= (M_i, \mu^i_j)\) und \(N_\bullet = (N_i, \nu^i_j)\) zwei gerichtete Systeme
	von \(A\)-Moduln über \(I\).
	\\
	Mit \(\mu^i\colon M^i \to M \coloneqq \varinjlim\limits_{i \in I} M_i\)
	und \(\nu^i\colon N^i \to N \coloneqq \varinjlim\limits_{i \in I} N_i\)
	bezeichnen wir die Strukturhomomorphismen der gerichteten Limiten.
	\begin{proposition}<+->
		Es existiert genau ein Homomorphismus \(\phi \coloneqq \varinjlim\limits_{i \in I}
		\phi_i\colon M \to N\) von \(A\)-Moduln, so daß
		\(\phi \circ \mu^i = \nu^i \circ \phi_i\colon M_i \to N\) für alle \(i \in I\).
		\qedhere
	\end{proposition}
\end{frame}

\begin{frame}{Exakte Sequenzen gerichteter Systeme}
	\begin{definition}<+->
		Seien \(A\) ein Ring und \(I\) eine gerichtete Menge. Eine Sequenz
		\(M_\bullet \xrightarrow{\phi_\bullet} N_\bullet \xrightarrow{\psi_\bullet}
		P_\bullet\) von gerichteten Systemen von \(A\)-Moduln über \(I\) heißt
		\emph{exakt}, falls die induzierten Sequenzen
		\(M_i \xrightarrow{\phi_i} N_i \xrightarrow{\psi_i} P_i\) für alle
		\(i \in I\) exakt sind.
	\end{definition}
\end{frame}

\begin{frame}{Exaktheit des gerichteten Limes}
	\begin{proposition}<+->
		Seien \(A\) ein Ring und \(I\) eine gerichtete Menge. Sei
		\(M_\bullet \xrightarrow{\phi_\bullet} N_\bullet \xrightarrow{\psi_\bullet}
		P_\bullet\) eine exakte Sequenz von gerichteten Systemen von \(A\)-Moduln über \(I\).
		Dann ist die induzierte Sequenz
		\(\varinjlim\limits_{i \in I} M_i \xrightarrow{\varinjlim\limits_{i \in I} \phi_i}
		\varinjlim\limits_{i \in I} N_i \xrightarrow{\varinjlim\limits_{i \in I} \psi_i}
		\varinjlim\limits_{i \in I} P_i\)
		exakt.
		\qed
	\end{proposition}
\end{frame}

\subsection{Tensorprodukte und gerichtete Limiten}

\begin{frame}{Tensorprodukte und gerichtete Limiten}
	Seien \(A\) ein Ring und \(I\) eine gerichtete Menge. Sei \(N\) ein \(A\)-Modul.
	Sei weiter \(M_\bullet = (M_i, \mu^i_j)\) ein gerichtetes System von \(A\)-Moduln über \(I\).
	\begin{example}<+->
		Zusammen mit den Abbildungen \(\mu^i_j \otimes \id_N\colon M_i \otimes N \to M_j \otimes N\) für
		alle \(i \le j\) wird \(M_\bullet \otimes N = (M_i \otimes N)\) zu einem gerichteten System über \(I\).
	\end{example}
	\begin{visibleenv}<+->
		Seien \(\mu^i\colon M_i \to M \coloneqq \varinjlim\limits_{i \in I} M_i\) 
		und \(\iota^i\colon M_i \otimes N \to P \coloneqq \varinjlim\limits_{i \in I} (M_i \otimes N)\)
		die Strukturhomomorphismen für \(i \in I\).
		\\
		Nach der universellen Eigenschaft des gerichteten Limes induzieren die \(\mu^i \otimes \id_N\colon M_i \otimes N \to
		M \otimes N\) eine eindeutige \(A\)-lineare Abbildung \(\psi\colon P
		\to M \otimes N\) mit \(\mu^i \otimes \id_N = \psi \circ \iota^i\) für alle \(i\).
	\end{visibleenv}
	\begin{proposition}<+->
		Die \(A\)-lineare Abbildung \(\psi\) ist ein Isomorphismus
		\(\varinjlim\limits_{i \in I} (M_i \otimes N) \isoto (\varinjlim\limits_{i \in I} M_i) \otimes N\).
		\qed
	\end{proposition}
\end{frame}

\subsection{Gerichtete Limiten von Ringen}

\begin{frame}{Gerichtete Limiten von Ringen}
	Sei \(I\) eine gerichtete Menge. Sei \((A_i)_{i \in I}\) eine Familie von Ringen über \(I\).
	Für \(i \le j\) sei \(\alpha^i_j\colon A_i \to A_j\) ein Ringhomomorphismus. 
	Die additiven Gruppen der Ringe \(A_i\) mögen ein gerichtetes System von \(\set Z\)-Moduln	
	\(A_\bullet = (A_i, \alpha^i_j)\) bilden.
	Seien die \(\alpha^i\colon A_i \to \varinjlim\limits_{i \in I} A_i\) die Strukturhomomorphismen,
	welche Homomorphismen abelscher Gruppen sind.
	\begin{proposition}<+->
		Auf \(\varinjlim\limits_{i \in I} A_i\) existiert genau eine Struktur eines Ringes, so daß die \(\alpha^i\colon A^i \to A\)
		Ringhomomorphismen werden.
		\qed
	\end{proposition}
	\begin{proposition}<+->
		Ist \(\varinjlim\limits_{i \in I} A_i = 0\), so existiert ein \(i \in I\) mit \(A_i = 0\).
		\qed
	\end{proposition}
\end{frame}

\begin{frame}{Tensorprodukte über beliebige Familien}
	Sei \(A\) ein kommutativer Ring. Sei \((B_i)_{i \in I}\) eine Familie kommutativer \(A\)-Algebren. Ist \(J \subset I\) eine
	endliche Teilmenge, so sei \(B_J = \bigotimes_{j \in J} B_j\).
	\\
	Ist \(K \subset I\) eine weitere endliche Teilmenge mit \(K \subset J\), so sei \(\beta^K_J\colon B_K \to B_J\) der kanonische
	Homomorphismus von \(A\)-Algebren, welcher \(\bigotimes\limits_{k \in K} b_k\) auf \(\bigotimes\limits_{j \in J} b_j\) abbildet,
	wobei wir \(b_j = 1\) für \(j \notin K\) setzen.
	\\
	Der gerichtete Limes \(B \coloneqq \bigotimes\limits_{i \in I} B_i \coloneqq \varinjlim_{J \subset I} B_J\) ist in kanonischer Weise
	eine \(A\)-
	Algebra, so daß die
	Strukturmorphismen \(B_J \to B\) Morphismen von \(A\)-Algebren werden.
	\begin{definition}<+->
		Die kommutative \(A\)-Algebra \(\bigotimes\limits_{i \in I} B_i\) heißt das \emph{Tensorprodukt über die Familie \((B_i)\)}.
	\end{definition}
\end{frame}



\lecture{Lokalisierungen von Ringen und Moduln}{Lokalisierung}
\part<article>{Lokalisierungen von Ringen und Moduln}
\mode<all>\setcounter{section}{19}
\mode<all>\section{Lokalisierungen von Ringen und Moduln}

\subsection{Lokalisierung eines Ringes}

\begin{frame}{Multiplikativ abgeschlossene Teilmengen}
	\begin{definition}<+->
		Sei \(A\) ein Ring. Eine \emph{multiplikativ abgeschlossene Teilmenge von \(A\)} ist eine Teilmenge \(S \subset A\)
		mit \(1 \in S\) und \(x y \in S\) für alle \(x, y \in S\).
	\end{definition}
	\begin{visibleenv}<+->
		Eine Teilmenge \(S \subset A\) ist also genau dann multiplikativ abgeschlossen, wenn sie ein Untermonoid des
		multiplikativen Monoides von \(A\) ist.
	\end{visibleenv}
	\begin{example}<+->
		Sei \(A\) ein kommutativer Ring. Dann ist \(A \setminus \{0\}\) genau dann multiplikativ abgeschlossen,
		wenn \(A\) ein Integritätsbereich ist.
	\end{example}
	\begin{example}<+->
		Sei \(\ideal a\) ein Ideal \(A\). Dann ist \(1 + \ideal a = \{1 + x \mid x \in \ideal a\} \subset A\) multiplikativ
		abgeschlossen.
	\end{example}
\end{frame}

\begin{frame}{Kürzungsregel}
	Sei \(A\) ein kommutativer Ring. Sei \(S \subset A\) multiplikativ abgeschlossen. Wir betrachten die
	Menge der Paare \((a, s) \in A \times S\).
	\\
	Wir nennen zwei Paare \((a, s), (b, t)\) \emph{äquivalent}, wenn \((at - bs) u = 0\) für ein \(u \in S\).
	\begin{proposition}<+->
		Die so definierte Relation ist eine Äquivalenzrelation.
	\end{proposition}
	\begin{proof}<+->
		\begin{enumerate}[<+->]
		\item<.->
			Die Relation ist offensichtlich reflexiv und symmetrisch.
		\item
			Sei \((a, s)\) äquivalent zu \((b, t)\) und \((b, t)\) äquivalent zu \((c, u)\). Damit
			existieren \(v, w \in S\) mit \((a t - b s) v = (b u - c t) w = 0\).
		\item
			Multiplizieren wir die linke Gleichung mit \(uw\) und die rechte mit \(sv\), können wir \(b\)
			eliminieren und erhalten \((au - cs) tvw = 0\).
		\item
			Da \(S\) multiplikativ abgeschlossen ist, ist \(tvw \in S\). Damit sind \((a, s)\) und \((c, u)\)
			äquivalent.
			\qedhere
		\end{enumerate}
	\end{proof}
\end{frame}

\begin{frame}{Brüche}
	Sei \(A\) ein kommutativer Ring. Sei \(S \subset A\) multiplikativ abgeschlossen.
	\\Die Äquivalenzklasse
	\((a, s) \in A \times S\) eines Paares nennen wir einen \emph{Bruch}. Wir schreiben die Äquivalenzklasse
	dieses Paares als \(\frac a s\).
	\\
	Mit \(S^{-1} A\) bezeichnen wir die Menge aller dieser Brüche. Auf dieser Menge definieren wir eine Addition
	durch
	\[
		\frac a s + \frac b t = \frac{at + bs}{st}
	\]
	und eine Multiplikation durch
	\[
		\frac a s \frac b t = \frac{ab}{st}.
	\]
	\begin{proposition}<+->
		Mit den so definierten Operationen wird \(S^{-1} A\) zu einem wohldefinierten kommutativen Ring.
		\qed
	\end{proposition}
\end{frame}

\begin{frame}{Die Lokalisierung eines kommutativen Ringes}
	Sei \(A\) ein kommutativer Ring. Sei \(S\) eine multiplikativ abgeschlossene Teilmenge.
	\\
	Die Abbildung \(\iota\colon A \to S^{-1} A, a \mapsto \frac a 1\) ist ein Ringhomomorphismus.
	\begin{definition}<+->
		Der kommutative Ring \(S^{-1} A\) heißt die \emph{Lokalisierung von \(A\) nach \(S\)} und
		\(\iota\colon A \to S^{-1} A\) ihr \emph{Strukturhomomorphismus}.
	\end{definition}
	\begin{proposition}[Universelle Eigenschaft der Lokalisierung]<+->
		Sei \(\phi\colon A \to B\) ein Ringhomomorphismus, so daß \(\phi(s) \in B^\units\) für alle \(s \in S\).
		Dann existiert genau ein Ringhomomorphismus \(\psi\colon S^{-1} A \to B\) mit \(\phi = \psi \circ \iota\).
	\end{proposition}
\end{frame}

\begin{frame}{Beweis der universellen Eigenschaft der Lokalisierung}
	\begin{proof}<+->
		\begin{enumerate}[<+->]
		\item<.->
			Eindeutigkeit: Sei \(\psi\colon S^{-1} A \to B\) mit \(\psi(\frac a 1) = \phi(a)\) für alle \(a \in A\). 
			Ist weiter \(s \in S\), so gilt dann 
			\(\psi(\frac a s) = \psi(\frac a 1 (\frac s 1)^{-1}) = \psi(\frac a 1) \psi(\frac s 1)^{-1}
			= \phi(a) \phi(s)^{-1}\).
		\item
			Existenz: Es ist zu zeigen, daß \(\psi\colon S^{-1} A \to B, \frac a s \mapsto \phi(a) \phi(s)^{-1}\) wohldefiniert
			ist. Dazu sei \(\frac a s = \frac{a'}{s'}\), also \((a s' - a' s) u = 0\) für ein \(u \in S\).
		\item
			Es folgt \((\phi(a) \phi(s') - \phi(a') \phi(s)) \phi(u) = 0\).
		\item
			Da \(\phi(u) \in B^\units\), folgt \(\phi(a) \phi(s)^{-1} = \phi(a') \phi(s')^{-1}\).
			\qedhere
		\end{enumerate}
	\end{proof}
\end{frame}

\subsection{Eigenschaften der Lokalisierung}

\begin{frame}{Eigenschaften der Lokalisierung}
	\begin{proposition}<+->
		Sei \(A\) ein kommutativer Ring. Sei \(S \subset A\) multiplikativ abgeschlossen. Sei \(\iota\colon A \to S^{-1} A\)
		der Strukturhomomorphismus. Dann gilt:
		\begin{enumerate}[<+->]
		\item<.->
			Für alle \(s \in S\) ist \(\iota(s)\) eine Einheit in \(S^{-1} A\).
		\item
			Ist \(a \in A\) mit \(\iota(a) = 0\), so ist \(a s = 0\) für ein \(s \in S\).
		\item
			Jedes Element in \(S^{-1} A\) ist von der Form \(\iota(a) \iota(s)^{-1}\) mit \(a \in A\) und \(s \in S\).
			\qed
		\end{enumerate}
	\end{proposition}
\end{frame}

\begin{frame}{Charakterisierung der Lokalisierung}
	\begin{corollary}[Universelle Eigenschaft der Lokalisierung]
		Sei \(A\) ein kommutativer Ring. Sei \(S \subset A\) multiplikativ abgeschlossen.
		Sei \(\iota\colon A \to S^{-1} A\) der Strukturhomomorphismus.
		Sei \(\phi\colon A \to B\) ein Homomorphismus kommutativer Ringe mit folgenden Eigenschaften:
		\begin{enumerate}[<+->]
		\item
			Für alle \(s \in S\) ist \(\phi(s)\) eine Einheit in \(B\).
		\item
			Ist \(a \in A\) mit \(\phi(a) = 0\), so ist \(a s = 0\) für ein \(s \in S\).
		\item
			Jedes Element von \(B\) ist von der Form \(\phi(a) \phi(s)^{-1}\) mit \(a \in A\) und \(s \in S\).
		\end{enumerate}
		\begin{visibleenv}<+->
			Dann existiert ein eindeutiger Isomorphismus \(\psi\colon S^{-1} A \to B\) mit \(\phi = \psi \circ \iota\).
		\end{visibleenv}
	\end{corollary}
\end{frame}

\begin{frame}{Beweis der Folgerung aus der universellen Eigenschaft der Lokalisierung}
	\begin{proof}<+->
		\begin{enumerate}[<+->]
		\item<.->
			Es ist zu zeigen, daß der aufgrund der ersten Eigenschaft wohldefinierte Ringhomomorphismus
			\(\psi\colon S^{-1} A \to B, \frac a s \mapsto \phi(a)\phi(s)^{-1}\) ein Isomorphismus ist.
		\item
			Da jedes Element in \(B\) von der Form \(\phi(a)\phi(s)^{-1}\) ist, ist \(\psi\) offensichtlich surjektiv.
		\item
			Sei schließlich \(\frac a s \in S^{-1} A\) im Kern von \(\psi\). Damit ist insbesondere \(\phi(a) = 0\), also
			\(a t = 0\) für ein \(t \in S\). Es folgt, daß \(\frac a s = \frac 0 1 = 0 \in S^{-1} A\). Also ist \(\psi\) auch
			injektiv.
			\qedhere
		\end{enumerate}
	\end{proof}
\end{frame}

\subsection{Beispiele von Lokalisierungen}

\begin{frame}{Lokalisierungen an Primidealen}
	Sei \(A\) ein kommutativer Ring. Sei \(\ideal p\) ein Primideal in \(A\).
	\begin{example}<+->
		Die Teilmenge \(A \setminus \ideal p \subset A\) ist multiplikativ abgeschlossen.
	\end{example}
	\begin{notation}<+->
		Wir schreiben \(A_{\ideal p} \coloneqq (A \setminus \ideal p)^{-1} A\) und nennen \(A_{\ideal p}\) die
		\emph{Lokalisierung von \(A\) bei \(\ideal p\)} oder den \emph{Halm von \(A\) an \(\ideal p\)}.
	\end{notation}
	\begin{example}<+->
		Es ist \(\ideal m \coloneqq \{\frac a s \mid a \in \ideal p, s \in A \setminus \ideal p\}\) ein echtes
		Ideal im Halm \(A_{\ideal p}\).
		\\
		Ist \(\frac b t \in A_{\ideal p} \setminus \ideal m\), so ist \(b \in A \setminus \ideal p\),
		also \(\frac b t \in (A_{\ideal p})^\units\).
		\\
		Es folgt, daß \(A_{\ideal p}\) ein lokaler Ring mit maximalem Ideal \(\ideal m\) ist.
		\\
		Es ist \(\ideal m = A_{\ideal p} \ideal p\).
	\end{example}
\end{frame}

\begin{frame}{Der Quotientenkörper}
	\begin{example}<+->
		Sei \(A\) ein Integritätsbereich. Dann ist \(S \coloneqq A \setminus \{0\}\) multiplikativ
		abgeschlossen, so daß wir die Lokalisierung \(S^{-1} A\) bilden können. Da \(S\) nur reguläre Elemente (und zwar
		alle) enthält, ist \(A \to S^{-1} A, a \mapsto \frac a 1\) ein injektiver Ringhomomorphismus, so daß
		wir \(A\) als Unterring von \(S^{-1} A\) auffassen können.
		\\
		Ist \(\frac p q \in S^{-1} A \setminus \{0\}\), so ist \((\frac p q)^{-1} = \frac q p\). Damit ist \(S^{-1} A\) ein
		Körper.
		\\
		Wir nennen \(S^{-1} A\) den \emph{Quotientenkörper von \(A\)}. Es ist der kleinste Körper, welcher \(A\) als
		Unterring enthält.
	\end{example}
	\begin{example}<+->
		Der Körper \(\set Q\) der rationalen Zahlen ist der Quotientenkörper des Ringes \(\set Z\) der ganzen Zahlen.
	\end{example}
\end{frame}

\begin{frame}{Lokalisierung außerhalb von Funktionen}
	Sei \(A\) ein kommutativer Ring.
	\begin{example}<+->
		Sei \(S \subset A\) eine multiplikativ abgeschlossene Teilmenge. Dann gilt \(S^{-1} A = 0\) genau dann, wenn
		\(0 \in S\).
	\end{example}
	\begin{example}<+->
		Sei \(f \in A\). Dann ist \(\{f^n \mid n \in \set N_0\}\) eine multiplikativ abgeschlossene Teilmenge.
	\end{example}
	\begin{notation}<+->
		Wir schreiben \(A[f^{-1}] \coloneqq \{f^n \mid n \in \set N_0\}^{-1} A\) und nennen \(A[f^{-1}]\) die
		\emph{Lokalisierung von \(A\) außerhalb von \(f\)}.
	\end{notation}
\end{frame}

\begin{frame}{Spezielle Lokalisierungen}
	\begin{example}<+->
		Sei \(p\) eine Primzahl. Dann ist \(\set Z_{(p)}\) die Menge aller rationalen Zahlen \(\frac m n\), wobei
		\(n\) teilerfremd zu \(p\) ist.
	\end{example}
	\begin{example}<+->
		Sei \(f \in \set Z \setminus \{0\}\). Dann ist \(\set Z[f^{-1}]\) die Menge der rationalen Zahlen der Form
		\(\frac m {f^n}\), wobei \(n \in \set N_0\).
	\end{example}
	\begin{example}<+->
		Sei \(A \coloneqq K[x_1, \dotsc, x_n]\) der Polynomring in \(n\) Variablen über einem Körper \(K\). Für jedes
		Primideal \(\ideal p\) in \(A\) ist dann \(A_\ideal p\) der Ring derjenigen rationalen Funktionen \(\frac f g\)
		in den \(x_i\) mit \(g \notin \ideal p\).
	\end{example}
\end{frame}

\subsection{Lokalisierung von Moduln}

\begin{frame}{Konstruktion der Lokalisierung eines Moduls}
	Sei \(A\) ein kommutativer Ring. Seien \(S \subset A\) multiplikativ abgeschlossen und
	\(M\) ein \(A\)-Modul.
	\\
	Wir nennen zwei Paare \((m, s), (m', s') \in M \times S\) \emph{äquivalent}, wenn \((ms' - m's) u = 0\) für
	ein \(u \in S\).
	\begin{proposition}<+->
		Die so definierte Relation ist eine Äquivalenzrelation.
		\qed
	\end{proposition}
	\begin{visibleenv}<+->
		Die Menge der Äquivalenzklasse \(\frac m s\) der Paare \((m, s)\) bezeichnen wir mit \(S^{-1} M\). 
	\end{visibleenv}
	\begin{proposition}<+->
		Mit der offensichtlichen Definition der Modulstruktur wird \(S^{-1} M\) zu einem \(S^{-1} A\)-Modul.
		\qed
	\end{proposition}
\end{frame}

\begin{frame}{Definition der Lokalisierung eines Moduls}
	Sei \(A\) ein kommutativer Ring. Seien \(S \subset A\) multiplikativ abgeschlossen und
	\(M\) ein \(A\)-Modul.
	\\
	Vermöge des Strukturhomomorphismus \(A \to S^{-1} A\) fassen wir \(S^{-1} A\) als \(A\)-Algebra auf.
	\\
	Die Abbildung \(\iota\colon M \to (S^{-1} M)^A, m \mapsto \frac m 1\)
	ist ein Homomomorphismus von \(A\)-Moduln.
	\begin{definition}<+->
		Der \(S^{-1} A\)-Modul \(S^{-1} M\) heißt die \emph{Lokalisierung von \(M\) nach \(S\)} und
		\(\iota\colon M \to (S^{-1} M)^A\) sein \emph{Strukturhomomorphismus}.
	\end{definition}
\end{frame}

\begin{frame}{Spezielle Lokalisierungen von Moduln}
	Sei \(A\) ein kommutativer Ring. Sei \(M\) ein \(A\)-Modul. 
	\begin{notation}<+->
		Sei \(\ideal p\) ein Primideal in \(A\).
		Wir schreiben \(M_\ideal p \coloneqq (A \setminus \ideal p)^{-1} M\) und nennen den \(A_{\ideal p}\)-Modul
		\(M_{\ideal p}\) die \emph{Lokalisierung von \(M\) bei \(\ideal p\)} oder den \emph{Halm von \(M\)
		an \(\ideal p\)}.

		Das Bild eines Schnittes \(m \in M\) in \(M_{\ideal p}\) unter dem Strukturhomomorphismus \(M \to M_{\ideal p}\)
		heißt der \emph{Keim von \(m\) an \(\ideal p\)}.
	\end{notation}
	\begin{notation}<+->
		Sei \(f \in A\). Wir schreiben \(M[f^{-1}] \coloneqq \{f^n \mid n \in \set N_0\}^{-1} M\) und nennen
		\(M[f^{-1}]\) die \emph{Lokalisierung von \(M\) außerhalb von \(f\)}.
		
		Das Bild eines Schnittes \(m \in M\) in \(M[f^{-1}]\) unter dem Strukturhomomorphismus \(M \to M[f^{-1}]\)
		heißt die \emph{Einschränkung von \(m\) außerhalb von \(f\)}.
	\end{notation}
\end{frame}

\subsection{Exaktheit der Lokalisierung}

\begin{frame}{Funktorialität der Lokalisierung}
	\begin{visibleenv}<+->
		Sei \(A\) ein kommutativer Ring. Seien \(S \subset A\) multiplikativ abgeschlossen und \(\phi\colon M \to N\)
		ein Homomorphismus von \(A\)-Moduln.
		\\
		Dann ist
		\[
			S^{-1}\phi\colon S^{-1} M \to S^{-1} N, \frac m s \mapsto \frac{\phi(m)} s
		\]
		ein Homomorphismus von \(S^{-1} A\)-Moduln.
	\end{visibleenv}
	\begin{proposition}<+->
		Sei \(\psi\colon N \to P\) ein weiterer Homomorphismus von \(A\)-Moduln. Dann ist
		\(S^{-1} (\psi \circ \phi) = S^{-1} \psi \circ S^{-1} \phi\colon S^{-1} M \to S^{-1} P\).
		\qed
	\end{proposition}
\end{frame}

\begin{frame}{Exaktheit der Lokalisierung}
	\begin{proposition}<+->
		Sei \(A\) ein kommutativer Ring. Sei \(S \subset A\) multiplikativ abgeschlossen. Ist \(M' \xrightarrow{\phi} M
		\xrightarrow{\psi} M''\) exakt bei \(M\), so ist auch \(S^{-1} M' \xrightarrow{S^{-1}\phi} S^{-1} M \xrightarrow{S^{-1} \psi}
		S^{-1} M''\) exakt.
	\end{proposition}
	\begin{proof}<+->
		\begin{enumerate}[<+->]
		\item<.->
			Wegen \(\psi \circ \phi = 0\) ist auch \((S^{-1} \psi) \circ (S^{-1} \phi) = S^{-1} 0 = 0\), also
			\(\im S^{-1} \phi \subset \ker S^{-1} \psi\).
		\item
			Sei \(\frac m s \in \ker S^{-1} \psi\), also \(\frac{\psi(m)} s = 0 \in S^{-1} M''\). Damit existiert ein
			\(t \in S\) mit \(\psi(t m) = t \psi(m) = 0 \in M''\).
		\item
			Damit existiert ein \(m' \in M'\) mit \(\phi(m') = t m\). Es folgt, daß \(\frac m s = \frac{\phi(m')}{st}
			= (S^{-1} \phi)(\frac{m'}{st})\).
			\qedhere
		\end{enumerate}
	\end{proof}
\end{frame}

\begin{frame}{Lokalisierung von Untermoduln}
	\begin{example}<+->
		Sei \(A\) ein kommutativer Ring. Sei \(S\) multiplikativ abgeschlossen. Sei
		\(M'\) ein Untermodul eines \(A\)-Moduls \(M\).
		\\
		Da die Lokalisierung exakt ist, ist die Lokalisierung \(S^{-1} M' \to S^{-1} M\) der
		Inklusion \(M' \injto M\) wieder injektiv.
		\\
		Damit können wir \(S^{-1} M'\) als Untermodul von \(S^{-1} M\) ansehen.
	\end{example}
\end{frame}

\begin{frame}{Lokalisierung kommutiert mit endlichen Summen und Quotienten}
	Sei \(A\) ein kommutativer Ring. Seien \(S \subset A\) multiplikativ abgeschlossen und
	\(M\) ein \(A\)-Modul. Seien \(N, P \subset M\) Untermoduln.
	\begin{corollary}<+->
		Es gilt \(S^{-1} (N + P) = S^{-1} N + S^{-1} P \subset S^{-1} M\).
	\end{corollary}
	\begin{proof}<+->
		Folgt sofort aus den Definitionen.
	\end{proof}
	\begin{corollary}<+->
		Die \(S^{-1} A\)-Moduln \(S^{-1}(M/N)\) und \((S^{-1} M)/(S^{-1} N)\) sind isomorph.
	\end{corollary}
	\begin{proof}<+->
		Aus der Exaktheit von \(0 \to N \to M \to M/N \to 0\) folgt die
		Exaktheit von \(0 \to S^{-1} N \to S^{-1} M \to S^{-1}(M/N) \to 0\).
	\end{proof}
\end{frame}

\begin{frame}{Lokalisierung kommutiert mit endlichen Schnitten}
	\begin{proposition}<+->
		Sei \(A\) ein kommutativer Ring. Seien \(S \subset A\) multiplikativ abgeschlossen und
		\(M\) ein \(A\)-Modul. Seien \(N, P \subset M\) Untermoduln.
		Es gilt \(S^{-1} (N \cap P) = S^{-1} N \cap S^{-1} P \subset S^{-1} M\).
	\end{proposition}
	\begin{proof}<+->
		\begin{enumerate}[<+->]
		\item<.->
			Die Inklusion \(S^{-1} (N \cap P) \subset S^{-1} N \cap S^{-1} P\) ist offensichtlich.
		\item
			Seien umgekehrt \(\frac y s = \frac z t\) mit \(y \in N, z \in P\) und \(s, t \in S\). Dann
			ist \(u (t y - s z) = 0\) für ein \(u \in S\). Es folgt, daß  \(w \coloneqq uty = usz \in N \cap P\).
			Damit ist \(\frac y s = \frac w{stu} \in S^{-1} (N \cap P)\).
			\qedhere
		\end{enumerate}
	\end{proof}
\end{frame}

\subsection{Lokalisierung als Basiswechsel}

\begin{frame}{Lokalisierung als Basiswechsel}
	\begin{proposition}<+->
		Sei \(A\) ein kommutativer Ring. Seien \(S \subset A\) multiplikativ abgeschlossen und
		\(M\) ein \(A\)-Modul. Dann existiert ein eindeutiger Isomorphismus
		\(\phi \colon S^{-1} A \otimes_A M \to S^{-1} M, \frac a s \otimes m \mapsto \frac{am} s\).
	\end{proposition}
	\begin{proof}<+->
		\begin{enumerate}[<+->]
		\item<.->
			Da \(\frac{am}{s}\) bilinear in \(\frac a s\) und \(m\) ist, ist \(\phi\) nach der universellen
			Eigenschaft des Tensorproduktes wohldefiniert und eindeutig.
		\item
			Die Surjektivität von \(\phi\) ist offensichtlich.
		\item
			Sei \(\sum_i \frac{a_i}{s_i} \otimes m_i \in S^{-1} A \otimes M\) ein beliebiges Element.
			Mit \(s \coloneqq \prod_i s_i \in S\) und \(t_i \coloneqq \prod_{j \neq i} s_j \in S\) haben wir
			\(\sum_i \frac{a_i}{s_i} \otimes m_i = \frac 1 s \otimes \sum_i a_i t_i m_i\), jedes Element von
			\(S^{-1} A \otimes M\) ist also von der Form \(\frac 1 s \otimes m\).
		\item
			Sei \(\phi(\frac 1 s \otimes m) = 0\). Dann ist \(\frac m s = 0 \in S^{-1} M\), also \(t m = 0 \in M\) für ein \(t \in S\).
			Damit ist \(\frac 1 s \otimes m = \frac 1 {st} \otimes t m = 0\). Also ist
			\(\phi\) injektiv.
		\qedhere
		\end{enumerate}
	\end{proof}
\end{frame}

\begin{frame}{Flachheit der Lokalisierung}
	\begin{proposition}<+->
		Sei \(A\) ein kommutativer Ring. Sei \(S \subset A\) multiplikativ abgeschlossen. Dann ist
		\(S^{-1} A\) eine flache \(A\)-Algebra.
	\end{proposition}
	\begin{proof}<+->
		\begin{enumerate}[<+->]
		\item<.->
			Tensorieren eines \(A\)-Moduls mit \(S^{-1} A\) über \(A\) entspricht Lokalisieren dieses \(A\)-Moduls an \(S\).
		\item
			Lokalisieren ist exakt.
			\qedhere
		\end{enumerate}
	\end{proof}
\end{frame}

\begin{frame}{Tensorprodukte von Lokalisierungen}
	Sei \(A\) ein kommutativer Ring. Seien \(M, N\) zwei \(A\)-Moduln. 
	\begin{proposition}<+->
		Sei \(S \subset A\) multiplikativ abgeschlossen. Dann existiert ein eindeutiger Isomorphismus
		\(\phi\colon S^{-1} M \otimes_{S^{-1} A} S^{-1} N \to S^{-1} (M \otimes_A N),
		\frac m s \otimes \frac n t \mapsto \frac{m \otimes n}{st}\)
		von \(S^{-1} A\)-Moduln.
	\end{proposition}
	\begin{proof}<+->
		Wir haben die Isomorphismen
		\(S^{-1} M \otimes_{S^{-1} A} S^{-1} N \isoto (S^{-1} A \otimes_A M) \otimes_{S^{-1} A} (S^{-1} A \otimes_A N)
		\isoto S^{-1} A \otimes_A (M \otimes_{S^{-1} A} S^{-1} A \otimes_A N)
		\isoto S^{-1} A \otimes_A (M \otimes_A N) \isoto S^{-1} (M \otimes_A N)\).
	\end{proof}
	\begin{example}<+->
		Sei \(\ideal p\) ein Primideal in \(A\). Dann gilt für die Halme \(M_{\ideal p} \otimes_{A_{\ideal p}}
		N_{\ideal p} = (M \otimes_A N)_{\ideal p}\).
	\end{example}
\end{frame}



\lecture{Lokale Eigenschaften}{Lokale Eigenschaften}
\mode<all>\setcounter{section}{20}
\mode<all>\section{Lokale Eigenschaften}

\subsection{Trivialität von Moduln}

\begin{frame}{Lokale Eigenschaften}
	Wir nennen eine Eigenschaft kommutativer Ringe (bzw.~Moduln über einem kommutativen Ring) \emph{lokal},
	falls folgendes gilt:
	
	Ein kommutativer Ring \(A\) (bzw.~ein Modul \(M\) über einem kommutativen Ring) hat die Eigenschaft genau dann, wenn
	alle seine Halme \(A_{\ideal p}\) (bzw.~\(M_{\ideal p}\)) an allen Primidealen \(\ideal p\) die Eigenschaft hat.
\end{frame}

\begin{frame}{Trivialität von Moduln}
	\begin{proposition}<+->
		Sei \(A\) ein kommutativer Ring. Sei \(M\) ein \(A\)-Modul. Dann sind folgende Aussagen äquivalent:
		\begin{enumerate}[<+->]
		\item<.->
			\(M = 0\).
		\item
			\(M_{\ideal p} = 0\) für alle Primideale \(\ideal p\) von \(A\).
		\item
			\(M_{\ideal m} = 0\) für alle maximalen Ideale \(\ideal m\) von \(A\).
		\end{enumerate}
	\end{proposition}
	\begin{proof}<+->
		\begin{enumerate}[<+->]
		\item<.->
			Aus der ersten folgt sicherlich die zweite Aussage und aus der zweiten die dritte. 
		\item
			Sei \(M_{\ideal m} = 0\) für alle maximalen Ideale \(\ideal m\) von \(A\). Sei \(x \in M\).
			Angenommen \(x \neq 0\).
			Dann ist das Ideal \(\ann(x)\) in einem maximalen Ideal \(\ideal m\) von
			\(A\) enthalten.
		\item
			Es ist \(\frac x 1 = 0 \in M_{\ideal m}\), also \(t x = 0\) für ein \(t \in A \setminus \ideal m\).
			Damit ist \(t \notin \ann(x)\), Widerspruch.
		\qedhere
		\end{enumerate}
	\end{proof}
\end{frame}

\subsection{Injektivität und Surjektivität}

\begin{frame}{Injektivität ist eine lokale Eigenschaft}
	\begin{proposition}<+->
		\label{prop:inj_is_local}
		Sei \(A\) ein kommutativer Ring. Sei \(\phi\colon M \to N\) ein
		Homomorphismus von \(A\)-Moduln. Dann
		sind folgende Aussagen äquivalent:
		\begin{enumerate}[<+->]
		\item<.->
			\(\phi\colon M \to N\) ist injektiv.
		\item
			\(\phi_{\ideal p}\colon M_{\ideal p} \to N_{\ideal p}\) ist injektiv für alle Primideale \(\ideal p\) von \(A\).
		\item
			\(\phi_{\ideal m}\colon M_{\ideal m} \to N_{\ideal m}\) ist injektiv für alle maximalen Ideale \(\ideal m\) von \(A\).
		\end{enumerate}
	\end{proposition}
	\begin{proof}<+->
		\begin{enumerate}[<+->]
		\item<.->
			Lokalisierung erhält Injektivität.
		\item
			Jedes maximale Ideal ist ein Primideal.
		\item
			Sei \(\phi_{\ideal m}\) für alle maximalen Ideale \(\ideal m\) injektiv. Sei \(M' = \ker \phi\).
			Dann ist \(0 \to M' \to M \to N\) exakt, also auch \(0 \to M'_{\ideal m} \to M_{\ideal m} \to N_{\ideal m}\).
			Damit ist \(M'_{\ideal m} = \ker \phi_{\ideal m} = 0\). Es folgt nach der letzten Aussage, daß \(M' = 0\).
			Also ist \(\phi\) injektiv.
			\qedhere
		\end{enumerate}
	\end{proof}
\end{frame}

\begin{frame}{Surjektivität ist eine lokale Eigenschaft}
	\begin{proposition}<+->
		Sei \(A\) ein kommutativer Ring. Sei \(\phi\colon M \to N\) ein Homomorphismus von \(A\)-Moduln. Dann
		sind folgende Aussagen äquivalent:
		\begin{enumerate}[<+->]
		\item<.->
			\(\phi\colon M \to N\) ist surjektiv.
		\item
			\(\phi_{\ideal p}\colon M_{\ideal p} \to N_{\ideal p}\) ist surjektiv für alle Primideale \(\ideal p\) von \(A\).
		\item
			\(\phi_{\ideal m}\colon M_{\ideal m} \to N_{\ideal m}\) ist surjektiv für alle maximalen Ideale \(\ideal m\) von \(A\).
			\qed
		\end{enumerate}
	\end{proposition}
\end{frame}

\subsection{Flachheit}

\begin{frame}{Flachheit ist eine lokale Eigenschaft}
	\begin{proposition}<+->
		Sei \(A\) ein kommutativer Ring. Sei \(M\) ein \(A\)-Modul. Dann sind folgende Aussagen äquivalent:
		\begin{enumerate}[<+->]
		\item<.->
			\(M\) ist ein flacher \(A\)-Modul.
		\item
			\(M_{\ideal p}\) ist ein flacher \(A_{\ideal p}\)-Modul für alle Primideale \(\ideal p\) von \(A\).
		\item
			\(M_{\ideal m}\) ist ein flacher \(A_{\ideal m}\)-Modul für alle maximalen Ideale \(\ideal m\) von \(A\).
		\end{enumerate}
	\end{proposition}
	\begin{proof}<+->
		\begin{enumerate}[<+->]
		\item<.->
			\(M_{\ideal p} = A_{\ideal p} \otimes_A M\) und
			Skalarerweiterung erhält Flachheit.
		\item
			Jedes maximale Ideal ist ein Primideal.
		\item
			Sei \(M_{\ideal m}\) flach über \(A_{\ideal m}\) für alle maximalen Ideale \(\ideal m\). Sei \(N \to P\) ein
			injektiver Homomorphismus von \(A\)-Moduln. Es folgt, daß \(N_{\ideal m} \to P_{\ideal m}\) injektiv ist.
			Nach Voraussetzung ist dann \((N \otimes_A M)_{\ideal m} = N_{\ideal m} \otimes_{A_{\ideal m}} M_{\ideal m} \to
			P_{\ideal m} \otimes_{A_{\ideal m}} M_{\ideal m} = (P \otimes_A M)_{\ideal m}\) injektiv. 
			Da \(\ideal m\) beliebig ist, folgt, daß \(N \otimes_A M \to P \otimes_A M\) injektiv ist.
			\qedhere
		\end{enumerate}
	\end{proof}
\end{frame}


\mode<all>\section{Idealerweiterungen und -kontraktionen in Lokalisierungen}

\subsection{Erweiterungen und Kontraktionen}

\begin{frame}{Erweiterung in einer Lokalisierung}
	Sei \(A\) ein kommutativer Ring. Sei \(S \subset A\) multiplikativ abgeschlossen.
	\begin{example}<+->
		Sei \(\ideal a\) ein Ideal in \(A\).
		Dann gilt für die Erweiterung \((S^{-1} A) \ideal a = S^{-1} \ideal a\), denn jedes Element in \((S^{-1} A) \ideal a\) ist
		von der Form \(\sum_i \frac {a_i}{s_i}\) mit \(a_i \in \ideal a\) und \(s_i \in S\), und Ausdrücke dieser Form
		können wir auf einen gemeinsamen Nenner bringen.
	\end{example}
	\begin{proposition}<+->
		Alle Ideale in \(S^{-1} A\) sind erweiterte Ideale, also von der Form \(S^{-1} \ideal a\).
	\end{proposition}
	\begin{proof}<+->
		Sei \(\ideal b\) ein Ideal in \(S^{-1} A\). Sei \(\frac x s \in \ideal b\). Dann ist \(\frac x 1\) in \(\ideal b\),
		also \(x \in A \cap \ideal b\). Damit folgt \(\frac x s \in S^{-1}(A \cap \ideal b)\).
	\end{proof}
\end{frame}

\begin{frame}{Kontraktionen von Erweiterungen in einer Lokalisierung}
	Sei \(A\) ein kommutativer Ring. Seien \(S \subset A\) multiplikativ abgeschlossen und \(\ideal a\) ein Ideal in \(A\).
	\begin{proposition}<+->
		Es ist \(A \cap (S^{-1} \ideal a) = \bigcup\limits_{u \in S} (\ideal a : u)\). 
	\end{proposition}
	\begin{proof}<+->
		\(x \in A \cap (S^{-1} \ideal a) \iff \exists a \in \ideal a \exists s \in S \colon \frac x 1 = \frac a s
		\iff \exists a \in \ideal a \exists s, t \in S \colon (x s - a) t = 0 \iff \exists u \in S\colon
		ux \in \ideal a \iff x \in \bigcup\limits_{u \in S} (\ideal a : u)\).
	\end{proof}	
	\begin{example}<+->
		Es \(S^{-1} \ideal a = (1)\) genau dann, wenn \(\ideal a \cap S \neq \emptyset\).
	\end{example}
\end{frame}
	
\begin{frame}{Kontrahierte Ideale und Lokalisierungen}
	\begin{proposition}<+->
		Sei \(A\) ein kommutativer Ring. Seien \(S \subset A\) multiplikativ abgeschlossen und \(\ideal a\) ein Ideal in \(A\).
		Es ist \(\ideal a\) genau dann ein kontrahiertes Ideal bezüglich \(A \to S^{-1} A\), also ein Ideal der Form
		\(A \cap \ideal b\), \(\ideal b \subset S^{-1} A\),
		wenn kein Element von \(S\) ein Nullteiler in \(A/\ideal a\) ist.
	\end{proposition}
	\begin{proof}<+->
		\(\ideal a\) ist genau dann kontrahiert, wenn \(A \cap (S^{-1} \ideal a) \subset \ideal a\), wenn also aus \(s x \in
		\ideal a\) mit \(s \in S, x \in A\) schon \(x \in \ideal a\) folgt. Das ist wiederum gleichbedeutend damit, daß
		kein \(s \in S\) ein Nullteiler in \(A/\ideal a\) ist.
	\end{proof}
\end{frame}

\begin{frame}{Lokalisierungen und endliche Summen, Produkte und Schnitte und Wurzeln von Idealen}
	\begin{proposition}<+->
		Sei \(A\) ein kommutativer Ring. Die Lokalisierung nach einer multiplikativ abgeschlossenen
		Teilmenge \(S \subset A\) kommutiert mit folgenden Operationen auf Idealen: endliche Summen,
		endliche Produkte, endliche Schnitte und Wurzeln.
	\end{proposition}
	\begin{proof}<+->
		\begin{enumerate}[<+->]
		\item<.->
			Endliche Summen und Produkte vertauschen mit Idealerweiterungen, und die Lokalisierung
			eines Ideals nach \(S\) entspricht der Idealerweiterung in \(S^{-1} A\).
		\item
			Endliche Schnitte von Idealen vertauschen mit Lokalisierung, denn dies gilt allgemeiner für Untermoduln.
		\item
			Sei \(\ideal a \subset A\) ein Ideal. Daß \(S^{-1} \sqrt{\ideal a} \subset \sqrt{S^{-1} \ideal a}\)
			folgt aus allgemeinen Eigenschaften der Erweiterung von Idealen. Die andere Inklusion ist einfach.
			\qedhere
		\end{enumerate}		
	\end{proof}
\end{frame}

\begin{frame}{Nilradikal der Lokalisierung}
	Sei \(A\) ein kommutativer Ring. Sei \(S \subset A\) multiplikativ abgeschlossen.
	\begin{corollary}<+->
		Ist \(\ideal n\) das Nilradikal von \(A\), so ist \(S^{-1} \ideal n\) das Nilradikal von \(S^{-1} A\).
		\qed
	\end{corollary}
\end{frame}

\subsection{Primideale und Lokalisierungen}

\begin{frame}{Primideale in Lokalisierungen}
	\begin{proposition}<+->
		Sei \(A\) ein kommutativer Ring. Sei \(S \subset A\) multiplikativ abgeschlossen.
		Durch \(\ideal q = S^{-1} \ideal p\) ist eine bijektive,
		ordnungserhaltende Korrespondenz zwischen den Primidealen
		\(\ideal p\) von \(A\) mit \(\ideal p \cap S = \emptyset\) und den Primidealen \(\ideal q\) von \(S^{-1} A\) gegeben.
	\end{proposition}
	\begin{proof}<+->
		\begin{enumerate}[<+->]
		\item<.->
			Ist \(\ideal q\) ein Primideal in \(S^{-1} A\), so ist \(\ideal p = A \cap \ideal q\) als Kontraktion eines
			Primideals wieder ein Primideal.
		\item
			Ist \(\ideal p\) ein Primideal in \(A\), so ist \(A/\ideal p\) ein Integritätsbereich. Ist \(\bar S\) das Bild von
			\(S\) in \(A/\ideal p\), so ist \(S^{-1} A/S^{-1} \ideal p \cong \bar S^{-1}(A/\ideal p)\) und damit entweder der Nullring
			oder wieder ein Integritätsbereich.
		\item
			Der erste Fall tritt genau dann ein, wenn \(S^{-1} \ideal p = (1)\), also genau dann, wenn
			\(\ideal p \cap S \neq \emptyset\).
			Der zweite Fall tritt genau dann ein, wenn \(S^{-1} \ideal p\) ein Primideal ist.
			\qedhere
		\end{enumerate}
	\end{proof}
\end{frame}

\begin{frame}{Bemerkungen zu Lokalisierungen und Idealen}
	\begin{remark}<+->
		Sei \(A\) ein kommutativer Ring. Daß zu einem nicht nilpotenten Element \(f \in A\) ein Primideal \(\ideal p\) von
		\(A\) mit \(f \notin \ideal p\) existiert, kann mittels Lokalisierung so bewiesen werden: Da \(f\) nicht nilpotent
		ist, ist \(A[f^{-1}] \neq 0\). Damit besitzt \(A[f^{-1}]\) ein maximales Ideal \(\ideal m\). Es folgt, daß
		\(\ideal p = A \cap \ideal m\) ein Primideal mit \(f \notin \ideal p\) ist.
	\end{remark}
\end{frame}

\begin{frame}{Primideale in Halmen}
	Sei \(\ideal p\) ein Primideal in einem kommutativen Ring \(A\).
	\begin{corollary}<+->
		Durch \(\ideal r = A_{\ideal p} \ideal q\) ist eine bijektive, ordnungserhaltende Korrespondenz zwischen den
		Primidealen \(\ideal q\) von \(A\) mit \(\ideal q \subset \ideal p\) und den Primidealen \(\ideal r\) im Halm \(A_{\ideal p}\)
		gegeben.
		\qed
	\end{corollary}
	\begin{remark}<+->
		Sei \(\ideal q \subset \ideal p\) ein weiteres Primideal.
		Lokalisieren bei \(\ideal p\) schneidet also alle Primideale außer denen heraus, die in \(\ideal p\) enthalten sind.
		\\
		Auf der anderen Seite schneidet der Wechsel von \(A\) nach \(A/\ideal q\) alle Primideale außer denen heraus, die
		\(\ideal q\) enthalten.
		\\
		Beim Übergang von \(A\) auf den Ring \(A_{\ideal p}/(A_{\ideal p} \ideal q) = (A/\ideal q)_{\bar{\ideal p}}\)
		beschränken wir unsere Betrachtung also auf alle Primideale zwischen \(\ideal q\) und \(\ideal p\) beschränken.
		Im Spezialfall \(\ideal p = \ideal q\) erhalten wir den \emph{Restklassenkörper \(A(\ideal p)\) von \(A\) an \(\ideal p\)},
		also den Restklassenkörper des Halmes \(A_{\ideal p}\) beziehungsweise den Quotientenkörper des Quotienten \(A/\ideal p\).
	\end{remark}
\end{frame}

\begin{frame}{Kontraktionen von Primidealen in beliebigen Ringhomomorphismen}
	\begin{proposition}<+->
		Sei \(\phi\colon A \to B\) ein Homomorphismus kommutativer Ringe. Dann ist ein Primideal \(\ideal p\) in
		\(A\) genau dann eine Kontraktion eines Primideals von \(B\), falls \(\ideal p = A \cap (B\ideal p)\).
	\end{proposition}
	\begin{proof}<+->
		\begin{enumerate}[<+->]
		\item<.->
			Ist \(\ideal p = A \cap \ideal q\) für ein Primideal \(\ideal q\) von \(B\), so ist \(A \cap (B \ideal p)
			= A \cap (B (A \cap \ideal q)) = A \cap \ideal q = \ideal p\).
		\item
			Sei umgekehrt \(\ideal p = A \cap (B \ideal p)\).
			Sei \(S\) das Bild von \(A \setminus \ideal p\) in \(B\).
			Dann ist \(S \cap (B \ideal p) = \emptyset\), also ist \(S^{-1} (B \ideal p)\) in einem maximalen
			Ideal \(\ideal m\) in \(S^{-1} B\) enthalten.
		\item
			Sei \(\ideal q \coloneqq B \cap \ideal m\). Es folgt, daß \(\ideal q\) ein Primideal ist. Weiter ist
			\(\ideal q \supset B \ideal p\) und \(\ideal q \cap S = \emptyset\). Damit ist \(A \cap \ideal q = \ideal p\).
			\qedhere
		\end{enumerate}
	\end{proof}
\end{frame}

\subsection{Lokalisierungen und der Annulator}

\begin{frame}{Lokalisierung des Annulators}
	\begin{proposition}<+->
		Sei \(A\) ein kommutativer Ring. Sei \(M\) ein endlich erzeugter \(A\)-Modul. Sei \(S \subset A\) multiplikativ abgeschlossen.
		Dann ist \(S^{-1} \ann(M) = \ann(S^{-1} M)\).
	\end{proposition}
	\begin{proof}<+->
		\begin{enumerate}[<+->]
		\item<.->
			Sei die Aussage wahr für zwei Untermoduln \(N\) und \(P\) von \(M\). Dann gilt sie auch für \(N + P\):
			\(S^{-1} \ann(N + P) = S^{-1} (\ann N \cap \ann P) = S^{-1} \ann N \cap S^{-1} \ann P
			= \ann(S^{-1} N) \cap \ann(S^{-1} P) = \ann(S^{-1} N + S^{-1} P) = \ann(S^{-1} (N + P))\).
		\item
			Damit reicht es die Aussage für einen Modul der Form \(M = A/\ideal a\), wobei \(\ideal a\) ein Ideal in \(A\) ist,
			zu beweisen. Es ist \(\ideal a = \ann(M)\). Weiter ist \(S^{-1} M \cong (S^{-1} A)/(S^{-1} \ideal a)\), also
			\(\ann(S^{-1} M) = S^{-1} \ideal a\).
			\qedhere
		\end{enumerate}
	\end{proof}
\end{frame}

\begin{frame}{Lokalisierungen und Quotienten}
	\begin{corollary}<+->
		Sei \(A\) ein kommutativer Ring. Seien \(N, P\) zwei Untermoduln eines \(A\)-Moduls \(M\). Sei \(S \subset A\) multiplikativ abgeschlossen.
		Ist \(P\) endlich erzeugt, so gilt \(S^{-1} (N : P) = (S^{-1} N : S^{-1} P)\).
	\end{corollary}
	\begin{proof}<+->
		Es ist \((N : P) = \ann((N + P)/N)\).
	\end{proof}
	\begin{example}<+->
		Seien \(\ideal a, \ideal b\) Ideale von \(A\). Ist \(\ideal b\) endlich erzeugt, so gilt
		\(S^{-1} (\ideal a : \ideal b) = (S^{-1} \ideal a : S^{-1} \ideal b)\).
	\end{example}
\end{frame}



\lecture{Primärzerlegung I}{Prim\"arzerlegung I}
\part<article>{Primärzerlegung}
\mode<all>\setcounter{section}{22}
\mode<all>\section{Primärzerlegung I}

\subsection{Primäre Ideale}

\mode<article>{Ein Primideal in einem kommutativen Ring ist in gewisser Weise eine Verallgemeinerung einer Primzahl.
Die entsprechende Verallgemeinerung einer Potenz einer Primzahl ist ein Primärideal.}

\begin{frame}{Definition primärer Ideale}
	Sei \(A\) ein kommutativer Ring.
	\begin{definition}
		Ein Ideal \(\ideal q\) in \(A\) heißt \emph{primär}, falls \(1 \notin \ideal q\) und falls aus \(x y \in
		\ideal q\) schon \(x \in \ideal q\) oder \(y^n \in \ideal q\) für ein \(n \in \set N_0\) folgt.
	\end{definition}
	\begin{visibleenv}<+->
		Das Ideal \(\ideal q\) ist also genau dann primär, falls die nilpotenten Elemente in \(A/\ideal q\) gerade die
		Nullteiler in \(A/\ideal q\) sind.
	\end{visibleenv}
	\begin{example}<+->
		Jedes Primideal in \(A\) ist auch primär.
	\end{example}
	\begin{example}<+->
		Sei \(\phi\colon A \to B\) ein Homomorphismus kommutativer Ringe. Ist dann \(\ideal q\) ein primäres Ideal in
		\(B\), ist die Kontraktion \(A \cap \ideal q\) primär in \(A\), denn \(A/(A \cap \ideal q)\) ist ein Unterring
		von \(B/\ideal q\).
	\end{example}
\end{frame}

\begin{frame}{Wurzeln primärer Ideale}
	Sei \(\ideal q\) ein primäres Ideal in einem kommutativen Ring \(A\). 
	\begin{proposition}<+->
		Es ist \(\sqrt{\ideal q}\) das kleinste Primideal \(\ideal p\) mit \(\ideal p \supset \ideal q\).
	\end{proposition}
	\begin{proof}<+->
		\begin{enumerate}[<+->]
		\item<.->
			Da \(\sqrt{\ideal q} = \bigcap\limits_{\ideal p' \supset \ideal q} \ideal p'\), wobei \(\ideal p'\) für ein
			Primideal von \(A\) steht, reicht es zu zeigen, daß \(\sqrt{\ideal q}\) ein Primideal ist.
		\item
			Sei dazu \(x y \in \sqrt{\ideal q}\). Dann ist \(x^m y^m = (x y)^m \in \ideal q\) für ein \(m \in \set N_0\).
		\item
			Da \(\ideal q\) primär ist, folgt \(x^m \in \ideal q\) oder \(y^{nm} \in \ideal q\) für ein \(n \in \set N_0\).
		\item
			Daraus folgt \(x \in \sqrt{\ideal q}\) oder \(y \in \sqrt{\ideal q}\).
			\qedhere
		\end{enumerate}
	\end{proof}
	\begin{definition}<+->
		Ist \(\ideal p = \sqrt{\ideal q}\), so heißt \(\ideal q\) ein \emph{\(\ideal p\)-primäres Ideal von \(A\)}.
	\end{definition}
\end{frame}

\begin{frame}{Beispiele primärer Ideale}
	\begin{example}<+->
		Die primären Ideale in \(\set Z\) sind die Ideale der Form \((0)\) und \((p^n)\), wobei \(p\) eine Primzahl ist, denn
		diese sind die einzigen Ideale, deren Wurzel ein Primideal ist, und es ist klar, daß diese Ideale primär sind.
	\end{example}
	\begin{example}<+->
		Sei \(A \coloneqq K[x, y]\) der Polynomring in zwei Variablen über einem Körper. Sei \(\ideal q \coloneqq (x, y^2)\). Dann ist
		\(A/\ideal q \cong k[y]/(y^2)\), ein Ring, in dem alle Nullteiler Vielfache von \(y\) sind, also nilpotent sind. Damit ist
		\(\ideal q\) ein primäres Ideal, und zwar mit Wurzel \(\ideal p \coloneqq (x, y)\).
		\\
		Da \(\ideal p^2 \subsetneq \ideal q \subsetneq \ideal p\) sehen wir, daß ein primäres Ideal im allgemeinen keine Potenz
		eines Primideals sein muß.
	\end{example}
\end{frame}

\begin{frame}{Nicht primäre Potenz eines Primideals}
	\begin{example}<+->
		Seien \(K\) ein Körper und \(A \coloneqq K[x, y, z]/(xy - z^2)\). Mit \(\bar x, \bar y, \bar z\) bezeichnen wir die
		Bilder von \(x, y, z\) in \(A\).
		\\
		Es ist \(\ideal p \coloneqq (\bar x, \bar z)\) ein Primideal in \(A\), denn \(A/\ideal p \cong K[y]\) ist ein Integritätsbereich.
		\\
		Wir haben \(\bar x \bar y = \bar z^2 \in \ideal p^2\), aber \(\bar x \notin \ideal p^2\) und \(\bar y \notin \sqrt{\ideal p^2}
		= \ideal p\), also ist \(\ideal p^2\) nicht primär.
	\end{example}
	\begin{visibleenv}<+->
		Wir sehen also, daß Potenzen von Primidealen nicht notwendigerweise primär sind.
	\end{visibleenv}
\end{frame}

\begin{frame}{Ideale, deren Wurzel ein maximales Ideal ist}
	Sei \(A\) ein kommutativer Ring.
	\begin{proposition}<+->
		Sei \(\ideal a\) ein Ideal in \(A\). Ist \(\ideal m \coloneqq \sqrt{\ideal a}\) ein maximales Ideal, so ist \(\ideal a\) ein
		\(\ideal m\)-primäres
		Ideal.
	\end{proposition}
	\begin{proof}<+->
		\begin{enumerate}[<+->]
		\item<.->
			Das Bild von \(\ideal m\) in \(A/\ideal a\) ist das Nilradikal von \(A/\ideal a\). Da jedes Primideal das Nilradikal enthält,
			besitzt \(A/\ideal a\) damit genau ein Primideal.
		\item
			Damit ist jedes Element in \(A/\ideal a\) entweder eine Einheit oder nilpotent, und damit ist jeder Nullteiler in
			\(A/\ideal a\) auch nilpotent.
			\qedhere
		\end{enumerate}
	\end{proof}
	\begin{example}<+->
		Ist \(\ideal m\) ein maximales Ideal in \(A\), so sind die Potenzen \(\ideal m^n\) mit \(n > 0\) alle \(\ideal m\)-primär.
	\end{example}
\end{frame}

\subsection{Schnitte und Idealquotienten primärer Ideale}

\begin{frame}{Schnitte primärer Ideale mit dem gleichen Wurzelideal}
	\begin{lemma}<+->
		\label{lem:primary1}
		Sei \(\ideal p\) ein Primideal in einem kommutativen Ring \(A\). Dann ist der Schnitt
		\(\ideal q \coloneqq \bigcap\limits_{i = 1}^n \ideal q_i\) endlich vieler \(\ideal p\)-primärer Ideale
		\(\ideal q_1, \dotsc, \ideal q_n\) wieder \(\ideal p\)-primär.
	\end{lemma}
	\begin{proof}<+->
		\begin{enumerate}[<+->]
		\item<.->
			\(\sqrt{\ideal q} = \sqrt{\bigcap\limits_i \ideal q_i} = \bigcap\limits_i \sqrt{\ideal q_i} = \ideal p\).
		\item
			Sei \(xy \in \ideal q\) mit \(x \notin \ideal q\). Dann ist \(xy \in \ideal q_i\) mit \(x \notin \ideal q_i\) für ein
			\(i\). Es folgt \(y \in \ideal p\), da \(\ideal q_i\) ein \(\ideal p\)-primäres Ideal ist.			
			\qedhere
		\end{enumerate}
	\end{proof}
\end{frame}

\begin{frame}{Idealquotienten primärer Ideale}
	\begin{lemma}<+->
		\label{lem:primary2}
		Sei \(\ideal p\) ein Primideal eines kommutativen Ringes \(A\). Sei \(\ideal q\) ein \(\ideal p\)-primäres
		Ideal und \(x \in A\). Dann gilt:
		\begin{enumerate}[<+->]
		\item<.->
			Ist \(x \in \ideal q\), so \((\ideal q : x) = (1)\).
		\item
			Ist \(x \notin \ideal q\), so ist \((\ideal q : x)\) ein \(\ideal p\)-primäres Ideal.
		\item
			Ist \(x \notin \ideal p\), so ist \((\ideal q : x) = \ideal q\).
		\end{enumerate}
	\end{lemma}
	\begin{proof}<+->
		\begin{enumerate}[<+->]
		\item<.->
			Die erste und die dritte Aussage folgt sofort aus den Definitionen.
		\item
			Sei \(x \notin \ideal q\). Für \(y \in (\ideal q : x)\) folgt dann \(x y \in \ideal q\), also \(y \in \ideal p\).
			Damit ist \(\ideal q \subset (\ideal q : x) \subset \ideal p\). Durch Wurzelziehen folgt \(\sqrt{(\ideal q : x)}
			= \ideal p\).
		\item
			Sei weiter \(yz \in (\ideal q : x)\) mit \(x \notin \ideal q\). Sei \(y \notin \ideal p\). Aus \(x y z \in \ideal q\)
			folgt dann \(xz \in \ideal q\), also \(z \in (\ideal q : x)\).
			\qedhere
		\end{enumerate}
	\end{proof}
\end{frame}

\subsection{Primärzerlegungen}

\begin{frame}{Definition der Primärzerlegung}
	\begin{definition}<+->
		Sei \(\ideal a\) ein Ideal in einem kommutativen Ring \(A\). Ein Ausdruck von \(\ideal a\) als endlicher Schnitt
		\(\ideal a = \bigcap\limits_{i = 1}^n \ideal q_i\) primärer Ideale \(\ideal q_i\) heißt \emph{Primärzerlegung von \(\ideal a\)}.
		\\
		Sind die \(\sqrt{\ideal q_i}\) paarweise verschieden und gilt \(\ideal q_i \not\supset \bigcap\limits_{j \neq i}
		\ideal q_j\) für alle \(i\), so heißt die Primärzerlegung \emph{minimal}.
	\end{definition}
	\begin{remark}<+->
		Nach dem vorletzten Hilfssatz können wir jede Primärzerlegung eines Ideals zu einer minimalen reduzieren.
	\end{remark}
	\begin{remark}<+->
		Im allgemeinen muß nicht jedes Ideal eine Primärzerlegung besitzen. Im Falle, daß es eine hat, heißt es \emph{zerlegbar}.
	\end{remark}
\end{frame}

\begin{frame}{Erster Eindeutigkeitssatz}
	\begin{theorem}[Der erste Eindeutigkeitssatz]<+->
		\label{thm:first_uniqueness}
		Sei \(\ideal a\) ein zerlegbares Ideal in einem kommutativen Ring \(A\). Sei \(\ideal a = \bigcap\limits_{i = 1}^n \ideal
		q_i\) eine minimale Primärzerlegung von \(\ideal a\). Sei \(\ideal p_i \coloneqq \sqrt{\ideal q_i}\). Dann sind die Ideale
		\(\ideal p_i\) genau diejenigen Primideale von \(A\), welche von der Form \(\sqrt{(\ideal a : x)}\) mit \(x \in A\) sind.
		\\
		Insbesondere sind die \(\ideal p_i\) unabhängig von der Primärzerlegung von \(A\). Sie heißen die 			\alert{zu \(\ideal a\)
		assoziierten Primideale}.
	\end{theorem}
	\begin{proof}<+->
		\begin{enumerate}[<+->]
		\item<.->
			\(\sqrt{(\ideal a : x)} = \sqrt{(\bigcap\limits_i \ideal q_i : x)} = \bigcap\limits_i
			\sqrt{(\ideal q_i : x)} = \bigcap\limits_{\ideal q_j \not\ni x} \ideal p_j\).
		\item
			Ist \(\ideal p \coloneqq \sqrt{(\ideal a : x)}\) für ein \(x \in A\) prim, folgt damit \(\ideal p = \ideal p_j\) für ein
			\(j\).
		\item
			Umgekehrt folgt aus der Minimalität der Zerlegung, daß für jedes \(i\) ein \(x \notin \ideal q_i\) mit
			\(x \in \bigcap\limits_{j \neq i} \ideal q_j\) existiert. Dafür gilt \(\sqrt{(\ideal a : x)} = \ideal p_i\).
			\qedhere
		\end{enumerate}
	\end{proof}
\end{frame}

\begin{frame}{Bemerkungen zum ersten Eindeutigkeitssatz}
	Sei \(\ideal a\) ein zerlegbares Ideal in einem kommutativen Ring \(A\).
	\begin{remark}<+->
		Sei \(\ideal p\) ein assoziiertes Primideal zu \(\ideal a\).
		Aus dem letzten Beweis und dem letzten Hilfssatz folgt, daß ein \(x \in A\) existiert, so daß
		\((\ideal a : x)\) ein \(\ideal p\)-primäres Ideal ist.
	\end{remark}
	\begin{remark}<+->
		Sehen wir \(A/\ideal a\) als \(A\)-Modul an, ist der Eindeutigkeitssatz äquivalent dazu zu sagen, daß die zu
		\(\ideal a\) assoziierten Primideale genau diejenigen Primideale sind, welche als Wurzeln von Annulatoren von Elementen von
		\(A/\ideal a\) auftauchen.
	\end{remark}
\end{frame}

\subsection{Isolierte und minimale Primideale}

\begin{frame}{Definition isolierter und eingebetteter Primideale}
	\begin{definition}<+->
		Sei \(\ideal a\) ein zerlegbares Ideal eines kommutativen Ringes \(A\). Die minimalen Elemente der Menge der zu
		\(\ideal a\) assoziierten Primideale heißen die \emph{isolierten Primideale zu \(\ideal a\)}. Die übrigen zu
		\(\ideal a\) assoziierten Primideale heißen die \emph{eingebetteten Primideale zu \(\ideal a\)}.
	\end{definition}
	\begin{example}<+->
		Sei \(K\) ein Körper. Sei \(A = K[x, y]\). Sei \(\ideal a = (x^2, xy)\). Dann ist \(\ideal a = \ideal p_1 \cap \ideal p_2^2\)
		mit \(\ideal p_1 = (x)\) und \(\ideal p_2 = (x, y)\). Da \(\ideal p_2\) maximal ist, ist \(\ideal p_2^2\) ein primäres
		Ideal. Damit sind \(\ideal p_1\) und \(\ideal p_2\) die zu \(\ideal a\) assoziierten Ideale.
		\\
		In diesem Beispiel ist \(\ideal p_1 \subset \ideal p_2\), weiter haben wir \(\sqrt{\ideal a} = \ideal p_1 \cap \ideal p_2 =
		\ideal p_1\), allerdings ist \(\ideal a\) selbst kein primäres Ideal.
		\\
		Es ist \(\ideal p_2\) ein eingebettetes Primideal zu \(\ideal a\).
	\end{example}
	\begin{visibleenv}<+->
		Die Begriffe "`isoliert"' und "`eingebettet"' kommen aus der algebraischen Geometrie.
	\end{visibleenv}
\end{frame}

\begin{frame}{Isolierte Primideale}
	\begin{proposition}<+->
		\label{prop:isolated_prime}
		Sei \(\ideal a\) ein zerlegbares Ideal eines kommutativen Ringes. Jedes Primideal \(\ideal p\) mit \(\ideal p \supset \ideal a\)
		enthält ein isoliertes Primideal zu \(\ideal a\).
	\end{proposition}
	Die isolierten Primideale zu \(\ideal a\) sind damit genau die minimalen Elemente der Menge aller Primideale, welche \(\ideal a\)
	umfassen.
	\begin{proof}<+->
		Sei \(\ideal a = \bigcap\limits_{i = 1}^n \ideal q_i\) eine Primärzerlegung von \(\ideal a\). Aus
		\(\ideal p \supset \bigcap\limits_i \ideal q_i\) folgt \(\ideal p = \sqrt{\ideal p}
		\supset \bigcap\limits_i \sqrt{\ideal q_i}\). Damit muß \(\ideal p \supset \sqrt{\ideal q_i}\) für ein \(i\) gelten.
		Damit umfaßt \(\ideal p\) ein isoliertes Primideal zu \(\ideal a\).
	\end{proof}
\end{frame}

\begin{frame}{Die primären Komponenten sind nicht eindeutig}
	\begin{remark}<+->
		Sei \(K\) ein Körper. Im Ring \(K[x, y]\) sind \((x^2, xy) = (x) \cap (x, y)^2\) und
		\((x) \cap(x^2, y)\) zwei verschiedene Primärzerlegungen von \((x^2, xy)\). Damit sind primären Komponenten
		eines zerlegbaren Ideals im allgemeinen nicht eindeutig.
		\\
		In Kürze werden wir allerdings gewisse Eindeutigkeitseigenschaften kennenlernen.
	\end{remark}
\end{frame}



\lecture{Primärzerlegung II}{Prim\"arzerlegung II}
\mode<all>\setcounter{section}{23}
\mode<all>\section{Primärzerlegung II}

\subsection{Primärzerlegung und das Nullideal} 

\begin{frame}{Vereinigung der zu einem Ideal assoziierten Primideale}
	 \begin{proposition}<+->
	 	\label{prop:union_of_assoc_primes}
		Sei \(\ideal a\) ein zerlegbares Ideal eines kommutativen Ringes \(A\). Sei \(\ideal a = \bigcap\limits_{i = 1}^n
		\ideal q_i\) eine minimale Primärzerlegung. Sei \(\ideal p_i \coloneqq \sqrt{\ideal q_i}\). Dann gilt
		\(\bigcup\limits_{i = 1}^n \ideal p_i = \{x \in A \mid (\ideal a : x) \neq \ideal a\}\).
	\end{proposition}
	\begin{corollary}<+->
		Ist das Nullideal in \(A\) zerlegbar, so ist die Menge \(D\) der Nullteiler von \(A\) die Vereinigung der zu \((0)\)
		assoziierten Primideale.
	\end{corollary}
	\begin{proof}[Beweis der Proposition]<+->
		Aus der Zerlegbarkeit von \(\ideal a\) folgt die Zerlegbarkeit von \((0) \subset A/\ideal a\), denn \((0) = \bigcap_{i \in I}
		\bar{\ideal q}_i\), wobei \(\bar{\ideal q}_i\) das Bild von \(\ideal q_i\) in \(A/\ideal a\) ist, welches primär ist.
		\\
		Damit reicht es, die Folgerung zu beweisen.
	\end{proof}
\end{frame}

\begin{frame}{Beweis der Folgerung}
	\begin{proof}[Beweis der Folgerung]<+->
		\begin{enumerate}[<+->]
		\item<.->
			Zunächst ist \(D = \bigcup\limits_{x \neq 0} \sqrt{(0 : x)}\).
			\\
			Auf der anderen Seite wissen wir aus dem Beweis des ersten Eindeutigkeitssatzes, daß
			\(\sqrt{(0 : x)} = \bigcap\limits_{\ideal q_j \not\ni x} \ideal p_j \subset \ideal p_i\) für ein \(i\). Damit
			gilt \(D \subset \bigcup\limits_{i = 1}^n \ideal p_i\).
		\item
			Auf der anderen Seite ist jedes \(\ideal p_i\) von der Form \(\sqrt{(0 : x)}\) mit \(x \neq 0\), also
			\(\bigcup\limits_{i = 1}^n \ideal p_i \subset D\).
			\qedhere
		\end{enumerate}
	\end{proof}
\end{frame}

\begin{frame}{Die Menge der Nullteiler und die Menge der nilpotenten Elemente}
	\begin{remark}<+->
		Für einen kommutativen Ring \(A\), in dem das Nullideal zerlegbar ist, gilt also:
		\begin{enumerate}[<+->]
		\item<.->
			Die Menge der Nullteiler von \(A\) ist die Vereinigung aller Primideale, die assoziiert zu \((0)\) sind.
		\item
			Die Menge der nilpotenten Elemente ist der Schnitt aller (isolierten) Primideale, die zu \((0)\) assoziiert sind.
		\end{enumerate}
	\end{remark}
\end{frame}

\subsection{Primäre Ideale und Lokalisierung}

\begin{frame}{Primäre Ideale in Lokalisierungen}
	\begin{proposition}<+->
		\label{prop:primaries_in_localisation}
		Sei \(\ideal p\) ein Primideal in einem kommutativen Ring \(A\). Sei \(\ideal q\) ein \(\ideal p\)-primäres Ideal.
		Sei \(S \subset A\) multiplikativ abgeschlossen. Dann gilt:
		\begin{enumerate}[<+->]
		\item<.->
			Ist \(S \cap \ideal p \neq \emptyset\), so folgt \(S^{-1} \ideal q = (1)\).
		\item
			Ist \(S \cap \ideal p = \emptyset\), so ist \(S^{-1} \ideal q\) ein \(S^{-1} \ideal p\)-primäres Ideal
			und für seine Kontraktion gilt \(A \cap S^{-1} \ideal q = \ideal q\).
		\end{enumerate}
	\end{proposition}
	\begin{proof}<+->
		\begin{enumerate}[<+->]
		\item<.->
			Ist \(s \in S \cap \ideal p\), so ist \(s^n \in S \cap \ideal q\) für ein \(n \in \set N_0\). Damit enthält
			\(S^{-1} \ideal q\) das Element \(\frac{s^n} 1\), welches eine Einheit in \(S^{-1} A\) ist.
		\item
			Ist \(S \cap \ideal p = \emptyset\), so folgt aus \(s \in S\) und \(as \in \ideal q\) schon \(a \in \ideal q\).
			Wegen \(A \cap S^{-1} \ideal q = \bigcup\limits_{s \in S} (\ideal q : s)\) folgt damit \(A \cap S^{-1} \ideal q
			= \ideal q\).
			Außerdem ist \(\sqrt{S^{-1} \ideal q} = S^{-1} \sqrt{\ideal q} = S^{-1} \ideal p\). Daß \(S^{-1} \ideal q\) primär ist,
			ist schließlich einfach.
			\qedhere
		\end{enumerate}
	\end{proof}
\end{frame}

\begin{frame}{Korrespondenz zwischen primären Idealen in einer Lokalisierung}
	\begin{corollary}<+->
		\label{cor:correspondence_for_primaries}
		Sei \(A\) ein kommutativer Ring. Sei \(S \subset A\) multiplikativ abgeschlossen.
		Durch \(\ideal r = S^{-1} \ideal q\) ist eine bijektive, ordnungserhaltende Korrespondenz
		zwischen den primären Idealen \(\ideal q\) von \(A\) mit \(\sqrt{\ideal q} \cap S = \emptyset\) und den
		primären Idealen von \(S^{-1} A\) gegeben. 
	\end{corollary}
	\begin{proof}<+->
		Folgt aus der letzten Proposition zusammen mit der Tatsache, daß die Kontraktion eines primären Ideals
		ein primäres Ideal ist.
	\end{proof}
\end{frame}

\subsection{Sättigung eines Ideals}

\begin{frame}{Definition der Sättigung eines Ideals}
	\begin{definition}<+->
		Sei \(A\) ein kommutativer Ring. Sei \(S \subset A\) multiplikativ
		abgeschlossen. Die \emph{Sättigung \(S(\ideal a)\) eines Ideals
		\(\ideal a\) von \(A\) bezüglich \(S\)} ist das Ideal
		\(S(\ideal a) \coloneqq A \cap S^{-1} \ideal a\), also die Kontraktion
		der Erweiterung des Ideals bezüglich der Lokalisierung \(S^{-1} A\).
	\end{definition}
\end{frame}

\begin{frame}{Sättigung zerlegbarer Ideale}
	\begin{proposition}<+->
		\label{prop:sat_decomp}
		Sei \(\ideal a\) ein zerlegbares Ideal in einem kommutativen Ring \(A\).
		Sei \(S \subset A\) multiplikativ abgeschlossen. Sei
		\(\ideal a = \ideal q_1 \cap \dotsb \cap \ideal q_n\) eine minimale
		Primärzerlegung von \(\ideal a\). Sei \(\ideal p_i \coloneqq
		\sqrt{\ideal q_i}\). Weiter seien die \(\ideal q_i\) so
		numeriert, daß \(S \cap \ideal p_i = \emptyset \iff
		i \leq m\) für ein \(0 \leq m \leq n\). Dann sind
		\(S^{-1} \ideal a = \bigcap\limits_{i = 1}^m S^{-1} \ideal q_i\)
		und \(S(\ideal a) = \bigcap\limits_{i = 1}^m \ideal q_i\)
		minimale Primärzerlegungen.
	\end{proposition}
	\begin{proof}<+->
		\begin{enumerate}[<+->]
		\item<.->
			\(S^{-1} \ideal a = \bigcap\limits_{i = 1}^n S^{-1} \ideal q_i
			= \bigcap\limits_{i = 1}^m S^{-1} \ideal q_i\), und
			\(S^{-1} \ideal q_i\) ist ein
			\(S^{-1} \ideal p_i\)-primäres Ideal für \(i \leq m\).
		\item
			Da die \(\ideal p_i\) paarweise verschieden sind, gilt dies auch
			für die \(S^{-1} \ideal p_i\) für \(i \leq m\), so daß
			\(\bigcap\limits_{i = 1}^m S^{-1} \ideal q_i\) minimal ist.
		\item
			Kontraktion liefert die entsprechende Aussage
			für \(S(\ideal a)\).
			\qedhere
		\end{enumerate}
	\end{proof}
\end{frame}

\begin{frame}{Isolierte Mengen assoziierter Primideale}
	Sei \(\ideal a\) ein zerlegbares Ideal in einem kommutativen Ring \(A\).
	\begin{definition}<+->
		Eine Menge \(\mathfrak S\) von zu \(\ideal a\) assoziierten Primidealen
		heißt \emph{isoliert}, falls sie folgende Bedingung erfüllt: Ist
		\(\ideal p'\) ein zu \(\ideal a\) assoziiertes Primideal und ist
		\(\ideal p' \subset \ideal p\) für ein \(\ideal p \in \mathfrak S\),
		so folgt \(\ideal p' \in \mathfrak S\).
	\end{definition}
	\begin{proposition}<+->
		Sei \(\mathfrak S\) eine isolierte Menge von an \(\ideal a\)
		assoziierten Primidealen. Dann ist \(S \coloneqq A \setminus
		\bigcup\limits_{\ideal p \in \mathfrak S} \ideal p\) multiplikativ
		abgeschlossen, und für jedes an \(\ideal a\) assoziierte Primideal
		\(\ideal p'\) gilt:
		\begin{enumerate}[<+->]
		\item<.->
			Ist \(\ideal p' \in \mathfrak S\), so gilt
			\(\ideal p' \cap S = \emptyset\).
		\item
			Ist \(\ideal p' \notin \mathfrak S\), so ist
			\(\ideal p' \cap S \neq \emptyset\).
			\qed
		\end{enumerate}
	\end{proposition}
\end{frame}

\begin{frame}{Der zweite Eindeutigkeitssatz}
	\begin{theorem}[Der zweite Eindeutigkeitssatz]<+->
		\label{thm:second_uniqueness}
		Sei \(\ideal a\) ein zerlegbares Ideal in einem kommutativen
		Ring \(A\). Sei \(\ideal a = \bigcap\limits_{i = 1}^n \ideal q_i\)
		eine minimale Primärzerlegung. Sei \(\ideal p_i \coloneqq
		\sqrt{\ideal q_i}\). Ist dann \(\{\ideal p_1, \dots, \ideal p_m\}\)
		eine isolierte Menge von an \(\ideal a\) assoziierten Primidealen,
		so ist \(\bigcap\limits_{i = 1}^m \ideal q_i\) unabhängig von der
		Zerlegung.
	\end{theorem}
	\begin{proof}<+->
		Sei \(S \coloneqq A \setminus \bigcup\limits_{i = 1}^m \ideal p_i\).
		Dann ist \(S(\ideal a) = \bigcap\limits_{i = 1}^m \ideal q_i\), also
		unabhängig von der Zerlegung, da die \(\ideal p_i\) nur von
		\(\ideal a\) abhängen.
	\end{proof}
\end{frame}

\begin{frame}{Eindeutigkeit der isolierten primären Komponenten}
	\begin{corollary}<+->
		\label{cor:second_uniqueness}
		Sei \(\ideal a\) ein zerlegbares Ideal in einem kommutativen
		Ring \(A\). Sei \(\ideal a = \bigcap\limits_{i = 1}^n \ideal q_i\)
		eine minimale Primärzerlegung. Sei \(\ideal p_i \coloneqq
		\sqrt{\ideal q_i}\). Dann sind die \(\ideal q_i\), für die \(\ideal p_i\)
		ein isoliertes Primideal von \(\ideal a\) ist, die \alert{isolierten
		primären Komponenten von \(\ideal a\)}, eindeutig bestimmt.
	\end{corollary}
	\begin{remark}<+->
		Die eingebetteten primären Komponenten sind dagegen im allgemeinen nicht
		eindeutig durch das Ideal bestimmt.
		\\
		Später werden wir Beispiele sehen, in denen es unendlich viele
		Möglichkeiten für jede eingebettete Komponente gibt.
	\end{remark}
\end{frame}



\lecture{Ganzheit}{Ganzheit}
\part<article>{Ganzheit und Bewertungen}
\mode<all>\setcounter{section}{24}
\mode<all>\section{Ganzheit}

\subsection{Ganze Elemente}

\begin{frame}{Definition ganzer Elemente}
	\begin{definition}<+->
		Sei \(B\) ein kommutativer Ring. Sei \(A \subset B\) ein Unterring.
		Ein Element \(x \in B\) heißt \emph{ganz über \(A\)}, falls es
		Nullstelle eines normierten Polynoms in \(A[x]\) ist, falls also
		\(x\) eine Gleichung der Form \(x^n + a_1 x^{n - 1} + \dotsb + a_n = 0\)
		mit \(a_i \in A\) erfüllt.
	\end{definition}
	\begin{visibleenv}<+->
		Insbesondere ist jedes Element aus \(A\) ganz über \(A\).
	\end{visibleenv}
	\begin{example}<+->
		Betrachte die Ringerweiterung \(\set Z \subset \set Q\). Sei eine
		rationale Zahl \(x = \frac r s\) mit \(r, s \in \set Z\) und
		\((r, s) = (1)\) ganz über \(\set Z\). Dann existieren
		\(a_i \in \set Z\) mit \(r^n + a_1 r^{n - 1} s + \dotsc + a_n s^n = 0\).
		Damit ist \(s\) Teiler von \(r^n\), also \(s = \pm 1\), also \(x \in
		\set Z\).
	\end{example}
\end{frame}

\begin{frame}{Charakterisierung ganzer Elemente}
	\begin{proposition}<+->
                \label{prop:characterization-integral-elements}
		Sei \(A \subset B\) eine Erweiterung kommutativer Ringe. Für ein
		Element \(x \in B\) sind folgende Aussagen äquivalent:
		\begin{enumerate}[<+->]
		\item
			Das Element \(x\) ist ganz über \(A\).
		\item
			Es ist \(A[x] \subset B\) als \(A\)-Modul endlich erzeugt.
		\item
			Es existiert ein Unterring \(C\) von \(B\) mit \(A[x] \subset C\),
			so daß \(C\) als \(A\)-Modul endlich erzeugt ist.
		\item
			Es existiert ein treuer \(A[x]\)-Modul \(M\), welcher als
			\(A\)-Modul endlich erzeugt ist.
		\end{enumerate}
	\end{proposition}
\end{frame}

\begin{frame}{Beweis zur Charakterisierung ganzer Elemente}
	\begin{proof}<+->
		\begin{enumerate}[<+->]
		\item<.->
			Erfüllt \(x\) die Gleichung \(x^n = - a_1 x^{n - 1} - \dotsb - 
			a_n\) mit \(a_i \in A\), so ist \(A[x]\) als \(A\)-Modul von
			\(1, x, \dotsc, x^{n - 1}\) erzeugt.
		\item
			Ist \(A[x]\) als \(A\)-Modul endlich erzeugt, so ist insbesondere
			\(C = A[x]\) ein Unterring von \(B\), welcher \(A[x]\)
			umfaßt und als \(A\)-Modul endlich erzeugt ist.
		\item
			Ist \(C \subset B\) ein \(A[x]\) umfassender Unterring, welcher
			als \(A\)-Modul endlich erzeugt ist, so ist insbesondere \(M = C\)
			ein treuer \(A[x]\)-Modul, denn aus \(y C = 0\) folgt
			insbesondere \(y = y \cdot 1 = 0\), und \(M\) ist als \(A\)-Modul
			endlich erzeugt.
		\item
			Sei \(M\) ein treuer \(A[x]\)-Modul, welcher als \(A\)-Modul
			endlich erzeugt ist. Sei \(\phi\colon M \to M, m \mapsto x m\).
			Da \(M\) als \(A\)-Modul endlich erzeugt ist, existieren
			\(a_i \in A\) mit \(\phi^n + a_1 \phi^{n -1} + \dotsc + a_n = 0\).
			Da \(M\) ein treuer \(A[x]\)-Modul ist, folgt daraus
			\(x^n + a_1 x^{n - 1} + \dotsc + a_n = 0\).
			\qedhere
		\end{enumerate}
	\end{proof}
\end{frame}

\begin{frame}{Von endlich vielen ganzen Elementen erzeugte Unterringe}
	\begin{corollary}<+->
		Sei \(A \subset B\) eine Erweiterung kommutativer Ringe. Seien
		\(x_1, \dotsc, x_n \in B\), welche jeweils ganz über \(A\) sind. Dann
		ist \(A[x_1, \dotsc, x_n]\) ein endlich erzeugter \(A\)-Modul.
	\end{corollary}
	\begin{proof}<+->
		\begin{enumerate}[<+->]
		\item<.->
			Der Fall \(n = 1\) ist ein Spezialfall der Proposition.
		\item
			Sei also \(n > 1\). Wir setzen
			\(A_r \coloneqq A[x_1, \dotsc, x_r]\). Nach Induktionsvoraussetzung
			ist \(A_{n - 1}\) als \(A\)-Modul endlich erzeugt.
		\item
			Da \(x_n\) insbesondere ganz über \(A_{n - 1}\) ist, ist
			\(A_n\) als \(A_{n - 1}\)-Modul endlich erzeugt.
		\item
			Es folgt, daß \(A_n\) auch als \(A\)-Modul endlich erzeugt ist.
			\qedhere
		\end{enumerate}
	\end{proof}
\end{frame}

\begin{frame}{Der Unterring der ganzen Elemente}
	\begin{corollary}<+->
		Sei \(A \subset B\) eine Erweiterung kommutativer Ringe. Dann ist
		die Menge \(C\) der über \(A\) ganzen Elemente von \(B\) ein Unterring
		von \(B\), welcher \(A\) enthält.
	\end{corollary}
	\begin{proof}<+->
		Seien \(x, y \in C\). Dann ist nach der letzten Folgerung \(A[x, y]\)
		ein endlich erzeugter \(A\)-Modul. Damit sind auch
		\(xy, x + y \in A[x, y]\) nach der dritten Aussage der Proposition
		ganz über \(A\). 
	\end{proof}
\end{frame}

\subsection{Ganzheit}

\begin{frame}{Ganzer Abschluß}
	\begin{definition}<+->
		Sei \(A \subset B\) eine Erweiterung kommutativer Ringe.
		\begin{enumerate}[<+->]
		\item<.->
			Der Unterring \(C\) aller über \(A\) ganzen Elemente von \(B\) heißt der
			\emph{ganze Abschluß von \(A\) in \(B\)}.
		\item
			Ist \(C = A\), so heißt \(A\) \emph{ganz abgeschlossen in \(B\)}.
		\item
			Ist \(C = B\), so heißt \(B\) \emph{ganz über \(A\)}.
		\end{enumerate}
	\end{definition}
	\begin{example}<+->
		Der Ring \(\set Z\) ist in \(\set Q\) ganz abgeschlossen.
	\end{example}
\end{frame}

\begin{frame}{Ganze Ringhomomorphismen}
	\begin{definition}<+->
		Sei \(\phi\colon A \to B\) ein Homomorphismus kommutativer Ringe, so daß
		\(B\) zu einer \(A\)-Algebra wird. Dann heißt \(\phi\) ganz und \(B\)
		eine \emph{ganze \(A\)-Algebra}, falls \(B\) ganz über dem Unterring
		\(\phi(A)\) ist.
	\end{definition}
	\begin{visibleenv}<+->
		Damit haben wir oben also gezeigt: Eine \(A\)-Algebra \(B\) ist genau
		dann endlich über \(A\), wenn sie endlich erzeugt und ganz über \(A\)
		ist.
	\end{visibleenv}	
\end{frame}

\begin{frame}{Transitivität der ganzen Abhängigkeit}
	\begin{corollary}<+->
		Seien \(A \subset B \subset C\) Erweiterungen kommutativer Ringe. 
		Ist dann \(B\) ganz über \(A\) und \(C\) ganz über \(B\), so ist
		auch \(C\) ganz über \(A\).
	\end{corollary}
	\begin{proof}<+->
		\begin{enumerate}[<+->]
		\item<.->
			Ist \(x \in C\), so existieren \(b_i \in B\) mit
			\(x^n + b_1 x^{n - 1} + \dotsb + b_n = 0\). Da die
			\(b_i\) ganz über \(A\) sind, ist \(B' = A[b_1, \dotsc, b_n]\)
			ein endlich erzeugter \(A\)-Modul.
		\item
			Da \(x\) ganz über \(B'\) ist, ist \(B'[x]\) ein endlich erzeugter
			\(B\)-Modul.
		\item
			Es folgt, daß \(B'[x]\) auch als \(A\)-Modul endlich erzeugt ist,
			so daß \(x\) nach der dritten Charakterisierung der Proposition
			endlich über \(A\) ist.
			\qedhere
		\end{enumerate}
	\end{proof}
\end{frame}

\begin{frame}{Ganze Abgeschlossenheit des ganzen Abschlusses}
	\begin{corollary}<+->
		Sei \(A \subset B\) eine Erweiterung kommutativer Ringe. Sei \(C\) der
		ganze Abschluß von \(A\) in \(B\). Dann ist \(C\) in \(B\) ganz
		abgeschlossen.
	\end{corollary}
	\begin{proof}<+->
		Sei \(x \in B\) ganz über \(C\). Da ganze Abhängigkeit transitiv ist,
		ist \(x\) damit auch ganz über \(A\). Damit ist \(x \in C\).
	\end{proof}
\end{frame}

\begin{frame}{Ganzheit in Quotienten und Lokalisierungen}
	\begin{proposition}<+->
		Sei \(A \subset B\) eine ganze Erweiterung kommutativer Ringe. Dann
		gilt:
		\begin{enumerate}[<+->]
		\item<.->
			Für jedes Ideal \(\ideal b\) von \(B\) ist \(B/\ideal b\) ganz über
			\(A/(A \cap \ideal b)\).
		\item
			Für jede multiplikativ abgeschlossene Teilmenge \(S \subset A\) ist
			\(S^{-1} B\) ganz über \(S^{-1} A\).
		\end{enumerate}
	\end{proposition}
	\begin{proof}<+->
		\begin{enumerate}[<+->]
		\item<.->
			Eine Gleichung \(x^n + a_1 x^{n - 1} + \dotsb + a_n = 0\) mit
			\(a_i \in A\) für ein \(x \in B\) können wir modulo \(\ideal b\)
			reduzieren.
		\item
			Sei \(\frac x s \in S^{-1} B\). Dann liefert die obige Gleichung
			\((\frac x s)^n + \frac{a_i} s (\frac x s)^{n - 1} + \dotsb +
			\frac{a_n} {s^n} = 0\)
			eine Ganzheitsbedingung über \(S^{-1} A\).
			\qedhere
		\end{enumerate}
	\end{proof}
\end{frame}

\subsection{Noethersche Normalisierung}

\begin{frame}{Ein schrecklicher Hilfssatz}
	\begin{lemma}<+->
		Sei \(M\) eine endliche Menge von Tupeln \(m = (m_1, \dotsc, m_n)
		\in \set N_0^n\). Dann existieren natürliche Zahlen \(w_1, \dotsc,
		w_n \in \set N\) mit \(w_n = 1\), so daß für alle
		\(m, m' \in M\) gilt: \(m \neq m' \implies
		w(m) \coloneqq
		\sum\limits_{i = 1}^n w_i m_i \neq
		w(m') = \sum\limits_{i = 1}^n w_i m_i\).
	\end{lemma}
	\begin{proof}<+->
		\begin{enumerate}[<+->]
		\item<.->
			Wir führen den Beweis nach Induktion über \(n\). Wir können
			\(n > 1\) annehmen. Ein \(m \in M\) schreiben wir als
			\(m = (m_1, m_{\ge 2})\) mit \(m_{\ge 2} = (m_2, \dotsc, m_n)\).
		\item
			Wir wenden die Induktionsvoraussetzung auf die vorkommenden
			\(m_{\ge 2}\) an und erhalten \(w_2, \dotsc, w_n \in \set N\) mit
			\(w_n = 1\).
		\item
			Wir wählen \(w_1 \in \set N\) so, daß
			\(w_1 > \sum\limits_{i = 2}^n w_i m_i\) für alle \(m \in M\).
			\qedhere
		\end{enumerate}
	\end{proof}
\end{frame}

\begin{frame}{Hilfssatz zur Noetherschen Normalisierung}
	\begin{lemma}<+->
		Sei \(K\) ein Körper. Sei \(f \in A \coloneqq K[x_1, \dotsc, x_n]\) mit
		\(f \neq 0\).
		Dann existiert ein Ringautomorphismus \(\phi\colon A \to A\) mit
		\(\phi(x_n) = x_n\), so daß
		\(
		\phi(f) = a_0 x_n^k + a_1 x_n^{k - 1} + \dotsb + a_k\)
		mit \(a_0 \in K^\units\) und \(a_1, \dotsc, a_k \in K[x_1, \dotsc, x_{n - 1}]\).
	\end{lemma}
\end{frame}

\begin{frame}{Beweis des Hilfssatzes}
	\begin{proof}<+->
		\begin{enumerate}[<+->]
		\item<.->
			Es gibt eine endliche Menge \(M\) von Tupeln \(m = (m_1, \dotsc,
			m_n) \in \set N_0^n\), so daß \(f = \sum\limits_{m \in M}
			a_m x_1^{m_1} \dotsm x_n^{m_n}\) mit \(a_m \in K^\units\). Wir
			wählen \(w_1, \dotsc, w_n \in \set N\) zu \(M\) wie im letzten
			Hilfssatz.
		\item
			Definiere \(\phi\colon A \to A\) mit \(\phi(x_i) = x_i + x_n^{w_i}\)
			für \(i < n\). Damit ist
			\(\phi(f) = f(x_1 + x_n^{w_1},
			\dotsc, x_{n - 1} + x_n^{w_{n - 1}}, x_n)\).
		\item
			Sei \(m \in M\) dasjenige Tupel, so daß \(\sum\limits_{i = 1}^n
			w_i m_i\) maximal wird. Dann ist der Term maximalen Grades in
			\(x_n\) in \(\phi(f)\) durch \(a_m x_n^{\sum w_i m_i}\) gegeben,
			und es ist \(a_m \in K^\units\).
		\qedhere
		\end{enumerate}
	\end{proof}
\end{frame}

\begin{frame}{Noethersche Normalisierung}
	\begin{proposition}<+->
		\label{prop:noether_norm}
		Seien \(K\) ein Körper und \(\ideal a \neq (1)\) ein Ideal in \(A \coloneqq K[x_1, \dotsc, x_n]\).
		Dann existieren ein Polynomring \(B \coloneqq K[y_1, \dotsc, y_n]\) und ein endlicher, injektiver
		Homomorphismus \(B \to A\) kommutativer \(K\)-Algebren und ein \(0 \leq r \leq n\), so daß
		\(B \cap \ideal a = (y_{r + 1}, \dotsc, y_n)\).
		\\
		Insbesondere folgt, daß \(K[y_1, \dotsc, y_r] \to A/\ideal a\) ein endlicher, injektiver Homomorphismus
		von \(K\)-Algebren ist.
	\end{proposition}
\end{frame}

\begin{frame}{Beweis der noetherschen Normalisierung}
	\begin{proof}<+->
		\begin{enumerate}[<+->]
		\item<.->
			Wir wenden Induktion über \(n\) an. Wir können \(n \ge 1\) und \(\ideal a \neq (0)\) annehmen.
		\item
			Sei \(f \in \ideal a \setminus \{0\}\). Nach dem Hilfssatz können wir
			davon ausgehen, daß \(f = 0\) eine
			Ganzheitsbedingung für \(x_n\) über \(A' \coloneqq K[x_1, \dotsc, x_{n - 1}]\) ist.
		\item
			Nach Induktionsvoraussetzung existiert ein endlicher, injektiver Homomorphismus \(B' \coloneqq K[y_1, \dotsc,
			y_{n - 1}] \to A'\), so daß \(B' \cap \ideal a = (y_{r + 1}, \dotsc, y_n)\) für ein \(r\). 
		\item
			Schließlich setzen wir \(B = B'[y_n] \to A = A'[x_n], y_n \mapsto f\).
			\qedhere
		\end{enumerate}
	\end{proof}
\end{frame}




\lecture{Ganz abgeschlossene Integritätsbereiche}
	{Ganz abgeschlossene Integrit\"atsbereiche}
\mode<all>\setcounter{section}{25}
\mode<all>\section{Erster Cohen--Seidenbergscher Satz}

\subsection{Körpererweiterungen}

\begin{frame}{Körper in einer ganzen Erweiterung}
	\begin{proposition}<+->
		\label{prop:fields_and_integral_extensions}
		Sei \(A \subset B\) eine ganze Erweiterung von Integritätsbereichen.
		Dann ist \(B\) genau dann ein Körper, wenn \(A\) ein Körper ist.
	\end{proposition}
	\begin{proof}<+->
		\begin{enumerate}[<+->]
		\item<.->
			Sei \(A\) ein Körper. Sei \(y \in B\) mit \(y \neq 0\). Sei
			\(y^n + a_1 y^{n - 1} + \dotsb + a_n = 0\) mit \(a_i \in A\) eine
			Ganzheitsbedingung minimalen Grades. Da \(B\) ein
			Integritätsbereich ist, ist \(a_n \neq 0\), also
			\(y^{-1} = -a_n^{-1} (y^{n - 1} + a_1 y^{n - 2} + \dotsb
			+ a_{n - 1}) \in B\). Damit ist \(B\) ein Körper.
		\item
			Sei umgekehrt \(B\) ein Körper. Sei \(x \in A\) mit \(x \neq 0\).
			Dann ist \(x^{-1} \in B\), also ganz über \(A\), so daß eine
			Ganzheitsbedingung \(x^{-n} + a_1 x^{-n + 1} + \dotsb + a_n = 0\)
			mit \(a_i \in A\) existiert. Es folgt, daß
			\(x^{-1} = -(a_1 + a_2 x + \dotsb + a_n x^{n - 1}) \in A\). Damit
			ist \(A\) ein Körper. 
			\qedhere
		\end{enumerate}
	\end{proof}
\end{frame}

\begin{frame}{Maximale Ideale in einer ganzen Erweiterung}
	\begin{corollary}<+->
		Sei \(A \subset B\) eine ganze Erweiterung kommutativer Ringe. Sei
		\(\ideal q\) ein Primideal in \(B\) und \(\ideal p \coloneqq A \cap
		\ideal q\). Dann ist \(\ideal q\) genau dann ein maximales Ideal, wenn
		\(\ideal p\) ein maximales Ideal ist.
	\end{corollary}
	\begin{proof}<+->
		Es ist \(A/\ideal p \subset B/\ideal q\) eine ganze Erweiterung
		von Integritätsbereichen. Es ist \(A/\ideal p\) genau dann ein Körper,
		wenn \(\ideal p\) maximal ist. Ebenso ist \(B/\ideal q\) genau dann
		ein Körper, wenn \(\ideal q\) maximal ist.
	\end{proof}
\end{frame}

\subsection{Primideale in ganzen Erweiterungen}

\begin{frame}{Primideale in einer ganzen Erweiterung}
	\begin{corollary}<+->
		\label{cor:equality_of_primes_in_integral_extension}
		Sei \(A \subset B\) eine ganze Erweiterung kommutativer Ringe. Seien
		\(\ideal q \subset \ideal q'\) zwei Primideale in \(B\) mit
		\(\ideal p \coloneqq A \cap \ideal q = A \cap \ideal q'\). Dann gilt
		\(\ideal q = \ideal q'\).
	\end{corollary}
	\begin{proof}<+->
		\begin{enumerate}[<+->]
		\item<.->
			Es ist \(B_{\ideal p}\) ganz über \(A_{\ideal p}\).
			Seien \(\ideal m \coloneqq A_{\ideal p} \ideal p\), \(\ideal n
			\coloneqq B_{\ideal p} \ideal q\) und
			\(\ideal n' \coloneqq B_{\ideal p} \ideal q'\).
		\item
			Dann ist \(\ideal m\) das maximale Ideal in \(A_{\ideal p}\).
			Weiter gilt \(\ideal n \subset \ideal n'\) und
			\(A_{\ideal p} \cap \ideal n = A_{\ideal p} \cap \ideal n' =
			\ideal m\).
		\item
			Aus der Maximalität von \(\ideal m\) folgt, daß \(\ideal n\)
			maximal ist, also \(\ideal n = \ideal n'\). Es folgt
			\(\ideal q = \ideal q'\).
			\qedhere
		\end{enumerate}
	\end{proof}
\end{frame}

\begin{frame}{Existenz von Primidealen in ganzen Ringerweiterungen}
	\begin{theorem}<+->
		\label{thm:existence_of_primes_in_integral_extensions}
		Sei \(A \subset B\) eine ganze Erweiterung kommutativer Ringe.
		Sei \(\ideal p\) ein Primideal von \(A\). Dann existiert ein
		Primideal \(\ideal q\) von \(B\) mit \(A \cap \ideal q = \ideal p\).
	\end{theorem}
	\begin{proof}<+->
		\begin{enumerate}[<+->]
		\item<.->
			Sei \(\ideal n\) ein maximales Ideal von \(B_{\ideal p}\). Da
			\(B_{\ideal p}\) ganz über \(A_{\ideal p}\) ist, ist
			\(\ideal m \coloneqq A_{\ideal p} \cap \ideal n\) ein maximales
			Ideal von \(A_{\ideal p}\) und damit das einzige maximale Ideal
			in \(A_{\ideal p}\).
		\item
			Sei \(\ideal q \coloneqq B \cap \ideal n\). Dann ist \(\ideal q\)
			ein Primideal in \(B\). Weiter ist
			\(A \cap \ideal q = A \cap (A_{\ideal p} \cap \ideal n)
			= A \cap \ideal m = \ideal p\).
			\qedhere
		\end{enumerate}
	\end{proof}
\end{frame}

\begin{frame}{Der erste Cohen--Seidenbergsche Satz}
	\begin{theorem}["`Going-up"']<+->
		Sei \(A \subset B\) eine ganze Erweiterung kommutativer Ringe. Sei
		\(\ideal p_1 \subset \dotsb \subset \ideal p_n\) eine Kette von
		Primidealen in \(A\) und
		\(\ideal q\colon \ideal q_1 \subset \dotsb \subset \ideal q_m\), \(m \le n\), eine
		Kette von Primidealen in \(B\) mit \(\ideal p_i = A \cap \ideal q_i\)
		für \(i \le m\). Dann kann die Kette \(\ideal q\) zu einer Kette
		\(\ideal q_1 \subset \dotsb \subset \ideal q_n\) mit \(A \cap
		\ideal q_i = \ideal p_i\) erweitert werden.
	\end{theorem}
	\begin{proof}<+->
		\begin{enumerate}[<+->]
		\item<.->
			Es reicht, den Fall \(m = 1\), \(n = 2\) zu behandeln. Seien \(\bar A
			\coloneqq A/\ideal p_1\) und \(\bar B \coloneqq B/\ideal q_1\).
			Dann ist \(\bar B\) ganz über \(\bar A\).
		\item
			 Damit existiert ein Primideal \(\bar{\ideal q}_2\) von \(\bar B\)
			 mit \(\bar A \cap \bar{\ideal q}_2 = \bar{\ideal p}_2
			 \coloneqq \bar A \ideal p_2\).
		\item
			Damit ist \(\ideal q_2 \coloneqq B \cap \bar{\ideal q}_2\) ein
			Primideal von \(B\) mit den gewünschten Eigenschaften.
			\qedhere
		\end{enumerate}
	\end{proof}
\end{frame}


\mode<all>\section{Der zweite Cohen--Seidenbergsche Satz}

\subsection{Ganz abgeschlossene Integritätsbereiche}

\begin{frame}{Lokalisierung des ganzen Abschlusses}
	\begin{proposition}<+->
		Sei \(A \subset B\) eine Erweiterung kommutativer Ringe.
		Sei \(C\) der ganze Abschluß von \(A\) in \(B\). Sei \(S \subset A\)
		multiplikativ abgeschlossen. Dann ist \(S^{-1} C\) der ganze Abschluß
		von \(S^{-1} A\) in \(S^{-1} B\).
	\end{proposition}
	\begin{proof}<+->
		\begin{enumerate}[<+->]
		\item<.->
			Wir haben schon gesehen, daß \(S^{-1} C\) ganz über \(S^{-1} A\) ist.
		\item
			Sei umgekehrt \(\frac b s \in S^{-1} B\) ganz über \(S^{-1} A\),
			das heißt, wir haben eine Gleichung der Form
			\((\frac b s)^n + \frac {a_1}{s_1} (\frac b s)^{n - 1}
			+ \dotsb + \frac{a_n}{s_n} = 0\) mit \(a_i \in A, s_i \in S\).
		\item
			Sei \(t \coloneqq s_1 \dotsm s_n\). Multiplizieren wir die
			Ganzheitsbedingung mit \((st)^n\) erhalten wir eine
			Ganzheitsbedingung für \(bt\) über \(A\).
		\item
			Damit ist \(bt \in C\), also
			\(\frac b s = \frac{bt}{st} \in S^{-1} C\).
			\qedhere
		\end{enumerate}
	\end{proof}
\end{frame}

\begin{frame}{Ganz abgeschlossene Integritätsbereiche}
	\begin{definition}<+->
		Ein Integritätsbereich heißt \emph{ganz abgeschlossen}, falls er
		in seinem Quotientenkörper ganz abgeschlossen ist.
	\end{definition}
	\begin{example}<+->
		Der Ring \(\set Z\) der ganzen Zahlen ist ganz abgeschlossen.
	\end{example}
	\mode<article>{Ganz ähnlich wird gezeigt:}
	\begin{example}<+->
		Ein Polynomring \(K[x_1, \dotsc, x_n]\) in mehreren Variablen über einem
		Körper ist ganz abgeschlossen.
	\end{example}
\end{frame}

\begin{frame}{Lokalität der ganzen Abgeschlossenheit}
	\begin{proposition}<+->
		Sei \(A\) ein Integritätsbereich. Die folgenden Eigenschaften sind
		äquivalent:
		\begin{enumerate}[<+->]
		\item<.->
			Es ist \(A\) ganz abgeschlossen.
		\item
			Für jedes Primideal \(\ideal p\) in \(A\) ist \(A_{\ideal p}\)
			ganz abgeschlossen.
		\item
			Für jedes maximale Ideal \(\ideal m\) in \(A\) ist \(A_{\ideal m}\)
			ganz abgeschlossen.
		\end{enumerate}
	\end{proposition}
	\begin{proof}<+->
		Seien \(K\) der Quotientenkörper von \(A\) und \(C\) der ganze Abschluß
		von \(A\) in \(K\). Dann ist \(A\) genau dann ganz abgeschlossen,
		wenn \(\phi\colon A \to C, x \mapsto x\) surjektiv ist. Es ist
		Surjektivität eine lokale Eigenschaft und Bilden des ganzen Abschlusses
		mit Lokalisierung verträglich.
	\end{proof}
\end{frame}

\subsection{Ganzheit über Idealen}

\begin{frame}{Definition der Ganzheit über Idealen}
	\begin{definition}<+->
		Sei \(A \subset B\) eine Erweiterung kommutativer Ringe. Sei
		\(\ideal a\) ein Ideal von \(A\).
		\begin{enumerate}[<+->]
		\item<.->
			Ein Element \(x \in B\) heißt
			\emph{ganz über \(\ideal a\)}, falls \(x\) eine Gleichung der
			Form \(x^n + a_1 x^{n - 1} + \dotsb + a_n = 0\) mit \(a_i \in \ideal a\)
			erfüllt.
		\item
			Der \emph{ganze Abschluß von \(\ideal a\) in \(B\)} ist die Menge
			aller \(x \in B\), die ganz über \(\ideal a\) sind.
		\end{enumerate}
	\end{definition}
\end{frame}

\begin{frame}{Erweiterungen in ganzen Abschlüssen}
	\begin{lemma}<+->
		Sei \(A \subset B\) eine Erweiterung kommutativer Ringe. Sei \(C\) der
		ganze Abschluß von \(A\) in \(B\). Sei \(\ideal a\) ein Ideal von
		\(A\). Dann ist der ganze Abschluß von \(\ideal a\) in \(B\)
		das Wurzelideal \(\sqrt{C \ideal a}\).
	\end{lemma}
	\begin{proof}<+->
		\begin{enumerate}[<+->]
		\item<.->
			Sei \(x \in B\) mit \(x^n + a_1 x^{n - 1} + \dotsb + a_n = 0\)
			für gewisse \(a_i \in \ideal a\).
			Es folgt \(x \in C\) und damit \(x^n \in C \ideal a\), also
			\(x \in \sqrt{C \ideal a}\).
		\item
			Sei umgekehrt \(x \in \sqrt{C \ideal a}\), also \(x^n = \sum_i
			a_i x_i\) für \(a_i \in \ideal a\) und \(x_i \in C\). Es ist
			\(M \coloneqq A[x_1, \dotsc, x_n]\) ein endlich erzeugter
			\(A\)-Modul mit \(x^n M \subset \ideal a M\). Damit erfüllt
			die Multiplikation mit \(x^n\) auf \(M\) eine Ganzheitsbedingung
			über \(\ideal a\). Also ist \(x^n\) und damit auch \(x\) ganz über
			\(\ideal a\).
			\qedhere
		\end{enumerate}
	\end{proof}
\end{frame}

\begin{frame}{Eine Proposition über über einem Ideal ganze Elemente}
	\begin{proposition}<+->
		Seien \(A \subset B\) eine Erweiterung von Integritätsbereichen, und
		sei \(A\) ganz abgeschlossen mit Quotientenkörper \(K\). Sei weiter
		\(x \in B\) ganz über einem Ideal \(\ideal a\) von \(A\). Dann ist
		\(x\) algebraisch über \(K\) und für sein Minimalpolynom
		\(f \in K[t]\) gilt \(f \in \sqrt{\ideal a}[t]\).
	\end{proposition}
	\begin{proof}<+->
		Da \(x\) ganz über \(\ideal a\) ist, ist \(x\) insbesondere algebraisch
		über \(K\). Sei \(L\) ein Zerfällungskörper von \(f\),
		und seien \(x_1, \dotsc, x_n\) die Nullstellen (mit Vielfachheit)
		von \(f\). Jedes \(x_i\) erfüllt dieselbe Ganzheitsbedingung wie \(x\),
		daher sind die \(x_i\) ganz über \(\ideal a\).
		Die Koeffizienten von \(f\) sind Polynome in den \(x_i\), also
		ebenfalls ganz über \(\ideal a\). Da \(A\) ganz abgeschlossen ist,
		müssen sie daher in \(\sqrt{\ideal a}\) liegen.
	\end{proof}
\end{frame}

\subsection{Der zweite Cohen--Seidenbergsche Satz}

\begin{frame}{Der zweite Cohen--Seidenbergsche Satz}
	\begin{theorem}["`Going-down"']<+->
		\label{thm:going_down}
		Sei \(A \subset B\) eine ganze Erweiterung von Integritätsbereichen,
		und sei \(A\) ganz abgeschlossen. Dann gilt: 
		Seien \(\ideal p_1 \supset \dotsb
		\supset \ideal p_n\) eine Kette von Primidealen in \(A\) und
		\(\ideal q \colon \ideal q_1 \supset \dotsb \supset \ideal q_m\),
		\(m \le n\)
		eine Kette von Primidealen von \(B\) mit \(\ideal p_i = A \cap \ideal
		q_i\) für \(i \le m\).
		Dann kann die Kette \(\ideal q\) zu einer Kette \(\ideal q_1 \supset
		\dotsb \supset \ideal q_n\) mit \(\ideal p_i = A \cap \ideal q_i\)
		erweitert werden.
	\end{theorem}
\end{frame}

\begin{frame}{Beweis des zweiten Cohen--Seidenbergschen Satzes}
	\begin{proof}<+->
		\begin{enumerate}[<+->]
		\item<.->
			Es reicht, den Fall \(m = 1\), \(n = 2\) zu behandeln. Damit ist zu
			zeigen, daß \(\ideal p_2\) die Kontraktion eines Primideals in
			\(B_{\ideal q_1}\) ist, das heißt, daß
			\(A \cap B_{\ideal q_1} \ideal p_2 = \ideal p_2\).
		\item
			Jedes \(x \in B_{\ideal q_1} \ideal p_2\) ist von der Form
			\(x = \frac y s\) mit \(y \in B \ideal p_2\) und \(s \in
			B \setminus \ideal q_1\). Dann ist \(y\) ganz über \(\ideal p_2\),
			die Ganzheitsbedingung minimalen Grades von \(y\) über dem
			Quotientenkörper \(K\) von \(A\)
			ist also von der Form \(y^r + u_1 y^{r - 1} + \dotsb + u_r = 0\)
			mit \(u_i \in \ideal p_2\).
		\item
			Sei zusätzlich \(x \in A\). Dann ist \(s = y x^{-1}\) mit \(x^{-1}
			\in K\). Damit ist \(s^r + v_1 s^{r - 1} + \dotsb + v_r = 0\) mit
			\(v_i = u_i x^{-i}\) die Ganzheitsbedingung minimalen Grades für
			\(s\) über \(K\). Wir halten \(x^i v_i = u_i \in \ideal p_2\) fest.
		\item
			Da \(s\) ganz über \((1)\) in \(A\) ist, folgt \(v_i \in A\).
			Angenommen, \(x \notin \ideal p_2\). Da \(\ideal p_2\) prim ist,
			muß dann \(v_i \in \ideal p_2\) gelten.
		\item
			Daraus folgt wiederum, daß \(s^r \in B \ideal p_2 \subset
			B \ideal p_1 \subset \ideal q_1\), also \(s \in \ideal q_1\),
			ein Widerspruch.
			\qedhere
		\end{enumerate}
	\end{proof}
\end{frame}

\begin{frame}{Ganzer Abschluß in separablen Erweiterungen}
	\begin{proposition}<+->
		Sei \(A\) ein ganz abgeschlossener Integritätsbereich mit Quotientenkörper
		\(K\). Sei \(L\) eine endliche separable Körpererweiterung von \(K\) und
		\(B\) der ganze Abschluß von \(A\) in \(L\). Dann existiert eine
		Basis \(v_1, \dotsc, v_n\) von \(L\) über \(K\) so daß
		\(B \subset \sum\limits_{j = 1}^n A v_j\).
	\end{proposition}
\end{frame}

\begin{frame}{Beweis der Proposition}
	\begin{proof}<+->
		\begin{enumerate}[<+->]
		\item<.->
			Ist \(v \in L\), erfüllt \(v\) eine Gleichung
			\(a_0 v^r + a_1 v^{r - 1} + \dotsb + a_n = 0\) mit \(a_i \in A, a_0
			\neq 0\).
			Durch Multiplizieren dieser Gleichung mit \(a_0^{r - 1}\) sehen wir,
			daß \(a_0 v\) ganz über \(A\) und damit in \(B\) ist. Damit können
			wir eine beliebige Basis von \(L\) über \(K\) so umnormieren,
			daß wir eine Basis \((u_1, \dotsc, u_n)\) mit \(u_i \in B\) erhalten.
		\item
			Da \(L\) über \(K\) separabel ist, existieren \(v_1, \dotsc, v_n
			\in L\) mit \(\tr_{L/K}(u_i v_j) = \kron_{ij}\).
		\item
			Sei \(x \in B\), etwa \(x = \sum\limits_j x_j v_j\) mit \(x_j \in K\).
			Weiter ist \(x u_i \in B\), also \(\tr_{L/K} (x u_i) \in A\), da die
			Spur eines Elementes ein Vielfaches eines Koeffizienten seines
			Minimalpolynomes ist.
		\item
			Es gilt \(\tr_{L/K} (x u_i) = \sum\limits_j x_j \tr_{L/K}(u_i v_j)
			= x_i\), also \(x_i \in A\). Folglich ist \(B \subset \sum\limits_j
			A v_j\).
			\qedhere
		\end{enumerate}
	\end{proof}
\end{frame}



\lecture{Bewertungsringe}{Bewertungsringe}
\mode<all>\setcounter{section}{27}
\mode<all>\section{Bewertungsringe}

\subsection{Definition und erste Eigenschaften von Bewertungsringen}

\begin{frame}{Definition eines Bewertungsringes}
	\begin{definition}<+->
		Sei \(B\) ein Integritätsbereich mit Quotientenkörper \(K\). Dann heißt
		\(B\) ein \emph{Bewertungsring für \(K\)}, falls 
		\(x \in B\) oder \(x^{-1} \in B\) (oder beides) für alle \(x \in K^\units\).
	\end{definition}
	\begin{example}<+->
		Seien \(K\) ein Körper und \(B\) ein Bewertungsring für \(K\). Ist
		dann \(B'\) ein Unterring von \(K\) mit \(B \subset B'\), so ist
		\(B'\) ein Bewertungsring für \(K\).
	\end{example}
\end{frame}

\begin{frame}{Bewertungsringe sind lokale Ringe}
	\begin{proposition}<+->
		Sei \(B\) ein Bewertungsring. Dann ist \(B\) ein lokaler Ring.
	\end{proposition}
	\begin{proof}<+->
		\begin{enumerate}[<+->]
		\item<.->
			Sei \(\ideal m \coloneqq B \setminus B^\units\). Ist \(K\) der
			Quotientenkörper von \(B\), so gilt damit für \(x \in K\), daß
			\(x \in \ideal m\) genau dann, wenn \(x = 0\) oder \(x^{-1} \notin B\).
		\item
			Ist \(a \in B\) und \(x \in \ideal m\), so ist \(ax \in \ideal m\),
			da ansonsten \((ax)^{-1} \in B\) und damit \(x^{-1} = a (ax)^{-1}
			\in B\).
		\item
			Sind \(x, y \in \ideal m\), so gilt \(xy^{-1} \in B\) oder
			\(x^{-1} y \in B\). Ohne Einschränkung sei \(xy^{-1} \in B\),
			also \(x + y = (1 + xy^{-1}) y \in B \ideal m \subset \ideal m\).
		\item
			Damit ist \(\ideal m\) ein Ideal in \(B\), und \(B\) daher ein
			lokaler Ring.
			\qedhere
		\end{enumerate}
	\end{proof}
\end{frame}

\begin{frame}{Ganze Abgeschlossenheit von Bewertungsringen}
	\begin{proposition}<+->
		Sei \(B\) ein Bewertungsring. Dann ist \(B\) ganz abgeschlossen.
	\end{proposition}
	\begin{proof}<+->
		Sei \(x \in K\) ein ganzes Element im Quotientenkörper \(K\) von \(B\),
		also \(x^n + b_1 x^{n - 1} + \dotsb + b_n = 0\) für gewisse \(b_i \in
		B\). Falls \(x \in B\), ist nichts zu zeigen. Ansonsten ist \(x^{-1}
		\in B\) und damit
		\(x = - (b_1 + b_2 x^{-1} + \dotsb + b_n x^{1 - n}) \in B\).
	\end{proof}
\end{frame}

\subsection{Existenz von Bewertungsringen}

\begin{frame}{Nicht fortsetzbare Homomorphismen in algebraisch abgeschlossene
	Körper}
	\begin{visibleenv}<+->
		Seien \(K\) ein Körper und \(L\) ein algebraisch abgeschlossener Körper.
		Wir wollen einen
		Ringhomomorphismus \(\phi\colon A \to L\) von einem Unterring \(A \subset K\)
		\emph{nicht fortsetzbar in \(K\)}
		nennen, falls für Unterringe \(A'\) von \(K\) mit \(A \subset
		A'\) und einem Ringhomomorphismus \(\phi'\colon A' \to L\) mit
		\(\phi'|_A = \phi\) schon \(A = A'\) gilt.
	\end{visibleenv}
	\begin{remark}<+->
		Aus dem Zornschen Lemma folgt, daß zu jedem Ringhomomorphismus
		\(\phi\colon A \to L\) ein Unterring \(B\) von \(K\) mit
		\(A \subset B\) und ein Ringhomomorphismus \(\psi\colon B \to K\)
		mit \(\psi|_A = \phi\) existiert, so daß \(\psi\) nicht fortsetzbar
		ist.
	\end{remark}
\end{frame}

\begin{frame}{Nicht fortsetzbare Homomorphismen sind auf lokalen Ringe definiert}
	\begin{lemma}<+->
		Seien \(K\) ein Körper und \(L\) ein algebraisch abgeschlossener Körper. Sei \(B
		\subset K\) ein	Unterring. Ist dann \(\psi\colon B \to L\) ein nicht
		in \(K\) fortsetzbarer Ringhomomorphismus, so ist \(B\) ein lokaler Ring mit
		maximalem Ideal \(\ideal m = \ker \psi\).
	\end{lemma}
	\begin{proof}<+->
		\begin{enumerate}[<+->]
		\item<.->
			Da \(\psi(B) \cong B/\ker \psi\) als Unterring der Körpers \(L\) ein
			Integritätsbereich ist, ist \(\ker \psi\) ein Primideal.
		\item
			Nach der universellen Eigenschaft der Lokalisierung faktorisiert
			\(\psi\) über einen Homomorphismus \(\psi_{\ideal m}\colon
			B_{\ideal m} \to L\).
		\item
			Da \(\psi\colon B \to L\) nicht fortsetzbar ist, folgt
			\(B = B_{\ideal m}\), also ist \(B\) ein lokaler Ring mit maximalem
			Ideal \(\ideal m\).
			\qedhere
		\end{enumerate}
	\end{proof}
\end{frame}

\begin{frame}{Ein Hilfssatz über nicht fortsetzbare Homomorphismen}
	\begin{lemma}<+->
		Seien \(K\) ein Körper und \(L\) ein algebraisch abgechlossener
		Körper. Sei \(B \subset K\) ein Unterring.
		Sei \(\psi\colon B \to L\) ein nicht in \(K\) fortsetzbarer Ringhomomorphismus.
		Sei \(\ideal m\) das maximale Ideal von \(B\). Sei weiter
		\(x \in K^\units\). Dann gilt
		\(\ideal m[x] \neq B[x]\) oder \(\ideal m[x^{-1}] \neq B[x^{-1}]\).
	\end{lemma}
	\begin{proof}<+->
		\begin{enumerate}[<+->]
		\item<.->
			Angenommen, \(\ideal m[x] = B[x]\) und \(\ideal m[x^{-1}]
			= B[x^{-1}]\). Dann gibt es Gleichungen
			\(u_0 + u_1 x + \dotsb + u_m x^m = 1\) und
			\(v_0 + v_1 x^{-1} + \dotsb + v_n x^{-n} = 1\) mit \(u_i, v_j \in
			\ideal m\). Seien weiter \(m, n\) minimal gewählt und ohne
			Einschränkung \(m \ge n\).
		\item
			Aus \((1 - v_0) x^n = v_1 x^{n - 1} + \dotsb + v_n\) und
			\((1 - v_0) \in B^\units\) folgt
			\(x^n = w_1 x^{n - 1} + \dotsb + w_n\)
			für gewisse \(w_j \in \ideal m\).
		\item
			Indem wir \(x^m = w_1 x^{m - 1} + \dotsb + w_n x^{m - n}\) in
			\(u_0 + u_1 x + \dotsb + u_m x^m = 1\) substituieren, erhalten wir,
			daß \(m\) nicht minimal gewählt ist. Widerspruch.
			\qedhere
		\end{enumerate}
	\end{proof}
\end{frame}

\begin{frame}{Nicht fortsetzbare Homomorphismen sind auf Bewertungsringen
	definiert}
	\begin{theorem}<+->
		\label{thm:existence_of_valuation_rings}
		Seien \(K\) ein Körper und \(L\) ein algebraisch abgeschlossener Körper.
		Sei \(B \subset K\) ein Unterring. Ist dann \(\psi\colon B \to L\)
		ein nicht in \(K\) fortsetzbarer Ringhomomorphismus, so ist \(B\) ein
		Bewertungsring für \(K\).
	\end{theorem}
\end{frame}

\begin{frame}{Beweis des Satzes über die Existenz von Bewertungsringen}
	\begin{proof}<+->
		\begin{enumerate}[<+->]
		\item<.->
			Mit \(\ideal m\) bezeichnen wir das maximale Ideal von \(B\).
			Sei \(x \in K^\units\). Wir haben \(x \in B\) oder \(x^{-1} \in B\)
			zu zeigen. Nach dem letzten Hilfssatz können wir ohne Einschränkung
			annehmen, daß \(\ideal m[x] \neq (1) = B' \coloneqq B[x]\).
		\item
			Damit ist \(\ideal m[x]\) in einem maximalen Ideal \(\ideal m'\) von
			\(B'\) enthalten. Da \(\ideal m' \cap B \subsetneq B\)
			folgt \(\ideal m' \cap B = \ideal m\).
		\item
			Die Einbettung \(B \to B'\) induziert damit eine Körpereinbettung
			\(k \coloneqq B/\ideal m \to k' \coloneqq B'/\ideal m'\).
			Sei \(\bar x\) das Bild von \(x\) in \(k'\). Damit
			\(k' = k[\bar x]\). Damit ist \(k[\bar x]\) ein Körper, \(\bar x\) also
			algebraisch über \(k\) und \(k'\) damit eine endlich algebraische
			Körpererweiterung von \(k\).
		\item
			Da \(\ideal m = \ker \psi\), induziert \(\psi\) einen Homomorphismus
			\(\bar\psi\colon k \to L\) von Körpern. Da \(L\) algebraisch
			abgeschlossen ist, können wir diesen zu einem Homomorphismus
			\(\bar\psi'\colon k' \to L\) fortsetzen. Verknüpfung mit
			\(B' \to k'\) liefert eine Fortsetzung von \(\psi\). Also \(B = B'\)
			und damit \(x \in B\).
			\qedhere
		\end{enumerate}
	\end{proof}
\end{frame}

\begin{frame}{Ganzer Abschluß als Schnitt über Bewertungsringe}
	\begin{corollary}<+->
		Sei \(A \subset K\) ein Unterring eines Körpers. Dann ist
		der ganze Abschluß \(\bar A\) von \(A\) in \(K\) der Schnitt aller
		Bewertungsringe \(B\) von \(K\) mit \(B \supset A\).
	\end{corollary}
	\begin{proof}<+->
		\begin{enumerate}[<+->]
		\item<.->
			Sei \(B\) ein Bewertungsring von \(K\) mit \(B \supset A\). Da
			\(B\) ganz abgeschlossen ist, folgt \(B \supset \bar A\).
		\item
			Sei umgekehrt \(x \notin \bar A\). Dann ist \(x \notin A' \coloneqq
			A[x^{-1}] \subset K\). Damit ist \(x^{-1}\) keine Einheit in \(A'\),
			also in einem maximalen Ideal \(\ideal m'\) von \(A'\) enthalten.
		\item
			Sei \(L\) ein algebraischer Abschluß des Körpers \(k' \coloneqq
			A'/\ideal m'\). Der Homomorphismus
			\(\phi\colon A \injto A' \surjto k' \injto L\) kann zu einem
			Bewertungsring \(B\) von \(K\) fortgesetzt werden. Da
			\(\phi(x^{-1}) = 0\) ist folglich \(x \notin B\).
			\qedhere
		\end{enumerate}
	\end{proof}
\end{frame}

\begin{frame}{Fortsetzbarkeit in endlich erzeugten Erweiterungen}
	\begin{proposition}<+->
		Sei \(A \subset B\) eine endlich erzeugte Erweiterung von
		Integritätsbereichen. Sei \(v \in B \setminus \{0\}\). Dann existiert
		ein \(u \in A \setminus \{0\}\) mit folgender Eigenschaft: Jeder
		Homomorphismus \(\phi\colon A \to L\) in einen algebraisch
		abgeschlossenen Körper mit \(\phi(u) \neq 0\) kann zu einem
		Homomorphismus \(\psi\colon B \to L\) mit \(\psi(v) \neq 0\)
		fortgesetzt werden.
	\end{proposition}
	\begin{proof}<+->
		\renewcommand\qedsymbol{}
		Mit Induktion über die Anzahl der Erzeuger von \(B\) über \(A\)
		können wir davon ausgehen, daß \(B\) von einem Element \(x\)
		erzeugt wird.
		\\
		Dann können zwei Fälle vorliegen: Entweder ist \(x\) über \(A\)
		transzendent, das heißt kein nicht
		verschwindendes Polynom über \(A\) besitzt \(x\) als Nullstelle.
		\\
		Oder \(x\) ist über \(A\) algebraisch, das heißt es gibt eine nicht
		triviale polynomielle Gleichung für \(x\) über dem Quotientenkörper von \(A\).
	\end{proof}
\end{frame}

\begin{frame}{Beweis für den transzendenten Fall}
	\begin{proof}[Transzendenter Fall]<+->
		\begin{enumerate}[<+->]
		\item<.->
			Sei \(x\) transzendent über \(A\).	
		\item
			Es existieren \(a_i \in A\) mit \(v = a_0 x^n + a_1 x^{n - 1} +
			\dotsb + a_n\). Wähle \(u = a_0\).
		\item
			Ist \(\phi\colon A \to L\) ein Homomorphismus mit
			\(\phi(u) \neq 0\), so existiert
			ein \(y \in L\) mit \(\phi(a_0) y^n + \phi(a_1) y^{n - 1} + \dotsb +
			\phi(a_n) \neq 0\), da \(L\) unendlich viele Elemente besitzt.
		\item
			Definiere \(\psi\colon B \to L\) durch \(\psi(x) = y\).
			\renewcommand\qedsymbol{}
			\qedhere
		\end{enumerate}
	\end{proof}
\end{frame}

\begin{frame}{Beweis für den algebraischen Fall}
	\begin{proof}[Algebraischer Fall]<+->
		\begin{enumerate}[<+->]
		\item<.->
			Sei \(x\) algebraisch über \(A\). Da \(v\) ein Polynom in \(x\)
			ist, ist damit auch \(v^{-1}\) algebraisch über \(x\). Damit
			existieren Gleichungen \(a_0 x^m + a_1 x^{m - 1} + \dotsb
			+ a_m = 0\) und \(a_0' v^{-n} + a_1' v^{1 - n} + \dotsb
			+ a_n' = 0\) mit \(a_i, a_j' \in A\) und \(u \coloneqq a_0 a_0'
			\neq 0\). 
		\item
			Sei \(\phi\colon A \to L\) mit \(\phi(u) \neq 0\). Dann kann
			\(\phi\) zunächst zu einem Homomorphismus \(A[u^{-1}] \to L\)
			fortgesetzt werden und danach zu einem Homomorphismus \(\chi\colon
			C \to L\), wobei \(C\) ein Bewertungsring mit
			\(C \supset A[u^{-1}]\) ist. 
		\item
			Aus der ersten der obigen Gleichungen folgt, daß \(x\) ganz über
			\(A[u^{-1}]\) ist, daß also \(x \in C\), so daß sogar \(C \supset
			B\), insbesondere also \(v \in C\).
		\item
			Analog ist \(v^{-1} \in C\), also ist \(v\) eine Einheit in \(C\),
			also ist \(\chi(v) \neq 0\). Definiere \(\psi\coloneqq \chi|_B\colon B \to L\).
			\qedhere
		\end{enumerate}
	\end{proof}
\end{frame}

\begin{frame}{Schwache Form des Hilbertschen Nullstellensatzes}
	Sei \(K\) ein Körper.
	\begin{corollary}<+->
		\label{cor:weak_hilbert1}
		Ist eine endlich erzeugte \(K\)-Algebra \(B\) ein Körper, so ist \(B\)
		eine endliche algebraische Erweiterung von \(K\).
	\end{corollary}
	\begin{proof}<+->
		Sei \(L\) ein algebraischer Abschluß von \(K\). Nach der Proposition
		läßt sich die Körpererweiterung \(K \subset L\) zu einer
		Körpererweiterung \(B \subset L\) fortsetzen. Damit ist \(B\)
		algebraisch über \(K\) und als endlich erzeugte Algebra damit auch
		endlich.
	\end{proof}
	\begin{corollary}[Schwacher Hilbertscher Nullstellensatz]<+->
		Ist \(\ideal m\) ein maximales Ideal einer endlich erzeugten
		\(K\)-Algebra \(A\), so ist \(A/\ideal m\)
		eine endliche algebraische Erweiterung von \(K\). Insbesondere
		ist \(A/\ideal m \cong K\), falls \(K\) algebraisch abgeschlossen ist.
		\qed
	\end{corollary}
\end{frame}



\lecture{Kettenbedingungen I}{Kettenbedingungen I}
\part<article>{Kettenbedingungen}
\mode<all>\setcounter{section}{28}
\mode<all>\section{Kettenbedingungen I}

\subsection{Kettenbedingungen}

\begin{frame}{Stationäre Folgen}
	\begin{proposition}<+->
		Sei \(X\) eine teilweise geordnete Menge. Dann sind folgende beide
		Bedingungen an \(X\) äquivalent:
		\begin{enumerate}[<+->]
		\item<.->
			Jede aufsteigende Folge \(x_1 \le x_2 \le \dotsb\) in \(X\) 
			ist stationär, das heißt es existiert ein \(n\) mit
			\(x_n = x_{n + 1} = \dotsb\).
		\item
			Jede nicht leere Teilmenge von \(X\) besitzt ein maximales Element.
		\end{enumerate}
	\end{proposition}
	\begin{proof}<+->
		\begin{enumerate}[<+->]
		\item<.->
			Angenommen, es gibt eine nicht leere Teilmenge ohne maximales
			Element. Dann können wir induktiv eine unendliche strikt
			aufsteigende Folge von Elementen in \(X\) konstruieren.
		\item
			Besitzt jede nicht leere Teilmenge ein maximales Element, so
			auch die Menge \((x_m)_{m \in \set N}\).
			\qedhere
		\end{enumerate}
	\end{proof}
\end{frame}

\begin{frame}{Kettenbedingungen für Untermoduln}
	Sei \(A\) ein Ring. Sei \(M\) ein \(A\)-Modul.
	\begin{definition}<+->
		Ist jede bezüglich der Inklusion aufsteigende Folge
		\(N_1 \subset N_2 \subset \dotsb\)
		von Untermoduln von \(M\) stationär, heißt \(M\) \emph{noethersch}.
	\end{definition}
	\begin{definition}<+->
		Ist jede bezüglich der Inklusion absteigende Folge
		\(N_1 \supset N_2 \supset \dotsb\) von Untermoduln von \(M\)
		stationär, heißt \(N\) \emph{artinsch}.
	\end{definition}
\end{frame}

\begin{frame}{Beispiele für noethersche und artinsche Moduln}
	\begin{example}<+->
		Jede endliche abelsche Gruppe ist als \(\set Z\)-Modul sowohl
		noethersch als auch artinsch.
	\end{example}
	\begin{example}<+->
		Der Ring \(\set Z\) der ganzen Zahlen ist als \(\set Z\)-Modul
		noethersch, aber nicht artinsch, denn wir haben
		\((a) \supsetneq (a^2) \supsetneq \dotsb\) für \(a \in \set Z
		\setminus \{0, \pm 1\}\).
	\end{example}
\end{frame}

\begin{frame}{Weitere Beispiele für noethersche und artinsche Moduln}
	Sei \(p\) eine Primzahl. Sei \(G\) die Gruppe der Elemente von
	\(\set Q/\set Z\) mit \(p\)-Potenzordnung. 
	\begin{example}<+->
		Es besitzt \(G\) genau
		eine Untergruppe \(G_n\) der Ordnung \(p^n\) für alle \(n \ge 0\)
		und \(G_0 \subsetneq G_1 \subsetneq \dotsb\), so daß \(G\) nicht 
		noethersch ist. Auf der anderen Seite sind die einzigen echten
		Untergruppen von \(G\) die \(G_n\), so daß \(G\) artinsch ist.
	\end{example}
	\begin{example}<+->
		Sei \(H\) die Gruppe aller rationalen Zahlen, in deren gekürzter
		Bruchdarstellung der Nenner eine \(p\)-Potenz ist. Dann ist \(H\)
		weder noethersch noch artinsch, denn es gibt eine exakte Sequenz
		\(0 \to \set Z \to H \to G \to 0\) und \(\set Z\) ist nicht artinsch
		und \(G\) ist nicht noethersch.
	\end{example}
\end{frame}

\begin{frame}{Beispiele mit Polynomringen}
	Sei \(K\) ein Körper.
	\begin{example}<+->
		Der Polynomring \(K[x]\) ist als Modul über sich
		selbst noethersch, das heißt, er erfüllt die Kettenbedingung für seine
		Ideale. Auf der anderen Seite ist er nicht artinsch.
	\end{example}
	\begin{example}<+->
		Der Polynomring \(K[x_1, x_2, \dotsc]\) in unendlich vielen Variablen
		ist weder noethersch noch artinsch: Die Folgen
		\((x_1) \subsetneq (x_1, x_2) \subsetneq \dotsb\) ist streng aufsteigend,
		die Folge \((x_1) \supsetneq (x_1^2) \supsetneq \dotsb\) ist streng
		absteigend.
	\end{example}
\end{frame}

\begin{frame}{Untermoduln noetherscher Moduln sind endlich erzeugt}
	\begin{proposition}<+->
		Sei \(A\) ein Ring. Ein \(A\)-Modul \(M\) ist genau dann noethersch,
		wenn alle seine Untermoduln endlich erzeugt sind.
	\end{proposition}
	\begin{proof}<+->
		\begin{enumerate}[<+->]
		\item<.->
			Sei \(N\) ein Untermodul eines noetherschen \(A\)-Moduls. Dann gibt es
			einen maximalen endlich erzeugten Untermodul \(N_0\) von \(N\). Aus der
			Maximalität folgt \(N_0 = N_0 + Ax\) für alle \(x \in N\), also \(N_0 = N\).
		\item
			Sei umgekehrt  jeder Untermodul von \(M\) endlich erzeugt, und sei
			\(M_1 \subset M_2 \subset \dotsb\) eine aufsteigende Kette von Untermoduln von \(M\).
			Dann ist \(N \coloneqq \bigcup\limits_{n = 1}^\infty M_i\) ein endlich erzeugter
			Untermodul. Alle Erzeuger sind schon in einem \(M_n\) enthalten, also \(M_n = N\),
			und damit ist die Kette stationär.
			\qedhere
		\end{enumerate}
	\end{proof}
\end{frame}

\mode<article>{Diese Proposition ist der Grund dafür, warum noethersche Moduln so wichtig sind.
Es stellt sich heraus, daß dies in der Praxis häufig die richtige Endlichkeitsbedingung ist.}

\begin{frame}{Noethersche und artinsche Moduln in exakten Sequenzen}
	\begin{proposition}<+->
		Sei \(A\) ein Ring. Sei \(0 \to M' \to M \to M'' \to 0\) eine exakte Sequenz
		von \(A\)-Moduln. Dann ist \(M\) genau dann noethersch, wenn \(M'\) und \(M''\)
		noethersch sind.
	\end{proposition}
	\begin{proof}<+->
		\begin{enumerate}[<+->]
		\item<.->
			Jede Kette von Untermoduln in \(M'\) bzw.\ \(M''\) liefert eine Kette
			von Untermoduln in \(M\). Ist sie dort stationär, so ist sie auch in
			\(M'\) bzw.\ \(M''\) stationär.
		\item
			Ist \((M_n)\) eine Kette von Untermoduln in \(M\) und ist ihr Urbild in \(M'\) und
			ihr Bild in \(M''\) stationär, so ist sie selbst stationär.
			\qedhere
		\end{enumerate}
	\end{proof}
	\begin{remark}<+->
		Eine entsprechende Aussage gilt auch für artinsche anstelle noetherscher Moduln.
	\end{remark}
\end{frame}

\begin{frame}{Endliche Summen noetherscher und artinscher Moduln}
	\begin{corollary}<+->
		Sei \(A\) ein Ring. Sind dann \(M_1, \dotsc, M_n\) noethersche \(A\)-Moduln,
		so ist auch \(\bigoplus\limits_{i = 1}^n M_i\) ein noetherscher \(A\)-Modul.
	\end{corollary}
	\begin{proof}<+->
		Wende Induktion und die Proposition auf die exakte Sequenz
		\[
			0 \to M_1 \to \bigoplus\limits_{i = 1}^n M_i \to \bigoplus_{i = 2}^n M_i \to 0
		\]
		an.
	\end{proof}
	\begin{remark}<+->
		Eine entsprechende Aussage gilt auch für artinsche anstelle noetherscher Moduln.
	\end{remark}
\end{frame}

\subsection{Noethersche und artinsche Ringe}

\begin{frame}{Definition noetherscher und artinscher Ringe}
	\begin{definition}<+->
		Ein Ring \(A\) heißt \emph{noethersch}, falls \(A\) als Modul über sich selbst
		noethersch ist.
	\end{definition}
	\begin{definition}<+->
		Ein Ring \(A\) heißt \emph{artinsch}, falls \(A\) als Modul über sich selbst
		artinsch ist.
	\end{definition}
	\begin{remark}<+->
		Eine Ring \(A\) ist also genau dann noethersch bzw.~artinsch, wenn jede Kette aufsteigender
		bzw.~absteigender Ideale in \(A\) stationär ist.
	\end{remark}
\end{frame}

\begin{frame}{Beispiele zu noetherschen und artinschen Ringen}
	\begin{example}<+->
		Jeder Körper ist sowohl ein noetherscher und artinscher Ring. Dasselbe gilt für die
		Ringe \(\set Z/(n)\) mit \(n > 0\).
	\end{example}
	\begin{example}<+->
		Der Ring \(\set Z\) der ganzen Zahlen ist noethersch, aber nicht artinsch.
	\end{example}
	\begin{example}<+->
		Jeder Hauptidealbereich ist noethersch, denn alle seine Ideale sind endlich erzeugt.
	\end{example}
\end{frame}

\begin{frame}{Beispiele nicht noetherscher Ringe}
	\begin{example}<+->
		Sei \(K\) ein Körper. Der Polynomring \(K[x_1, x_2, \dotsc]\) in unendlich vielen
		Variablen ist nicht noethersch. Als Integritätsbereich ist er aber in einem Körper
		enthalten. Damit sehen wir, daß ein Unterring eines noetherschen Ringes im allgemeinen
		nicht wieder noethersch sein muß.
	\end{example}
	\begin{example}<+->
		Sei \(X\) ein kompakter Hausdorffraum, und sei \(F_1 \supsetneq F_2 \supsetneq \dotsb\)
		eine streng absteigende Folge abgeschlossener Teilmengen. Sei \(\Cont(X)\) der Ring
		der stetigen reellwertigen Funktionen auf \(X\). Dann definieren die
		\(\ideal a_n \coloneqq \{f \in \Cont(X) \mid f|_{F_n} = 0\}\) eine streng aufsteigende
		Folge von Idealen in \(\Cont(X)\), es ist \(\Cont(X)\) also nicht noethersch.
	\end{example}
\end{frame}

\begin{frame}{Endlich erzeugte Moduln noetherscher und artinscher Ringe}
	\begin{proposition}<+->
		Sei \(A\) ein noetherscher Ring. Ist dann \(M\) ein endlich erzeugter
		\(A\)-Modul, so ist \(M\) noethersch.
	\end{proposition}
	\begin{proof}<+->
		Es ist \(M\) ein Quotient des \(A\)-Moduls \(A^n\), und Quotienten und Summen
		noetherscher Moduln sind wieder noethersch.
	\end{proof}
	\begin{remark}<+->
		Eine entsprechende Aussage gilt auch für artinsche anstelle von noetherschen
		Ringen.
	\end{remark}
\end{frame}

\begin{frame}{Quotienten noetherscher und artinscher Ringe}
	\begin{proposition}<+->
		Sei \(\ideal a\) ein Ideal eines noetherschen Ringes \(A\). Dann
		ist auch \(A/\ideal a\) ein noetherscher Ring.
	\end{proposition}
	\begin{proof}<+->
		Da Quotienten noetherscher Moduln noethersch sind, ist \(A/\ideal a\)
		ein noetherscher \(A\)-Modul. Damit ist \(A/\ideal a\) aber auch
		noethersch über sich selbst.
	\end{proof}
	\begin{remark}<+->
		Eine entsprechende Aussage gilt auch für artinsche anstelle von noetherschen
		Ringen.
	\end{remark}
\end{frame}



\lecture{Kettenbedingungen II}{Kettenbedingungen II}
\mode<all>\setcounter{section}{29}
\mode<all>\section{Kettenbedingungen II}

\subsection{Kompositionsreihen und Länge eines Moduls}

\begin{frame}{Ketten von Untermoduln}
	Sei \(A\) ein Ring.
	\begin{definition}<+->
		Sei \(M\) ein \(A\)-Modul. Eine \emph{Untermodulkette von
		\(M\)} ist eine Kette von Untermoduln von \(M\) der Form
		\(M_\bullet\colon M = M_0 \supsetneq M_1 \supsetneq \dotsb \supsetneq M_n = 0\).
		Es heißt \(n\) die \emph{Länge von \(M_\bullet\)}.
	\end{definition}
	\begin{definition}<+->
		Ein \(A\)-Modul \(M\) heißt \emph{einfach}, falls er außer \(0\) und
		sich selbst keine weiteren Untermoduln besitzt.
	\end{definition}
	\begin{definition}<+->
		Sei \(M\) ein \(A\)-Modul. Eine Untermodulkette \(M_\bullet\) von \(M\)
		heißt eine \emph{Kompositionsreihe}, falls jeder Quotient
		\(M_i/M_{i + 1}\) ein einfacher \(A\)-Modul ist.
	\end{definition}
	\begin{visibleenv}<+->
		Eine Kompositionsreihe ist also eine maximale Untermodulkette.
	\end{visibleenv}
\end{frame}

\begin{frame}{Länge eines Moduls}
	\begin{definition}<+->
		Sei \(A\) ein Ring und \(M\) ein \(A\)-Modul.
		Die \emph{Länge \(\ell(M)\) von \(M\)} ist das Infimum über die Längen aller
		Kompositionsreihen von \(M\).
	\end{definition}
	\begin{visibleenv}<+->
		Es ist also \(\ell(M) = \infty\) genau dann, wenn \(M\) keine Kompositionsreihe
		besitzt.
	\end{visibleenv}
\end{frame}

\begin{frame}{Längen echter Untermoduln}
	\begin{lemma}<+->
		Sei \(A\) ein Ring und \(M\) ein \(A\)-Modul endlicher Länge.
		Für jeden echten Untermodul \(N\) von \(M\) gilt \(\ell(N) < \ell(M)\).
	\end{lemma}
	\begin{proof}<+->
		\begin{enumerate}[<+->]
		\item<.->
			Sei \(M_\bullet\) eine Kompositionsreihe von \(M\) minimaler Länge \(n\). Sei \(N_i
			\coloneqq N \cap M_i\). Dann ist \(N_i/N_{i + 1} \subset M_i/M_{i + 1}\).
		\item
			Da \(M_i/M_{i + 1}\) einfach ist, ist \(N_i/N_{i + 1} = M_i/M_{i + 1}\) oder
			\(N_i = N_{i + 1}\). Indem wir also wiederholte Moduln herausnehmen, erhalten
			wir eine Kompositionsreihe von \(N\) und insbesondere \(\ell(N) \leq \ell(M)\).
		\item
			Angenommen, \(\ell(N) = \ell(M)\). Dann ist \(N_i/N_{i + 1} = M_i/M_{i + 1}\),
			also \(M_{n - 1} = N_{n - 1}, M_{n - 2} = N_{n - 2}, \dotsc, M_0 = N_0\) und
			damit \(M = N\).
			\qedhere
		\end{enumerate}
	\end{proof}
\end{frame}

\begin{frame}{Längen von Untermodulketten}
	\begin{lemma}<+->
		Sei \(A\) ein Ring. Sei \(M\) ein \(A\)-Modul endlicher Länge. Die Länge einer jeden
		Untermodulkette von \(M\) ist höchstens \(\ell(M)\).
	\end{lemma}
	\begin{proof}<+->
		Ist \(M = M_0 \supsetneq M_1 \supsetneq \dotsb \supsetneq M_k = 0\) eine Untermodulkette
		der Länge \(k\), so haben wir nach dem letzten Hilfssatz, daß
		\(\ell(M) > \ell(M_1) > \dotsb > \ell(M_k)\), also \(\ell(M) \ge k\).
	\end{proof}
\end{frame}

\begin{frame}{Invarianz der Länge}
	\begin{proposition}<+->
		Sei \(A\) ein Ring. Sei \(M\) ein \(A\)-Modul, welcher eine Kompositionsreihe
		der Länge \(n\) besitze. Dann hat jede Kompositionsreihe die Länge \(n\),
		und jede Untermodulkette von \(M\) läßt sich zu einer Kompositionsreihe
		erweitern.
	\end{proposition}
	\begin{proof}<+->
		\begin{enumerate}[<+->]
		\item<.->
			Sei \(M_\bullet\) eine Kompositionsreihe von \(M\) der Länge \(k\).
			Nach dem letzten Hilfssatz wissen wir, daß \(k \le \ell(M) \leq n\). Nach
			Definition von \(\ell(M)\) folgt daraus \(k = \ell(M) \leq n\).
		\item
			Sei \(M_\bullet\) eine beliebige Untermodulfolge von \(M\) der Länge \(k\).
			Ist \(k < \ell(M)\), so ist \(M_\bullet\) keine Kompositionsreihe, also nicht
			maximal, also kann \(M_\bullet\) zu einer Folge der Länge \(k + 1\)
			erweitert werden.
		\item
			Ist \(k = \ell(M)\), so muß \(M_\bullet\) schon eine Kompositionsreihe sein,
			denn ansonsten könnte \(M_\bullet\) zu einer Folge der Länge
			\(\ell(M) + 1\) erweitert werden.
			\qedhere
		\end{enumerate}
	\end{proof}
\end{frame}

\subsection{Moduln endlicher Länge}

\begin{frame}{Moduln mit Kompositionsreihen}
	\begin{proposition}<+->
		Sei \(A\) ein kommutativer Ring. Ein \(A\)-Modul \(M\)
		besitzt genau dann eine Kompositionsreihe, wenn \(M\)
		noethersch und artinsch ist.
	\end{proposition}
	\begin{proof}<+->
		\begin{enumerate}[<+->]
		\item<.->
			Besitzt \(M\) eine Kompositionsreihe, so sind alle streng
			monotonen Ketten von Untermoduln endlich und damit \(M\)
			sowohl noethersch als auch artinsch.
		\item
			Sei \(M\) noethersch und artinsch. Da \(M_0 \coloneqq M\) noethersch
			ist, existiert ein maximaler echter Untermodul \(M_1 \subsetneq M_0\).
			Analog besitzt \(M_1\) einen maximalen echten Untermodul \(M_2 \subsetneq M_1\),
			usw.
			\\
			Wir erhalten eine streng absteigende Kette \(M_0 \supsetneq M_1
			\supsetneq \dotsb\), welche endlich sein muß, da \(M\) artinsch ist.
			Dies ist damit eine Kompositionsreihe von \(M\).
			\qedhere
		\end{enumerate}
	\end{proof}
\end{frame}

\mode<article>{Folglich definieren wir:}

\begin{frame}{Moduln endlicher Länge}
	Sei \(A\) ein Ring.
	\begin{definition}<+->
		Ist ein \(A\)-Modul \(M\) noethersch und artinsch, so heißt \(M\) \emph{von endlicher Länge}.
	\end{definition}
	\begin{remark}[Jordan--Hölderscher Satz]<+->
		Sind \(M_\bullet\) und \(M'_\bullet\) zwei Kompositionsreihen eines \(A\)-Moduls
		\(M\) endlicher Länge \(n\), so existiert ein \(\sigma \in \SG_n\) mit
		\(M_{i - 1}/M_i \cong M'_{\sigma(i) - 1}/M'_{\sigma(i)}\).
	\end{remark}
\end{frame}

\begin{frame}{Die Länge ist eine additive Funktion}
	\begin{proposition}<+->
		Sei \(A\) ein Ring.
		Die Länge ist eine additive Funktion auf der Klasse aller \(A\)-Moduln
		endlicher Länge.
	\end{proposition}
	\begin{proof}<+->
		\begin{enumerate}[<+->]
		\item<.->
			Sei \(0 \to M' \xrightarrow{\phi} M \xrightarrow{\psi} M'' \to 0\) eine exakte Sequenz von
			\(A\)-Moduln endlicher Länge. Wir müssen \(\ell(M') + \ell(M'') = \ell(M)\) zeigen.
		\item
			Sei \(M'_0 \supsetneq M'_1 \supsetneq \dotsb \supsetneq M'_m\) eine Kompositionsreihe in \(M'\)
			und \(M''_0 \supsetneq M''_1 \supsetneq \dotsb \supsetneq M''_n\) eine Kompositionsreihe in \(M''\).
			Dann ist \(\psi^{-1}(M''_0) \supsetneq \dotsb \supsetneq \psi^{-1}(M''_n)
			= \phi(M'_0) \supsetneq \dotsb \supsetneq \phi(M'_m)\) eine Kompositionsreihe von \(M\).
			\qedhere
		\end{enumerate}
	\end{proof}
\end{frame}

\begin{frame}{Vektorräume endlicher Länge}
	\begin{proposition}<+->
		Für einen Vektorraum \(V\) über einem Körper \(K\) sind folgende
		Aussagen äquivalent.
		\begin{enumerate}[<+->]
		\item<.->
			Es ist \(V\) endlich-dimensional.
		\item
			Es hat \(V\) endliche Länge.
		\item
			Es ist \(V\) noethersch.
		\item
			Es ist \(V\) artinsch.
		\end{enumerate}
		\begin{visibleenv}<+->
			In diesen Fällen gilt außerdem \(\ell(V) = \dim V\).
		\end{visibleenv}
	\end{proposition}
\end{frame}

\begin{frame}{Beweis der Proposition über Vektorräume endlicher Länge}
	\begin{proof}<+->
		\begin{enumerate}[<+->]
		\item<.->
			Besitzt \(V\) eine Basis \((e_1, \dotsc, e_n)\), so bilden die Unterräume
			\(U_i \coloneqq \gen{e_{i + 1}, \dotsc, e_n}\) eine Kompositionsreihe von \(V\).
		\item
			Nach Definition ist \(V\) noethersch und artinsch, wenn \(V\) von endlicher Länge ist.
		\item
			Sei \(V\) ein unendlich-dimensionaler Vektorraum, das heißt \(V\) besitzt
			eine unendliche Familie \((v_n)_{n \in \set N}\) linear unabhängiger Elemente.
			Dann bilden die Unterräume \(U_i \coloneqq \gen{v_1, \dotsc, v_i}\) eine
			nicht stationäre aufsteigende Kette und die Unterräume
			\(W_i \coloneqq \gen{v_{i + 1}, v_{i + 2}, \dotsc}\) eine nicht stationäre absteigende Kette.
			\qedhere
		\end{enumerate}
	\end{proof}
\end{frame}

\begin{frame}{Kommutative Ringe, deren Nullideal Produkt maximaler Ideale ist}
	\begin{corollary}<+->
		\label{cor:zero_is_prod_of_max}
		Seien \(\ideal m_1, \dotsc, \ideal m_n\) maximale Ideale in einem kommutativen
		Ring \(A\), so daß \(\ideal m_1 \dotsm \ideal m_n = (0)\). Dann ist
		\(A\) genau dann noethersch, wenn \(A\) artinsch ist.
	\end{corollary}
	\begin{proof}<+->
	\begin{enumerate}[<+->]
	\item<.->
		Der Ring \(A\) besitzt die Idealkette \(A \supset \ideal m_1 \supset
		\ideal m_1 \ideal m_2 \supset \dotsb \supset \ideal m_1 \dotsm \ideal m_n = (0)\),
		und jeder Faktor \((\ideal m_1 \dotsm \ideal m_{i - 1})/(\ideal m_1 \dotsm \ideal m_i)\)
		ist ein Vektorraum über dem Körper \(A/\ideal m_i\).
	\item
		Damit ist jeder Faktor genau dann artinsch, wenn er noethersch ist.
	\item
		Es ist \(A\) genau dann noethersch bzw.~artinsch, wenn jeder Faktor noethersch
		bzw.~artinsch ist.
		\qedhere
	\end{enumerate}
	\end{proof}
\end{frame}


\lecture{Noethersche Ringe}{Noethersche Ringe}
\part<article>{Noethersche Ringe}
\mode<all>\setcounter{section}{30}
\mode<all>\section{Noethersche Ringe}

\subsection{Elementare Eigenschaften noetherscher Ringe}

\begin{frame}{Bilder noetherscher Ringe}
	Wir erinnern an folgende Möglichkeiten, einen Ring \(A\) als noethersch zu
	charakterisieren:
	\begin{enumerate}[<+->]
	\item
		Jede nicht leere Menge von Idealen von \(A\) besitzt ein maximales Element.
	\item
		Jede aufsteigende Kette von Idealen in \(A\) ist stationär.
	\item
		Jedes Ideal in \(A\) ist endlich erzeugt.
	\end{enumerate}
	\begin{proposition}<+->
		Sei \(\phi\colon A \to B\) ein Homomorphismus von Ringen. Ist \(A\)
		noethersch, so ist auch \(\phi(A)\) noethersch.
	\end{proposition}
	\begin{proof}<+->
		Nach dem Homomorphiesatz ist \(\phi(A) \cong A/\ideal \ker \phi\), und
		Quotienten noetherscher Ringe sind noethersch.
	\end{proof}
\end{frame}

\begin{frame}{Endliche Erweiterungen noetherscher Ringe}
	\begin{proposition}<+->
		Sei \(A \subset B\) eine endliche Erweiterung kommutativer Ringe.
		Sei \(A\) noethersch. Dann ist auch \(B\) noethersch.
	\end{proposition}
	\begin{proof}<+->
		Da \(B\) als \(A\)-Modul endlich erzeugt ist und \(A\) ein noetherscher Ring
		ist, ist \(B\) als \(A\)-Modul noethersch. Damit ist \(B\) auch als
		Modul über sich selbst noethersch.
	\end{proof}
	\begin{example}<+->
		Der Ring \(\set Z[i]\) der ganzen Gaußschen Zahlen eine endliche Erweiterung
		von \(\set Z\) und damit noethersch.
	\end{example}
	\begin{remark}<+->
		Allgemeiner ist ganze Abschluß von \(\set Z\) in einer beliebigen endlichen
		algebraischen Erweiterung von \(\set Q\) ein noetherscher Ring.
	\end{remark}
\end{frame}

\begin{frame}{Lokalisierungen noetherscher Ringe}
	Sei \(A\) ein noetherscher kommutativer Ring. 
	\begin{proposition}<+->
		Sei \(S \subset A\) multiplikativ
		abgeschlossen. Dann ist \(S^{-1} A\) noethersch.
	\end{proposition}
	\begin{proof}<+->
		\begin{enumerate}[<+->]
		\item<.->
			Jedes Ideal in \(S^{-1} A\) ist von der Form \(S^{-1} \ideal a\), wobei
			\(\ideal a\) ein Ideal in \(A\) ist.
		\item
			Erzeugen \(x_1, \dotsc, x_n\) das Ideal \(\ideal a\), so wird \(S^{-1} \ideal a\)
			von \(\frac {x_1} 1, \dotsc, \frac{x_n} 1\) erzeugt.
			\qedhere
		\end{enumerate}
	\end{proof}
	\begin{corollary}<+->
		Die Halme \(A_{\ideal p}\) von \(A\) sind noethersch.
		\qed
	\end{corollary}
\end{frame}

\subsection{Der Hilbertsche Basissatz}

\begin{frame}{Der Hilbertsche Basissatz}
	\begin{theorem}[Hilbertscher Basissatz]<+->
		Ist \(A\) ein noetherscher kommutativer Ring, so ist auch der Polynomring \(A[x]\) noethersch.
	\end{theorem}
\end{frame}

\begin{frame}{Beweis des Hilbertschen Basissatzes}
	\begin{proof}<+->
		\begin{enumerate}[<+->]
		\item<.->
			Sei \(\ideal a\) ein Ideal in \(A[x]\). Die führenden Koeffizienten der
			Polynome in \(\ideal a\) erzeugen ein Ideal \(\ideal l\) in \(A\), welches
			nach Voraussetzung von endlich vielen Elementen \(a_1, \dotsc, a_n\) erzeugt ist.
		\item
			Zu jedem \(a_i\) wählen wir ein Polynom \(f_i \in \ideal a\) mit Leitmonom
			\(a_i x^{r_i}\). Sei \(r\) das Maximum der \(r_i\). Die \(f_i\)
			erzeugen ein Ideal \(\ideal a' \subset \ideal a\).
		\item
			Sei \(f \in \ideal a\) ein Polynom mit Leitmonom \(a x^m\).
			Ist \(m \ge r\), so existieren \(u_i \in A\) mit \(a = \sum\limits_i u_i a_i\).
			Dann ist \(f - \sum\limits_i u_i f_i x^{m - r_i} \in \ideal a\) ein Polynom echt kleineren Grades 
			als \(f\). Indem wir so fortfahren, können wir \(f = g + h\) mit \(h \in \ideal a'\)
			schreiben, wobei \(\deg g < r\).
		\item
			Ist \(M\) der von \(1, x, \dots, x^{r - 1}\) erzeugte \(A\)-Modul, so haben
			wir \(\ideal a = (\ideal a \cap M) + \ideal a'\) gezeigt.
		\item
			Es ist \(\ideal a \cap M\) als Untermodul eines noetherschen Moduls endlich erzeugt, 
			etwa von \(g_1, \dotsc, g_m\). Damit ist \(\ideal a\) von
			\(g_1, \dotsc, g_m, f_1, \dotsc, f_n\) erzeugt.
			\qedhere
		\end{enumerate}
	\end{proof}
\end{frame}

\begin{frame}{Folgerungen aus dem Hilbertschen Basissatz}
	\begin{corollary}<+->
		Ist \(A\) ein noetherscher kommutativer Ring, so ist auch der Polynomring \(A[x_1, \dotsc, x_n]\) in
		endlich vielen Variablen noethersch.
	\end{corollary}
	\begin{proof}<+->
		Folgt aus dem Hilbertschen Basissatz mittels Induktion über \(n\).
	\end{proof}
	\begin{corollary}<+->
		Sei \(A\) ein noetherscher Ring. Dann ist jede endlich erzeugte kommutative
		\(A\)-Algebra \(B\) noethersch.
	\end{corollary}
	\begin{proof}<+->
		Es ist \(B\) ein Bild eines Polynomringes der Form \(A[x_1, \dotsc, x_n]\) unter einem
		Ringhomomorphismus.
	\end{proof}
\end{frame}

\begin{frame}{Zwischenringe in endlich erzeugten Erweiterungen}
	\begin{proposition}<+->
		\label{prop:intermediate_ring}
		Seien \(A \subset B \subset C\) Erweiterungen kommutativer Ringe. Seien \(A\) noethersch
		und \(C\) endlich erzeugt als \(A\)-Algebra. Sei weiter \(C\) endlich als \(B\)-Algebra.
		Dann ist \(B\) endlich erzeugt als \(A\)-Algebra.
	\end{proposition}
	\begin{remark}<+->
		In der Situation der Proposition ist es äquivalent zu fordern, daß \(C\) ganz als \(B\)-Algebra ist.
	\end{remark}
\end{frame}

\begin{frame}{Beweis der Proposition über Zwischenringe}
	\begin{proof}<+->
		\begin{enumerate}[<+->]
		\item<.->
			Seien \(x_1, \dotsc, x_m\) Erzeuger von \(C\) als \(A\)-Algebra. Seien weiter
			\(y_1, \dotsc, y_n\) Erzeuger von \(C\) als \(B\)-Modul. Damit existieren
			\(b_i^j \in B\) mit \(x_i = \sum\limits_j b_i^j y_j\) und \(b_{ij}^k \in B\)
			mit \(y_i y_j = \sum\limits_k b_{ij}^k y_k\). Sei \(B_0\) die von den \(b_i^j\) und
			\(b_{ij}^k\) erzeugte \(A\)-Unteralgebra von \(B\). Da \(A\)
			noethersch, ist auch \(B_0\) noethersch.
		\item
			Jedes Element in \(C\) ist ein Polynom in den \(x_i\) mit Koeffizienten in \(A\).
			Durch wiederholtes Einsetzen der Ausdrücke für \(x_i\) und \(y_i y_j\) erhalten wir,
			daß jedes Element in \(C\) eine Linearkombination der \(y_j\) mit Koeffizienten
			in \(B_0\) ist, und damit ist \(C\) eine endliche \(B_0\)-Algebra.
		\item
			Da \(B_0\) noethersch ist, ist damit \(C\) ein noetherscher \(B_0\)-Modul.
			Da \(B\) Untermodul von	\(C\), ist \(B\) als \(B_0\)-Modul endlich erzeugt.
		\item
			Schließlich ist \(B_0\) als \(A\)-Algebra endlich erzeugt, also ist
			auch \(B\) als \(A\)-Algebra endlich erzeugt.
		\qedhere
		\end{enumerate}
	\end{proof}
\end{frame}



\lecture{Primärzerlegung in noetherschen Ringen}{Prim\"arzerlegung in noetherschen Ringen}
\mode<all>\setcounter{section}{31}
\mode<all>\section{Primärzerlegung in noetherschen Ringen}

\subsection{Irreduzible Ideale}

\begin{frame}{Definition eines irreduziblen Ideals}
	Sei \(A\) ein kommutativer Ring.
	\begin{definition}<+->
		Ein Ideal \(\ideal a \neq (1)\) in \(A\) heißt \emph{irreduzibel}, falls
		für je zwei weitere Ideale \(\ideal b, \ideal c\) von \(A\) gilt:
		\(\ideal a = \ideal b \cap \ideal c \implies (\ideal a = \ideal b
		\lor \ideal a = \ideal c)\).
	\end{definition}
	\begin{lemma}<+->
		\label{lem:lasker1}
		Ist \(A\) noethersch, so ist jedes Ideal \(\ideal a\) ein endlicher Schnitt
		irreduzibler Ideale.
	\end{lemma}
	\begin{proof}<+->
		Angenommen, es gibt ein Ideal, welches sich nicht als endlicher Schnitt
		irreduzibler Ideale schreiben läßt. Dann gibt es auch ein solches
		maximales \(\ideal a\). Da \(\ideal a\) selbst nicht irreduzibel sein kann,
		existieren Ideale \(\ideal b, \ideal c\) mit \(\ideal a = \ideal b \cap \ideal c\),
		aber \(\ideal a \subsetneq \ideal b, \ideal c\). Damit sind \(\ideal b, \ideal c\)
		jeweils Schnitte endlich vieler irreduzibler Ideale, also auch \(\ideal a\).
		Widerspruch.
	\end{proof}
\end{frame}

\begin{frame}{Irreduzible Ideale in noetherschen Ringen}
	\begin{lemma}<+->
		\label{lem:lasker2}
		Sei \(\ideal a\) ein irreduzibles Ideal in einem noetherschen kommutativen
		Ring \(A\). Dann ist \(\ideal a\) ein Primärideal.
	\end{lemma}
	\begin{proof}<+->
		\begin{enumerate}[<+->]
		\item<.->
			Indem wir von \(A\) zu \(A/\ideal a\) übergehen, können wir uns auf den
			Fall beschränken, daß das Nullideal primär ist, wenn es irreduzibel ist.
		\item
			Sei also \(xy = 0\) mit \(y \neq 0\). Die Kette \(\ann(x) \subset \ann(x^2)
			\subset \dotsb\) ist stationär, also ist \(\ann(x^n) = \ann(x^{n + 1})\) für
			ein \(n\).
		\item
			Sei \(a \in (x^n) \cap (y)\). Dann ist \(a = b x^n\) für ein \(b \in A\) und
			\(a x = b x^{n + 1} = 0\). Insbesondere \(b \in \ann(x^{n + 1}) = \ann(x^{n})\),
			also \(a = b x^n = 0\). Folglich \((x^n) \cap (y) = (0)\).
		\item
			Ist \((0)\) irreduzibel, so folgt wegen \((y) \neq (0)\) damit \((x^n) = 0\),
			also \(x^n = 0\).
			\qedhere
		\end{enumerate}
	\end{proof}
\end{frame}

\subsection{Existenz der Primärzerlegung in noetherschen Ringen}

\begin{frame}{Existenz der Primärzerlegung in noetherschen Ringen}
	Sei \(A\) ein noetherscher kommutativer Ring.
	\begin{theorem}<+->
		In \(A\) ist jedes Ideal zerlegbar.
		\qed
	\end{theorem}
	\begin{proposition}<+->
		Jedes Ideal \(\ideal a\) in \(A\) enthält eine Potenz seines Wurzelideals.
	\end{proposition}
	\begin{proof}<+->
		\begin{enumerate}[<+->]
		\item<.->
			Seien \(x_1, \dotsc, x_k\) Erzeuger von \(\sqrt{\ideal a}\). Dann gilt
			\(x_i^{n_i} \in \ideal a\) für gewisse \(n_i\). Sei \(m \coloneqq
			\sum\limits_i (n_i - 1) + 1\).
		\item
			Es wird \(\sqrt{\ideal a}^m\) von Produkten \(x_1^{r_1} \dotsm x_k^{r_k}\) mit
			\(\sum\limits_i r_i = m\) erzeugt. Nach Definition von \(m\) gilt
			\(r_i \ge n_i\) für mindestens ein \(i\) und damit
			\(x_1^{r_1} \dotsm x_k^{r_k} \in \ideal a\).
			\qedhere
		\end{enumerate}
	\end{proof}
\end{frame}

\begin{frame}{Das Nilradikal in noetherschen Ringen ist nilpotent}
	Sei \(A\) ein noetherscher kommutativer Ring.
	\begin{corollary}<+->
		Das Nilradikal in \(A\) ist nilpotent.
	\end{corollary}
	\begin{proof}<+->
		Eine Potenz von \(\sqrt{(0)}\) ist in \((0)\) enthalten.
	\end{proof}
\end{frame}

\begin{frame}{Primäre Ideale zu maximalen Idealen in noetherschen Ringen}
	\begin{corollary}<+->
		Für ein maximales Ideal \(\ideal m\) und ein weiteres Ideal \(\ideal q\) eines
		noetherschen kommutativen Ringes \(A\) sind
		äquivalent:
		\begin{enumerate}[<+->]
		\item<.->
			Es ist \(\ideal q\) ein \(\ideal m\)-primäres Ideal.
		\item
			Es ist \(\sqrt{\ideal q} = \ideal m\).
		\item
			Es ist \(\ideal m^n \subset \ideal q \subset \ideal m\) für \(n \gg 0\).
		\end{enumerate}
	\end{corollary}
	\begin{proof}<+->
		Die Äquivalenz der ersten beiden Aussagen gilt in beliebigen kommutativen Ringen. Daß aus der
		zweiten die dritte folgt, ist die Aussage der Proposition. Die zweite folgt aus der dritten so:
		\(\ideal m = \sqrt{\ideal m^n} \subset \sqrt{\ideal q} \subset \sqrt{\ideal m} = \ideal m\).
	\end{proof}
\end{frame}

\begin{frame}{Assoziierte Primideale in noetherschen Ringen}
	\begin{proposition}<+->
		Sei \(\ideal a\) ein Ideal in einem noetherschen kommutativen Ring \(A\). Dann sind die
		zu \(\ideal a\) assoziierten Primideale genau die Primideale in \(A\) von der Form
		\((\ideal a : x)\) mit \(x \in A\).
	\end{proposition}
	\begin{proof}<+->
		Es reicht, die Aussage für \(A/\ideal a\) und das Nullideal zu beweisen, da die eigentliche
		Aussage dann durch Kontraktion nach \(A\) folgt. Wir zeigen also, daß in einem noetherschen
		Ring die zu \((0)\) assoziierten Primideale gerade die Primideale der Form \(\ann(x) = (0 : x)\)
		mit \(x \in A\) sind.
		\renewcommand{\qedsymbol}{}
	\end{proof}
\end{frame}

\begin{frame}{Fortsetzung des Beweises zu assoziierten Primidealen}
	\begin{proof}[Fortsetzung]<+->
		\begin{enumerate}[<+->]
		\item<.->
			Sei \(\bigcap\limits_{i = 1}^n \ideal q_i = (0)\) eine minimale Primärzerlegung. Sei \(\ideal p_i
			\coloneqq \sqrt{\ideal q_i}\). Sei \(\ideal a_i = \bigcap\limits_{j \neq i} \ideal q_j \neq (0)\).
			Wir hatten schon einmal \(\sqrt{\ann(x)} = \ideal p_i\) für alle \(x \in \ideal a_i \setminus \{0\}\)
			gezeigt, also \(\ann(x) \subset \ideal p_i\).%
			% XXX Das sollte vielleicht zu einem Hilfssatz werden
		\item
			Da \(\ideal p_i = \sqrt{\ideal q_i}\), ist \(\ideal p_i^m \subset \ideal q_i\) für \(m \gg 0\),
			also \(\ideal a_i \ideal p_i^m \subset \ideal a_i \cap \ideal p_i^m \subset \ideal a_i \cap \ideal q_i = (0)\).
			Sei \(m\) jetzt die kleinste natürliche Zahl mit \(\ideal a_i \ideal p_i^m = (0)\) und \(x \in \ideal a_i
			\ideal p_i^{m - 1} \setminus \{0\}\).
		\item
			Dann ist \(\ideal p_i x = 0\), also \(\ann(x) \supset \ideal p_i \), also \(\ann(x) = \ideal p_i\).
		\item
			Ist umgekehrt \(\ann(x)\) ein Primideal, so haben wir schon gezeigt, daß \(\ann(x) = \sqrt{\ann(x)}\) ein
			assoziiertes Primideal ist.
			\qedhere
		\end{enumerate}
	\end{proof}
\end{frame}



\lecture{Artinsche Ringe}{Artinsche Ringe}
\part<article>{Artinsche Ringe}
\mode<all>\setcounter{section}{32}
\mode<all>\section{Artinsche Ringe}

\subsection{Elementare Eigenschaften Artinscher Ringe}

\begin{frame}{Primideale in artinschen Ringen}
	Wir erinnern an folgende Möglichkeiten, einen Ring \(A\) als artinsch zu
	charakterisieren:
	\begin{enumerate}[<+->]
	\item
		Jede nicht leere Menge von Idealen von \(A\) besitzt ein minimales Element.
	\item
		Jede absteigende Kette von Idealen in \(A\) ist stationär.
	\end{enumerate}
	\mode<article>{Die anscheinende Symmetrie mit noetherschen Ringen ist
	irreführend. Und zwar werden wir sehen, daß jeder artinsche Ring
	notwendigerweise ein noetherscher Ring und zudem von besonders einfacher
	Gestalt ist.}
	\begin{proposition}<+->
		Sei \(A\) ein artinscher kommutativer Ring. Dann ist jedes Primideal
		in \(A\) maximal.
	\end{proposition}
	\begin{proof}<+->
		Ist \(\ideal p\) ein Primideal in \(A\), so ist
		\(B \coloneqq A/\ideal p\) ein artinscher Integritätsbereich.
		Sei \(x \in B \setminus \{0\}\). Dann ist \((x^n) = (x^{n + 1})\) für
		ein \(n \gg 0\), da \(B\) artinsch ist, also \(x^n = x^{n + 1} y\)
		für \(y \in B\). Kürzen mit \(x^n\) liefert \(1 = xy\), also ist
		\(x \in B^\units\). Damit ist \(B\) ein Körper und \(\ideal p\) ein
		maximales Ideal.
	\end{proof}
\end{frame}

\begin{frame}{Maximale Ideale in artinschen Ringen}
	\begin{corollary}<+->
		In einem artinschen kommutativen Ring ist das Nilradikal gleich dem
		Jacobsonschen Radikal.
		\qed
	\end{corollary}
	\begin{proposition}<+->
		Ein artinscher kommutativer Ringe hat nur endlich viele maximale Ideale.
	\end{proposition}
	\begin{proof}<+->
		\begin{enumerate}[<+->]
		\item<.->
			Die Menge aller endlichen Schnitte maximaler Ideale hat ein minimales
			Element, etwa \(\ideal m_1 \cap \dotsb \cap \ideal m_n\), wobei die
			\(\ideal m_i\) maximale Ideale sind.
		\item
			Für jedes weitere maximale Ideal
			\(\ideal m\) haben wir damit \(\ideal m \supset
			\ideal m_1 \cap \dotsb \cap \ideal m_n\).
		\item
			Da \(\ideal m\) ein Primideal
			ist, folgt \(\ideal m \supset \ideal m_i\) für ein \(i\), aufgrund
			der Maximalität also \(\ideal m = \ideal m_i\).
			\qedhere
		\end{enumerate}
	\end{proof}
\end{frame}

\begin{frame}{Nilpotentes Nilradikal in artinschen Ringen}
	\begin{proposition}<+->
		In einem kommutativen artinschen Ring ist das Nilradikal \(\ideal n\)
		nilpotent.
	\end{proposition}
	\begin{proof}<+->
		\begin{enumerate}[<+->]
		\item<.->
			Da der Ring artinsch ist, existiert ein \(k \gg 0\) mit
			\(\ideal a \coloneqq \ideal n^k = \ideal n^{k + 1} = \dotsb\).
			Angenommen, \(\ideal a \neq (0)\).
		\item
			Die Menge der Ideale \(\ideal b\) mit \(\ideal a \ideal b \neq (0)\)
			ist nicht leer, da insbesondere \(\ideal a\) ein solches Ideal ist,
			besitzt also ein minimales Element \(\ideal c\).
		\item
			Es existiert ein \(x \in \ideal c\) mit \(x \ideal a \neq (0)\).
			Weiter ist \((x) \subset \ideal c\), also \((x) = \ideal c\)
			aufgrund der Minimalität von \(\ideal c\).
		\item
			Weiter ist \((x \ideal a) \ideal a = x \ideal a^2 = x \ideal a
			\neq (0)\) und sowieso \(x \ideal a \subset (x)\), also
			\(x \ideal a = (x)\) aufgrund der Minimalität von \((x)\).
		\item
			Damit ist \(x = xy\) für ein \(y \in \ideal a\), also
			\(x = xy = xy^2 = \dotsb = xy^n\). Da \(y\) insbesondere nilpotent
			ist, ist folglich \(x = 0\), Widerspruch.
			\qedhere
		\end{enumerate}
	\end{proof}
\end{frame}

\subsection{Der Struktursatz für artinsche kommutative Ringe}

\begin{frame}{Dimension kommutativer Ringe}
	Sei \(A\) ein kommutativer Ring.
	\begin{definition}<+->
		Eine \emph{Primidealkette in \(A\)} ist eine streng aufsteigende
		Kette \(\ideal p_0 \subsetneq \ideal p_1 \subsetneq \dotsb
		\subsetneq \ideal p_n\) von Primidealen in \(A\). Es heißt \(n\)
		die \emph{Länge} der Kette.
	\end{definition}
	\begin{definition}<+->
		Die \emph{Dimension \(\dim A\) von \(A\)} ist das Supremum der Längen
		aller Primidealketten in \(A\).
	\end{definition}
	\begin{example}<+->
		Es ist \(\dim A \ge 0\) genau dann, wenn \(A\) nicht der Nullring ist.
	\end{example}
\end{frame}

\begin{frame}{Beispiele für Dimensionen}
	\begin{example}<+->
		Die Dimension eines Körpers ist \(0\).
	\end{example}
	\begin{example}<+->
		Für den Ring der ganzen Zahlen gilt \(\dim \set Z = 1\).
	\end{example}
\end{frame}

\begin{frame}{Das Nullideal in artinschen kommutativen Ringen}
	\begin{lemma}<+->
		\label{lem:zero_is_prod_of_max}
		Seien \(\ideal m_1, \dotsc, \ideal m_n\) die maximalen Ideale
		eines artinschen kommutativen Ringes \(A\). Dann ist
		\(\prod\limits_{i = 1}^n \ideal m_i^k = (0)\) für \(k \gg 0\).
	\end{lemma}
	\begin{proof}<+->
		\begin{enumerate}[<+->]
		\item<.->
			Da jedes Primideal in \(A\) maximal ist, gilt \(\ideal n = \ideal m_1
			\cap \dotsb \cap \ideal m_n\) für das Nilradikal \(\ideal n\) von
			\(A\).
		\item
			Da das Nilradikal in \(A\) nilpotent ist, gilt
			\(\prod\limits_{i = 1}^n \ideal m_i^k
			\subset (\bigcap\limits_{i = 1}^n \ideal m_i)^k = \ideal n^k = (0)\)
			für \(k \gg 0\).
			\qedhere
		\end{enumerate}
	\end{proof}
\end{frame}

\begin{frame}{Dimensionscharakterisierung artinscher kommutativer Ringe}
	\begin{theorem}<+->
		Ein kommutativer Ring \(A \neq 0\) ist genau dann artinsch, wenn er
		noethersch ist und \(\dim A = 0\) gilt.
	\end{theorem}
\end{frame}

\begin{frame}{Beweis der Dimensionscharakterisierung}
	\begin{proof}<+->
		\begin{enumerate}[<+->]
		\item<.->
			Ist \(A\) artinsch, so sind alle Primideale maximal, also \(\dim
			A = 0\). Weiter ist nach dem letzten Lemma das Nullideal Produkt
			von maximalen Idealen. Damit sind äquivalent, daß \(A\) artinsch und
			noethersch ist.
		\item
			Sei \(A\) noethersch mit \(\dim A = 0\). Da das Nullideal eine
			Primärzerlegung besitzt, gibt es nur endlich viele minimale
			Primideale. Wegen \(\dim A = 0\) sind alle diese maximal.
		\item
			Das Nilradikal ist also Schnitt endlich vieler maximaler Ideale.
			Da das Nilradikal in noetherschen Ringen nilpotent ist, ist
			das Nullideal Produkt endlich vieler
			maximaler Ideale. Damit sind äquivalent, daß \(A\) artinsch und
			noethersch ist.
			\qedhere
		\end{enumerate}
	\end{proof}
\end{frame}

\begin{frame}{Ein kommutativer Ring mit nur einem Primideal}
	\begin{example}<+->
		Ein kommutativer Ring mit genau einem Primideal muß nicht
		notwendigerweise noethersch (und damit auch nicht artinsch) sein: Sei
		etwa \(A = K[x_1, x_2, \dotsc]\) der Polynomring in unendlich vielen
		Variablen über einem Körper \(K\).
		\\
		Sei \(\ideal a \coloneqq (x_1, x_2^2, \dotsc)\). Dann besitzt
		\(B \coloneqq A/\ideal a\) genau ein Primideal, nämlich das Bild von
		\((x_1, x_2, \dotsc)\).
		\\
		Damit ist \(B\) ein lokaler Ring der Dimension \(0\). Allerdings ist
		\(B\) nicht noethersch, denn z.B.~sein Primideal ist nicht endlich
		erzeugt.
	\end{example}
\end{frame}

\begin{frame}{Artinsche lokale Ringe}
	Ist \((A, \ideal m)\) ein artinscher lokaler Ring, so ist \(\ideal m\)
	das einzige Primideal von \(A\) und damit gleich dem Nilradikal von \(A\).
	\\
	Insbesondere ist jedes Element aus \(\ideal m\) nilpotent, und \(\ideal m\)
	selbst ist nilpotent.
	\\
	Jedes Element aus \(A\) ist damit entweder eine Einheit oder nilpotent.
	\begin{example}<+->
		Sei \(p\) ein Primzahl und \(n \ge 1\). Dann ist \(\set Z/(p^n)\) ein
		artinscher lokaler Ring mit maximalem Ideal \(\set Z/(p^n) (p)\).
	\end{example}
\end{frame}

\begin{frame}{Nilpotenz des maximalen Ideals charakterisiert lokale artinsche Ringe}
	\begin{proposition}<+->
		Sei \((A, \ideal m)\) ein noetherscher lokaler Ring. Dann tritt genau
		einer der beiden folgenden Fälle ein:
		\begin{enumerate}[<+->]
		\item<.->
			Für alle \(n \in \set N_0\) ist \(\ideal m^n \supsetneq
			\ideal m^{n + 1}\).
		\item
			Es ist \(\ideal m^n = (0)\) für \(n \gg 0\), in welchem Falle
			\(A\) artinsch ist.
		\end{enumerate}
	\end{proposition}
	\begin{proof}<+->
		Ist \(\ideal m^n = \ideal m^{n + 1}\), so folgt mit dem
		Nakayamaschen Lemma, daß \(\ideal m^n = 0\).
		\\
		Insbesondere ist dann
		\(\ideal m^n \subset \ideal p\) für ein Primideal \(\ideal p\). Nach
		Wurzelziehen erhalten wir \(\ideal m = \ideal p\), also besitzt \(A\)
		dann nur ein Primideal, ist also artinsch.
	\end{proof}
\end{frame}

\begin{frame}{Der Struktursatz für artinsche kommutative Ringe}
	\begin{theorem}[Struktursatz für artinsche kommutative Ringe]<+->
		\label{thm:artin_structure}
		Ein artinscher kommutativer Ring ist (eindeutig bis auf Isomorphie
		der Faktoren) ein direktes Produkt artinscher lokaler
		Ringe.
	\end{theorem}
	\begin{proof}[Beweis der Existenz]<+->
		\renewcommand{\qedsymbol}{}
		\begin{enumerate}[<+->]
		\item<.->
			Seien \(\ideal m_1, \dotsb, \ideal m_n\) die verschiedenen maximalen
			Ideale eines artinschen Ringes \(A\). Nach dem letzten Hilfssatz
			ist \(\prod\limits_{i = 1}^n \ideal m_i^k = (0)\) für \(k \gg 0\).
			Da die \(\ideal m_i = \sqrt{\ideal m_i^k}\) paarweise koprim sind,
			sind auch die \(\ideal m_i^k\) paarweise koprim.
		\item
			Damit gilt \(\bigcap\limits_i \ideal m_i^k
			= \prod\limits_i \ideal m_i^k = (0)\). Damit ist insbesondere
			der kanonische Homomorphismus \(A \to \prod\limits A/\ideal m_i^k\)
			ein Isomorphismus. Da die \(A/\ideal m_i^k\) jeweils lokale
			artinsche Ringe sind, ist \(A\) direktes Produkt artinscher lokaler
			Ringe.
			\qedhere
		\end{enumerate}
	\end{proof}
\end{frame}

\begin{frame}{Eindeutigkeit im Struktursatz}
	\begin{proof}[Beweis der Eindeutigkeit]<+->
		\begin{enumerate}[<+->]
		\item<.->
			Sei umgekehrt \(A \cong \prod\limits_{i = 1}^n A_i\) für einen
			kommutativen Ring \(A\), wobei die
			\(A_i\) lokale artinsche kommutative Ringe sind. Seien
			\(\pi_i\colon A \to A_i\) die Projektionen und \(\ideal a_i
			\coloneqq \ker\pi_i\). Die \(\ideal a_i\) sind paarweise koprim
			mit \(\bigcap\limits_i \ideal a_i = (0)\).
		\item
			Sei \(\ideal q_i\) das einzige Primideal in \(A_i\), und sei
			\(\ideal p_i \coloneqq A \cap \ideal q_i\). Dann ist \(\ideal p_i\)
			ein maximales Primideal in \(A\).
		\item
			Da \((0)\) in \(A_i\) ein \(\ideal q_i\)-primäres Ideal ist, ist \(\ideal a_i\) in \(A\)
			ein \(\ideal p_i\)-primäres Ideal. Damit ist \(\bigcap\limits_i
			\ideal a_i = (0)\) eine Primärzerlegung.
		\item
			Da die \(\ideal a_i\) paarweise koprim sind, sind die \(\ideal p_i\)
			ebenso paarweise koprim, also die zu \((0)\)
			assoziierten isolierten Primideale.
		\item
			Damit sind alle Primkomponenten von \(A\) isoliert, also nach
			dem zweiten Eindeutigkeitssatz durch \(A\) eindeutig bestimmt.
			Also sind die \(A_i \cong A/\ideal a_i\) durch \(A\) eindeutig
			bestimmt.
			\qedhere
		\end{enumerate}
	\end{proof}
\end{frame}

\subsection{Artinsche lokale Ringe}

\begin{frame}{Der Zariskische Kotangentialraum}
	Sei \((A, \ideal m, F)\) ein lokaler Ring. Dann ist
	die spezielle Faser \(\ideal m/\ideal m^2\) von \(\ideal m\) ein \(F\)-Vektorraum.
	\begin{definition}<+->
		Der \(F\)-Vektorraum \(\ideal m/\ideal m^2\) ist der
		\emph{Zariskische Kotangentialraum von \(A\)}.
	\end{definition}
	\begin{visibleenv}<+->
		Ist \(\ideal m\) endlich erzeugt (z.B.~weil \(A\) noethersch ist),
		ist auch \(\ideal m/\ideal m^2\) als \(F\)-Vektorraum endlich erzeugt.
		Es folgt, daß in diesem Falle \(\dim_F \ideal m/\ideal m^2 < \infty\).
	\end{visibleenv}
	
	\begin{visibleenv}<+->
		Wir erinnern an die Tatsache, daß sich jede Basis des Zariskischen
		Kotangentialraums \(\ideal m/\ideal m^2\) zu einem Erzeugendensystem von \(\ideal m\)
		hochheben läßt.
	\end{visibleenv}
\end{frame}

\begin{frame}{Artinsche lokale Ringe, deren maximales Ideal ein Hauptideal ist}
	\begin{lemma}<+->
		Sei \((A, \ideal m)\) ein artinscher lokaler Ring, so daß \(\ideal m\) ein
		Hauptideal ist. Dann ist jedes Ideal \(\ideal a\) von \(A\) ein Hauptideal.
	\end{lemma}
	\begin{proof}<+->
		\begin{enumerate}[<+->]
		\item<.->
			Sei \(\ideal m = (x)\) für ein \(x \in A\).
			Wir können \(\ideal a \neq (0)\) annehmen. Da \(\ideal m\) auch das Nilradikal
			ist, ist \(\ideal m\) nilpotent. Damit existiert ein \(r \in \set N_0\) mit
			\(\ideal a \subset \ideal m^r\), aber \(\ideal a \subsetneq \ideal m^{r + 1}\).
		\item
			Folglich existiert ein \(a \in A\) mit \(y \coloneqq ax^r \in \ideal a\), aber
			\(y \notin (x^{r + 1})\). Es folgt \(a \notin (x)\), also ist \(a\) eine Einheit in
			\(A\).
		\item
			Damit ist \(x^r \in \ideal a\), also \(\ideal m^r = (x^r) \subset \ideal a\),
			also \(\ideal a = \ideal m^r = (x^r)\). Also ist \(\ideal a\) ein Hauptideal.
			\qedhere
		\end{enumerate}
	\end{proof}
\end{frame}

\begin{frame}{Artinsche lokale Ringe}
	\begin{proposition}<+->
		Für einen artinschen lokalen Ring \((A, \ideal m, F)\) sind folgende Aussagen äquivalent:
		\begin{enumerate}[<+->]
		\item<.->
			Jedes Ideal in \(A\) ist ein Hauptideal.
		\item
			Das maximale Ideal \(\ideal m\) ist ein Hauptideal.
		\item
			Es ist \(\dim_F \ideal m/\ideal m^2 \leq 1\).
		\end{enumerate}
	\end{proposition}
	\begin{proof}<+->
		\begin{enumerate}[<+->]
		\item<.->
			Daß aus der ersten die zweite und aus der zweite die dritte Aussage folgt ist offensichtlich.
		\item
			Ist \(\dim_F \ideal m/\ideal m^2 = 0\), also \(\ideal m = \ideal m^2\), so folgt \(\ideal m = 0\)
			nach dem Nakayamaschen Lemma. Damit ist \(A\) ein Körper und nichts weiter zu zeigen.
		\item
			Ist \(\dim_F \ideal m/\ideal m^2 = 1\), so wird \(\ideal m\) von einem Element \(x \in A\) erzeugt.
			Dann schließen wir mit dem Hilfssatz.
			\qedhere
		\end{enumerate}
	\end{proof}
\end{frame}

\begin{frame}{Beispiele zu lokalen artinschen Ringen}
	Sei \(K\) ein Körper.
	\begin{example}<+->
		Die Ringe \(\set Z/(p^n)\) und \(K[x]/(f^n)\), wobei \(p\) eine Primzahl ist und \(f\) ein
		irreduzibles Polynom, sind artinsche lokale Ringe, deren maximales Ideal
		von einem Element (nämlich dem Bild von \(p\) beziehungsweise \(f\)) erzeugt wird.
	\end{example}
	\begin{example}<+->
		Im artinschen lokalen Ring \(K[x^2, x^3]/(x^4)\) ist das maximale Ideal \(\ideal m\)
		von den zwei Elementen \(x^2\) und \(x^3\) modulo \(x^4\) erzeugt. Damit ist
		\(\ideal m^2 = (0)\), also \(\dim_K \ideal m/\ideal m^2 = 2\).
	\end{example}
\end{frame}




\lecture{Diskrete Bewertungsringe}{Diskrete Bewertungsringe}
\part<article>{Diskrete Bewertungsringe und Dedekindsche Bereiche}
\mode<all>\setcounter{section}{33}
\mode<all>\input{eindimensionale}
\mode<all>\section{Diskrete Bewertungsringe}

\subsection{Eindimensionale noethersche Integritätsbereiche}

\begin{frame}{Eindeutige Faktorisierung in Primärideale}
	\begin{proposition}<+->
		Sei \(A\) ein eindimensionaler noetherscher Integritätsbereich. Dann kann
		jedes Ideal \(\ideal a \neq (0)\) von \(A\) eindeutig als Produkt primärer
		Ideale mit paarweise verschiedenen Wurzelidealen geschrieben werden.
	\end{proposition}
	\begin{proof}<+->
		\begin{enumerate}[<+->]
		\item<.->
			Da \(A\) noethersch ist, existiert eine  minimale Primärzerlegung
			\(\ideal a = \bigcap\limits_{i = 1}^n \ideal q_i\). Die \(\ideal p_i
			\coloneqq \sqrt{\ideal q_i}\) sind Primideale ungleich dem Nullideal, welches
			selbst ein Primideal ist. Wegen \(\dim A = 1\) sind die
			\(\ideal p_i\) paarweise verschiedene maximale Ideale.
		\item
			Es folgt, daß die \(\ideal q_i\) paarweise koprim sind, also
			\(\ideal a = \bigcap\limits_i \ideal q_i = \prod\limits_i \ideal q_i\).
		\item
			Ist umgekehrt \(\ideal a = \prod\limits_i \ideal q_i\) mit
			\(\ideal p_i\)-primären Idealen \(\ideal q_i\), so folgt analog \(\ideal a = \bigcap\limits_i \ideal q_i\).
			Dies ist eine minimale Primärzerlegung von \(\ideal a\), in dem die
			\(\ideal q_i\) isolierte Komponenten sind.
			\qedhere
		\end{enumerate}
	\end{proof}
\end{frame}

\begin{frame}{Eindeutige Faktorisierung in Primideale}
	\begin{remark}<+->
		Ist \(A\) ein eindimensionaler noetherscher Integritätsbereich, in dem jedes Primärideal
		Potenz eines Primideals ist, so folgt aus der Proposition, daß jedes nicht verschwindende
		Ideal in \(A\) eine eindeutige Faktorisierung in Primideale besitzt.
		\\
		Für jedes Primideal \(\ideal p \neq (0)\) besitzt der Halm \(A_{\ideal p}\) dann dieselben
		Eigenschaften, in \(A_{\ideal p}\) kann jedes nicht verschwindende Ideal also als eindeutige
		Potenz des maximalen Ideals geschrieben werden.
	\end{remark}
\end{frame}

\subsection{Diskrete Bewertungsringe}

\begin{frame}{Definition einer diskreten Bewertung}
	\begin{definition}<+->
		Eine \emph{diskrete Bewertung auf einem Körper \(K\)} ist eine surjektive Abbildung
		\(\nu\colon K \surjto \set Z \cup \{\infty\}\) mit folgenden Eigenschaften:
		\begin{enumerate}[<+->]
		\item<.->
			\(\nu(x) = \infty \iff x = 0\) für alle \(x \in K\).
		\item
			\(\nu(xy) = \nu(x) + \nu(y)\) für alle \(x, y \in K\).
		\item
			\(\nu(x + y) \ge \min(\nu(x), \nu(y))\) für alle \(x, y \in K\).
		\end{enumerate}
	\end{definition}
	\begin{visibleenv}<+->
		Die Menge aller \(x \in K\) mit \(\nu(x) \ge 0\) ist ein Unterring von \(K\)
		und heißt der \emph{Bewertungsring von \(K\) (zu \(\nu\))}. Dieser ist in der Tat ein
		Bewertungsring.
	\end{visibleenv}
\end{frame}

\begin{frame}{Beispiele diskreter Bewertungen}
	\begin{example}<+->
		Sei \(p\) eine Primzahl. Jedes \(x \in \set Q^\units\) kann eindeutig in der Form
		\(x = p^a y\) geschrieben werden, wobei \(a \in \set Z\) ist und \(y\) ein Bruch ist,
		dessen Zähler und Nenner nicht durch \(p\) teilbar sind.
		\\
		Durch die Setzung \(\nu(x) \coloneqq a\) wird eine diskrete Bewertung auf \(\set Q\) definiert,
		deren Bewertungsring der lokale Ring \(\set Z_{(p)}\) ist.
	\end{example}
	\begin{example}<+->
		Seien \(F\) ein Körper und \(K \coloneqq F(x)\) der Körper der rationalen Funktion über \(F\) in \(x\).
		Sei \(f \in F[x]\) ein irreduzibles Polynom. Jedes \(g \in K^\units\) kann eindeutig in der
		Form \(g = f^a h\) geschrieben werden, wobei \(a \in \set Z\) ist und \(h\) ein Bruch ist,
		dessen Zähler und Nenner nicht durch \(f\) teilbar sind.
		\\
		Durch die Setzung \(\nu(g) \coloneqq a\) wird eine diskrete Bewertung auf \(K\) definiert,
		deren Bewertungsring der lokale Ring \(F[x]_{(f)}\) ist.
	\end{example}
\end{frame}

\begin{frame}{Definition eines diskreten Bewertungsringes}
	\begin{definition}<+->
		Ein Integritätsbereich \(A\) heißt \emph{diskreter Bewertungsring}, falls
		\(A\) der Bewertungsring einer diskreten Bewertung auf seinem Quotientenkörper \(K\) ist.
	\end{definition}
	\begin{visibleenv}<+->
		Ist \(A\) ein diskreter Bewertungsring zur Bewertung \(\nu\colon K \to \set Z \cup \{\infty\}\),
		so ist \(A\) insbesondere ein lokaler Ring mit maximalem Ideal \(\ideal m \coloneqq
		\{x \in A \mid \nu(x) > 0\}\).
	\end{visibleenv}
\end{frame}

\begin{frame}{Hauptideale in einem diskreten Bewertungring}
	\begin{proposition}<+->
		Sei \(A\) ein diskreter Bewertungsring mit Bewertung \(\nu\colon K \to \set Z \cup \{\infty\}\).
		Seien \(x, y \in A\). Dann gilt \(\nu(x) = \nu(y)\) genau dann, wenn \((x) = (y)\).
	\end{proposition}
	\begin{proof}<+->
		Aus \(\nu(xy^{-1}) = 0\) folgt, daß \(xy^{-1}\) eine Einheit in \(A\) ist.
		\\
		Umgekehrt genauso.
	\end{proof}
\end{frame}

\begin{frame}{Diskrete Bewertungsringe sind noethersch}
	Sei \(A\) ein diskreter Bewertungsring mit Bewertung \(\nu\colon K \to \set Z \cup \{\infty\}\).
	\begin{proposition}<+->
		Jedes Ideal in \(A\) ist von der Form \(\ideal m_k = \{x \in A \mid \nu(x) \ge k\}\) mit
		\(k \in \set N_0 \cup \{\infty\}\).
	\end{proposition}
	\begin{proof}<+->
		Zu einem Ideal \(\ideal a\) sei \(k \coloneqq \inf\{\nu(x) \mid x \in \ideal a\}\). Dann gilt
		\(\ideal a = \ideal m_k\), denn ist \(x \in \ideal a\), so ist auch \(y \in \ideal a\), falls
		\(\nu(y) \ge \nu(x)\).
	\end{proof}
	\begin{corollary}<+->
		Diskrete Bewertungsringe sind noethersche lokale Ringe.
	\end{corollary}
	\begin{proof}<+->
		Die einzigen Ideale in \(A\) bilden die absteigende Folge \((1) = \ideal m_0 \supset \ideal m_1
		\supset \ideal m_2 \supset \dotsb\).
	\end{proof}
\end{frame}

\begin{frame}{Diskrete Bewertungsringe sind eindimensional}
	\begin{proposition}<+->
		Sei \(A\) ein diskreter Bewertungsring.
		Dann ist \(A\) ein eindimensionaler noetherscher lokaler Integritätsbereich,
		in dem jedes (nicht verschwindende) Ideal eine Potenz des
		maximalen Ideals \(\ideal m\) ist.
	\end{proposition}
	\begin{proof}<+->
		Da die Bewertung \(\nu\colon K \to \set Z \cup \{\infty\}\) von \(A\) surjektiv ist, existiert ein \(x \in \ideal m\)
		mit \(\nu(x) = 1\). Damit ist \(\ideal m = (x)\), und alle weiteren Ideale sind von der Form \(\ideal m_k = (x^k) = \ideal m^k\).
	\end{proof}
\end{frame}

\subsection{Charakterisierungen diskreter Bewertungsringe}

\begin{frame}{Charakterisierungen diskreter Bewertungsringe}
	\begin{proposition}<+->
		Sei \((A, \ideal m, F)\) ein eindimensionaler noetherscher lokaler Integritätsbereich.
		Dann sind folgende Aussagen äquivalent:
		\begin{enumerate}[<+->]
		\item<.->
			\(A\) ist ein diskreter Bewertungsring.
		\item
			\(A\) ist ganz abgeschlossen.
		\item
			\(\ideal m\) ist ein Hauptideal.
		\item
			\(\dim_F \ideal m/\ideal m^2 = 1\).
		\item
			Jedes (nicht verschwindende) Ideal von \(A\) ist eine Potenz von \(\ideal m\).
		\item
			Es existiert ein \(x \in A\), so daß jedes (nicht verschwindende) Ideal von der Form
			\((x^k)\) mit \(k \ge 0\) ist.
		\end{enumerate}
	\end{proposition}
\end{frame}

\begin{frame}{Beweis der Charakterisierungen}
	\begin{proof}[Beweis]<+->
		\renewcommand{\qedsymbol}{}
		\begin{enumerate}[<+->]
		\item<.->
			Bewertungsringe sind ganz abgeschlossen.
		\item
			Sei \(A\) ganz abgeschlossen. Sei \(a \in \ideal m \setminus \{0\}\). Dann ist \(\ideal a \coloneqq (a)\)
			ein \(\ideal m\)-primäres Ideal (da \(\ideal m\) das einzige nicht verschwindende
			Primideal ist), also \(\sqrt{\ideal (a)} = \ideal m\). Da \(A\) noethersch ist,
			existiert ein \(n\) mit \(\ideal m^n \subset \ideal a\). Wir wählen \(n\) minimal.
			\\
			Wähle ein \(b \in \ideal m^{n - 1}\) mit \(b \notin \ideal a\). Sei \(x \coloneqq \frac a b\).
			Damit ist \(x^{-1} \notin A\), also nicht ganz über \(A\). Damit ist \(x^{-1} \ideal m \subsetneq \ideal m\),
			denn sonst wäre \(\ideal m\) ein treuer \(A[x^{-1}]\)-Modul, der endlich erzeugt ist als \(A\)-Modul.
			\\
			Aber \(x^{-1} \ideal m \subset A\) nach Konstruktion, also \(x^{-1} \ideal m = A\), also \(\ideal m = Ax = (x)\).
			\\
			Damit ist \(\ideal m\) ein Hauptideal.
		\item
			Sei \(\ideal m\) ein Hauptideal. Dann ist \(\dim_F \ideal m/\ideal m^2 \leq 1\). Da \(\ideal m \neq \ideal m^2\),
			da \(\dim A > 0\) folgt \(\dim_F \ideal m/\ideal m^2 = 1\).
			\qedhere
		\end{enumerate}
	\end{proof}
\end{frame}

\begin{frame}{Fortsetzung des Beweises}
	\begin{proof}[Fortsetzung des Beweises]<+->
		\renewcommand{\qedsymbol}{}
		\begin{enumerate}[<+->]
		\item<.->
			Sei \(\dim_F \ideal m/\ideal m^2 = 1\). Sei \(\ideal a\) ein nicht verschwindendes Ideal von \(A\).
			Wir können uns auf den Fall \(\ideal a \neq (1)\) beschränken.
			Wir haben schon gesehen, daß \(\ideal m^n \subset \ideal a\) für ein \(n\). Es ist \(A/\ideal m^n\) ein
			lokaler artinscher Ring, dessen Zariskischer Kotangentialraum höchstens eindimensional ist.
			\\
			Damit ist \((A/\ideal m^n) \ideal a\) notwendigerweise eine Potenz von \((A/\ideal m^n) \ideal m\), wie wir schon
			gezeigt haben.
			\\
			Folglich ist \(\ideal a\) eine Potenz von \(\ideal m\).
		\item
			Sei jedes nicht verschwindende Ideal von \(A\) eine Potenz von \(\ideal m\). Da \(\dim A > 0\), ist \(\ideal m
			\neq \ideal m^2\), also existiert ein \(x \in \ideal m \setminus \ideal m^2\).
			\\
			Nach Voraussetzung ist \((x) = \ideal m^r\) für ein \(r \ge 0\), also \(r = 1\). Damit ist \((x) = \ideal m\) und
			\((x^k) = \ideal m^k\) für alle \(k \ge 0\).
			\\
			Damit sind alle nicht verschwindenden Ideale von der Form \((x^k)\).
			\qedhere
		\end{enumerate}
	\end{proof}
\end{frame}

\begin{frame}{Ende des Beweises}
	\begin{proof}[Ende des Beweises]<+->
		\begin{enumerate}
		\item<.->
			Seien alle nicht verschwindenden Ideale von der Form \((x^k)\) mit \(x \in A\). Da \(\ideal m\) maximal ist,
			folgt \(\ideal m = (x)\). Da \(\dim A > 0\), damit \((x^k) \neq (x^{k + 1})\). Ist also \(a \in A \setminus \{0\}\),
			so gilt \((a) = (x^k)\) für genau ein \(k \in \set N_0\). Durch die Setzung \(\nu(a) \coloneqq k\) wird eine
			diskrete Bewertung auf dem Quotientenkörper von \(A\) mit Bewertungsring \(A\) definiert.
			\\
			Folglich ist \(A\) ein diskreter Bewertungsring.
			\qedhere
		\end{enumerate}
	\end{proof}
\end{frame}



\lecture{Dedekindsche bereiche}{Dedekindsche Bereiche}
\mode<all>\setcounter{section}{35}
\mode<all>\section{Dedekindsche Bereiche}

\subsection{Charakterisierung Dedekindscher Bereiche}

\begin{frame}{Definition Dedekindscher Bereiche}
	\begin{theorem}<+->
		Für einen eindimensionalen noetherschen Integritätsbereich \(A\) sind folgende
		Aussagen äquivalent:
		\begin{enumerate}[<+->]
		\item
			\(A\) ist ganz abgeschlossen.
		\item
			Jedes Primärideal in \(A\) ist eine Potenz eines Primideals.
		\item
			Für jedes Primideal \(\ideal p \neq (0)\) von \(A\)
			ist der Halm \(A_{\ideal p}\) ein diskreter Bewertungsring.
		\end{enumerate}
	\end{theorem}
	\begin{visibleenv}<+->
		Erfüllt \(A\) die drei äquivalenten Aussagen des Satzes, so heißt \(A\)
		ein \emph{Dedekindscher Bereich}.
	\end{visibleenv}
\end{frame}

\begin{frame}{Beweis der Charakterisierung Dedekindscher Bereiche}
	\begin{proof}<+->
		\begin{enumerate}[<+->]
		\item<.->
			Daß ein Ring ganz abgeschlossen ist, ist eine lokale Eigenschaft. Damit
			folgt die Äquivalenz der ersten mit der dritten Aussage aus den
			Charakterisierungen diskreter Bewertungsringe.
		\item
			Auch die zweite Aussage beschreibt eine lokale Eigenschaft, denn Primärideale
			und Idealpotenzen verhalten sich gut unter Lokalisierung.
			\\
			Damit folgt die Äquivalenz der zweiten mit der dritten Aussage ebenfalls aus den
			Charakterisierungen diskreter Bewertungsringe, denn in noetherschen eindimensionalen
			lokalen Ringen sind alle Ideale \(\ideal a \neq (0)\) Primärideale.
			\qedhere
		\end{enumerate}
	\end{proof}
\end{frame}

\begin{frame}{Eindeutige Faktorisierung in Dedekindschen Bereichen}
	\begin{corollary}<+->
		In einem Dedekindschen Bereich läßt sich jedes nicht verschwindende Ideal als
		eindeutiges Produkt von Primidealen schreiben.
	\end{corollary}
	\begin{proof}<+->
		\begin{enumerate}[<+->]
		\item<.->
			Jeder Dedekindsche Bereich ist ein eindimensionaler noetherscher Integritätsbereich,
			in welchem sich jedes Ideal eindeutig als Produkt von Primäridealen schreiben läßt.
		\item
			Die Primärideale in einem Dedekindschen Bereich sind alle (eindeutige) Potenzen von Primidealen.
			\qedhere
		\end{enumerate}
	\end{proof}
\end{frame}

\subsection{Beispiele Dedekindscher Bereiche}

\begin{frame}{Hauptidealbereiche}
	\begin{example}<+->
		Sei \(A\) ein Hauptidealbereich. Da jedes Ideal offensichtlich
		endlich erzeugt ist, ist \(A\) ein noetherscher Ring.
		\\
		Da in Hauptidealbereichen jedes nicht verschwindende Primideal maximal
		ist, gilt weiter \(\dim A = 1\).
		\\
		Ist \(\ideal p \neq (0)\) ein Primideal, so ist \(A_{\ideal p}\)
		ein lokaler Hauptidealbereich, nach unseren Charakterisierungen
		diskreter Bewertungsbereiche also ein solcher.
		\\
		Damit ist \(A\) ein Dedekindscher Bereich, das heißt Hauptidealbereiche
		sind spezielle Dedekindsche Bereiche.
	\end{example}
\end{frame}

\begin{frame}{Ringe ganzer Zahlen in Zahlkörpern}
	\begin{theorem}<+->
		Sei \(K\) ein Zahlkörper, das heißt eine endliche Erweiterung von
		\(\set Q\). Dann ist der Ring \(A\) der ganzen Zahlen in \(K\), das
		heißt der ganze Abschluß von \(\set Z\) in \(K\), ein Dedekindscher
		Bereich.
	\end{theorem}
	\begin{proof}<+->
		\begin{enumerate}[<+->]
		\item<.->
			Da \(\set Q\) die Charakteristik \(0\) hat, 
			ist \(K\) eine separable Erweiterung von
			\(\set Q\). Damit existiert eine Basis \(v_1, \dotsc, v_n\) von
			\(K\) über \(\set Q\) mit \(A \subset \sum\limits_i \set Z v_i\).
		\item
			Damit ist \(A\) als \(\set Z\)-Modul endlich erzeugt, also
			noethersch. Als ganzer Abschluß in \(K\) ist \(A\) selbst ganz
			abgeschlossen.
		\item
			Es bleibt zu zeigen, daß jedes Primideal \(\ideal p \neq (0)\) von
			\(A\) maximal ist: Zunächst ist \(\set Z \cap \ideal p \neq (0)\),
			da \(A\) ganz über \(\set Z\) ist. Damit ist \(\set Z \cap \ideal p\)
			maximal. Wieder weil \(A\) ganz über \(\set Z\) ist, ist damit
			auch \(\ideal p\) maximal.
			\qedhere
		\end{enumerate}
	\end{proof}
\end{frame}



\mode<all>\section{Gebrochene Ideale}

\subsection{Gebrochene Ideale}

\begin{frame}{Gebrochene Ideale}
	Sei \(A\) ein Integritätsbereich mit Quotientenkörper \(K\).
	\begin{definition}<+->
		Ein \(A\)-Untermodul \(\ideal r\) von \(K\) heißt \emph{gebrochenes Ideal
		von \(A\)}, falls ein \(x \in A \setminus \{0\}\) mit \(x \ideal r
		\subset A\) existiert.
	\end{definition}
	\begin{example}<+->
		Ein ganzes (d.h.\ gewöhnliches) Ideal in \(A\) ist insbesondere
		ein gebrochenes Ideal.
	\end{example}
	\begin{example}<+->
		Jedes \(u \in K\) definiert ein gebrochenes Ideal \((u) \coloneqq
		A u \subset K\), ein \emph{gebrochenes Hauptideal von \(A\)}.
	\end{example}
\end{frame}

\begin{frame}{Gebrochene Ideale in noetherschen Integritätsbereichen}
	Sei \(A\) ein Integritätsbereich mit Quotientenkörper \(K\).
	\begin{proposition}<+->
		Ist \(\ideal r\) ein endlich erzeugter \(A\)-Untermodul von \(K\),
		so ist \(\ideal r\) ein gebrochenes Ideal.
	\end{proposition}
	\begin{proof}<+->
		Ist \(\ideal r\) von \(x_1, \dotsc, x_n\) erzeugt, so können wir
		\(x_i = \frac{y_i}z\) mit \(y_i \in A\) mit einem \(z \in A\) schreiben.
		Dann ist \(z \ideal r \subset A\).
	\end{proof}
	\begin{proposition}<+->
		Ist \(A\) noethersch, so ist jedes gebrochene Ideal \(\ideal r\)
		als \(A\)-Modul endlich erzeugt.
	\end{proposition}
	\begin{proof}<+->
		Ist \(x \in A \setminus \{0\}\), so daß \(\ideal a \coloneqq x \ideal r\)
		ein ganzes Ideal von \(A\) ist, so ist mit \(\ideal a\) auch
		\(\ideal r = x^{-1} \ideal a\) endlich erzeugt.
	\end{proof}
\end{frame}

\subsection{Invertierbare Ideale}

\begin{frame}{Invertierbare Ideale}
	\begin{definition}<+->
		Sei \(A\) ein Integritätsbereich mit Quotientenkörper \(K\).
		Ein \(A\)-Untermodul \(\ideal r\) von \(K\) heißt \emph{invertierbares
		Ideal}, falls ein \(A\)-Untermodul \(\ideal s\) mit \(\ideal r
		\ideal s = A\) existiert.
	\end{definition}
	\begin{visibleenv}<+->
		Ist \(\ideal r\) ein \(A\)-Untermodul von \(K\), so schreiben wir
		\((1 : \ideal r)\) für den \(A\)-Modul derjenigen \(x \in K\) mit
		\(x \ideal r \subset A\).
		\\
		(Diese Schreibweise möge nicht zur Verwechslung mit dem gewöhnlichen
		Idealquotienten führen.)
	\end{visibleenv}
\end{frame}

\begin{frame}{Inversen invertierbarer Ideale}
	\begin{proposition}<+->
		Sei \(A\) ein Integritätsbereich mit Quotientenkörper \(K\).
		Ist dann \(\ideal r\) ein invertierbares Ideal von \(K\), so existiert
		genau ein \(A\)-Untermodul \(\ideal s\) mit \(\ideal r \ideal s = (1)\),
		nämlich \((1 : \ideal r)\).
	\end{proposition}
	\begin{visibleenv}<.->
		In diesem Falle schreiben wir \(\ideal r^{-1} \coloneqq (1 : \ideal r)\).
	\end{visibleenv}
	\begin{proof}<+->
		Sei \(\ideal s\) mit \(\ideal r\ideal s = (1)\). Dann gilt
		\(\ideal s \subset (1 : \ideal r) = (1 : \ideal r) \ideal r \ideal s
		\subset \ideal s\).
	\end{proof}
\end{frame}

\begin{frame}{Invertierbare Ideale sind gebrochene Ideale}
	\begin{proposition}<+->
		Sei \(A\) ein Integritätsbereich. Jedes invertierbare Ideal
		\(\ideal r\) von \(A\) ist als \(A\)-Modul endlich erzeugt, insbesondere
		also ein gebrochenes Ideal von \(A\).
	\end{proposition}
	\begin{proof}<+->
		\begin{enumerate}[<+->]
		\item<.->
			Aus \(\ideal r \ideal r^{-1} = (1)\) folgt die Existenz endlich vieler
			\(x_i \in \ideal r, y_i \in \ideal r^{-1}\) mit \(\sum\limits_i
			x_i y_i = 1\).
		\item
			Für jedes \(x \in \ideal r\) gilt damit \(x = \sum\limits_i
			(y_i x) x_i\).
		\item
			Wegen \(y_i x \in A\) wird \(\ideal r\) damit als \(A\)-Modul von
			\(x_1, \dotsc, x_n\) erzeugt.
			\qedhere
		\end{enumerate}
	\end{proof}
\end{frame}

\begin{frame}{Die Gruppe der invertierbaren Ideale}
	Sei \(A\) ein Integritätsbereich mit Quotientenkörper \(K\).
	\begin{example}<+->
		Ist \(u \in K^\units\), so ist das Hauptideal \((u)\) ein invertierbares
		Ideal mit \((u)^{-1} = (u^{-1})\).
	\end{example}
	\begin{remark}<+->
		Bezüglich der Idealmultiplikation bilden die invertierbaren Ideale
		eine Gruppe, deren neutrales Element durch \((1)\) gegeben ist.
	\end{remark}
\end{frame}

\begin{frame}{Invertierbarkeit ist eine lokale Eigenschaft}
	\begin{proposition}<+->
		Sei \(A\) ein Integritätsbereich. Für ein gebrochenes Ideal
		\(\ideal r\) von \(A\) sind dann folgende Aussagen äquivalent:
		\begin{enumerate}[<+->]
		\item<.->
			Es ist \(\ideal r\) invertierbar.
		\item
			Es ist \(\ideal r\) endlich erzeugt, und für alle Primideale
			\(\ideal p\) von \(A\) ist \(\ideal r_{\ideal p}\) ein
			invertierbares Ideal von \(A_{\ideal p}\).
		\item
			Es ist \(\ideal r\) endlich erzeugt, und für alle maximalen
			Ideale \(\ideal m\) von \(A\) ist \(\ideal r_{\ideal m}\) ein
			invertierbares Ideal von \(A_{\ideal m}\).
		\end{enumerate}
	\end{proposition}
\end{frame}

\begin{frame}{Beweis, daß Invertierbarkeit eine lokale Eigenschaft ist}
	\begin{proof}<+->
		\begin{enumerate}[<+->]
			\item<.->
				Es ist \(A_{\ideal p} = (\ideal r \cdot (1 : \ideal r))_{\ideal p}
				= \ideal r_{\ideal p} \cdot (1 : \ideal r_{\ideal p})\), da
				\(\ideal r\) als invertierbares Ideal endlich erzeugt ist.
				Damit folgt die zweite aus der ersten Aussage.
			\item
				Die dritte Aussage folgt trivialerweise aus der zweiten Aussage.
			\item
				Um aus der dritten Aussage, die erste zu folgern, betrachten wir
				\(\ideal a = \ideal r \cdot (1 : \ideal r)\), ein ganzes Ideal.
				Für jedes maximale Ideal \(\ideal m\) haben wir
				\(\ideal a_{\ideal m} = \ideal r_{\ideal m}
				\cdot (1 : \ideal r_{\ideal m}) = A_{\ideal m}\), wenn \(\ideal r\) endlich erzeugt
				und \(\ideal r_{\ideal m}\) invertierbar ist. Damit ist
				\(\ideal a = (1)\) und damit \(\ideal r\) invertierbar.
				\qedhere
		\end{enumerate}
	\end{proof}
\end{frame}

\subsection{Gebrochene Ideale in Dedekindschen Bereichen}

\begin{frame}{Gebrochene Ideale in diskreten Bewertungsbereichen}
	\begin{proposition}<+->
		Ein lokaler Integritätsbereich \(A\), welcher kein Körper ist,
		ist genau dann ein diskreter
		Bewertungsring, wenn jedes nicht verschwindende gebrochene Ideal
		von \(A\) invertierbar ist.
	\end{proposition}
	\begin{proof}<+->
		\renewcommand{\qedsymbol}{}
		Sei zunächst \(A\) ein diskreter Bewertungsring. Sei \(x\) ein
		Erzeuger des maximalen Ideals von \(A\). Sei \(\ideal r \neq (0)\)
		ein gebrochenes Ideal. Dann existiert ein \(y \in A\) mit
		\(y \ideal r \subset A\), also ist \(y \ideal r = (x^r)\) für ein
		\(r \in \set N_0\). Damit folgt \(\ideal r = (x^{r - s})\), wenn
		\((y) = (x^s)\). Also ist \(\ideal r\) invertierbar.
	\end{proof}
\end{frame}

\begin{frame}{Fortsetzung des Beweises}
	\begin{proof}[Fortsetzung des Beweises]<+->
		\begin{enumerate}[<+->]
		\item<.->
			Sei umgekehrt jedes nicht verschwindende gebrochene Ideal
			invertierbar. Damit ist insbesondere jedes nicht triviale ganze Ideal
			invertierbar, also auch endlich erzeugt, also ist \(A\) noethersch.
		\item
			Damit reicht es zu zeigen, daß jedes nicht verschwindende Ideal
			eine Potenz des maximalen Ideals \(\ideal m\) ist. Angenommen, dies ist falsch.
			Dann existiert ein maximales nicht verschwindendes Ideal \(\ideal a\),
			welches keine Potenz von \(\ideal m\) ist.
		\item
			Da \(\ideal a \subset \ideal m\) ist \(\ideal m^{-1} \ideal a\) ein
			ganzes Ideal mit \(\ideal m^{-1} \ideal a \supset \ideal a\).
		\item
			Wäre \(\ideal m^{-1} \ideal a = \ideal a\), dann auch \(\ideal a = \ideal m \ideal a\),
			also \(\ideal a = 0\) nach dem Nakayamaschen Lemma.
		\item
			Daher ist \(\ideal m^{-1} \ideal a \supsetneq \ideal a\), also ist
			\(\ideal m^{-1} \ideal a\) aufgrund der Maximalität von \(\ideal a\)
			eine Potenz von \(\ideal m\), also auch \(\ideal a\). Widerspruch.
			\qedhere
		\end{enumerate}
	\end{proof}
\end{frame}

\begin{frame}{Gebrochene Ideale in Dedekindschen Bereichen}
	\begin{theorem}<+->
		Ein Integritätsbereich \(A\), welcher kein Körper ist, ist genau
		dann ein Dedekindscher Bereich, wenn jedes nicht verschwindende
		gebrochene Ideal invertierbar ist.
	\end{theorem}
	\begin{proof}<+->
		\begin{enumerate}[<+->]
		\item<.->
			Sei \(A\) ein Dedekindscher Bereich. Ist \(\ideal r \neq (0)\) ein
			gebrochenes Ideal, so ist \(\ideal r\) endlich erzeugt, da
			\(A\) noethersch ist. Für jedes Primideal \(\ideal p \neq (0)\) ist
			\(\ideal r_{\ideal p} \neq (0)\) ein gebrochenes, also invertierbares
			Ideal im diskreten Bewertungsring \(A_{\ideal p}\). Damit ist
			\(\ideal r\) invertierbar.
		\item
			Sei umgekehrt jedes nicht verschwindende gebrochene Ideal invertierbar.
			Wie im letzten Beweis folgt, daß \(A\) noethersch ist. Damit reicht es
			zu zeigen, daß \(A_{\ideal p}\) für alle Primideal \(\ideal p \neq (0)\)
			ein diskreter Bewertungsring ist. Dazu reicht es zu zeigen, daß
			jedes (ganze) Ideal \(\ideal b \neq (0)\) in \(A_{\ideal p}\)
			invertierbar ist.
		\item
			Es ist aber \(\ideal a = A \cap \ideal b\) invertierbar, und damit
			auch \(\ideal b = \ideal a_{\ideal p}\).
			\qedhere
		\end{enumerate}
	\end{proof}
\end{frame}

\subsection{Anmerkungen zur Idealklassengruppe}

\begin{frame}{Idealgruppe eines Dedekindschen Bereiches}
	\begin{corollary}<+->
		In einem Dedekindschen Bereich \(A\) bilden die nicht verschwindenden
		gebrochenen Ideale eine Gruppe bezüglich der Multiplikation, die
		\emph{Idealgruppe von \(A\)}.
	\end{corollary}
	\begin{remark}<+->
		In diesem Zusammenhang können wir den Satz über die eindeutige
		Faktorisierbarkeit von nicht verschwindenden Idealen in Dedekindschen
		Bereichen in Primideale auch so formulieren:
		\\
		Die Idealgruppe eines Dedekindschen Bereiches ist eine freie abelsche
		Gruppe, welche durch die nicht verschwindenden Primideale von \(A\)
		erzeugt wird.
	\end{remark}
\end{frame}

\begin{frame}{Idealklassengruppe eines Dedekindschen Bereiches}
	Sei \(A\) ein Dedekindscher Bereich mit Quotientenkörper \(K\), dessen
	Idealgruppe \(\IdealG(A)\) sei.
	\\
	Wir haben einen Gruppenhomomorphismus \(K^\units \to \IdealG(A),
	u \mapsto (u)\), dessen Bild die gebrochenen Hauptideale sind.
	\begin{definition}<+->
		Die Faktorgruppe \(\ClassG(A) \coloneqq \IdealG(A)/K^\units\) heißt die
		\emph{Idealklassengruppe von \(A\)}.
	\end{definition}
	\begin{remark}<+->
		Im Kern von \(K^\units \to \IdealG(A)\) liegen genau die \(u \in K^\units\) mit
		\((u) = (1)\), also die Einheiten \(u \in A^\units\). Wir erhalten damit
		eine exakte Sequenz
		\[1 \to A^\units \to K^\units \to \IdealG(A) \to \ClassG(A) \to 1\]
		(multiplikativ geschriebener) abelscher Gruppen.
	\end{remark}
\end{frame}

\begin{frame}{Idealklassengruppe von Ringen ganzer Zahlen}
	Seien \(K\) ein Zahlkörper und \(A\) sein Ring ganzer Zahlen, ein
	Dedekindscher Bereich.
	\begin{remark}<+->
		In der algebraischen Zahlentheorie wird gezeigt, daß \(\ClassG(A)\)
		eine endliche Gruppe ist, ihre Ordnung heißt die \emph{Klassenzahl von
		\(K\)}.
		\\
		Es ist genau dann \([\ClassG(A) : 1] = 1\), wenn jedes invertierbare
		Ideal in \(A\) ein gebrochenes Hauptideal ist. Und dies ist genau dann
		der Fall, wenn \(A\) ein faktorieller Ring ist.
	\end{remark}
	\begin{remark}<+->
		Es ist \(A^\units\) eine endlich erzeugte abelsche Gruppe. Die Elemente
		endlicher Ordnung sind gerade die Einheitswurzeln \(\rou(K)\) von \(K\).
		Der Rang der freien abelschen Gruppe \(A^\units/\rou(K)\) ist \(r_1
		+ r_2 - 1\), wobei \(r_1\) die Anzahl der reellen und
		\(2 r_2\) die Anzahl der echt komplexen Einbettungen von \(K\) in \(\set C\)
		ist.
	\end{remark}
\end{frame}

\begin{frame}{Beispiele zu Ringen ganzer Zahlen}
	\begin{example}<+->
		Der Zahlkörper \(\set Q(\sqrt{-1})\) besitzt nur zwei echt komplexe Einbettungen.
		Sein Ring \(\set Z[i]\) ganzer Zahlen besitzt damit nur endlich viele Einheiten,
		nämlich alle Einheitswurzeln: \(\pm 1, \pm i\).
	\end{example}
	\begin{example}<+->
		Der Zahlkörper \(\set Q(\sqrt{2})\) besitzt nur zwei reelle Einbettungen.
		Sein Ring \(\set Z[\sqrt{2}]\) ganzer Zahlen hat damit unendlich viele
		Einheiten, und zwar \(\pm (1 + \sqrt{2})^n\) mit \(n \in \set Z\).
	\end{example}
\end{frame}



\lecture{Vervollständigungen I}{Vervollst\"andigungen I}
\part<article>{Vervollständigungen}
\mode<all>\setcounter{section}{37}
\mode<all>\section{Vervollständigungen I}

\subsection{Topologische Gruppen}

\begin{frame}{Definition topologischer Gruppen}
	\begin{definition}<+->
		Sei \(G\) eine (additiv geschriebene) Gruppe, deren zugrundeliegende Menge von Elementen
		ein topologischer Raum ist. Dann heißt \(G\) eine \emph{topologische
		Gruppe}, falls die Addition \(G \times G \to G, (g, h) \mapsto g + h\)
		und die Negation \(G \to G, g \mapsto -g\) stetige Abbildungen sind.
	\end{definition}
	\begin{example}<+->
		Jede Gruppe ist bezüglich der diskreten Topologie eine topologische.
	\end{example}
	\begin{proposition}<+->
		Sei \(G\) eine topologische Gruppe, so daß \(\{0\} \subset G\) 
		abgeschlossen ist. Dann ist \(G\) hausdorffsch.
	\end{proposition}
	\begin{proof}<+->
		Die Diagonale \(\{(g, g) \mid g \in G\} \subset G \times G\)
		ist als Urbild von \(\{0\}\) unter der stetigen Abbildung \(G \times G \to G,
		(x, y) \mapsto x - y\) abgeschlossen.
	\end{proof}
\end{frame}

\begin{frame}{Unter- und Faktorgruppen topologischer Gruppen}
	\begin{example}<+->
		Sei \(H\) eine Untergruppe einer topologischen Gruppe \(G\).
		Dann ist \(H\) mit der Teilraumtopologie eine topologische
		Gruppe.
	\end{example}
	\begin{example}<+->
		Seien \(G\) ein topologische Gruppe und \(N\) ein Normalteiler in \(G\).
		Versehen mit der Quotiententopologie ist \(G/N\) eine topologische Gruppe.
	\end{example}
\end{frame}

\begin{frame}{Uniformität topologischer Gruppen}
	\begin{remark}<+->
		Sei \(G\) eine topologische Gruppe. Für jedes \(a \in G\) ist die
		Verschiebung \(\tau_a\colon G \to G, g \mapsto a + g\) eine stetige
		Abbildung mit Inverse \(\tau_{-a}\). Damit ist \(\tau_a\) ein
		Homöomorphismus.
		\\
		Ist also \(U \subset G\) eine offene Umgebung von \(0\), so ist
		\(a + U\) eine offene Umgebung von \(a\). Umgekehrt ist jede
		offene Umgebung von \(a\) von dieser Form.
		\\
		Damit definieren die offenen Umgebungen um \(0\) die Topologie auf \(G\).
	\end{remark}
\end{frame}

\begin{frame}{Hausdorffsche topologische Gruppen}
	\begin{lemma}<+->
		Sei \(G\) eine topologische Gruppe.
		Sei \(H\) der Schnitt aller (offenen) Umgebungen \(U\) von \(0\) in \(G\). Dann
		gilt:
		\begin{enumerate}[<+->]
		\item<.->
			Es ist \(H\) ein Normalteiler in \(G\).
		\item
			Es ist \(H\) der topologische Abschluß von \(\{0\}\).
		\item
			Die Faktorgruppe \(G/H\) ist hausdorffsch.
		\item
			Es ist \(G\) genau dann hausdorffsch, wenn \(H = 0\).
		\end{enumerate}
	\end{lemma}
\end{frame}

\begin{frame}{Beweis des Hilfssatzes über hausdorffsche topologische Gruppen}
	\begin{proof}<+->
		\begin{enumerate}[<+->]
		\item<.->
			Aus der Stetigkeit der Gruppenoperationen folgt, daß \(H\) ein Normalteiler ist.
		\item
			Es ist \(\pm x \in H\) genau dann, wenn \(0 \in \mp x + U\) für alle \(U\), wenn also \(\mp x \in \closure{\{0\}}\).
		\item
			Damit sind insbesondere auch die Nebenklassen von \(H\) abgeschlossen, also sind
			Punkte in \(G/H\) abgeschlossen, also ist \(G/H\) hausdorffsch.
		\item
			Schließlich ist \(G \cong G/H\), wenn \(H = 0\) ist, insbesondere also hausdorffsch. Umgekehrt
			folgt aus \(G\) hausdorffsch, daß \(H = 0\), da \(\{0\}\) dann abgeschlossen ist.
			\qedhere
		\end{enumerate}
	\end{proof}
\end{frame}

\subsection{Vervollständigungen topologischer Gruppen}

\begin{frame}{Cauchy-Folgen}
	\begin{visibleenv}<+->
		Der Einfachheit halber setzen ab sofort voraus, daß alle unsere topologischen Gruppen \(G\) das
		erste Abzählbarkeitsaxiom erfüllen, daß also eine Folge \(U_0 \supset U_1 \supset \dotsb\)
		von Umgebungen von \(0 \in G\) existiert, so daß für jede
		Umgebung \(U\) von \(0\) gilt, daß \(U_n \subset U\) für \(n \gg 0\).
		\\
		Eine solche Folge heißt \emph{Umgebungsbasis von \(0\)}.
	\end{visibleenv}
	\begin{definition}<+->
		Sei \(G\) eine topologische Gruppe. Eine \emph{Cauchy-Folge \((g_n)_{n \in \set N_0}\) in \(G\)}
		ist eine Folge von Elementen in \(G\), so daß für alle (offenen) Umgebungen \(U\) von \(0 \in G\)
		ein \(N \in \set N_0\) mit \(g_n - g_m \in U\) für \(n, m \ge N\) existiert.
	\end{definition}
\end{frame}

\begin{frame}{Definition der Vervollständigung}
	Sei \(G\) eine topologische Gruppe.
	\begin{definition}<+->
		Zwei Cauchy-Folgen \((g_n), (h_n)\) in \(G\) heißen
		\emph{äquivalent}, wenn \(\lim\limits_{n \to \infty} (g_n - h_n) = 0\).
	\end{definition}
	\begin{visibleenv}<+->
		Mit \(\hat G\) bezeichnen wir die Menge aller Äquivalenzklassen von Cauchy-Folgen in \(G\).
		Sind \((g_n), (h_n)\) beliebige Cauchy-Folgen, so ist auch \((g_n + h_n)\) eine Cauchy-Folge, deren
		Äquivalenzklasse nur von den Klassen von \((g_n)\) und \((h_n)\) abhängt. Dies definiert eine
		Addition auf \(\hat G\), welche \(\hat G\) eindeutig zu einer Gruppe macht.
	\end{visibleenv}
	\begin{definition}<+->
		Die Gruppe \(\hat G\) heißt die \emph{Vervollständigung von \(G\)}.
	\end{definition}
	\begin{remark}<+->
		Ist \(G\) abelsch, so auch \(\hat G\).
	\end{remark}
\end{frame}

\begin{frame}{Topologie der Vervollständigung}
	Sei \(G\) eine topologische Gruppe.
	\begin{remark}<+->
		Für jede offene Umgebung \(U\) von
		\(0 \in G\) definieren wir \(\hat U \subset \hat G\) als die Menge der
		Äquivalenzklassen von Cauchy-Folgen \((g_n)\) in \(G\) mit \(g_n \in U\)
		für \(n \gg 0\).
		\\
		Alle Teilmengen der Form \(a + \hat U\) mit \(a \in \hat G\) bilden dann
		die Basis einer Topologie auf \(\hat G\). Mit dieser Definition wird
		\(\hat G\) zu einer topologischen Gruppe.
	\end{remark}
	\begin{visibleenv}<+->
		Ist \(U\) eine mit der Teilraumtopologie versehene Untergruppe, so stimmt \(\hat U\)
		mit der Vervollständigung von \(U\) überein.
	\end{visibleenv}
\end{frame}

\begin{frame}{Die gewöhnliche Vervollständigung der rationalen Zahlen}
	\begin{example}<+->
		Betrachten wir \(\set Q\) mit der gewöhnlichen Topologie als topologische Gruppe,
		so ist \(\hat{\set Q} = \set R\).
	\end{example}
\end{frame}

\begin{frame}{Kern der kanonischen Abbildung in die Vervollständigung}
	\begin{visibleenv}<+->
		Sei \(G\) eine topologische Gruppe. Jedes Element \(g \in G\) definiert die konstante
		Cauchy-Folge \((g)_{n \in \set N_0}\). Dies definiert eine kanonische 
		Abbildung \(\phi\colon G \to \hat G\), welche ein Gruppenhomomorphismus ist.
	\end{visibleenv}
	\begin{example}<+->
		Die kanonische Abbildung \(G \to \hat G\) ist stetig, und ihr Bild ist dicht
		in \(\hat G\).
	\end{example}
	\begin{proposition}<+->
		Es ist \(\ker \phi = \overline{\{0\}}\), und damit ist \(\phi\) genau dann injektiv, wenn
		\(G\) hausdorffsch ist.
	\end{proposition}
	\begin{proof}<+->
		Es ist \(g \in \ker \phi\), wenn \(g = \lim\limits_n g = 0\), wenn also \(g \in \overline{\{0\}}\).
	\end{proof}
\end{frame}

\begin{frame}{Funktorialität der Vervollständigung}
	\begin{visibleenv}<+->
		Sei \(\phi\colon G \to H\) ein stetiger Homomorphismus topologischer Gruppen.
		Ist dann \((g_n)\) eine Cauchy-Folge in \(G\), so ist \(\phi(g_n)\) eine
		Cauchy-Folge in \(H\), deren Äquivalenzklasse nur von der Klasse von \((g_n)\)
		abhängt. Damit definiert \(\phi\) eine kanonische Abbildung
		\(\hat \phi\colon \hat G \to \hat H\).
	\end{visibleenv}
	\begin{proposition}<+->
		Es ist \(\hat\phi\colon \hat G \to \hat H\) ein stetiger Gruppenhomomorphismus.
		\\
		Ist \(\psi\colon H \to K\) ein weiterer stetiger Homomorphismus topologischer
		Gruppen, so ist \(\widehat{\psi \circ \phi} = \hat \psi \circ \hat\phi\colon \hat G \to \hat K\).
		\qed
	\end{proposition}
\end{frame}

\subsection{Inverse Limiten}

\begin{frame}{Inverse Systeme}
	\begin{definition}<+->
		Eine Sequenz von Gruppen und Gruppenhomomorphismen der Form
		\(\dotsb \xrightarrow{\theta_3} A_2 \xrightarrow{\theta_2} A_1 \xrightarrow{\theta_1} A_0\)
		heißt ein \emph{inverses System (von Gruppen)}.
	\end{definition}
	\begin{example}<+->
		Sei \(p\) ein Primzahl. Dann ist
		\(\dotsb \to \set Z/(p^2) \surjto \set Z/(p) \surjto 0\), wobei die Abbildungen die
		kanonischen sind, ein inverses System abelscher Gruppen.
	\end{example}
	\begin{definition}<+->
		Sei \(A_\bullet\colon \dotsb \xrightarrow{\theta_2} A_1 \xrightarrow{\theta_1} A_0\) ein inverses
		System von Gruppen. Eine Folge \((\xi_n)_{n \in \set N_0}\) von Elementen \(\xi_n \in A_n\)
		heißt \emph{kohärent (in \(A_\bullet\))}, wenn \(\theta_n(\xi_n) = \xi_{n - 1}\) für alle \(n \in \set N\).
	\end{definition}
	\begin{visibleenv}<+->
		Die Menge der kohärenten Systeme bezeichnen wir mit \(\varprojlim\limits_n A_n\).
	\end{visibleenv}
\end{frame}

\begin{frame}{Der inverse Limes}
	Sei \(A_\bullet\colon \dotsb \to A_1 \to A_0\) ein inverses System von Gruppen. Sind \((\xi_n), (\eta_n)\)
	zwei kohärente Folgen, so ist auch \((\xi_n) + (\eta_n) \coloneqq (\xi_n + \eta_n)\) eine
	kohärente Folge. Dies definiert eine Addition auf der Menge \(\varprojlim\limits_n A_n\)
	der kohärenten Folgen. Mit dieser Definition wird \(\varprojlim\limits_n A_n\) zu
	einer Gruppe.
	\begin{definition}<+->
		Die Gruppe \(\varprojlim\limits_n A_n\) heißt der \emph{inverse Limes des Systems
		\(A_\bullet\)}.
	\end{definition}
	\begin{visibleenv}<+->
		Die kanonischen Abbildungen \(\varprojlim\limits_n A_n \to A_i, (\xi_n) \mapsto \xi_i\) sind
		für alle \(i\) Gruppenhomomorphismen.
	\end{visibleenv}
	\begin{remark}<+->
		Sind die \(A_n\) abelsche Gruppen, so ist auch \(\varprojlim\limits_n A_n\) abelsch.
	\end{remark}
\end{frame}

\begin{frame}{Der inverse Limes topologischer Gruppen}
	\begin{remark}<+->
		Sei \(\dotsb \to A_1 \to A_0\) ein \emph{inverses System topologischer Gruppen}, das heißt
		die \(A_i\) sind topologische Gruppen und die Homomorphismen \(A_n \to A_{n - 1}\) sind
		außerdem stetig.
		\\
		Die Menge \(\prod\limits_n A_n\) aller Folgen versehen wir mit der Produkttopologie.
		Damit können wir der Menge \(\varprojlim\limits_n A_n \subset \prod\limits_n A_n\) der kohärenten
		Folgen die Teilraumtopologie geben.
		\\
		Mit dieser Setzung wird \(\varprojlim\limits_n A_n\) zu einer topologischen Gruppe.
	\end{remark}
	\begin{example}<+->
		Ist \(\dotsb \to A_1 \to A_0\) ein inverses System von Gruppen, so fassen wir es
		als System topologischer Gruppen auf, indem die \(A_n\) mit der diskreten Topologie versehen werden.
		\\
		Die Topologie des inversen Limes \(\varprojlim\limits_n A_n\) ist i.a.\ nicht diskret.
	\end{example}
\end{frame}

\subsection{Topologische Gruppen mit neutralen Umgebungsbasen aus Normalteilern}

\begin{frame}{Neutrale Umgebungsbasis aus Normalteilern}
	\begin{proposition}<+->
		Sei \(G\) eine topologische Gruppe. Sei \(G_0 \supset G_1 \supset \dotsb\)
		eine Umgebungsbasis von \(0\) aus Normalteilern. Dann sind die \(G_n\) sowohl offen als auch
		abgeschlossen in \(G\).
	\end{proposition}
	\begin{proof}<+->
		\begin{enumerate}[<+->]
		\item<.->
			Ist \(g \in G_n\), so ist \(g + G_n\) eine Umgebung von \(g\). Da \(g + G_n \subset G_n\)
			folgt, daß \(G_n\) offen ist.
		\item
			Damit ist auch \(\bigcup\limits_{h \notin G_n} (h + G_n)\) offen. Das Komplement, nämlich 
			\(G_n\), ist damit abgeschlossen.
			\qedhere
		\end{enumerate}
	\end{proof}
\end{frame}

\begin{frame}{Beispiele zu neutralen Umgebungsbasen aus Normalteilern}
	\begin{example}<+->
		Sei \(G\) eine  Gruppe. Sei \(G_0 \supset G_1 \supset \dotsb\) eine
		Folge von Normalteilern in \(G\). Dann gibt es genau eine Topologie
		auf \(G\), so daß \(G_0 \supset G_1 \supset \dotsb\) eine Umgebungsbasis
		von \(0\) ist und mit der \(G\) zu einer topologischen Gruppe wird.
	\end{example}
	\begin{example}<+->
		Für jede Primzahl \(p\) wird
		\(\set Z\) mit der Umgebungsbasis \((1) \supset (p) \supset (p^2) \dotsb\)
		zu einer topologischen Gruppe. Die zugehörige Topologie nennen wir
		die \emph{\(p\)-adische Topologie auf \(\set Z\)}.
	\end{example}
\end{frame}

\begin{frame}{Vervollständigungen und inverse Limiten}
	\begin{enumerate}[<+->]
	\item
		Sei \(G\) eine topologische Gruppe mit einer Umgebungsbasis \(G_0 \supset G_1 \supset \dotsb\)
		von \(0\) aus Normalteilern. Wir setzen \(A_n \coloneqq G/G_n\) für alle \(n \in \set N_0\).
	\item
		Vermöge der kanonischen Homomorphismen \(A_n = G/G_n \surjto A_{n - 1} = G/G_{n - 1}\) wird
		\(A_\bullet\colon \dotsb \to A_2 \to A_1 \to A_0\) zu einem inversen System topologischer Gruppen.
	\item
		Ist dann \((x_{\nu})\) eine Cauchy-Folge in \(G\), so hängt die Äquivalenzklasse \(\xi_n \in A_n\) von
		\(x_{\nu}\) modulo \(G_n\) für \(\nu \gg 0\) nicht von \(\nu\) ab.
	\item
		Die Folge \((\xi_n)\) ist offensichtlich eine kohärente Folge in \(A_\bullet\). Wir erhalten also
		eine kanonische Abbildung \(\hat G \to \varprojlim\limits_n A_n, [x_\nu] \mapsto (\xi_n)\),
		welche ein Gruppenhomomorphismus ist.
	\end{enumerate}
\end{frame}

\begin{frame}{Die Vervollständigung als inverser Limes}
	\begin{proposition}<+->
		Sei \(G\) eine topologische Gruppe mit einer Umgebungsbasis \(G_0 \supset G_1 \supset \dotsb\).
		Der kanonische Gruppenhomomorphismus \(\hat G \to \varprojlim\limits_n G/G_n\) ist ein Isomorphismus
		topologischer Gruppen, das heißt ein Gruppenisomorphismus, welcher gleichzeitig ein Homöomorphismus ist.
	\end{proposition}
	\begin{proof}<+->
		Wir geben die Umkehrabbildung an: Ist \((\xi_n) \in \varprojlim\limits_n G/G_n\) eine kohärente Folge,
		so definieren wir \((x_\nu)\), indem wir für jedes \(\nu\) ein \(x_\nu\) mit \(\xi_\nu = x_\nu + G_\nu\)
		wählen.
	\end{proof}
\end{frame}




\lecture{Vervollständigungen II}{Vervollst\"andigungen II}
\mode<all>\setcounter{section}{38}
\mode<all>\section{Vervollständigungen II}

\subsection{Exaktheitseigenschaften inverser Limiten}

\begin{frame}{Surjektive Systeme}
	\begin{definition}<+->
		Ein inverses System \(\dotsb \to A_1 \to A_0\) von Gruppen heißt \emph{surjektives System}, falls
		die Gruppenhomomorphismen \(A_n \to A_{n - 1}\) surjektiv sind.
	\end{definition}
	\begin{example}<+->
		Sei \(G\) eine topologische Gruppe mit einer Umgebungsbasis \(G_0 \supset G_1 \supset \dotsb\) von
		\(0\) aus Normalteilern. Dann ist das inverse System \(\dotsb \to G/G_n \to \dotsb \to G/G_1 \to G/G_0\)
		ein surjektives.
	\end{example}
\end{frame}

\begin{frame}{Homomorphismen inverser Systeme}
	\begin{definition}<+->
		Seien \(A_\bullet\colon \dotsb \xrightarrow{\alpha_1} A_1 \xrightarrow{\alpha_0} A_0\) und
		\(B_\bullet\colon \dotsb \xrightarrow{\beta_1} B_1 \xrightarrow{\beta_0} B_0\) inverse Systeme
		topologischer Gruppen. Eine Familie \((\phi_n\colon A_n \to B_n)_n\) stetiger Gruppenhomomorphismen
		ist ein \emph{Homomorphismus inverser Systeme}, wenn \(\beta_n \circ \phi_n = \phi_{n - 1} \circ \alpha_n\colon A_n \to
		B_{n - 1}\) für alle \(n \ge 1\).
	\end{definition}
	\begin{example}<+->
		Sind \(\phi_\bullet\colon A_\bullet \to B_\bullet\)
		und \(\psi_\bullet\colon B_\bullet \to C_\bullet\) zwei Homomorphismen inverser Systeme, so ist ihre
		\emph{Verknüpfung} \(\psi_\bullet \circ \phi_\bullet \coloneqq (\psi_n \circ \phi_n)\colon A_\bullet \to
		C_\bullet\) wieder ein Homomorphismus inverser Systeme.
	\end{example}
	\begin{example}<+->
		Ist \(A_\bullet\) ein inverses System, so ist \(\id_{A_\bullet} \coloneqq (\id_{A_n})\colon
		A_\bullet \to A_\bullet\) ein Homomorphismus inverser Systeme.
	\end{example}
\end{frame}

\begin{frame}{Homomorphismen inverser Systeme und der inverse Limes}
	Sei \(\phi_\bullet\colon A_\bullet \to B_\bullet\) ein Homomorphismus inverser Systeme. Dann definiert
	\[
		\varprojlim\limits_n \phi_n\colon \varprojlim\limits_n A_n \to \varprojlim\limits_n B_n,
		(\xi_n)_n \mapsto (\phi_n(\xi_n))_n
	\]
	einen stetigen Gruppenhomomorphismus.
	\begin{example}<+->
		Ist \(\psi_\bullet\colon B_\bullet \to C_\bullet\) ein weiterer Homomorphismus inverser Systeme,
		so ist
		\((\varprojlim\limits_n \psi_n) \circ (\varprojlim\limits_n \phi_n) = \varprojlim\limits_n (\psi_n \circ \phi_n)\colon
		\varprojlim\limits_n A_n \to \varprojlim\limits_n C_n\).
	\end{example}
	\begin{example}<+->
		Es ist \(\varprojlim\limits_n \id_{A_n} = \id_{\varprojlim\limits A_n}\).
	\end{example}
\end{frame}

\begin{frame}{Exakte Sequenzen inverser Systeme}
	\begin{definition}<+->
		Eine Sequenz \(\dotsb \to A_\bullet^{i - 1} \xrightarrow{\phi_\bullet^{i - 1}} A_\bullet^i \xrightarrow{\phi_\bullet^i}
		A_\bullet^{i + 1} \to \dotsb\) inverser Systeme abelscher Gruppen \(A_\bullet^i\)
		heißt \emph{exakt bei \(A_\bullet^i\)},
		falls die induzierten Sequenzen \(A_n^{i - 1} \xrightarrow{\phi_n^{i - 1}} A_n^i \xrightarrow{\phi_n^i} A_n^{i + 1} \to \dotsb\)
		für alle \(n \in \set N_0\) exakt sind.
	\end{definition}
	\begin{visibleenv}<+->
		Den Begriff der kurzen exakten Sequenz inverser Systeme abelscher Gruppen definieren wir auf die offensichtliche Art
		und Weise.
	\end{visibleenv}
	\begin{remark}<+->
		Ist \(\dotsb \to A_\bullet^{i - 1} \xrightarrow{\phi_\bullet^{i - 1}} A_\bullet^i \xrightarrow{\phi_\bullet^i}
		A_\bullet^{i + 1} \to \dotsb\) eine exakte Sequenz inverser Systeme abelscher Gruppen, so ist die
		induzierte Sequenz
		\(\dotsb \to \varprojlim\limits_n A_n^{i - 1} \to \varprojlim\limits_n A_n^i \to \varprojlim\limits_n A_n^{i + 1} \to \dotsb\)
		abelscher Gruppen im allgemeinen nicht mehr exakt.
	\end{remark}
\end{frame}

\begin{frame}{Linksexaktheit des inversen Limes}
	\begin{proposition}<+->
		Sei \(0 \to A_\bullet \to B_\bullet \to C_\bullet \to 0\) eine exakte Sequenz inverser Systeme abelscher Gruppen. Dann
		ist die induzierte Sequenz \(0 \to \varprojlim\limits_n A_n \to \varprojlim\limits_n B_n \to \varprojlim\limits_n C_n\)
		exakt.
	\end{proposition}
\end{frame}

\begin{frame}{Beweis der Linksexaktheit}
	\begin{proof}<+->
		\begin{enumerate}[<+->]
		\item<.->
			Sei \(A_\bullet\colon \dotsb \xrightarrow{\alpha_2} A_1 \xrightarrow{\alpha_1} A_0\).
			Sei \(A \coloneqq \prod\limits_n A_n\). Setze \(d^A\colon A \to A, (a_n)_n \mapsto
			(a_n - \alpha_{n + 1}(a_{n + 1}))_n\). Dann ist \(\ker d^A = \varprojlim\limits_n A_n\).
		\item
			Definiere \(B, C\) und \(d^B, d^C\) analog. Dann ist
			\[
				\begin{CD}
					0 @>>> A @>>> B @>>> C @>>> 0 \\
					& & @V{d^A}VV @V{d^B}VV @V{d^C}VV \\
					0 @>>> A @>>> B @>>> C @>>> 0
				\end{CD}
			\]
			ein kommutatives Diagram mit exakten Reihen.
		\item
			Nach dem Schlangenlemma ist \(0 \to \ker d^A \to \ker d^B \to \ker d^C \to {\varprojlim\limits_n}^1 A_n\)
			mit \({\varprojlim\limits_n}^1 A_n \coloneqq \coker d^A\) exakt.
			\qedhere
		\end{enumerate}
	\end{proof}
\end{frame}

\begin{frame}{Exaktheit bei surjektiven Systemen}
	\begin{proposition}<+->
		Sei \(0 \to A_\bullet \to B_\bullet \to C_\bullet \to 0\) eine exakte Sequenz inverser Systeme abelscher Gruppen. 
		Sei weiter \(A_\bullet\) ein surjektives System.
		Dann
		ist die induzierte Sequenz \(0 \to \varprojlim\limits_n A_n \to \varprojlim\limits_n B_n \to \varprojlim\limits_n C_n \to 0\)
		exakt.
	\end{proposition}
	\begin{proof}<+->
		Sei \(A_\bullet\colon \dotsb \xrightarrow{\alpha_2} A_1 \xrightarrow{\alpha_1} A_0\).
		Es ist zu zeigen, daß \({\varprojlim\limits_n}^1 A_n = 0\), daß also \(d^A\) surjektiv ist. Dies folgt aus der Tatsache,
		daß sich die Gleichungen \(\eta_n - \alpha_{n + 1}(\eta_{n + 1}) = \xi_n\) induktiv für \(\eta_n \in A_n\) lösen lassen, wenn
		die \(\xi_n \in A\) gegeben sind.
	\end{proof}
\end{frame}

\subsection{Vollständige topologische Gruppen}

\begin{frame}{Exaktheit der Vervollständigung}
	\begin{corollary}<+->
		Sei \(0 \to G' \xrightarrow{\iota} G \xrightarrow{\pi} G'' \to 0\) eine exakte Sequenz
		abelscher Gruppen. Definiere eine Folge \(G_0 \supset G_1 \supset \dotsb\) von
		Untergruppen von \(G\) eine Topologie auf \(G\). Versehen wir \(G'\) und \(G''\)
		mit den durch die Folgen \(\iota^{-1} G_0 \supset \iota^{-1} G_1 \supset \dotsb\) bzw.\
		\(\pi(G_0) \supset \pi(G_1) \supset \dotsb\) definierten Topologien, so ist
		die induzierte Sequenz
		\(0 \to \hat G' \to \hat G \to \hat G'' \to 0\) exakt.
	\end{corollary}
	\begin{proof}<+->
		Die Sequenzen \(0 \to G'/\iota^{-1} (G_n) \to G/G_n \to G/\pi(G_n) \to 0\) sind exakt
		und bilden eine exakte Sequenz surjektiver Systeme. Damit ist ihr inverser Limes
		\(0 \to \hat G' \to \hat G \to \hat G'' \to 0\) exakt.
	\end{proof}
\end{frame}

\begin{frame}{Folgerung aus der Exaktheit}
	\begin{corollary}<+->
		Sei \(G\) eine abelsche topologische Gruppe. Sei \(G_0 \supset G_1 \supset \dotsb\) eine
		Umgebungsbasis von \(0\) aus Untergruppen. Für jedes \(n\) induziert der kanonische Homomorphismus
		\(G \to \hat G\) dann einen Isomorphismus \(G/G_n \to \hat G/\hat G_n\).
	\end{corollary}
	\begin{proof}<+->
		Es ist \(0 \to G_n \to G \to G/G_n \to 0\) eine exakte Sequenz. Versehen wir \(G_n\) mit
		der Teilraumtopologie und \(G/G_n\) mit der diskreten Topologie,
		können wir die vorherige Proposition anwenden und erhalten die exakte Sequenz
		\(0 \to \hat{G_n} \to \hat G \to \widehat{G/G_n} = G/G_n \to 0\).
	\end{proof}
\end{frame}

\begin{frame}{Vollständige topologische Gruppen}
	\begin{definition}<+->
		Eine topologische Gruppe heißt \emph{vollständig}, wenn der kanonische Homomorphismus
		\(G \to \hat G\) ein Isomorphismus ist.
	\end{definition}
	\begin{remark}<+->
		Eine vollständige topologische Gruppe ist also insbesondere hausdorffsch.
	\end{remark}
	\begin{proposition}<+->
		Sei \(G\) eine abelsche topologische Gruppe. Sei \(G_0 \supset G_1 \supset \dotsb\) eine
		Umgebungsbasis von \(0\) aus Untergruppen. Dann ist die Vervollständigung \(\hat G\) vollständig.
	\end{proposition}
	\begin{proof}<+->
		Es ist \(\hat G \cong \varprojlim\limits_n G/G_n \cong \varprojlim\limits_n \hat G/\hat G_n
		\cong \hat{\hat G}\).
	\end{proof}
\end{frame}

\subsection{Topologische Ringe und Moduln}

\begin{frame}{Topologische Ringe}
	\begin{definition}<+->
		Ein \emph{topologischer Ring} ist ein Ring \(A\), dessen additive Gruppe eine topologische Gruppe ist, so
		daß die Multiplikation \(A \times A \to A, (x, y) \mapsto x y\) eine stetige Abbildung ist.
	\end{definition}
	\begin{example}<+->
		Sei \(\ideal a\) ein Ideal eines Ringes \(A\). Dann gibt es genau eine Topologie auf
		der additiven Gruppe von \(A\), so daß \((1) \supset \ideal a \supset \ideal a^2 \supset \dotsb\) zu einer
		Umgebungsbasis von~\(0\) wird, die \emph{\(\ideal a\)-adische Topologie}.
		\\
		Da die \(\ideal a^n\) Ideale sind, läßt sich zeigen, daß \(A\) damit zu einem topologischen Ring wird.
	\end{example}
	\begin{remark}<+->
		Die Vervollständigung \(\hat A\) eines topologischen Ringes \(A\) als topologische Gruppe ist wieder in kanonischer Weise
		ein Ring. Außerdem ist die kanonische Abbildung \(A \to \hat A\) ein stetiger Ringhomomorphismus.
	\end{remark}
\end{frame}

\begin{frame}{Vervollständigung topologischer Ringe}
	Sei \(\ideal a\) ein Ideal eines Ringes \(A\). Wir versehen \(A\) mit der \(\ideal a\)-adischen Topologie.
	\begin{example}<+->
		Es ist \(A\) genau dann hausdorffsch, wenn \(\bigcap\limits_n \ideal a^n = (0)\).
	\end{example}
	\begin{visibleenv}<+->
		Mit \(\hat A = \hat A_{\ideal a}\) bezeichnen wir die Vervollständigung von \(A\).
	\end{visibleenv}
	\begin{definition}<.->
		Der Ring \(\hat A_{\ideal a}\) heißt die \emph{\(\ideal a\)-adische Vervollständigung von \(A\)}.
	\end{definition}
\end{frame}

\begin{frame}{Topologische Moduln}
	Sei \(A\) ein topologischer Ring. 
	\begin{definition}<+->
		Ein \emph{topologischer \(A\)-Modul \(M\)} ist ein \(A\)-Modul \(M\), dessen
		additive Gruppe eine topologische Gruppe ist, so daß die Multiplikation \(A \times M \to M, (a, x) \mapsto a x\)
		eine stetige Abbildung ist.
	\end{definition}
	\begin{remark}<+->
		Die Vervollständigung \(\hat M\) eines topologischen \(A\)-Moduls \(M\) als topologische Gruppe ist in kanonischer Weise
		ein \(\hat A\)-Modul. Außerdem ist die kanonische Abbildung \(M \to \hat M\) ein stetiger Homomorphismus
		\(M \to \hat M^{A}\) von \(A\)-Moduln.
	\end{remark}
\end{frame}

\begin{frame}{Vervollständigung topologischer Moduln}
	Sei \(\ideal a\) ein Ideal eines Ringes \(A\). Sei \(M\) ein \(A\)-Modul. 
	\begin{example}<+->
		Sei \(\ideal a\) ein Ideal eines Ringes \(A\). Sei \(M\) ein \(A\)-Modul. Dann gibt es genau eine Topologie
		auf der additiven Gruppe von \(M\), so daß \(M \supset \ideal a M \supset \ideal a^2 M \supset \dotsb\) zu einer
		Umgebungsbasis von \(0\) wird, die \emph{\(\ideal a\)-adische Topologie}.
		\\
		In der Tat wird \(M\) damit zu einem topologischen \(A\)-Modul, wenn \(M\) mit der \(\ideal a\)-adischen Topologie
		versehen wird.
	\end{example}
	\begin{visibleenv}<+->
		Wir versehen \(A\) und \(M\) mit der
		\(\ideal a\)-adischen Topologie.
		\\
		Mit \(\hat M = \hat M_{\ideal a}\) bezeichnen wir die Vervollständigung von \(M\).
	\end{visibleenv}
	\begin{definition}<+->
		Der \(\hat A_{\ideal a}\)-Modul \(\hat M_{\ideal a}\) heißt die \emph{\(\ideal a\)-adische Vervollständigung von \(M\)}.
	\end{definition}
\end{frame}

\begin{frame}{Vervollständigung von Homomorphismen}
	Sei \(\ideal a\) ein Ideal eines Ringes \(A\). Sei \(\phi\colon M \to N\) ein Homomorphismus von \(A\)-Moduln.
	\\
	Dann gilt \(\phi(\ideal a^n M) \subset \ideal a^n \phi(M) \subset \ideal a^n N\), es ist
	\(\phi\) damit stetig bezüglich der \(\ideal a\)-adischen Topologien auf \(M\) und \(N\).
	\\
	Damit definiert \(\phi\) eine Abbildung \(\hat\phi = \hat\phi_{\ideal a}\colon \hat M_{\ideal a} \to \hat N_{\ideal a}\).
	\begin{proposition}<+->
		Die Abbildung \(\hat\phi_{\ideal a}\colon \hat M_{\ideal a} \to \hat N_{\ideal a}\) ist ein stetiger
		Homomorphismus topologischer Ringe.
		\qed
	\end{proposition}
\end{frame}

\begin{frame}{Beispiele Vervollständigungen topologischer Ringe}
	\begin{example}<+->
		Sei \(K\) ein Körper. Vervollständigen wir \(K[x]\) bezüglich der \((x)\)-adischen Topologie, so erhalten
		wir \(\widehat{K[x]}_{(x)} = \ps K x\) den Potenzreihenring über \(K\) als Vervollständigung.
	\end{example}
	\begin{example}<+->
		Sei \(p\) eine Primzahl. Die Vervollständigung \(\set Z_p \coloneqq \hat{\set Z}_{(p)}\) heißt der
		\emph{Ring der \(p\)-adischen Zahlen}. Elemente in \(\set Z\) können wir als Reihen
		\(\sum_{n = 0}^\infty a_n p^n\) mit \(0 \le a_n < p\) darstellen.
		\\
		Es gilt \(\lim\limits_n p^n = 0\) in diesem Ring.
	\end{example}
\end{frame}



\lecture{Gewichtete Ringe und Moduln}{Gewichtete Ringe und Moduln}
\mode<all>\setcounter{section}{39}
\mode<all>\section{Filtrationen}

\subsection{Filtrationen}

\begin{frame}{Definition einer Filtration}
	Sei \(A\) ein Ring. Sei \(M\) ein \(A\)-Modul.
	\begin{definition}<+->
		\begin{enumerate}[<+->]
		\item<.->
			Eine (unendliche) Folge
			\(M_\bullet\colon M = M_0 \supset M_1 \supset M_2 \supset \dotsb\) von Untermoduln von
			\(M\) heißt eine \emph{Filtration von \(M\)}.
		\item
			Sei \(\ideal a\) ein Ideal von \(A\). Die Filtration \(M_\bullet\) heißt eine
			\emph{\(\ideal a\)-Filtration}, falls \(\ideal a M_n \subset M_{n + 1}\) für alle \(n\).
		\item
			Eine \(\ideal a\)-Filtration \(M_\bullet\) heißt \emph{stabil}, falls
			\(\ideal a M_n = M_{n + 1}\) für \(n \gg 0\).
		\end{enumerate}
	\end{definition}
	\begin{example}<+->
		Für jedes Ideal \(\ideal a\) von \(A\) ist \(M \supset \ideal a M \supset \ideal a^2 M \supset \dotsb\)
		eine stabile \(\ideal a\)-Filtration von \(M\), die \emph{\(\ideal a\)-adische Filtration von \(M\)}.
	\end{example}
\end{frame}

\begin{frame}{Topologie durch \(\ideal a\)-stabiler Filtrationen}
	\begin{lemma}<+->
		Sei \(\ideal a\) ein Ideal eines Ringes \(A\). Sei \(M\) ein \(A\)-Modul. Je
		zwei stabile \(\ideal a\)-adische Filtrationen \(M_\bullet\) und \(M_\bullet'\) 
		von \(M\) haben eine beschränkte Differenz, das heißt es existiert ein \(n_0 \in \set N_0\) mit
		\(M_{n + n_0} \subset M_n'\) und \(M'_{n + n_0} \subset M_n\) für alle \(n \ge 0\).
	\end{lemma}
	\begin{visibleenv}<+->
		Damit sind zwei stabile \(\ideal a\)-adische Filtrationen Umgebungsbasen von \(0\) ein- und derselben
		Topologie auf \(M\).
	\end{visibleenv}
	\begin{proof}<+->
		Für \(n_0 \gg 0\) gilt
		\(M'_{n + n_0} = \ideal a^n M'_{n_0} \subset \ideal a^n M = \ideal a^n M_0 \subset \ideal a^{n - 1} M_1 \subset \dotsb
		\subset M_n\).
	\end{proof}
\end{frame}


\mode<all>\section{Gewichtete Ringe und Moduln I}

\subsection{Definition gewichteter Ringe und Moduln}

\begin{frame}{Gewichtete Ringe}
	\begin{definition}<+->
		Ein \emph{gewichteter Ring} ist ein Ring \(A\) zusammen mit einer
		Familie \((A_n)_{n \in \set N_0}\) von Untergruppen der additiven
		Gruppen von \(A\), so daß
		\(\bigoplus\limits_{n \in \set N_0} A_n \to A, (a_n) \mapsto
		\sum\limits_n a_n\) ein Gruppenisomorphismus ist und
		\(A_m A_n \subset A_{n + m}\) für alle \(m, n \in \set N_0\).
	\end{definition}
	\begin{visibleenv}<+->
		Aus der Definition folgt insbesondere, daß \(A_0\) ein Unterring
		von \(A\) ist und daß jedes \(A_n\) ein \(A_0\)-Modul ist.
	\end{visibleenv}
\end{frame}

\begin{frame}{Der Polynomring als gewichteter Ring}
	\begin{example}<+->
		Sei \(K\) ein Körper. Sei \(A \coloneqq K[X_1, \dotsc, X_r]\) der
		Polynomring in \(r\) Variablen. 
		\\
		Für alle \(n \in \set N_0\) sei \(A_n\) die Untergruppe der homogenen
		Polynome vom Grad \(n\) (inklusive des Nullpolynoms).
		\\
		Mit dieser Setzung wird \(A\) zu einem kommutativen gewichteten Ring.
	\end{example}
\end{frame}

\begin{frame}{Das irrelevante Ideal}
	\begin{visibleenv}<+->
		Sei \(A\) ein gewichteter Ring. Die Menge
		\(A_+ \coloneqq \sum\limits_{n > 0} A_n\) ist ein Ideal von \(A\).
	\end{visibleenv}
	\begin{notation}<+->
		Wir nennen \(A_+\) das \emph{irrelevante Ideal von \(A\)}.
	\end{notation}
	\begin{proposition}<+->
		Es ist \(A_0 \to A/A_+, x \mapsto [x]_{A_+}\) ein Ringisomorphismus.
		\qed
	\end{proposition}
\end{frame}

\begin{frame}{Gewichtete Moduln}
	\begin{definition}<+->
		Sei \(A\) ein gewichteter Ring. Ein \emph{gewichteter \(A\)-Modul} ist
		ein \(A\)-Modul \(M\) zusammen mit einer Familie \((M_n)_{n \in \set N_0}\)
		von Untergruppen der additiven Gruppe von \(M\), so daß
		\(\bigoplus\limits_n M_n \to M, (m_n) \mapsto \sum\limits_n m_n\)
		ein Gruppenisomorphismus ist und \(A_m M_n \subset M_{m + n}\)
		für alle \(m, n \in \set N_0\).
	\end{definition}
	\begin{visibleenv}<+->
		Aus der Definition folgt insbesondere, daß jedes \(M_n\) ein \(A_0\)-Modul
		ist.
	\end{visibleenv}
\end{frame}

\begin{frame}{Homogene Komponenten}
	\begin{definition}<+->
		Sei \(A\) ein gewichteter Ring. Sei \(M\) ein gewichteter \(A\)-Modul. 
		Sei \(n \in \set N_0\). Ein Element \(x \in M\) heißt \emph{homogen vom
		Gewicht \(n\)}, falls \(x \in M_n\).
	\end{definition}
	\begin{visibleenv}<+->
		Offensichtlich kann jedes Element \(x \in M\) eindeutig als Summe
		\(x = \sum\limits_n x_n\) geschrieben werden, wobei die \(x_n\) jeweils
		homogen vom Gewicht \(n\) sind und fast alle \(x_n\) verschwinden
		(d.h.\ die Summe ist endlich).
		\\
		Die nicht verschwindenden \(x_n\) heißen die \emph{homogenen Komponenten
		von \(x\)}.
	\end{visibleenv}
\end{frame}

\begin{frame}{Homomorphismen gewichteter Moduln}
	\begin{definition}<+->
		Sei \(A\) ein gewichteter Ring. Seien \(M, N\) zwei gewichtete \(A\)-Moduln.
		Ein \emph{Homomorphismus \(\phi\colon M \to N\) gewichteter \(A\)-Moduln}
		ist ein Homomorphismus \(\phi\colon M \to N\) von \(A\)-Moduln mit
		\(\phi(M_n) \subset N_n\) für alle \(n \in \set N_0\).
	\end{definition}
\end{frame}

\begin{frame}{Noethersche gewichtete Ringe}
	\begin{proposition}<+->
		Ein kommutativer gewichteter Ring ist genau dann noethersch, wenn
		\(A_0\) noethersch ist und \(A\) als \(A_0\)-Algebra endlich erzeugt ist.
	\end{proposition}
\end{frame}

\begin{frame}{Beweis zu noetherschen gewichteten Ringen}
	\begin{proof}<+->
		\begin{enumerate}[<+->]
		\item<.->
			Ist \(A_0\) noethersch und \(A\) als \(A_0\)-Algebra endlich erzeugt,
			so ist \(A\) nach dem Hilbertschen Basissatz noethersch.
		\item
			Sei umgekehrt \(A\) noethersch. Damit ist \(A_0 \cong A/A_+\) als
			Quotient ebenfalls noethersch. Es bleibt, endlich viele Erzeuger
			von \(A\) als \(A_0\)-Algebra zu finden. Zunächst ist
			\(A_+\) als Ideal von \(A\) endlich erzeugt, etwa von \(x_1, \dotsc,
			x_r\). Ohne Einschränkung seien die \(x_i\) jeweils homogen von
			den Gewichten \(d_i > 0\).
		\item
			Sei \(A'\) die von den \(x_i\) über \(A_0\) erzeugte Unteralgebra
			von \(A\). Wir zeigen per Induktion, daß \(A_n \subset A'\) für
			alle \(n\). Wir können \(n > 0\) annehmen. Sei \(y \in A_n\).
		\item
			Da \(y \in A_+\), existieren homogene \(a_i \in A\) 
			mit \(y = \sum\limits_i a_i x_i\). Die Gewichte der \(a_i\) sind echt
			kleiner als \(n\). Nach Induktionsvoraussetzung ist daher \(a_i \in A'\)
			und damit auch \(y \in A'\).
			\qedhere
		\end{enumerate}
	\end{proof}
\end{frame}

\subsection{Reessche Ringe und Moduln}

\begin{frame}{Der Reessche Ring}
	Sei \(\ideal a\) ein Ideal in einem kommutativen Ring \(A\). Mit
	\(\Rees_{\ideal a}(t) = \Rees_{\ideal a}(A, t)\) bezeichnen wir die Teilmenge aller derjenigen Polynome
	\(a_n t^n + a_{n - 1} t^{n - 1} + \dotsb + a_0 \in A[t]\) mit
	\(a_i \in \ideal a^i\).
	\\
	Durch die Setzung \(\Rees_{\ideal a}(t)_n = \ideal a^n t^n\) wird
	\(\Rees_{\ideal a}(t)\) zu einem kommutativen gewichteten Ring.
	\begin{definition}<+->
		Der gewichtete Ring \(\Rees_{\ideal a}(A, t)\) heißt der \emph{Reessche
		Ring von \(A\) bezüglich \(\ideal a\)}.
	\end{definition}
\end{frame}

\begin{frame}{Reessche Ringe noetherscher Ringe}
	\begin{proposition}<+->
		Sei \(\ideal a\) ein Ideal in einem kommutativen Ring \(A\). Ist
		\(A\) noethersch, so ist auch der Reessche Ring
		\(\Rees_{\ideal a}(A, t)\) noethersch.
	\end{proposition}
	\begin{proof}<+->
		\begin{enumerate}[<+->]
		\item<.->
			Ist \(A\) noethersch, so ist insbesondere das Ideal \(\ideal a\)
			endlich erzeugt, etwa von \(x_1, \dotsc, x_r\).
		\item
			Damit ist \(\Rees_{\ideal a}(t)\) als \(A\)-Algebra von
			\(x_1 t, \dotsc, x_r t\) erzeugt. Nach dem Hilbertschen Basissatz
			ist \(\Rees_{\ideal a}(t)\) damit auch noethersch.
			\qedhere
		\end{enumerate}
	\end{proof}
\end{frame}

\begin{frame}{Der Reessche Modul}
	\begin{visibleenv}<+->
		Sei \(\ideal a\) ein Ideal in einem kommutativen Ring \(A\). Sei
		\(M\) ein \(A\)-Modul zusammen mit einer \(\ideal a\)-Filtrierung
		\(M_\bullet\colon M = M_0 \supset M_1 \supset \dotsb\). 
		\\
		Mit \(\Rees(M_\bullet, t)\) bezeichnen wir die Teilmenge aller derjenigen
		Polynome \(m_n t^n + m_{n - 1} t^{n - 1} + \dotsb + m_0 \in M[t]\)
		mit \(m_i \in M_i\).
		\\
		Durch die Setzung \(\Rees(M_\bullet, t)_n \coloneqq M_n t^n\) wird
		\(\Rees(M_\bullet, t)\) wegen \(\ideal a^m M_n \subset M_{m + n}\)
		zu einem gewichteten \(\Rees_{\ideal a}(A, t)\)-Modul.
	\end{visibleenv}
	\begin{definition}<+->
		Der \(\Rees_{\ideal a}(A, t)\)-Modul \(\Rees(M_\bullet, t)\) heißt der
		\emph{Reessche Modul zur Filtration \(M_\bullet\)}.
	\end{definition}
	\begin{notation}<+->
		Im Falle der \(\ideal a\)-adischen Filtrierung \(M_n = \ideal a^n M\) schreiben wir
		\(\Rees_{\ideal a}(M, t) = \Rees(M_\bullet, t)\).
	\end{notation}
\end{frame}

\begin{frame}{Endliche erzeugte Reessche Moduln}
	\begin{proposition}<+->
		Sei \(\ideal a\) ein Ideal in einem noetherschen kommutativen Ring
		\(A\). Sei \(M\) ein endlich erzeugter \(A\)-Modul zusammen mit einer
		\(\ideal a\)-Filtration \(M_\bullet\). Dann ist
		\(\Rees(M_\bullet, t)\) genau dann ein endlich erzeugter \(\Rees_{\ideal a}(A, t)\)-Modul,
		wenn die Filtration \(M_\bullet\) stabil ist.
	\end{proposition}
	\begin{proof}<+->
		\begin{enumerate}[<+->]
		\item<.->
			Die \(A\)-Moduln \(M_n\) sind endlich erzeugt. Damit ist auch
			\(Q_n \coloneqq \sum\limits_{r = 0}^n M_r t^r \subset \Rees(M_\bullet, t)\)
			ein endlich erzeugter \(A\)-Modul.
		\item
			Der von \(Q_n\) in \(\Rees(M_\bullet, t)\) erzeugte \(\Rees_{\ideal a}(t)\)-Untermodul
			\(Q^*_n\) ist damit als \(\Rees_{\ideal a}(t)\)-Modul endlich erzeugt.
			Es ist \(\Rees(M_\bullet, t) = \sum\limits_n Q^*_n\).
		\item
			Da \(\Rees_{\ideal a}(t)\) noethersch ist, ist \(\Rees(M_\bullet, t)\)
			genau dann als \(\Rees_{\ideal a}(t)\)-Modul endlich erzeugt, wenn
			\(\Rees(M_\bullet, t) = Q^*_{n_0}\) für ein \(n_0 \in \set N_0\), wenn also
			\(M_{n_0 + n} = \ideal a^n M_{n_0}\) für alle \(n \ge 0\), wenn
			die Filtration also stabil ist.
			\qedhere
		\end{enumerate}
	\end{proof}
\end{frame}

\subsection{Das Artin--Reessche Lemma}

\begin{frame}{Das Artin--Reessche Lemma}
	\begin{proposition}<+->
		Sei \(\ideal a\) ein Ideal in einem noetherschen kommutativen Ring
		\(A\). Sei \(M\) ein endlich erzeugter \(A\)-Modul zusammen mit einer
		stabilen \(\ideal a\)-Filtration \(M_\bullet\). Für jeden Untermodul
		\(M'\) von \(M\) ist dann \(M' \cap M_\bullet\colon M' = M' \cap M_0
		\supset M' \cap M_1 \supset M' \cap M_2 \supset \dotsb\) eine
		stabile \(\ideal a\)-Filtration von \(M'\).
	\end{proposition}
	\begin{proof}<+->
		\begin{enumerate}[<+->]
		\item<.->
			Da \(\ideal a (M' \cap M_n) \subset \ideal a M' \cap \ideal a M_n
			\subset M' \cap M_{n + 1}\), ist \(M' \cap M_\bullet\) eine
			\(\ideal a\)-Filtration.
		\item
			Der Reessche Modul \(Q^*\) zur Filtration \(M' \cap M_\bullet\)
			von \(M'\) ist ein gewichteter \(\Rees_{\ideal a}(t)\)-Modul, und zwar
			ein Untermodul des endlich erzeugten \(\Rees_{\ideal a}(t)\)-Moduls
			\(\Rees(M_\bullet, t)\).
		\item
			Da \(\Rees_{\ideal a}(t)\) noethersch ist, ist damit auch
			\(Q^*\) endlich erzeugt. Nach dem letzten Hilfssatz ist damit
			\(M' \cap M_\bullet\) eine stabile Filtration.
			\qedhere
		\end{enumerate}
	\end{proof}
\end{frame}

\begin{frame}{Das spezielle Artin--Reessche Lemma}
	\begin{corollary}<+->
		Sei \(\ideal a\) ein Ideal in einem noetherschen kommutativen Ring \(A\).
		Sei \(M\) ein endlich erzeugter \(A\)-Modul. Dann existiert für jeden Untermodul \(M' \subset M\)
		ein \(n_0 \in \set N_0\),
		so daß
		\[
			(\ideal a^n M) \cap M' = \ideal a^{n - n_0} ((\ideal a^{n_0} M) \cap M')
		\]
		für alle \(n \ge n_0\).
	\end{corollary}
	\begin{proof}<+->
		Ist die Aussage der Proposition, wenn wir die \(\ideal a\)-adische Filtration
		auf \(M\) wählen.
	\end{proof}
\end{frame}

\begin{frame}{\(\ideal a\)-adische Topologien auf Untermoduln}
	\begin{theorem}<+->
		Sei \(A\) ein noetherscher kommutativer Ring. Sei \(\ideal a\) ein
		Ideal in \(A\). Sei \(M\) ein endlich erzeugter \(A\)-Modul und
		\(M'\) ein Untermodul von \(M\). Dann haben die Filtrationen
		\(M' \supset \ideal a M' \supset \ideal a^2 M' \supset \dotsb\) und
		\(M' \supset (\ideal a M) \cap M' \supset (\ideal a^2 M) \cap M'
		\supset \dotsb\) beschränkte Differenz.
	\end{theorem}
	\begin{visibleenv}<+->
		Insbesondere stimmt die \(\ideal a\)-adische Topologie auf \(M'\) mit
		der von der \(\ideal a\)-adischen Topologie auf \(M\) induzierten
		Teilraumtopologie überein.
	\end{visibleenv}
	\begin{proof}<+->
		Nach dem Artin--Reesschen Lemma ist die zweite Filtration eine stabile.
		Die erste ist es trivialerweise. Damit haben sie beschränkte Differenz.
	\end{proof}
\end{frame}



\lecture{Der assoziierte gewichtete Ring}{Der Krullsche Satz}
\mode<all>\setcounter{section}{41}
\mode<all>\section{Gewichtete Ringe und Moduln II}

\subsection{Exaktheit der Vervollständigung}

\begin{frame}{Exaktheit der Vervollständigung}
	\begin{proposition}<+->
		Sei \(A\) ein noetherscher kommutativer Ring. Sei \(0 \to M' \to M \to
		M'' \to 0\) eine exakte Sequenz endlich erzeugter \(A\)-Moduln.
		Für jedes \(\ideal a\) von \(A\) ist dann die induzierte Sequenz
		\(0 \to \hat M'_{\ideal a} \to \hat M_{\ideal a} \to \hat M''_{\ideal a}
		\to 0\) wieder exakt.
	\end{proposition}
	\begin{proof}<+->
		\begin{enumerate}[<+->]
		\item<.->
			Die \(\ideal a\)-adische Topologie auf \(M'\) ist nach dem 
			letzten Satz die von
			der \(\ideal a\)-adischen Topologie auf \(M\) induzierte. 
		\item
			Da \(\phi(\ideal a^n M) = \ideal a^n \phi(M) = \ideal a^n M''\)
			ist außerdem die \(\ideal a\)-adische Topologie auf \(M''\) die
			von der \(\ideal a\)-adischen Topologie auf \(M\) induzierte.
		\item
			Damit ist die Vervollständigung der exakten Sequenz bezüglich dieser
			Topologien wieder exakt.
			\qedhere
		\end{enumerate}
	\end{proof}
\end{frame}

\begin{frame}{Vervollständigung von Moduln}
	\begin{remark}<+->
		Sei \(A\) ein kommutativer Ring. Sei \(\ideal a\) ein
		Ideal von \(A\). Durch den kanonischen Homomorphismus \(A \to \hat A_{\ideal a}\)
		können wir die Vervollständigung in kanonischer Weise als \(A\)-Modul
		auffassen.
		\\
		Zu jedem \(A\)-Modul \(M\) können wir damit die Skalarerweiterung
		\(M_{\hat A_{\ideal a}}\) definieren. Der kanonische Homomorphismus
		\(M \to \hat M_{\ideal a}^A\) von \(A\)-Moduln definiert damit einen
		kanonischen \(\hat A_{\ideal a}\)-Modulhomomorphismus
		\[
			M_{\hat A_{\ideal a}}
			= M \otimes_A \hat A_{\ideal a} \to \hat M_{\ideal a} \otimes_A
			\hat A_{\ideal a} \to \hat M_{\ideal a} \otimes_{\hat A_{\ideal a}}
			\hat A_{\ideal a} \to \hat M.
		\]
		Für allgemeines \(A\) und \(M\) ist dieser kanonische Homomorphismus
		im allgemeinen weder injektiv noch surjektiv.
	\end{remark}
\end{frame}

\begin{frame}{Vervollständigung von Moduln als Skalarerweiterung}
	\begin{proposition}<+->
		\label{prop:compl_as_scalar_ext}
		Sei \(\ideal a\) ein Ideal in einem kommutativen Ring \(A\). Ist
		\(M\) ein endlich erzeugter \(A\)-Modul, so ist die kanonische
		Abbildung \(M_{\hat A_{\ideal a}} \to \hat M_{\ideal a}\) surjektiv.
		\\
		Ist \(A\) außerdem noethersch, so ist \(M_{\hat A_{\ideal a}} \to
		\hat M_{\ideal a}\) ein Isomorphismus.
	\end{proposition}
	\begin{proof}  <+->
		\begin{enumerate}[<+->]
		\item<.->
			Es ist leicht zu sehen, daß die \(\ideal a\)-adische
			Vervollständigung mit direkten Summen kommutiert. Für einen \(A\)-Modul
			der Form \(F \cong A^n\) gilt daher \(F_{\hat A} = F \otimes_A \hat A \cong \hat F\).
		\item
			Da \(M\) endlich erzeugt ist, existiert eine exakte Sequenz der
			Form \(0 \to N \to F \to M \to 0\) mit \(F \cong A^n\).
			\renewcommand{\qedsymbol}{}
			\qedhere
		\end{enumerate}
	\end{proof}
\end{frame}

\begin{frame}{Fortsetzung des Beweises zur Vollständigkeit von Moduln}
	\begin{proof}[Fortsetzung des Beweises]<+->
		\begin{enumerate}[<+->]
		\item<.->
			Die erste Zeile des kommutativen Diagrams
			\[
				\begin{CD}
					& & N_{\hat A} @>>> F_{\hat A} @>>> M_{\hat A} @>>> 0 \\
					& & @V{\gamma}VV @V{\beta}VV @V{\alpha}VV \\
					0 @>>> \hat N @>>> \hat F @>{\psi}>> \hat M @>>> 0
				\end{CD}
			\]
			ist exakt aufgrund der Rechtsexaktheit des Tensorproduktes.
		\item
			Es ist \(\psi\) surjektiv, da \(M\) die von \(F\) induzierte
			Topologie trägt. Da \(\beta\) surjektiv ist, folgt daraus die
			Surjektivität von \(\alpha\).
		\item
			Ist zusätzlich \(A\) noethersch, so ist \(N\) endlich erzeugt.
			Wir haben schon gezeigt, daß \(\gamma\) damit surjektiv ist.
			Außerdem ist dann die untere Zeile nach der letzten Proposition exakt.
		\item
			Eine Diagrammjagd zeigt, daß \(\alpha\) dann auch injektiv sein muß.
			\qedhere
		\end{enumerate}
	\end{proof}
\end{frame}

\begin{frame}{Flachheit der Vervollständigung}
	Sei \(A\) ein noetherscher kommutativer Ring. 
	\begin{proposition}<+->
		Für jedes Ideal
		\(\ideal a\) von \(A\) ist \(\hat A_{\ideal a}\) eine
		flache \(A\)-Algebra.
	\end{proposition}
	\begin{proof}<+->
		\begin{enumerate}[<+->]
		\item<.->
			Für endlich erzeugte \(A\)-Moduln \(M\) ist der Funktor
			\(M \mapsto \hat A \otimes_A M \cong \hat M\) exakt, da die
			Vervollständigung exakt ist.
		\item
			Wir haben schon allgemein gezeigt, daß dies Flachheit, also
			die Exaktheit für auch nicht endlich erzeugte Moduln, impliziert.
			\qedhere
		\end{enumerate}
	\end{proof}
	\begin{remark}<+->
		Für nicht endlich erzeugte \(A\)-Moduln ist der Funktor \(M \mapsto
		\hat M\) nicht exakt. Der gute Funktor ist daher \(M \mapsto M_{\hat A}\).
	\end{remark}
\end{frame}

\subsection{Vervollständigungen von Ringen}

\begin{frame}{Die Vervollständigung eines Ideals}
	\begin{proposition}<+->
		Sei \(\ideal a\) ein Ideal in einem noetherschen kommutativen Ring \(A\).
		Dann gilt für alle \(n \in \set N_0\):
		\begin{enumerate}[<+->]
		\item<.->
			\(\hat{\ideal a}_{\ideal a} = \hat A_{\ideal a} \ideal a
			\cong {\ideal a}_{\hat A_{\ideal a}}\).
		\item
			\(\widehat{\ideal a^n}_{\ideal a} = {\hat{\ideal a}_{\ideal a}}^n\).
		\item
			\(\ideal a^n/\ideal a^{n + 1}
			\cong \hat{\ideal a}_{\ideal a}^n/\hat{\ideal a}_{\ideal a}^{n + 1}\).
		\end{enumerate}
	\end{proposition}
	\begin{proof}<+->
		\begin{enumerate}[<+->]
		\item<.->
			Da \(A\) noethersch ist, ist \(\ideal a\) endlich erzeugt, womit
			die Abbildung \(\ideal a_{\hat A} = \ideal a \otimes_A \hat A \to
			\hat {\ideal a}\) ein Isomorphismus (mit Bild \(\hat A \ideal a\)) ist.
		\item
			\(\widehat{\ideal a^n} = \hat A \ideal a^n = (\hat A \ideal a)^n
			= \hat{\ideal a}^n\).
		\item
			\(\ideal a^n/\ideal a^{n + 1} = \ker(A/\ideal a^{n + 1} \to
			A/\ideal a^n) \cong \ker (\hat A/\hat{\ideal a}^{n + 1} \to
			\hat A/\hat{\ideal a}^n) = \hat{\ideal a}^n/\hat{\ideal a}^{n + 1}\).
			\qedhere
		\end{enumerate}
	\end{proof}
\end{frame}

\begin{frame}{Die Vervollständigung eines Ideals liegt im Jacobsonschen Ideal}
	\begin{proposition}<+->
		Sei \(\ideal a\) ein Ideal in einem noetherschen kommutativen Ring
		\(A\). Dann liegt \(\hat{\ideal a}_{\ideal a}\) im Jacobsonschen
		Radikal von \(\hat A_{\ideal a}\).
	\end{proposition}
	\begin{proof}<+->
		\begin{enumerate}[<+->]
		\item<.->
			Da \(\widehat{\ideal a^n} = \hat{\ideal a}^n\), ist
			\(\hat A\) vollständig bezüglich der \(\hat{\ideal a}\)-adischen
			Topologie.
		\item
			Für jedes \(x \in \hat{\ideal a}\) konvergiert
			\((1 - x)^{-1} = 1 + x + x^2 + \dotsb\) damit in \(\hat A\), so
			daß \(1 - x\) eine Einheit in \(\hat A\) ist.
		\item
			Damit liegt \(\hat{\ideal a}\) im Jacobsonschen Radikal von \(\hat A\).
			\qedhere
		\end{enumerate}
	\end{proof}
\end{frame}

\begin{frame}{Die Vervollständigung eines lokalen Ringes}
	\begin{proposition}<+->
		Sei \((A, \ideal m, F)\) ein noetherscher lokaler Ring. Dann ist
		\((\hat A_{\ideal m}, \hat{\ideal m}_{\ideal m}, F)\) ein lokaler Ring.
	\end{proposition}
	\begin{proof}<+->
		\begin{enumerate}[<+->]
		\item<.->
			Es ist \(\hat A/\hat{\ideal m} = A/\ideal m = F\), also ein Körper.  Damit ist
			\(\hat {\ideal m}\) ein maximales Ideal.
		\item
			Da \(\hat {\ideal m}\) im Jacobsonschen Ideal liegt und maximal ist, ist es das einzige
			maximale Ideal. Damit ist \(\hat A\) ein lokaler Ring.
			\qedhere
		\end{enumerate}
	\end{proof}
\end{frame}

\subsection{Der Krullsche Satz}

\begin{frame}{Krullscher Satz}
	\begin{theorem}[Krullscher Satz]<+->
		\label{thm:krull}
		Sei \(\ideal a\) ein Ideal in einem noetherschen kommutativen Ring \(A\). Sei
		\(M\) ein endlich erzeugter \(A\)-Modul. Der Kern \(K \coloneqq \bigcap\limits_{n = 1}^\infty
		\ideal a^n M\) des kanonischen Homomorphismus \(M \to \hat M_{\ideal a}\) besteht genau
		aus den \(x \in M\) mit \((1 + \ideal a) \cap \ann(x) \neq \emptyset\).
	\end{theorem}
	\begin{proof}<+->
		\begin{enumerate}[<+->]
		\item<.->
			Da \(K\) der Schnitt aller Umgebungen von \(0\) ist, ist die induzierte Topologie auf \(K\)
			die Klumpentopologie. Da die induzierte Topologie auch die \(\ideal a\)-adische ist,
			ist damit auch die \(\ideal a\)-adische Topologie auf \(K\)
			die Klumpentopologie, es ist also \(\ideal a K = K\).
		\item
			Damit existiert ein \(x \in \ideal a\) mit \((1 - x) K = 0\). 
		\item
			Ist umgekehrt \((1 - x) y = 0\) für ein \(x \in \ideal a\) und ein \(y \in M\), 
			so folgt \(y = x y = x^2 y = \dotsb \in K\).
			\qedhere
		\end{enumerate}
	\end{proof}
\end{frame}

\begin{frame}{Lokalisierung und Vervollständigung}
	\begin{remark}<+->
		Sei \(\ideal a\) ein Ideal in einem noetherschen kommutativen Ring \(A\). Sei \(S \coloneqq 1 + \ideal a\).
		Dann ist \(S\) multiplikativ abgeschlossen.
		\\
		Nach dem Krullschen Satz stimmen die Kerne der beiden kanonischen
		Abbildungen \(A \to \hat A_{\ideal a}\) und \(A \to S^{-1} A\) überein.
		\\
		Für jedes \(x \in \hat {\ideal a}_{\ideal a}\) gilt weiter \((1 - x)^{-1} = 1 + x + x^2 + \dotsb\), so daß
		jedes Element in \(S\) unter \(A \to \hat A_{\ideal a}\) zu einer Einheit wird.
		\\
		Die universelle Eigenschaft von \(S^{-1} A\) impliziert damit, daß \(S^{-1} A \to \hat A_{\ideal a},
		\frac a s \mapsto s^{-1} a\) zu einem wohldefinierten (injektiven) Homomorphismus von Ringen wird.
		\\
		Wir können (und werden) also \(S^{-1} A\) mit einem Unterring von \(\hat A_{\ideal a}\) identifizieren.
	\end{remark}
\end{frame}

\begin{frame}{Gegenbeispiel zum Krullschen Satz}
	\begin{example}<+->
		Ohne die Voraussetzung, daß der Ring noethersch ist, ist der Krullsche Satz im allgemeinen falsch.
		Sei etwa \(A\) der Ring aller \(\Cont^\infty\)-Funktionen auf \(\set R\) und \(\ideal a\) das Ideal der
		an \(0\) verschwindenden Funktionen (wegen \(A/\ideal a \cong \set R\) ist \(\ideal a\) maximal).
		\\
		Aus der Existenz der Taylorreihenentwicklung folgt, daß \(\ideal a\) von der identischen Funktion \(x\) erzeugt wird
		und daß \(\bigcap\limits_{n = 0}^\infty \ideal a^n\) die Menge aller \(f \in A\) ist, deren Taylorreihe in \(0\) trivial
		ist.
		\\
		Auf der anderen Seite ist \((1 + g) f = 0\) für ein \(g \in \ideal a\) genau dann Null, wenn \(f \in A\) einer ganzen
		Umgebung um \(0\) verschwindet.
		\\
		Damit liegt die Funktion \(\exp(-x^{-2})\) im Kern von \(A \to \hat A_{\ideal a}\) aber nicht im Kern von
		\(A \to (1 + \ideal a)^{-1} A\).
		\\
		Folglich ist \(A\) nicht noethersch.
	\end{example}
\end{frame}

\begin{frame}{Krullscher Schnittsatz}
	\begin{corollary}<+->
		\label{cor:krull}
		Sei \(A\) ein noetherscher Integritätsbereich. Für jedes echte Ideal \(\ideal a \neq (1)\) von \(A\) gilt
		dann \(\bigcap\limits_{n = 0}^\infty \ideal a^n = (0)\).
	\end{corollary}
	\begin{proof}<+->
		Nach dem Krullschen Satz ist \(\bigcap\limits_{n = 0}^\infty \ideal a^n\) der Kern von \(A \to (1 + \ideal a)^{-1} A\).
		Dieser ist trivial, da \(A\) ein Integritätsbereich ist.
	\end{proof}
\end{frame}

\begin{frame}{Hausdorffeigenschaft der \(\ideal a\)-adischen Topologie}
	\begin{corollary}<+->
		Sei \(A\) ein noetherscher kommutativer Ring. Sei \(\ideal a\) ein Ideal von \(A\), welches im Jacobsonschen
		Radikal von \(A\) enthalten ist. Für jeden endlich erzeugten \(A\)-Modul ist die \(\ideal a\)-adische Topologie auf
		\(M\) dann hausdorffsch.
	\end{corollary}
	\begin{visibleenv}<.->
		Es ist also \(\bigcap\limits_{n = 0}^\infty \ideal a^n M = (0)\).
	\end{visibleenv}
	\begin{proof}<+->
		Jedes Element der Form \(1 + x\) ist eine Einheit, wenn \(x\) in Jacobsonschen Radikal von \(A\) liegt.
	\end{proof}
	\begin{corollary}<+->
		Sei \((A, \ideal m)\) ein noetherscher lokaler Ring. Für jeden endlich erzeugten \(A\)-Modul ist die
		\(\ideal m\)-adische Topologie hausdorffsch. Insbesondere ist die \(\ideal m\)-adische Topologie auf \(A\)
		hausdorffsch.
		\qed
	\end{corollary}
\end{frame}

\begin{frame}{Der Schnitt aller primären Ideale zu einem Primideal}
	\begin{corollary}<+->
		Sei \(\ideal p\) ein Primideal in einem noetherschen kommutativen Ring \(A\). Dann ist der Kern des
		kanonischen Homomorphismus \(A \to A_{\ideal p}\) durch den Schnitt aller \(\ideal p\)-primären Ideale von \(A\)
		gegeben.
	\end{corollary}
	\begin{proof}<+->
		\begin{enumerate}[<+->]
		\item<.->
			Ist \(A\) lokal mit maximalem Ideal \(\ideal m\), so sind alle \(\ideal m\)-primären Ideale alle
			zwischen \(\ideal m\)
			und Idealen der Form \(\ideal m^n\) liegende Ideale. Damit zeigt die letzte Folgerung, daß der Schnitt
			aller \(\ideal m\)-primären Ideale in einem lokalen Ring das Nullideal ist.
		\item
			Ist \(A\) ein beliebiger kommutativer Ring erhalten wir, daß der Schnitt aller
			\(\ideal m \coloneqq A_{\ideal p} \ideal p\)-primären Ideale
			des lokalen Ringes \(A_{\ideal p}\) das Nullideal ist.
		\item
			Aus der Tatsache, daß die \(\ideal p\)-primären Ideale von \(A\) gerade die Kontraktionen der
			\(\ideal m\)-primären Ideale von \(A_{\ideal p}\) sind, erhalten wir die Behauptung.
			\qedhere
		\end{enumerate}
	\end{proof}
\end{frame}



\lecture{Der assoziierte gewichtete Ring}{Der assoziierte gewichtete Ring}
\mode<all>\setcounter{section}{42}
\mode<all>\section{Der assoziierte gewichtete Ring}

\subsection{Definition und grundlegende Eigenschaften des assoziierten gewichteten Ringes}

\begin{frame}{Definition des assoziierten gewichteten Ringes}
	\begin{definition}<+->
		Sei \(\ideal a\) ein Ideal in einem kommutativen Ring \(A\). Dann heißt der
		gewichtete kommutative Ring
		\[
			\Graded_{\ideal a}(t) = \Graded_{\ideal a}(A, t)
			\coloneqq \Rees_{\ideal a}(A, t)/t^{-1} \Rees_{\ideal a}(A, t)_+ 
			\cong \bigoplus\limits_{n = 0}^\infty \ideal a^n/\ideal a^{n + 1} t^n
		\]
		der \emph{assoziierte gewichtete Ring zur \(\ideal a\)-adischen Filtrierung von \(A\)}.
	\end{definition}
	\begin{visibleenv}<+->
		Ist \(x \in \ideal a^n\), so schreiben wir \(\bar x\) für die Restklasse modulo \(\ideal a^{n + 1}\). Damit können
		wir die Multiplikation auf \(\Graded_{\ideal a}(t)\) folgendermaßen beschreiben:
		\\
		Es ist \((\bar x t^m) \cdot (\bar y t^n) = \overline{xy} t^{m + n}\) für \(x \in \ideal a^m, y \in \ideal a^n\).
	\end{visibleenv}
\end{frame}

\begin{frame}{Assozierte gewichtete Moduln}
	Sei \(\ideal a\) ein Ideal in einem kommutativen Ring. 
	Sei \(M\) ein \(A\)-Modul zusammen mit einer \(\ideal a\)-Filtrierung \(M_\bullet\). 
	\begin{definition}<+->
		Der gewichtete \(A\)-Modul
		\[
			\Graded(M_\bullet, t) = \Rees(M_\bullet, t)/t^{-1} \Rees(M_\bullet, t)_+
			\cong \bigoplus\limits_{n = 0}^\infty M_n/M_{n + 1} t^n
		\]
		heißt der \emph{assoziierte gewichtete Modul zur Filtrierung \(M_\bullet\)}.
	\end{definition}
	\begin{visibleenv}<+->
		In kanonischer Weise ist \(\Graded(M_\bullet, t)\) sogar ein \(\Graded_{\ideal a}(t)\)-Modul. Schreiben wir
		wieder \(\bar\cdot\) für Restklasen, so ist die Multiplikation auf \(\Graded(M_\bullet, t)\) durch
		\((\bar a t^m) \cdot (\bar x t^n) = \overline{ax} t^{m + n}\) für \(a \in \ideal a^m, x \in M_n\) gegeben.
	\end{visibleenv}
	\begin{notation}<+->
		Ist \(M_\bullet\) die \(\ideal a\)-adische Filtrierung auf \(M\), so schreiben wir
		\(\Graded_{\ideal a}(M, t) = \Graded(M_\bullet, t)\).
	\end{notation}
\end{frame}

\begin{frame}{Morphismen zwischen assoziierten gewichteten Moduln}
	Sei \(\ideal a\) ein Ideal in einem kommutativen Ring \(A\).
	Seien \(M, N\) zwei \(A\)-Moduln jeweils zusammen mit einer \(\ideal a\)-Filtrierung \(M_\bullet\) bzw.\
	\(N_\bullet\).
	\\
	Sei \(\phi\colon M \to N\) ein filtrierter Homomorphismus von \(A\)-Moduln, das heißt \(\phi(M_n) \subset N_n\)
	für alle \(n\). Dann induziert \(\phi\) einen Homomorphismus
	\[
		\Graded(\phi)\colon \Graded(M_\bullet, t) \to \Graded(N_\bullet, t), \bar x t^n \mapsto \overline{\phi(x)} t^n,
	\]
	wobei wieder \(\bar x \in M_n/M_{n + 1}\) das Bild eines \(x \in M_n\) modulo \(M_{n + 1}\) ist.
\end{frame}

\begin{frame}{Endlichkeitseigenschaften des assoziierten gewichteten Ringes}
	\begin{proposition}<+->
		Sei \(A\) ein noetherscher kommutativer Ring. Für ein Ideal \(\ideal a\) in \(A\) gilt:
		\begin{enumerate}[<+->]
		\item<.->
			Der Ring \(\Graded_{\ideal a}(A, t)\) ist noethersch.
		\item
			Es sind \(\Graded_{\ideal a}(A, t)\) und \(\Graded_{\hat{\ideal a}_{\ideal a}}(\hat A_{\ideal a}, t)\)
			als gewichtete Ringe isomorph.
		\item
			Für jeden endlich erzeugten \(A\)-Modul \(M\) zusammen mit einer stabilen \(\ideal a\)-Filtrierung
			\(M_\bullet\) ist \(\Graded(M_\bullet, t)\) ein endlich erzeugter \(\Graded_{\ideal a}(A, t)\)-Modul.
		\end{enumerate}
	\end{proposition}
	\begin{proof}<+->
		\begin{enumerate}[<+->]
		\item<.->
			Da \(A\) noethersch ist, ist \(\ideal a\) durch endlich viele \(x_1, \dotsc, x_n \in A\) erzeugt.
			Ist \(\bar x_i\) das Bild von \(x_i\) in \(\ideal a/\ideal a^2\), so ist \(\Graded_{\ideal a}(t)\) als
			\(A\)-Algebra von \(\bar x_1 t, \dotsc, \bar x_n t\) erzeugt. Damit ist
			\(\Graded_{\ideal a}(t)\) noethersch.
		\item
			\(\Graded_{\ideal a}(t) \cong \bigoplus\limits_n \ideal a^n/\ideal a^{n + 1} t^n \cong \bigoplus\limits_n
			\hat{\ideal a}^n/\hat{\ideal a}^{n + 1} t^n \cong \Graded_{\hat{\ideal a}}(t)\).
			\renewcommand{\qedsymbol}{}
			\qedhere
		\end{enumerate}
	\end{proof}
\end{frame}

\begin{frame}{Fortsetzung des Beweises zu Endlichkeitseigenschaften}
	\begin{proof}[Beweis, daß \(\Graded(M_\bullet, t)\) endlich erzeugt ist]<+->
		\begin{enumerate}[<+->]
		\item<.->
			Sei jetzt \(M\) ein endlich erzeugter \(A\)-Modul und \(M_\bullet\) eine stabile \(\ideal a\)-Filtrierung.
			Dann existiert ein \(n_0 \in \set N_0\) mit \(M_{n_0 + r} = \ideal a^r M_{n_0}\) für alle \(r \ge 0\).
		\item
			Folglich ist \(\Graded(M_\bullet, t)\) als \(\Graded_{\ideal a}(t)\)-Modul durch
			\(\bigoplus\limits_{n = 0}^{n_0} M_n/M_{n + 1} t^n\) erzeugt. Die \(M_n/M_{n + 1}\) sind endlich erzeugte
			\(A\)-Moduln, da \(A\) noethersch ist und \(M_n\) Untermodul des endlich erzeugten Moduls \(M\) ist.
		\item
			Da \(\ideal a \subset \ann M_n/M_{n + 1}\), ist \(M_n/M_{n + 1}\) auch ein endlich erzeugter
			\(A/\ideal a\)-Modul.
			Damit ist \(\bigoplus_{n = 0}^{n_0} M_n/M_{n + 1} t^n\) ein endlich erzeugter \(A/\ideal a\)-Modul.
		\item
			Es folgt, daß \(\Graded(M_\bullet, t)\) ein endlich erzeugter \(\Graded_{\ideal a}(t)\)-Modul ist.
			\qedhere
		\end{enumerate}
	\end{proof}
\end{frame}

\subsection{Endlichkeitseigenschaften der Vervollständigung}

\begin{frame}{Injektivität und Surjektivität von vervollständigten Homomorphismen}
	\begin{lemma}<+->
		Seien \(A\) und \(B\) zwei abelsche Gruppen (d.h.\ \(\set Z\)-Moduln) jeweils zusammen mit einer
		Filtrierung \(A_\bullet\) bzw.\ \(B_\bullet\), welche jeweils eine Umgebungsbasis um \(0\) einer
		Topologie auf \(A\) bzw.\ \(B\) bilden.
		\\
		Ist dann \(\phi\colon A \to B\) ein filtrierter Homomorphismus,
		so gilt:
		\begin{enumerate}[<+->]
		\item<.->
			Ist \(\Graded(\phi)\colon \Graded(A_\bullet, t) \to \Graded(B_\bullet, t)\) injektiv,
			so ist auch \(\hat\phi\colon \hat A \to \hat B\) injektiv.
		\item
			Ist \(\Graded(\phi)\colon \Graded(A_\bullet, t) \to \Graded(B_\bullet, t)\) surjektiv,
			so ist auch \(\hat\phi\colon \hat A \to \hat B\) surjektiv.
		\end{enumerate}
	\end{lemma}
\end{frame}

\begin{frame}[Beweis zur Injektivität und Surjektivität von vervollständigten Homomorphismen]
	\begin{proof}<+->
		\begin{enumerate}[<+->]
		\item<.->
			Die Reihen im folgenden Diagramm sind exakt:
			\[
				\begin{CD}
					0 @>>> A_n/A_{n + 1} @>>> A/A_{n + 1} @>>> A/A_n @>>> 0 \\
					& & @V{\psi_n}VV @V{\phi_{n + 1}}VV @V{\phi_n}VV \\
					0 @>>> B_n/B_{m + 1} @>>> B/B_{n + 1} @>>> B/B_n @>>> 0.
				\end{CD}
			\]
			Nach dem Schlangenlemma existiert damit eine exakte Sequenz
			\(0 \to \ker \psi_n \to \ker \phi_{n + 1} \to \ker \phi_n \to \coker \psi_n \to \coker \phi_{n + 1} \to \coker \phi_n \to 0	
			\).
		\item
			Sind die \(\psi_n\) injektiv bzw.\ surjektiv, folgt durch Induktion nach \(n\), daß die \(\phi_n\) injektiv bzw.\ surjektiv
			sind. Im letzteren Fall ist außerdem \((\ker \phi_n)_n\) ein surjektives System.
		\item
			Im ersten Fall ist \(\hat \phi\) injektiv, da der inverse Limes linksexakt ist.
		\item
			Im zweiten Falle ist \(\hat \phi\) surjektiv, da in diesem Falle \({\varprojlim\limits_n}^1 (\ker \phi_n) = 0\).
			\qedhere
		\end{enumerate}
	\end{proof}
\end{frame}

\begin{frame}{Moduln mit endlich erzeugtem assoziierten gewichteten Modul}
	\begin{proposition}<+->
		\label{prop:weighted_mod_is_ft}
		Sei \(\ideal a\) ein Ideal in einem kommutativen Ring \(A\). Sei \(M\) ein \(A\)-Modul zusammen mit
		einer \(\ideal a\)-Filtrierung \(M_\bullet\). Sei \(A\) vollständig bezüglich der \(\ideal a\)-adischen Topologie,
		und sei \(M\) in seiner Filtrationstopologie hausdorffsch, also \(\bigcap\limits_n M_n = 0\). Sei schließlich
		\(\Graded(M_\bullet, t)\) als \(\Graded_{\ideal a}(A, t)\)-Modul endlich erzeugt. Dann ist \(M\) ein endlich
		erzeugter \(A\)-Modul.
	\end{proposition}
\end{frame}

\begin{frame}{Beweis der Proposition}
	\begin{proof}<+->
		\begin{enumerate}[<+->]
		\item<.->
			Seien \(x_1, \dotsc, x_r\) mit \(x_i \in M_{n(i)}\) für \(n(i) \in \set N_0\), so daß die Bilder
			\(\bar x_i t^{n(i)} \in M_{n(i)}/M_{n(i) + 1}\) den \(\Graded_{\ideal a}(t)\)-Modul \(\Graded(M_\bullet)\) erzeugen.
		\item
			Für jedes \(i\) sei \(F^i\) der \(A\)-Modul \(A\) zusammen mit der stabilen \(\ideal a\)-Filtrierung \(F^i_\bullet\)
			mit	\(F^i_k = \ideal a^{k + n(i)}\). Seien \(F \coloneqq\bigoplus\limits_{i = 1}^r F^i, F_\bullet
			\coloneqq \bigoplus\limits_{i = 1}^r F^i_\bullet\). Dann ist
			\(\phi\colon F \to M, (a_1, \dotsc, a_r) \mapsto a_1 x_1 + \dotsb a_r x_r\) ein Homomorphismus filtrierter Gruppen.
		\item
			Der induzierte Homomorphismus \(\Graded(\phi)\colon \Graded(F_\bullet) \to \Graded(M_\bullet)\) ist nach
			Konstruktion surjektiv. Damit ist der Homomorphismus \(\hat\phi\colon \hat F \to \hat M\) zwischen den
			Vervollständigungen surjektiv.
		\item
			Da \(F \cong A^r\) als topologische \(A\)-Moduln und \(A\) vollständig ist, ist auch \(F \cong \hat F\).
			Damit ist \(F \to \hat F \to \hat M\) surjektiv. Da \(M \to \hat M\) aufgrund der Hausdorffeigenschaft von
			\(M\) injektiv ist, muß damit \(\phi\colon F \to M\) surjektiv sein. Folglich ist \(M\)
			als \(A\)-Modul von den \(x_i\) erzeugt.
			\qedhere
		\end{enumerate}
	\end{proof}
\end{frame}

\begin{frame}{Moduln mit noetherschen assoziierten gewichteten Moduln}
	\begin{corollary}<+->
		Sei \(\ideal a\) ein Ideal in einem kommutativen Ring \(A\). Sei \(M\) ein \(A\)-Modul zusammen mit
		einer \(\ideal a\)-Filtrierung \(M_\bullet\). Sei \(A\) vollständig bezüglich der \(\ideal a\)-adischen Topologie,
		und sei \(M\) in seiner Filtrationstopologie hausdorffsch, also \(\bigcap\limits_n M_n = 0\). Sei schließlich
		\(\Graded(M_\bullet, t)\) ein noetherscher \(\Graded_{A, \ideal a}(t)\)-Modul. Dann ist \(M\) ein noetherscher
		\(A\)-Modul.
	\end{corollary}
	\begin{proof}<+->
		\begin{enumerate}[<+->]
		\item<.->
			Es ist zu zeigen, daß jeder Untermodul \(M'\) von \(M\) endlich erzeugt ist. Es \(M'_\bullet \coloneqq
			M' \cap M_\bullet\) eine \(\ideal a\)-Filtration auf \(M'\), und die Einbettung \(M' \to M\) induziert
			eine Einbettung \(\Graded(M'_\bullet, t) \to \Graded(M_\bullet, t)\).
		\item
			Da \(\Graded(M_\bullet, t)\) noethersch ist, ist \(\Graded(M'_\bullet, t)\) endlich erzeugt.
		\item
			Wegen \(\bigcap\limits_n M'_n \subset \bigcap\limits_n M_n = (0)\) ist \(M'\) schließlich hausdorffsch.
			Nach der Proposition ist \(M'\) damit endlich erzeugt.
			\qedhere
		\end{enumerate}
	\end{proof}
\end{frame}

\begin{frame}{Endlichkeitseigenschaft der Vervollständigung}
	\begin{theorem}<+->
		\label{thm:compl_is_noeth}
		Sei \(A\) ein noetherscher kommutativer Ring. Für jedes Ideal \(\ideal a\) von \(A\) ist
		die Vervollständigung \(\hat A_{\ideal a}\) noethersch.
	\end{theorem}
	\begin{proof}<+->
		Es ist \(\Graded_{\ideal a}(t) \cong \Graded_{\hat {\ideal a}}(t)\), und diese Ringe sind noethersch.
		Damit können wir die letzte Folgerung auf den vollständigen Ring \(\hat A\) und den \(\hat A\)-Modul
		\(\hat A\) mit der \(\hat{\ideal a}\)-adischen Filtrierung anwenden (welche hausdorffsch ist) und erhalten,
		daß \(\hat A\) ein noetherscher \(\hat A\)-Modul ist, also ein noetherscher Ring.
	\end{proof}
\end{frame}

\begin{frame}{Beispiel des Potenzreihenrings}
	\begin{corollary}<+->
		Für jeden noetherschen kommutativen Ring \(A\) ist der Potenzreihenring \(\ps A{X_1, \dotsc, X_n}\) in \(n\)
		Variablen noethersch.
	\end{corollary}
	\begin{proof}<+->
		Nach dem Hilbertschen Basissatz ist \(A[X_1, \dotsc, X_n]\) noethersch. Es ist \(\ps A{X_1, \dotsc, X_n}\) die
		\((X_1, \dotsc, X_n)\)-adische Vervollständigung von \(A[X_1, \dotsc, X_n]\).
	\end{proof}
	\begin{example}<+->
		Für jeden Körper \(K\) ist \(\ps K{X_1, \dotsc, X_n}\) noethersch.
	\end{example}
\end{frame}



\lecture{Hilbertfunktionen}{Hilbertfunktionen}
\part<article>{Dimensionstheorie}
\mode<all>\setcounter{section}{43}
\mode<all>\section{Hilbertfunktionen}

\subsection{Poincarésche Reihe}

\begin{frame}{Endlich erzeugte gewichte Moduln}
	\begin{visibleenv}<+->
		Sei \(A = \bigoplus_{n \in \set N_0} A_n\) ein noetherscher gewichteter kommutativer Ring. Wir haben
		gesehen, daß \(A_0\) dann ein noetherscher Ring ist und daß \(A\) als
		\(A_0\)-Algebra endlich erzeugt ist.
	\end{visibleenv}
	\begin{proposition}
		Sei
		\(M = \bigoplus_{n \in \set N_0} M_n\) ein endlich erzeugter gewichteter \(A\)-Modul. Dann ist
		\(M_n\) für alle \(n \in \set N_0\) ein endlich erzeugter \(A_0\)-Modul.
	\end{proposition}
	\begin{proof}<+->
		\begin{enumerate}[<+->]
		\item<.->
			Es existieren endlich
			viele \(x_1, \dotsc, x_s \in A\), welche \(A\) als
			\(A_0\)-Algebra erzeugen. Wir können annehmen, daß die \(x_i\)
			homogen von Gewichten \(k_i\) sind.
		\item
			Es existieren homogene Erzeuger \(m_1, \dotsc, m_t \in M\) von \(M\) als
			\(A\)-Modul. Seien die Gewichte der \(m_j\) durch \(r_j\) gegeben.
		\item
			Damit wird  \(M_n\) als \(A_0\)-Modul durch alle Terme der Form
			\(g_j(x) m_j\) erzeugt, wobei \(g_j(x)\) ein Monom in den \(x_i\) vom
			Totalgrad \(n - r_j\) ist.
			\qedhere
		\end{enumerate}
	\end{proof}
\end{frame}

\begin{frame}{Definition der Poincaréschen Reihe}
	\begin{visibleenv}<+->
		Sei \(A\) ein noetherscher gewichteter kommutativer Ring. Sei
		\(\lambda\) eine (\(\set Z\)-wertige) additive Funktion auf der Klasse
		aller endlich erzeugten \(A_0\)-Moduln.
		\\
		Mit den letzten Ergebnissen ist die Reihe
		\[
			\lambda(M, t) \coloneqq \sum\limits_{n = 0}^\infty \lambda(M_n) t^n
			\in \ps{\set Z} t
		\]
		für jeden endlich erzeugten gewichteten \(A\)-Modul \(M\)
		wohldefiniert.
	\end{visibleenv}
	\begin{definition}<.->
		Die Reihe \(\lambda(M, t)\) heißt die \emph{Poincarésche Reihe
		von \(M\) (zu \(\lambda\))}.
	\end{definition}
\end{frame}

\subsection{Der Hilbert--Serresche Satz}

\begin{frame}{Der Hilbert--Serresche Satz}
	\begin{theorem}[Hilbert--Serrescher Satz]<+->
		\label{thm:hilbert_serre}
		Sei \(A\) ein noetherscher gewichteter kommutativer Ring. Sei \(M\)
		ein endlich erzeugter gewichteter \(A\)-Modul. Für jede additive Funktion
		\(\lambda\) auf der Klasse der endlich erzeugten \(A_0\)-Moduln ist dann
		\(\lambda(M, t)\) eine rationale Funktion der Form
		\(f/\prod\limits_{i = 1}^s (1 - t^{k_i})\) mit \(f \in \set Z[t]\) und
		\(k_i \in \set N\).
	\end{theorem}
	\begin{remark}<+->
		Wir werden im Beweis sehen, daß wir alle \(k_i = 1\) wählen können,
		wenn \(A\) als \(A_0\)-Algebra von \(A_1\) erzeugt wird.
	\end{remark}
\end{frame}

\begin{frame}{Beweis des Hilbert--Serreschen Satzes}
	\begin{proof}<+->
		\begin{enumerate}[<+->]
		\item<.->
			Wir führen Induktion über die Anzahl \(s\) der Erzeuger von \(A\)
			als \(A_0\)-Algebra. Im Falle von \(s = 0\) ist \(A = A_0\) und damit
			ist \(M\) ein endlich erzeugter \(A_0\)-Modul. Folglich ist
			\(M_n = 0\) für \(n \gg 0\), also ist \(\lambda(M, t)\) in diesem
			Falle ein Polynom.
		\item
			Sei also \(s > 0\). Seien \(x_1, \dotsc, x_s\) homogene
			Erzeuger von \(A\) als \(A_0\)-Modul mit Gewichten \(k_i\). Sei
			\(\xi_s\colon M_n \to M_{n + k_s}, m \mapsto x_s m\).
			Sei \(0 \to K_n \to M_n \xrightarrow{\xi_s} M_{n + k_s} \to L_{n + k_s}
			\to 0\) eine exakte Sequenz von \(A_0\)-Moduln. Es sind \(K = \bigoplus\limits_n K_n\)
			und \(L = \bigoplus\limits_n L_n\) endlich erzeugte \(A\)-Moduln.
		\item
			Aus der Additivität von \(\lambda\) folgt
			\(\lambda(K_n) - \lambda(M_n) + \lambda(M_{n + k_s}) -
			\lambda(L_{n + k_s}) = 0\), also
			\((1 - t^{k_s}) \lambda(M, t) = \lambda(L, t) -
			t^{k_s} \lambda(K, t) + g\) für ein \(g \in \set Z[t]\).
		\item
			Da \(x_s\) auf \(K\) und \(L\) trivial wirkt, können wir die
			Induktionsvoraussetzung auf \(A_0[x_1, \dotsc, x_{s - 1}]\) anwenden.
			\qedhere
		\end{enumerate}
	\end{proof}
\end{frame}

\begin{frame}{Die Größe eines gewichteten Moduls}
	Sei \(A\) ein noetherscher gewichteter kommutativer Ring.
	Sei \(M\) ein endlich erzeugter gewichteter \(A\)-Modul. Sei
	\(\lambda\) eine additive Funktion auf der Klasse der endlich erzeugten
	\(A_0\)-Moduln.
	\begin{definition}<+->
		Die Polordnung von \(\lambda(M, t)\) an \(t = 1\) heißt die \emph{Größe \(\size_\lambda(M)\)
		von \(M\) (zu \(\lambda\))}.
	\end{definition}
	\begin{example}<+->
		Da \(A\) ein endlich erzeugter gewichteter Modul über sich selbst ist, ist inbesondere
		die Größe \(\size_\lambda(A)\) von \(A\) definiert.
	\end{example}
\end{frame}

\begin{frame}{Polynomielles Verhalten}
	\begin{proposition}<+->
		Sei \(A\) ein noetherscher gewichteter kommutativer Ring, der als \(A_0\)-Algebra von \(A_1\) erzeugt wird.
		Sei \(M\) ein endlich erzeugter gewichteter \(A\)-Modul.
		Sei \(\lambda\) eine additive Funktion auf der Klasse der endlich erzeugten \(A\)-Moduln.
		Dann existiert ein Polynom \(p \in \set Q[n]\) vom Grad \(\size_\lambda(M, t) - 1\) mit \(\lambda(M_n) = p(n)\)
		für \(n \gg 0\).
	\end{proposition}
	\begin{visibleenv}<.->
		Dem Nullpolynom sei hier der Grad \(-1\) zugeordnet.
	\end{visibleenv}
	\begin{visibleenv}<+->
		Das Polynom \(p\) aus der Proposition heißt die
		\emph{Hilbertfunktion von \(M\) (zu \(\lambda\))}. Diese ist durch
		ihre Eigenschaften eindeutig bestimmt.
	\end{visibleenv}
\end{frame}

\begin{frame}{Beweis der Proposition über das polynomielle Verhalten}
	\begin{proof}<+->
		\begin{enumerate}[<+->]
		\item<.->
			Nach dem Hilbert--Serrschen Satz existiert ein
			\(f = \sum\limits_{k = 0}^N a_k t^k \in \set Z[t]\), so daß \(\lambda(M_n)\)
			der Koeffizient von \(t^n\) in \(f(t) \cdot (1 - t)^{-s}\) ist. Durch
			Kürzen können wir erreichen, daß
			\(s = \size_\lambda(M)\) und \(f(1) \neq 0\).
		\item
			Aus \((1 - t)^{-s} = \sum\limits_{k = 0}^\infty \binom{s + k - 1}{s - 1} t^k\) folgt
			\(\lambda(M_n) = \sum\limits_{k = 0}^N a_k \binom{s + n - k - 1}{s - 1}\) für \(n \ge N\).
		\item
			Die rechte Seite ist ein Polynom in \(n\) mit führendem Term
			\(f(1) n^{s - 1}/(s - 1)! \neq 0\).
			\qedhere
		\end{enumerate}
	\end{proof}
\end{frame}

\begin{frame}{Numerische Polynome}
	\begin{remark}<+->
		Ein Polynom \(p \in \set Q[n]\), welches ganzzahlige Werte \(p(n)\) für
		\(n \gg 0\) annimmt, heißt auch \emph{numerisches Polynom}.
	\end{remark}
	\begin{example}<+->
		Es gibt Polynome \(p \in \set Q[n]\) mit \(p(n) \in \set Z\) für alle
		\(n \in \set Z\), welche aber nicht in \(\set Z[n]\) liegen, etwa
		\(p(n) = \frac 1 2 x(x + 1)\).
	\end{example}
\end{frame}

\begin{frame}{Die Größe regulärer Quotienten}
	\begin{proposition}<+->
		Sei \(A\) ein noetherscher gewichteter kommutativer Ring. Sei
		\(M\) ein endlich erzeugter gewichteter \(A\)-Modul. Sei \(\lambda\) eine additive
		Funktion auf der Klasse der endlich erzeugten \(A_0\)-Moduln. Ist dann
		\(x \in A_k\), \(k \in \set N_0\), regulär in \(M\), das heißt
		\(x m = 0 \implies m = 0\) für alle \(m \in M\), so gilt
		\(\size_\lambda(M/x M) = \size_\lambda(M) - 1\).
	\end{proposition}
	\begin{proof}<+->
		\begin{enumerate}[<+->]
		\item<.->
			Es existieren exakte Sequenzen \(0 \to M_n \xrightarrow{\xi} 
			M_{n + k} \to (M/x M)_{n + k} \to 0\) mit \(\xi\colon M_n \to M_{n + k},
			m \mapsto x m\).
		\item
			Es folgt \(- \lambda(M_n) + \lambda(M_{m + k}) - \lambda((M/xM)_{n + k}) = 0\),
			also \((1 - t^k) \lambda(M, t) = \lambda(M/xM, t) + g\) für ein \(g \in \set Z[t]\).
		\item
			Damit ist \(\size_\lambda(M/x M) = \size_\lambda(M) - 1\).
			\qedhere
		\end{enumerate}
	\end{proof}
\end{frame}

\begin{frame}{Beispiel zu einer Poincaréschen Reihen}
	\begin{example}<+->
		Sei \(A_0\) ein artinscher Ring, z.B.\ ein Körper. Dann ist insbesondere die Länge \(\ell\) von
		\(A_0\)-Moduln eine additive Funktion auf den endlich erzeugten \(A_0\)-Moduln.
		\\
		Sei \(A = A[X_1, \dotsc, X_s]\) der Polynomring mit der kanonischen Gewichtung.
		Dann ist \(A_n\) ein freier \(A_0\)-Modul mit einer Basis bestehend
		aus allen Monomen \(X_1^{m_1} \dotsm X_s^{m_s}\) vom Totalgrad \(n\).
		\\
		Von diesen gibt es genau \(\binom{s + n - 1}{s - 1}\), daher ist
		\(\ell(A, t) = \frac{1}{(1 - t)^s}\).
	\end{example}
\end{frame}

\subsection{Das charakterische Polynom primärer Ideale}

\begin{frame}{Ein Hilfssatz über endliche Länge für Quotienten in einer stabilen Filtration}
	\begin{lemma}<+->
		Sei \((A, \ideal m)\) ein noetherscher lokaler Ring. Sei \(\ideal q\) ein \(\ideal m\)-primäres Ideal.
		Sei \(M\) ein endlich erzeugter \(A\)-Modul und \(M_\bullet\) eine stabile \(\ideal q\)-Filtration.
		Dann ist \(M/M_n\) für alle \(n \in \set N_0\) von endlicher Länge mit
		\(\ell(M/M_n) = \sum\limits_{r = 0}^{n - 1} \ell(M_r/M_{r + 1})\).
	\end{lemma}
\end{frame}

\begin{frame}{Beweis des Hilfssatzes über die endliche Länge}
	\begin{proof}<+->
		\begin{enumerate}[<+->]
		\item<.->
			Da \(A\) noethersch ist und \(M_\bullet\) eine stabile Filtration eines endlich erzeugten Moduls über \(A\),
			ist \(\Graded_{\ideal q}(t) \cong \bigoplus\limits_n \ideal q^n/\ideal q^{n + 1} t^n\) noethersch und
			\(\Graded(M_\bullet, t) \cong \bigoplus\limits_n M_n/M_{n + 1} t^n\) ist ein endlich erzeugter Modul über
			\(\Graded_{\ideal q}(t)\).
		\item
			Es ist \(\Graded_{\ideal q}(t)_0 \cong A/\ideal q\) noethersch der Dimension \(0\) und damit artinsch.
		\item
			Die \(\Graded(M_\bullet, t)_n \cong M_n/M_{n + 1}\) sind noethersche \(A\)-Moduln, deren Annihilator \(\ideal q\)
			umfaßt, also sogar noethersche \(A/\ideal q\)-Moduln und damit von endlicher Länge.
		\item
			Aus \(\ell(M/M_{r + 1}) - \ell(M/M_r) = \ell(M_r/M_{r + 1})\) folgt die Endlichkeit der Länge von
			\(\ell(M/M_n)\) und die angegebene Formel in Termen von \(\ell(M_r/M_{r + 1})\).
			\qedhere
		\end{enumerate}
	\end{proof}
\end{frame}

\begin{frame}{Die Länge von Quotienten in einer stabilen Filtration}
	\begin{proposition}<+->
		Sei \((A, \ideal m)\) ein noetherscher lokaler Ring. Sei \(\ideal q\) ein \(\ideal m\)-primäres Ideal, welches von
		minimal \(s\) Elementen erzeugt wird.
		Sei \(M\) ein endlich erzeugter \(A\)-Modul zusammen mit einer stabilen \(\ideal q\)-Filtration.
		Dann existiert genau ein Polynom \(g \in \set Q[n]\) vom Grad höchstens \(s\)
		mit \(\ell(M/M_n) = g(n)\) für \(n \gg 0\). Grad und Leitkoeffizient von \(g\) hängen nur von \(M\) und \(\ideal q\), aber
		nicht von der gewählten Filtrierung ab.
	\end{proposition}
	\begin{proof}[Beweis der Existenz]<+->
		\begin{enumerate}[<+->]
		\item<.->
			Seien \(x_1, \dotsc, x_s\) Erzeuger von \(\ideal q\), deren Bilder in \(\ideal q/\ideal q^2\) mit \(\bar x_i\) bezeichnet
			seien. Dann wird \(\Graded_{\ideal q}(t)\) von den \(s\) Elementen \(\bar x_i t\) im Gewicht \(1\)
			als \(\Graded_{\ideal q}(t)_0\)-Algebra erzeugt, woraus \(\ell(M_n/M_{n + 1}) = f(n)\) für \(n \gg 0\) für ein
			Polynom \(f(n) \in \set Q[n]\) vom Grad höchstens \(s - 1\) folgt.
		\item
			Wegen \(\ell(M/M_{n + 1}) - \ell(M/M_{n}) = f(n)\) für \(n \gg 0\) ist damit \(\ell(M/M_{n})\) für \(n \gg 0\)
			ein Polynom in \(n\) vom Grad höchstens \(s\).
			\renewcommand{\qedsymbol}{}
			\qedhere	
		\end{enumerate}
	\end{proof}
\end{frame}

\begin{frame}{Beweis der Eindeutigkeit von Grad und Leitkoeffizient}
	\begin{proof}[Beweis der Eindeutigkeit von Grad und Leitkoeffizient]<+->
		\begin{enumerate}[<+->]
		\item<.->
			Ist \(\tilde M_\bullet\) eine weitere stabile \(\ideal q\)-Filtration von \(M\), so sei \(\tilde g \in \set Q[n]\) mit
			\(\tilde g(n) = \ell(M/\tilde M_n)\) für \(n \gg 0\). Da die Filtrationen beschränkte Differenz haben,
			existiert ein \(n_0 \in \set N_0\) mit \(M_{n + n_0} \subset \tilde M_n\) und \(\tilde M_{n + n_0} \subset M_n\)
			für alle \(n \ge 0\).
		\item
			Folglich ist \(g(n + n_0) \ge \tilde g(n)\) und \(\tilde g(n + n_0) \ge g(n)\) für \(n \gg 0\). Da \(g, \tilde g\) Polynome
			sind, folgt \(\lim\limits_{n \to \infty} g(n)/\tilde g(n) = 1\), womit
			\(g, \tilde g\) denselben Grad und Leitkoeffizient haben.
			\qedhere
		\end{enumerate}
	\end{proof}
\end{frame}

\begin{frame}{Das charakteristische Polynom}
	Sei \((A, \ideal m)\) ein noetherscher lokaler Ring. Sei \(\ideal q\) ein \(\ideal m\)-primäres Ideal, welches von
	minimal \(s\) Elementen erzeugt wird.
	Sei \(M\) ein endlich erzeugter \(A\)-Modul zusammen mit einer stabilen \(\ideal q\)-Filtration.
	Mit \(\charpoly_{\ideal q}^{M_\bullet} \in \set Q[n]\) bezeichnen wir dasjenige Polynom
	mit \(\charpoly_{\ideal q}^M(n) = \ell(M/M_n)\) für
	\(n \gg 0\).
	\begin{definition}<+->
		Das Polynom \(\chi_{\ideal q}^{M_\bullet}\) heißt das \emph{charakteristische Polynom von \(\ideal q\) über \(M_\bullet\)}.
	\end{definition}
	\begin{example}<+->
		Im Falle von \(M = A\) zusammen mit der \(\ideal q\)-adischen Filtrierung schreiben
		wir \(\charpoly_{\ideal q} = \charpoly_{\ideal q}^{M_\bullet}\) und nennen \(\charpoly_{\ideal q}\) das
		\emph{charakteristische Polynom von \(\ideal q\)}. Nach der letzten Proposition ist \(\charpoly_{\ideal q}\) ein Polynom, dessen
		Grad höchstens \(s\) ist, wobei \(s\) die minimale Anzahl von Erzeugern von \(\ideal q\) ist.
	\end{example}
\end{frame}

\begin{frame}{Grad des charakteristischen Polynome zu primären Idealen}
	\begin{proposition}<+->
		Sei \((A, \ideal m)\) ein noetherscher lokaler Ring. Sei \(\ideal q\) ein \(\ideal m\)-primäres Ideal. Dann ist
		\(\deg \charpoly_{\ideal q} = \deg \charpoly_{\ideal m}\).
	\end{proposition}
	\begin{visibleenv}<+->
		Der Grad \(\size(A)\) des charakteristischen Polynoms hängt also nicht vom gewählten \(\ideal m\)-primären Ideal ab. 
	\end{visibleenv}
	\begin{proof}<+->
		Es ist \(\ideal m \supset \ideal q \supset \ideal m^r\) für ein \(r\), also \(\ideal m^n \supset \ideal q^n
		\supset \ideal m^{rn}\)
		für \(n \ge 0\), also \(\charpoly_{\ideal m}(n) \le \charpoly_{\ideal q}(n) \le \charpoly_{\ideal m}(rn)\) für alle \(n \gg 0\).
		Es folgt \(\deg \charpoly_{\ideal m} \le \deg \charpoly_{\ideal q} \le \deg\charpoly_{\ideal m}\).
	\end{proof}
	\begin{remark}<+->
		Es ist insbesondere \(\size(A) = \size_\ell(\Graded_{\ideal m}(A, t))\), die vorher definierte Größe eines noetherschen
		gewichteten homogenen Ringes (zur additiven Funktion der Länge).
	\end{remark}
\end{frame}



\lecture{Dimensionstheorie noetherscher lokaler Ringe}{Dimensionstheorie noetherscher lokaler Ringe}
\mode<all>\setcounter{section}{44}
\mode<all>\section{Dimensionstheorie noetherscher lokaler Ringe}

\subsection{Die Größe regulärer Quotienten}

\begin{frame}{Größen eines noetherschen lokalen Ringes}
	\begin{visibleenv}<+->
		Einem noetherschen lokalen Ring \((A, \ideal m)\) können wir folgende Größen zuordnen:
		\begin{enumerate}[<+->]
		\item<.->
			Die minimale Anzahl \(\updelta(A)\) von Elementen \(x_1, \dotsc, x_{\updelta(A)} \in A\), so daß
			\((x_1, \dotsc, x_{\updelta(A)})\) ein \(\ideal m\)-primäres Ideal ist.
		\item
			Der Grad \(\size(A)\) des charakteristischen Polynoms \(\chi_{\ideal m}\) von \(A\), also
			die Ordnung plus \(1\), mit der \(\ell(A/\ideal m^n)\) für
			\(n \gg 0\) wächst.
		\item
			Die Dimension \(\dim A\), also das Supremum der Längen aller Primidealketten in \(A\).
		\end{enumerate}
	\end{visibleenv}
	\begin{visibleenv}<+->
		Im folgenden zeigen wir, daß \(\updelta(A) \ge \size(A) \ge \dim A \ge \updelta(A)\), daß also alle drei Größen
		übereinstimmen.
	\end{visibleenv}
	\begin{proposition}<+->
		Es ist \(\delta(A) \ge \size(A)\).
		\qed
	\end{proposition}
\end{frame}

\begin{frame}{Die Größe eines regulären Quotienten eines Moduls}
	\begin{proposition}<+->
		Sei \((A, \ideal m)\) ein noetherscher lokaler Ring. Sei \(\ideal q\) ein \(\ideal m\)-primäres
		Ideal. Sei \(M\) ein endlich erzeugter \(A\)-Modul. Seien \(x \in A\) regulär in \(M\) und
		\(M'' \coloneqq M/xM\). Dann gilt: \(\deg \chi_{\ideal q}^{M''} \leq  \deg \chi_{\ideal q}^M - 1\).
	\end{proposition}
	\begin{proof}<+->
		\begin{enumerate}[<+->]
		\item<.->
			Da \(x\) regulär in \(M\) ist, ist \(M \to N \coloneqq x M, m \mapsto x m\) ein Isomorphismus von
			\(A\)-Moduln. Sei \(N_n \coloneqq N \cap \ideal q^n M\) für \(n \in \set N_0\). Es existieren exakte Sequenzen
			\(0 \to N/N_n \to M/\ideal q^n M \to M''/\ideal q^n M'' \to 0\).
		\item
			Sei \(g(n) = \ell(N/N_n)\) für \(n \gg 0\), so \(g(n) - \chi_{\ideal q}^M(n) + \chi_{\ideal q}^{M''}(n) = 0\).
			Nach Artin--Rees ist \((N_n)\) eine stabile \(\ideal q\)-Filtration von \(N\), wir können daher
			\(g \in \set Q[n]\) annehmen.
			Da \(N \cong M\) müssen Grad und Leitkoeffizient von \(g\) und \(\chi_{\ideal q}^M\) übereinstimmen.
			\qedhere
		\end{enumerate}
	\end{proof}
\end{frame}

\begin{frame}{Die Größe eines regulären Quotienten}
	\begin{corollary}<+->
		Sei \(A\) ein noetherscher lokaler Ring. Sei \(x \in A\) ein reguläres Element. Dann ist
		\(\size(A/(x)) \le \size(A) - 1\).
	\end{corollary}
	\begin{proof}<+->
		Wir wenden die Proposition auf den \(A\)-Modul \(M = A\) an.
	\end{proof}
\end{frame}

\subsection{Die Dimension noetherscher lokaler Ringe}

\begin{frame}{Größe und Dimension}
	\begin{proposition}<+->
		Sei \((A, \ideal m)\) ein noetherscher lokaler Ring. Dann ist \(\size(A) \ge \dim A\).
	\end{proposition}
	\begin{proof}[Induktionsanfang]<+->
		Der Beweis erfolge mit Induktion über \(d \coloneqq \size(A)\). Ist \(d = 0\), so ist
		\(\ell(A/\ideal m^n)\) für \(n \gg 0\) konstant, also ist \(\ideal m^n = \ideal m^{n + 1}\) für
		\(n \gg 0\), also ist \(A\) artinsch, also \(\dim A = 0\).
		\renewcommand{\qedsymbol}{}
	\end{proof}
\end{frame}

\begin{frame}{Fortsetzung des Beweises}
	\begin{proof}[Induktionsschritt]<+->
		\begin{enumerate}[<+->]
		\item<.->
			Sei \(d > 0\). Sei \(\ideal p_0 \subsetneq \ideal p_1 \subsetneq \dotsb \subsetneq \ideal p_r\) eine
			Primidealkette in \(A\). Sei \(x \in \ideal p_1 \setminus \ideal p_0\). Sei \(A' \coloneqq A/\ideal p_0\),
			und sei \(x'\) das Bild von \(x \in A'\). Dann ist \(x'\) regulär,
			also \(\size(A'/(x')) \leq \size(A') - 1\).
		\item
			Sei \(\ideal m'\) das maximale Ideal in \(A'\). Dann ist \(A'/(\ideal m')^n\) homomorphes Bild von
			\(A/\ideal m^n\), also ist \(\ell(A/\ideal m^n) \ge \ell(A'/(\ideal m')^n)\), also \(\size(A) \ge \size(A')\),
			also \(\size(A'/(x')) \leq d - 1\).
		\item
			Nach Induktionsvoraussetzung ist die Länge einer Primidealkette in \(A'/(x')\) damit höchstens \(d - 1\). Das
			Bild der Kette \(\ideal p_1 \subsetneq \dotsb \subsetneq \ideal p_r\) in \(A'/(x')\) ist eine Kette der Länge
			\(r - 1\), also \(r - 1 \le d - 1\), also \(r \le d\), also \(\dim A \le d\).
		\qedhere
		\end{enumerate}
	\end{proof}
\end{frame}

\begin{frame}{Endlichkeit der Dimension}
	\begin{corollary}<+->
		Sei \(A\) ein noetherscher lokaler Ring. Dann ist \(\dim A < \infty\).
		\qed
	\end{corollary}
	\begin{visibleenv}<+->
		Die Längen von Primidealketten in \(A\) sind also beschränkt.
	\end{visibleenv}
\end{frame}

\begin{frame}{Die Höhe von Primidealen}
	\begin{definition}<+->
		Sei \(A\) ein kommutativer Ring. Die \emph{Höhe \(\height \ideal p\) eines Primideals \(\ideal p\)} von
		\(A\) ist das Supremum der Längen von Primidealketten der Form \(\ideal p_0 \subsetneq \ideal p_1 \subsetneq
		\dotsb \subsetneq \ideal p_r = \ideal p\).
	\end{definition}
	\begin{visibleenv}<+->
		Es ist also \(\height \ideal p = \dim A_{\ideal p}\).
	\end{visibleenv}
	\begin{corollary}<+->
		Ist \(A\) ein noetherscher kommutativer Ring, so hat jedes Primideal \(\ideal p\) von \(A\) endliche Höhe.
		\qed
	\end{corollary}
	\begin{visibleenv}<+->
		Die Menge der Primideale in einem noetherschen kommutativen Ring erfüllt damit die absteigende Kettenbedingung.
	\end{visibleenv}
\end{frame}

\begin{frame}{Die Tiefe von Primidealen}
	\begin{definition}<+->
		Sei \(A\) ein kommutativer Ring. Die \emph{Tiefe \(\depth \ideal p\) eine Primideals \(\ideal p\)} von
		\(A\) ist das Supremum der Längen von Primidealketten der Form \(\ideal p = \ideal p_0 \subsetneq \ideal p_1
		\subsetneq \dotsb \subsetneq \ideal p_r\).
	\end{definition}
	\begin{visibleenv}<+->
		Es ist also \(\depth \ideal p = \dim A/\ideal p\).
	\end{visibleenv}
	\begin{remark}<+->
		Selbst wenn \(A\) noethersch ist, kann die Tiefe eines  Primideals unendlich sein --- außer \(A\) ist zudem lokal.
	\end{remark}
\end{frame}

\begin{frame}{Dimension und Erzeuger primärer Ideale}
	\begin{proposition}<+->
		Sei \((A, \ideal m)\) ein noetherscher lokaler Ring der Dimension \(d\). Dann existiert ein von
		\(d\) Elementen \(x_1, \dotsc, x_d \in A\) erzeugtes \(\ideal m\)-primäres Ideal, also \(\dim A \ge \updelta(A)\).
	\end{proposition}
\end{frame}

\begin{frame}{Beweis der Proposition}
	\begin{proof}<+->
		\begin{enumerate}[<+->]
		\item<.->
			Wir konstruieren induktiv \(x_1, \dotsc, x_d \in A\), so daß jedes Primideal, welches
			\(\ideal a_i \coloneqq (x_1, \dotsc, x_i)\) enthält,
			mindestens Höhe \(i\) hat. Sei \(i > 0\) und seien \(x_1, \dotsc, x_{i - 1}\) schon konstruiert.
		\item
			Seien \(\ideal p_1, \dotsc, \ideal p_s\) die minimalen Primideale mit \(\ideal p_j \supset \ideal a_{i - 1}\) und
			\(\height \ideal p_j = i - 1\).
			Dann ist \(\ideal m \neq \ideal p_j\) für alle \(j\), da \(\height \ideal m = d > i - 1 = \height \ideal p_j\).
			Es folgt \(\ideal m \neq \bigcup\limits_{j = 1}^s \ideal p_j\), also können wir ein \(x_i \in \ideal m\) mit
			\(x_i \notin \ideal p_j\) wählen.
		\item
			Sei \(\ideal q\) ein Primideal, welches \(\ideal a_i\) enthält. Dann enthält \(\ideal q\) ein minimales Primideal
			\(\ideal p\) mit \(\ideal p \supset \ideal a_{i - 1}\). Ist \(\ideal p = \ideal p_j\) für ein \(j\),
			so folgt \(\ideal q \supsetneq \ideal p\), also \(\height \ideal q \ge i\).
		\item
			Ist dagegen \(\ideal p \neq \ideal p_j\) für alle \(j\), so ist \(\height \ideal p \ge i\),
			also \(\height \ideal q \ge i\).
		\item
			Ist schließlich \(\ideal p\) ein minimales Primideal mit \(\ideal p \supset \ideal a_d\), so hat \(\ideal p\) damit
			Höhe \(d\), also \(\ideal p = \ideal m\). Also ist \(\ideal a_d\) ein \(\ideal m\)-primäres Ideal.
			\qedhere
		\end{enumerate}
	\end{proof}
\end{frame}

\subsection{Der Dimensionssatz}

\begin{frame}{Der Dimensionssatz}
	\begin{theorem}[Dimensionssatz]<+->
		Sei \((A, \ideal m)\) ein noetherscher lokaler Ring. Dann sind folgende Größen gleich:
		\begin{enumerate}[<+->]
		\item<.->
			Das maximale Länge \(\dim A\) von Primidealketten in \(A\).
		\item
			Der Grad \(\size(A)\) des charakteristischen Polynoms \(\chi_{\ideal m}\) von \(A\).
		\item
			Die minimale Anzahl von Erzeugern \(\ideal m\)-primärer Ideale von \(A\).
			\qed
		\end{enumerate}
	\end{theorem}
\end{frame}

\begin{frame}{Beispiel zur Dimension}
	\begin{example}<+->
		Sei \(K\) ein Körper. Sei \(A \coloneqq K[X_1, \dotsc, X_n]_{\ideal m}\) der Polynomring in \(n\) Variablen über \(K\)
		lokalisiert am maximalen Ideal \(\ideal m = (X_1, \dotsc, X_n)\). Dann ist \(\Graded_{\ideal m}(A, t) \cong
		K[\bar X_1 t, \dotsc, \bar X_n t]\), wobei die \(\bar X_i\) die Bilder der \(X_i\) in \(\Graded_{\ideal m}(A, t)\) sind.
		\\
		Damit ist die Poincarésche Reihe von \(\Graded_{\ideal m}(A, t)\) durch \((1 - t)^{- n}\) gegeben, also
		\(\dim A = \size(A) = \size(\Graded_{\ideal m}(A, t)) = n\).
	\end{example}
\end{frame}

\begin{frame}{Vergleich mit der Dimension des Kotangentialraums}
	\begin{corollary}<+->
		Sei \((A, \ideal m, F)\) ein noetherscher lokaler Ring. Dann ist \(\dim A \leq \dim_F \ideal m/\ideal m^2\).
	\end{corollary}
	\begin{proof}<+->
		Seien \(x_1, \dotsc, x_s \in \ideal m\), so daß ihre Bilder in \(\ideal m/\ideal m^2\) eine Basis über \(F\) bilden.
		Nach dem Nakayamaschen Lemma erzeugen die \(x_i\) damit das Ideal \(\ideal m\). Also ist \(\dim A = \updelta(A) \le s
		= \dim_F \ideal m/\ideal m^2\).
	\end{proof}
\end{frame}

\begin{frame}{Höhe minimaler Primideale}
	\begin{corollary}<+->
		Sei \(A\) ein kommutativer noetherscher Ring. Seien \(x_1, \dotsc, x_r \in A\). Ist dann \(\ideal p\) ein
		minimales Primideal mit \(\ideal p \supset (x_1, \dotsc, x_r)\), so gilt \(\height \ideal p \le r\).
	\end{corollary}
	\begin{proof}<+->
		Indem wir von \(A\) auf \(A_{\ideal p}\) übergehen, 
		können wir davon ausgehen, daß \(A\) ein lokaler Ring mit maximalem Ideal \(\ideal p\) ist, in dem \((x_1, \dotsc,
		x_r)\) ein \(\ideal p\)-primäres Ideal ist. Damit ist \(r \ge \updelta(A) = \dim A = \height \ideal p\).
	\end{proof}
\end{frame}

\begin{frame}{Der Krullsche Hauptidealsatz}
	\begin{corollary}<+->
		Sei \(A\) ein kommutativer noetherscher Ring. Sei weiter \(x \in A\) ein reguläres Element.
		Dann gilt für jedes minimale Primideal \(\ideal p\) mit \(\ideal p \supset (x)\), daß \(\height \ideal p = 1\).
	\end{corollary}
	\begin{proof}<+->
		Nach der letzten Folgerung ist \(\height \ideal p \le 1\). Wäre \(\height \ideal p = 0\),
		so wäre \(\ideal p\) assoziiert zu \((0)\).
		Damit besteht \(\ideal p\) nur aus Nullteilern. Widerspruch.
	\end{proof}
\end{frame}

\begin{frame}{Dimension regulärer Quotienten}
	\begin{corollary}<+->
		\label{cor:dim_of_reg_quot}
		Sei \((A, \ideal m)\) ein noetherscher lokaler Ring. Sei \(x \in \ideal m\) regulär. Dann ist
		\(\dim A/(x) = \dim A - 1\).
	\end{corollary}
	\begin{proof}<+->
		\begin{enumerate}[<+->]
		\item<.->
			Sei \(d \coloneqq \dim A/(x)\). Dann ist \(d = \size(A/(x)) \le \size(A) - 1 \le \dim A - 1\). 
		\item
			Seien auf der anderen Seite \(x_1, \dotsc, x_d \in \ideal m\), deren Bilder in \(A/(x)\) ein
			\(\ideal m/(x)\)-primäres Ideal erzeugen. Dann ist \((x, x_1, \dotsc, x_d)\) ein \(\ideal m\)-primäres Ideal
			in \(A\), also \(d + 1 \ge \updelta(A) = \dim A\).
		\qedhere
		\end{enumerate}
	\end{proof}
\end{frame}

\begin{frame}{Die Dimension der Vervollständigung eines lokalen Ringes}
	\begin{corollary}<+->
		Sei \((A, \ideal m)\) ein lokaler noetherscher Ring. Sei \(\hat A\) seine \(\ideal m\)-adische Vervollständigung.
		Dann ist \(\dim A = \dim \hat A\).
	\end{corollary}
	\begin{proof}<+->
		Sei \(\hat{\ideal m}\) das maximale Ideal von \(\hat A\).
		Es ist \(A/\ideal m^n \cong \hat A/\hat{\ideal m}^n\), also \(\chi_{\ideal m} = \chi_{\hat{\ideal m}}\). Damit
		ist \(\dim A = \size(A) = \size(\hat A) = \dim \hat A\).
	\end{proof}
\end{frame}

\subsection{Parametersysteme}

\begin{frame}{Parametersysteme}
	\begin{definition}<+->
		Sei \((A, \ideal m)\) ein noetherscher lokaler Ring der Dimension \(d\). Sind dann \(x_1, \dotsc, x_d\) Erzeuger eines
		\(\ideal m\)-primären Ideals von \(A\), so heißt \((x_1, \dotsc, x_d)\) ein Parametersystem von \(A\).
	\end{definition}
\end{frame}

\begin{frame}{Unabhängigkeit der Parameter}
	\begin{proposition}<+->
		Sei \((A, \ideal m)\) ein noetherscher lokaler Ring. Sei \((x_1, \dotsc, x_d)\) ein Parametersystem für \(A\).
		Sei \(\ideal q \coloneqq (x_1, \dotsc, x_d)\) das erzeugte \(\ideal m\)-primäre Ideal.
		Ist \(f \in A[X_1, \dotsc, X_d]\) homogen vom Grad \(s\) mit \(f(x_1, \dotsc, x_d) \in \ideal q^{s + 1}\), so folgt
		\(f \in \ideal m[X_1, \dotsc, X_d]\).
	\end{proposition}
	\begin{proof}<+->
		\begin{enumerate}[<+->]
		\item<.->
			Es ist \(\alpha\colon A/\ideal q[X_1, \dotsc, X_d] \to \Graded_{\ideal q}(t), X_i \mapsto \bar x_i t\),
			wobei \(\bar x_i\) das Bild von \(x_i\) in \(\ideal q/\ideal q^2\) ist, ein surjektiver 
			Homomorphismus gewichteter Ringe. 
		\item
			Nach Voraussetzung an \(f\) ist das Bild \(\bar f\) von \(f\) im Kern von \(\alpha\). Angenommen, ein 
			Koeffizient von
			\(f\) ist eine Einheit. Dann ist \(\bar f\) regulär. Dann gilt:
			\(\size(\Graded_{\ideal q}(t)) \leq \size(A/\ideal q[X_1, \dotsc, X_d]/(\bar f))
			= \size(A/\ideal q[X_1, \dotsc, X_d]) - 1 = d - 1\).
		 \item
		 	Aber es ist \(\size(\Graded_{\ideal q}(t)) = \size(A) = d\), ein Widerspruch.
			\qedhere
		\end{enumerate}
	\end{proof}
\end{frame}

\begin{frame}{Algebraische Unabhängigkeit von Parametern}
	\begin{corollary}<+->
		Sei \(K\) ein Körper. Sei \((A, \ideal m)\) eine lokale \(K\)-Algebra, so daß \(K\) isomorph auf \(A/\ideal m\)
		abgebildet wird. Ist dann \((x_1, \dotsc, x_d)\) ein Parametersystem für \(A\), so sind die \(x_i\) algebraisch
		unabhängig über \(K\).
	\end{corollary}
	\begin{proof}<+->
		\begin{enumerate}[<+->]
		\item<.->
			Sei \(f \in K[X_1, \dotsc, X_n]\) mit \(f(x_1, \dotsc, x_d) = 0\). Angenommen, \(f \neq 0\).
		\item
			Dann können wir \(f = g + h\) schreiben, wobei \(g \neq 0\) ein homogenes Polynom ist und \(h\) echt größeren
			Grad als \(g\) hat.
		\item
			Anwenden der Proposition liefert, daß \(g\) Koeffizienten in \(\ideal m\) hat.
			Da aber \(g\) ein Polynom über \(K\) ist, folgt \(g = 0\), ein Widerspruch.
			\qedhere
		\end{enumerate}
	\end{proof}
\end{frame}



\lecture{Reguläre lokale Ringe und transzendente Dimension}{Regul\"are lokale Ringe und transzendente Dimension}
\mode<all>\setcounter{section}{45}
\mode<all>\section{Reguläre lokale Ringe}

\subsection{Charakterisierung regulärer lokaler Ringe}

\begin{frame}{Reguläre lokale Ringe}
	\begin{theorem}<+->
		Sei \((A, \ideal m, F)\) ein noetherscher lokaler Ring der Dimension \(d\). Dann sind äquivalent:
		\begin{enumerate}[<+->]
		\item<.->
			\(\Graded_{\ideal m}(A, t)\) und \(F[X_1, \dotsc, X_d]\) sind als gewichtete \(F\)-Algebren isomorph.
		\item
			Für die Dimension des Zariskischen Kotangentialraumes von \(A\) gilt \(\dim_F \ideal m/\ideal m^2 = d\).
		\item
			Es existiert ein Parametersystem \((x_1, \dotsc, x_d)\) von \(A\) mit \(\ideal m = (x_1, \dotsc, x_d)\).
		\end{enumerate}
	\end{theorem}
	\begin{visibleenv}<+->
		Ein \emph{regulärer lokaler Ring \(A\)} ist ein noetherscher lokaler Ring, welcher die Bedingungen
		des Satzes erfüllt.
	\end{visibleenv}
\end{frame}

\begin{frame}{Beweis des Satzes über reguläre lokale Ringe}
	\begin{proof}<+->
		\begin{enumerate}[<+->]
		\item<.->
			Aus der ersten folgt sicherlich die zweite Aussage. Die dritte folgt aus der zweiten
			mit dem Nakayamaschen Lemma.
		\item
			Sei also \(\ideal m = (x_1, \dotsc, x_m)\). Dann ist \(\alpha\colon F[X_1, \dotsc, X_d] \to
			\Graded_{\ideal m}(t), X_i \mapsto \bar x_i t\) ein surjektiver Homomorphismus gewichteter
			Algebren.
		\item
			Der Homomorphismus ist aufgrund der Unabhängigkeitseigenschaft eines Parametersystems injektiv.
			\qedhere
		\end{enumerate}
	\end{proof}
\end{frame}

\begin{frame}{Reguläre lokale Ringe sind Integritätsbereiche}
	\begin{lemma}<+->
		Sei \(A\) ein kommutativer Ring. Sei \(\ideal a\) ein Ideal in \(A\) mit \(\bigcap\limits_n \ideal a^n = (0)\).
		Ist dann \(\Graded_{\ideal a}(A, t)\) ein Integritätsbereich, so auch \(A\).
	\end{lemma}
	\begin{proof}<+->
		Seien \(x, y \in A \setminus \{0\}\). Dann existieren \(r, s \in \set N_0\) mit \(x \in \ideal a^r \setminus \ideal a^{r + 1}\)
		und \(y \in \ideal a^s \setminus \ideal a^{s + 1}\). Dann sind \(\bar x t^r, \bar y t^s \neq 0 \in \Graded_{\ideal a}(t)\),
		also \(\bar x \cdot \bar y t^{r + s} \neq 0\), also \(x y \neq 0 \in A\).
	\end{proof}
	\begin{corollary}<+->
		Ein regulärer lokaler Ring ist ein Integritätsbereich.
	\end{corollary}
\end{frame}

\begin{frame}{Bemerkungen zu regulären lokalen Ringen}
	\begin{remark}<+->
		Ein regulärer lokaler Ring \((A, \ideal m, F)\) der Dimension \(1\) ist ein Integritätsbereich mit \(\dim_F \ideal m/\ideal m^2
		= 1\), also ein diskreter Bewertungsbereich.
		\\
		Umgekehrt ist ein diskreter Bewertungsbereich ein regulärer lokaler Ring der Dimension \(1\).
	\end{remark}
	\begin{remark}<+->
		Sei \((A, \ideal m)\) ein lokaler Ring. Ist dann \(\Graded_{\ideal m}(A, t)\) ein ganz abgeschlossener Integritätsbereich,
		so kann gezeigt werden, daß auch \(A\) ganz abgeschlossen ist. Jeder regulärer lokale Ring ist also ganz abgeschlossen.
		\\
		Es existieren aber ganz abgeschlossene lokale Integritätsbereiche mit Dimension größer als \(1\), welche nicht regulär sind.
	\end{remark}
\end{frame}

\subsection{Regularität als analytische Eigenschaft}

\begin{frame}{Regularität als analytische Eigenschaft}
	\begin{proposition}<+->
		Sei \((A, \ideal m)\) ein noetherscher lokaler Ring. Dann ist \(A\)
		genau dann regulär, wenn seine \(\ideal m\)-adische Vervollständigung
		\(\hat A\) ein regulärer noetherscher lokaler Ring ist.
	\end{proposition}
	\begin{proof}<+->
		\begin{enumerate}[<+->]
		\item<.->
			Wir haben schon gezeigt, daß \((\hat A, \hat{\ideal m})\) ein noetherscher 
			lokaler Ring ist, wenn \((A, \ideal m)\) ein noetherscher lokaler Ring
			ist.
		\item
			Da \(A\) noethersch ist, gilt außerdem
			\(\Graded_{\ideal m}(A, t) \cong \Graded_{\hat{\ideal m}}(\hat A, t)\). 
		\item
			Schließlich ist \(\dim A = \dim{\hat A}\).
			\qedhere
		\end{enumerate}
	\end{proof}
\end{frame}

\begin{frame}{Geometrische Interpretation}
	\begin{remark}<+->
		Sei \((A, \ideal m)\) ein lokaler Ring. Die \(\ideal m\)-adische
		Vervollständigung \((\hat A, \hat{\ideal m})\) heißt auch der
		\emph{analytische Halm von \(A\)}. 
		\\
		Wir haben damit gezeigt, daß der analytische Halm eines regulären
		lokalen Ringes \(A\) ein Integritätsbereich ist, was geometrisch so
		ausgedrückt wird, daß \(A\) nur einen analytischen Zweig habe. 
	\end{remark}
\end{frame}

\begin{frame}{Analytischer Halm regulärer lokaler Ringe im geometrischen Fall}
	\begin{example}<+->
		Sei \(K\) ein Körper und \((A, \ideal m, F)\) eine reguläre lokale
		\(K\)-Algebra, so daß \(K\) isomorph auf \(F\) abgebildet wird. 
		Sei \(d \coloneqq \dim A\).
		\\
		Dann ist \(\Graded_{\ideal m}(A, t) \cong K[X_1, \dotsc, X_d]\),
		woraus \(\hat A = \ps K{X_1, \dotsc, X_d}\) folgt.
		\\
		Damit hängt der analytische Halm in dieser Situation nur von der Dimension
		ab.
	\end{example}
\end{frame}

\begin{frame}{Regularität des Polynomrings}
	\begin{example}
		Sei \(K\) ein Körper. Sei \(\ideal m = (X_1 - x_1, \dotsc, X_n - x_n)\) ein
		maximales Ideal des Polynomringes \(A \coloneqq K[X_1, \dotsc, X_n]\).
		Dann ist \(A_{\ideal m}\) ein regulärer lokaler Ring der
		Dimension \(n\), denn
		\(\Graded_{\ideal m}(A, t)\) ein Polynomring in \(n\) Variablen.
	\end{example}
\end{frame}


\mode<all>\section{Transzendente Dimension}

\subsection{Transzendente Dimension}

\begin{frame}{Transzendente Dimension}
	Sei \(K\) ein Körper. Sei \(A\) ein endlich erzeugter Integritätsbereich über \(K\).
	\begin{definition}<+->
		Der Transzendenzgrad des Quotientenkörpers \(K(A)\) von \(A\) über \(K\) heißt die \emph{(transzendente)
		Dimension \(\trdim_K A\) von \(A\)}.
	\end{definition}
	\begin{remark}<+->
		Sind \(x_1, \dotsc, x_n\) Erzeuger von \(A\) über \(K\), so erzeugen die
		\(x_i\) auch den Quotientenkörper \(K(A)\) über \(A\). Damit muß
		\(\trdim_K A \leq n\) gelten. Die transzendente Dimension von \(A\) ist
		also endlich.
	\end{remark}
	\begin{visibleenv}<+->
		Im folgenden wollen wir zeigen, daß \(\trdim_K A = \dim A\).
	\end{visibleenv}
\end{frame}

\begin{frame}{Ein Lemma über Höhe und Tiefe in ganzen Erweiterungen}
	\begin{lemma}<+->
		Sei \(A \subset B\) eine ganze Erweiterung von
		Integritätsbereichen. Sei \(A\) weiter ganz abgeschlossen.
		Sei \(\ideal q\) ein Primideal in \(B\). Sei \(\ideal p \coloneqq A \cap \ideal q\).
		Dann gilt \(\height \ideal p = \height \ideal q\) und \(\depth \ideal p = \depth \ideal q\).
	\end{lemma}
	\begin{proof}<+->
		\begin{enumerate}[<+->]
		\item<.->
			Ist \(\ideal q' \subsetneq \ideal q''\) eine echte Inklusion von
			Primidealen in \(B\), so ist \(A \cap \ideal q' \subsetneq A
			\cap \ideal q''\) eine echte Inklusion von Primidealen in
			\(A\). Damit folgt
			\(\height \ideal p \ge \height \ideal q\), \(\depth \ideal p \ge \depth \ideal q\).
		\item
			Nach dem "`Going-Down"'-Satz kann jede absteigende Primidealkette in \(A\)
			zu einer Primidealkette in \(B\) hochgehoben werden. Damit folgt
			\(\height \ideal q \ge \height \ideal p\).
		\item
			Nach dem "`Going-Up"'-Satz kann jede aufsteigende Primidealkette in
			\(A\) zu einer Primidealkette in \(B\) hochgehoben werden. Damit
			folgt \(\depth \ideal q \ge \depth \ideal p\).
			\qedhere
		\end{enumerate}
	\end{proof}
\end{frame}

\begin{frame}{Dimension des Polynomrings}
	\begin{lemma}<+->
		Sei \(K\) ein Körper. Für jedes maximale Ideal \(\ideal m\) des
		Polynomringes \(A \coloneqq K[X_1, \dotsc, X_n]\) ist dann
		\(\dim A_{\ideal m} = n\).
	\end{lemma}
	\begin{proof}<+->
		\begin{enumerate}[<+->]
		\item<.->
			Sei \(L\) ein algebraischer Abschluß von \(K\).
			Dann ist \(B \coloneqq L[X_1, \dotsc, X_n]\) der ganze
			Abschluß von \(A\) in \(L\). Dann existiert ein maximales
			Ideal \(\ideal n\) von \(B\) mit \(\ideal m = A \cap \ideal n\).
		\item		
			Nach dem Hilbertschen Basissatz ist
			\(\ideal n = (X_1 - b_1, \dotsc, X_n - b_n)\) für gewisse
			\(b_i \in B\). Damit ist \(\dim B_{\ideal n} = n\) nach den
			schon angestellten Überlegungen.
		\item
			Daraus folgt nach dem letzten Hilfssatz, daß \(\dim A_{\ideal m}
			= \height \ideal m = \height \ideal n = \dim B_{\ideal n} = n\).
			\qedhere
		\end{enumerate}
	\end{proof}
\end{frame}

\begin{frame}{Transzendente Dimension und lokale Dimension}
	\begin{theorem}<+->
		Sei \(K\) ein Körper. Sei \(A\) ein endlich erzeugter
		Integritätsbereich über \(K\). Dann ist \(\dim A_{\ideal m} = \trdim_K A\) für
		alle maximalen Ideale \(\ideal m\) von \(A\).
	\end{theorem}
	\begin{proof}<+->
		\begin{enumerate}[<+->]
		\item<.->
			Nach der noetherschen Normalisierung existiert eine ganze
			Ringerweiterung der Form \(A' \coloneqq K[X_1, \dotsc, X_d] \subset A\).
			Damit ist \(\trdim_K A = \trdim_K A' = d\).
		\item
			Ist weiter \(\ideal m' \coloneqq A' \cap \ideal m\), so folgt
			\(\dim A_{\ideal m} = \dim A'_{\ideal m'} = d\).
			\qedhere
		\end{enumerate}
	\end{proof}
\end{frame}

\begin{frame}{Dimension und lokale Dimension}
	\begin{corollary}<+->
		Sei \(K\) ein Körper. Sei \(A\) ein endlich erzeugter Integritätsbereich
		über \(K\). Dann ist \(\dim A = \dim A_{\ideal m}\) für jedes maximale
		Ideal \(\ideal m\) von \(A\).
	\end{corollary}
	\begin{proof}<+->
		Es ist \(\dim A = \sup\limits_{\ideal m} \dim A_{\ideal m}\), wobei
		\(\ideal m\) alle maximalen Ideale von \(A\) durchläuft. Nach dem
		Satz haben aber alle \(A_{\ideal m}\) dieselbe Dimension, nämlich die
		transzendente.
	\end{proof}
\end{frame}

\begin{frame}{Dimension und Kodimension}
	\begin{theorem}<+->
		Sei \(K\) ein Körper. Sei \(A\) ein endlich erzeugter Integritätsbereich
		über \(K\). Dann gilt \(\height \ideal p + \dim A/\ideal p = \dim A\)
		für alle Primideale \(\ideal p\) von \(A\).
	\end{theorem}
	\begin{proof}<+->
		\begin{enumerate}[<+->]
		\item<.->
			Nach der noetherschen Normalisierung existiert ein endlicher,
			injektiver Homomorphismus \(A' \coloneqq K[X_1, \dotsc, X_n] \to A\).
			Sei \(\ideal m\) ein maximales Ideal, \(\ideal m' \coloneqq A' \cap \ideal m\). Dann ist \(\dim A = \dim A_{\ideal m} =
			\dim A'_{\ideal m'} = n\). Sei \(\ideal p' = A' \cap \ideal p\). Da \(\height \ideal p = \height \ideal p'\)
			und \(\depth \ideal p = \depth \ideal p'\), reicht es, den Satz für \(A'\) und \(\ideal p'\) zu beweisen.
		\item
			Wieder aufgrund der noetherschen Normalisierung können wir davon ausgehen, daß \(\ideal p' = (X_{r + 1},
			\dotsc, X_n)\). Damit ist
			\(\dim A' \ge \height \ideal p' + \depth \ideal p' \ge (n - r) + r = n = \dim A'\).
			\qedhere
		\end{enumerate}
	\end{proof}
\end{frame}



\clearpage

\mode<article>{
	\appendix
	\part{\appendixname}}

%\lecture{Tor und Ext}{Tor und Ext}
%\mode<all>\section{Tor und Ext}



%\lecture{Moduln über Dedekindschen Bereichen}{Moduln über Dedekindschen Bereichen}
%\mode<all>\setcounter{section}{1}
%\mode<all>\section{Moduln über Dedekindschen Bereichen}

\subsection{Vorüberlegungen}

\mode<article>{In diesem Kapitel wollen wir die Moduln über einem Dedekindschen Bereich klassifizieren. Da jeder
Hauptidealbereich insbesondere ein Dedekindscher Bereich ist, ist die Klassifikation in diesem Kapitel eine
Verallgemeinerung der Klassifikation von Moduln über Hauptidealbereichen.}

\begin{frame}{Exponenten gebrochener Ideale}
	Sei \(A\) ein Dedekindscher Bereich. Wir erinnern, daß die heißt, daß \(A\) ein ganz abgeschlossener
	Integritätsbereich der Dimension \(1\) ist. Insbesondere läßt sich in \(A\) jedes nicht verschwindende
	gebrochene Ideal \(\ideal r\) eindeutig als Produkt
	\(\ideal r = \prod\limits_{\ideal p} \ideal p^{n_{\ideal p}}\)
	von (ganzzahligen) Potenzen von Primidealen von \(A\) schreiben.
	\begin{definition}<+->
		Die Zahl \(\ord_{\ideal p}(\ideal r) \coloneqq n_{\ideal p}\) heißt der \emph{Exponent von \(\ideal r\) an
		\(\ideal p\)}.
	\end{definition}
	\begin{visibleenv}<+->
		Wir setzen \(\ord_{\ideal p}((0)) \coloneqq \infty\) für alle \(\ideal p\) und
		\(\ord_{\ideal p} f = \ord_{\ideal p}(f)\) für eine Funktion \(f\).
	\end{visibleenv}
	\begin{remark}<+->
		Der Expontent \(\ord_{\ideal p} \ideal a\) von \(\ideal a\) an \(\ideal p\) ist
		gerade durch dasjenige \(n \in \set Z\) gegeben, so daß \(A_{\ideal p} \ideal a = (x^n)\)
		gilt, wobei \(x\) Erzeuger des maximalen Ideals im diskreten lokalen Bewertungsring \(A_{\ideal p}\) ist.
	\end{remark}
\end{frame}

\begin{frame}{Lösungen von Idealgleichungen}
	\begin{lemma}<+->
		Sei \(A\) ein Dedekindscher Bereich. Seien \(\ideal a, \ideal b\) zwei ganze Ideale von \(A\) mit
		\(\ideal b \neq (0)\).
		Dann existieren ein ganzes Ideal \(\ideal c\) und ein \(r \in A \setminus \{0\}\)
		mit \(\ideal a + \ideal c = (1)\) und \(\ideal b \ideal c = (r)\).
	\end{lemma}
	\begin{visibleenv}<+->
		Ist \(\ideal b\) nur ein gebrochenes Ideal, so gilt die Aussage des Hilfssatzes ebenfalls, es ist dann
		allerdings im allgemeinen \(r \in K^\units\).
	\end{visibleenv}
	\begin{proof}<+->
		\begin{enumerate}[<+->]
		\item<.->
			Seien \(\ideal a = \prod\limits_{i = 1}^t (\ideal p_i)^{a_i}\)
			und \(\ideal b = \prod\limits_{i = 1}^t (\ideal p_i)^{b_i}\)
			Primidealzerlegungen von \(\ideal a\) und \(\ideal b\).
			Wir wählen \(r_i \in \ideal p_i^{b_i} \setminus \ideal p_i^{b_i + 1}\).
		\item
			Da die \(\ideal p_i\) paarweise koprim sind, existiert ein \(r \in A\) mit \(r = r_i\)
			modulo \(\ideal p_i^{b_i + 1}\). Es folgt \(r \in \bigcup\limits_{i = 1}^t
			\ideal p_i^{b_i} = \ideal b\).
		\item
			Damit ist \((r) + \ideal a \ideal b = \ideal b\), denn der Exponent der linken Seite an einem Primideal
			\(\ideal p\) ist gerade \(b_i\).
		\item
			Setze schließlich \(\ideal c \coloneqq (r) \ideal b^{-1}\). Dann ist \(\ideal c \ideal b + \ideal a \ideal b
			= \ideal b\), also \(\ideal c + \ideal a = (1)\).
			\qedhere
		\end{enumerate}
	\end{proof}
\end{frame}

\begin{frame}{Direkte Summen gebrochener Ideale}
	\begin{lemma}<+->
		Seien \(\ideal a, \ideal b\) zwei Ideale in einem Dedekindschen Bereich \(A\). Dann existiert
		ein Isomorphismus \(\ideal a \oplus \ideal b \cong (1) \oplus \ideal a \ideal b\) von \(A\)-Moduln.
	\end{lemma}
	\begin{proof}<+->
		\begin{enumerate}[<+->]
		\item<.->
			Sind \(\ideal a, \ideal b\) ganz und koprim, so ist \(\ideal a \oplus \ideal b \to (1), (a, b) \mapsto
			a + b\) ein surjektiver Ringhomomorphismus mit Kern \(\ideal a \ideal b\).
		\item
		\item
		\end{enumerate}
	\end{proof}
\end{frame}



\mode<article>

\section{Aufgaben}

\subsection{Ringe und Ideale}

\begin{exercise}
	\label{exer:sum_unit_nilp}
	Sei \(x\) ein nilpotentes Element eines kommutativen Ringes \(A\). Zeige,
	daß \(1 + x\) eine Einheit von \(A\) ist. Folgere, daß die Summe eines
	nilpotenten Elementes mit einer Einheit wieder eine Einheit ist.
\end{exercise}

\begin{exercise}
	\label{exer:polys}
	Sei \(A\) ein kommutativer Ring und \(A[x]\) der Polynomring in der
	Variablen \(x\) über \(A\). Sei \(f = a_0 + a_1 x + \dotsb + a_m x^m
	\in A[x]\). Zeige:
	\begin{enumerate}
	\item
		Das Polynom \(f\) ist genau dann eine Einheit in \(A[x]\), wenn \(a_0\)
		eine Einheit in \(A\) und die \(a_1, \dotsc, a_m\) nilpotent sind.
		
		(Sei \(b_0 + b_1 x + \dotsb + b_n x^n \in A[x]\) eine Inverse von \(f\).
		Zeige per Induktion über \(r\), daß \(a_m^{r + 1} b_{n - r} = 0\).
		Folgere daraus, daß \(a_m\) nilpotent ist und nutze
		dann~\prettyref{exer:sum_unit_nilp}.)
	\item
		Das Polynom \(f\) ist genau dann nilpotent, wenn die \(a_0, \dotsc, a_m\)
		nilpotent sind.
	\item
		Das Polynom \(f\) ist genau dann ein Nullteiler, wenn ein \(a \in A \setminus
		\{0\}\) mit \(a f = 0\) existiert.
		
		(Sei \(g = b_0 + b_1 x + \dotsb + b_n x^n \in A[x] \setminus \{0\}\) ein
		Polynom minimalen Grades mit \(g f = 0\). Dann ist \(a_m b_n = 0\), und damit
		auch \(a_m g = 0\), denn \((a_m g) f = 0\) und \(a_m g\) hat echt kleineren
		Grad als \(g\). Folgere dann per Induktion über \(r\), daß
		\(a_{m - r} g = 0\).)
	\item
		Das Polynom \(f \in A[x]\) heißt \emph{primitiv}, wenn
		\((a_0, \dotsc, a_m) = (1)\). Sei \(g \in A[x]\) ein weiteres Polynom.
		
		Dann ist \(fg\) genau dann primitiv, wenn \(f\) und \(g\) primitiv sind.
	\end{enumerate}
\end{exercise}

\begin{exercise}
	Verallgemeinere die Aussagen der \prettyref{exer:polys} auf einen Polynomring
	\(A[x_1, \dotsc, x_n]\) in mehreren Variablen.
\end{exercise}

\begin{exercise}
	Sei \(A\) ein kommutativer Ring. Zeige, daß im Polynomring \(A[x]\) das 
	Jacobsonsche Radikal gleich dem Nilradikal ist.
\end{exercise}

\begin{exercise}
	Sei \(A\) ein kommutativer Ring. Sei \(\ps A x\) der Ring der formalen
	Potenzreihen \(f = \sum\limits_{m = 0}^\infty a_m x^m\) mit Koeffizienten in
	\(A\). Zeige:
	\begin{enumerate}
	\item
		Die Potenzreihe \(f\) ist genau dann eine Einheit in \(\ps A x\), wenn
		\(a_0\) eine Einheit in \(A\) ist.
	\item
		Ist \(f\) nilpotent, ist \(a_m\) für alle \(m \in \set N_0\) nilpotent.
		
		Gilt auch die Umkehrung? (Vergleiche mit \prettyref{exer:nilp_powerseries}.)
	\item
		Die Potenzreihe \(f\) liegt genau dann im Jacobsonschen Radikal von
		\(\ps A x\), wenn \(a_0\) im Jacobsonschen Ideal von \(A\) liegt.
	\item
		Sei \(\ideal m\) ein maximales Ideal in \(\ps A x\). Dann ist
		die Kontraktion \(\ideal m_0 \coloneqq A \cap \ideal m\) ein maximales Ideal
		in \(A\) und \(\ideal m\) ist das von \(\ideal m_0\) und \(x\) in \(\ps A x\)
		erzeugte Ideal.
	\item
		Jedes Primideal von \(A\) ist die Kontraktion eines Primideals von
		\(\ps A x\).
	\end{enumerate}
\end{exercise}

\begin{exercise}
	Sei \(A\) ein kommutativer Ring, in dem jedes nicht im Nilradikal enthaltene
	Ideal ein nicht triviales Idempotentes enthält, das heißt, ein Element
	\(e \neq 0\) mit \(e^2 = e\). Zeige, daß das Nilradikal und das Jacobsonsche
	Radikal von \(A\) übereinstimmen.
\end{exercise}

\begin{exercise}
	Sei \(A\) ein kommutativer Ring, in dem jedes Element \(x\) eine Gleichung der
	Form \(x^n = x\) für ein (von \(x\) abhängiges) \(n > 1\) erfüllt. Zeige, daß
	jedes Primideal von \(A\) maximal ist.
\end{exercise}

\begin{exercise}
	Sei \(A\) ein kommutativer Ring, welcher nicht der Nullring ist. Zeige, daß \(A\)
	ein bezüglich der Inklusion minimales Primideal besitzt.
\end{exercise}

\begin{exercise}
	\label{exer:radicals}
	Sei \(\ideal a \neq (1)\) ein echtes Ideal eines kommutativen Ringes \(A\). Zeige,
	daß \(\ideal a\) genau dann mit seinem Wurzelideal übereinstimmt, wenn
	\(\ideal a\) ein Schnitt von Primidealen ist.
\end{exercise}

\begin{exercise}
	Sei \(A\) ein kommutativer Ring mit Nilradikal \(\ideal n\). Zeige, daß folgende
	Aussagen äquivalent sind:
	\begin{enumerate}
	\item
		Der Ring \(A\) besitzt genau ein Primideal.
	\item
		Jedes Element von \(A\) ist entweder eine Einheit oder nilpotent.
	\item
		Der Quotientenring \(A/\ideal n\) ist ein Körper.
	\end{enumerate}
\end{exercise}

\begin{exercise}
	Sei \(A\) ein \emph{Boolescher Ring}, das heißt ein kommutativer Ring, in dem
	\(x^2 = x\) für alle \(x \in A\) gilt. Zeige:
	\begin{enumerate}
	\item
		Für alle \(x \in A\) gilt \(2 x = 0\).
	\item
		Jedes Primideal \(\ideal p\) von \(A\) ist maximal und \(A/\ideal p\) ist
		ein Körper mit zwei Elementen.
	\item
		Jedes endlich erzeugte Ideal von \(A\) ist ein Hauptideal.
	\end{enumerate}
\end{exercise}

\begin{exercise}
	Sei \(A\) ein lokaler kommutativer Ring. Zeige, daß \(A\) außer \(0\) und \(1\)
	keine idempotenten Elemente enthält.
\end{exercise}

\begin{exercise}
	\label{exer:ideals_of_zero_divs}
	Sei \(A\) ein kommutativer Ring. Sei \(\mathfrak S\) die Menge aller Ideale von \(A\),
	in denen jedes Element ein Nullteiler ist. Zeige, daß im Falle \(A \neq 0\) die Menge \(\mathfrak S\) bezüglich
	der Inklusion maximale Elemente besitzt und daß jedes maximale Element von \(\mathfrak S\) ein
	Primideal ist. Folgere, daß die Menge der Nullteiler von \(A\) eine Vereinigung von Primidealen
	ist.
\end{exercise}


\subsection{Moduln}

\begin{exercise}
	Zeige, daß \((\set Z/(m)) \otimes_{\set Z} (\set Z/(n)) = 0\), falls \(m, n\) teilerfremd
	sind.
\end{exercise}

\begin{exercise}
	\label{exer:tensor_with_quotient}
	Sei \(A\) ein kommutativer Ring. Seien \(\ideal a\) ein Ideal in \(A\) und \(M\) ein \(A\)-Modul.
	Zeige, daß der \(A\)-Modul \((A/\ideal a) \otimes_A M\) isomorph zu \(M/\ideal a M\) ist.
	
	(Tip: Tensoriere die exakte Sequenz \(0 \to \ideal a \to A \to A/\ideal a \to 0\) mit \(M\).)
\end{exercise}

\begin{exercise}
	Sei \(A\) ein lokaler, kommutativer Ring. Seien \(M\) und \(N\) zwei endlich erzeugte \(A\)-Moduln.
	Zeige, daß aus \(M \otimes N = 0\) schon \(M = 0\) oder \(N = 0\) folgt.
	
	(Tip: Sei \(\ideal m\) das maximale Ideal von \(A\) und \(k = A/\ideal m\) der Restklassenkörper.
	Nach~\prettyref{exer:tensor_with_quotient} ist die Skalarerweiterung von \(M\) auf \(k\) gerade die
	Faser \(M_k = k^A \otimes_A M \cong M/\ideal m M = M(\ideal m)\). Folgere aus dem Nakajamaschen Lemma,
	daß aus \(M_k = 0\) schon \(M = 0\) folgt. Ist \(M \otimes_A N = 0\) folgt auch \((M \otimes_A N)_k
	= M_k \otimes_k N_k = 0\). Da \(k\) ein Körper ist, muß daher schon \(M_k = 0\) oder \(N_k = 0\) folgen.)
\end{exercise}

\begin{exercise}
	\label{exer:flatness_of_direct_sum}
	Sei \(A\) ein kommutativer Ring.
	Sei \((M_i)_{i \in I}\) eine Familie von \(A\)-Moduln. Sei \(M \coloneqq \bigoplus\limits_{i \in I}
	M_i\) ihre direkte Summe. Zeige, daß der \(A\)-Modul \(M\) genau dann flach ist, wenn alle \(M_i\) flach sind.
\end{exercise}

\begin{exercise}
	Sei \(A\) ein kommutativer Ring. Zeige, daß die Polynomalgebra \(A[x]\) eine flache \(A\)-Algebra ist.
	
	(Tip: Benutze~\prettyref{exer:flatness_of_direct_sum}.)
\end{exercise}

\begin{exercise}
	\label{exer:poly_over_mod}
	Sei \(A\) ein kommutativer Ring. Für jeden \(A\)-Modul \(M\) sei \(M[x]\) die Menge aller Polynome in
	\(x\) mit Koeffizienten in \(M\), also Ausdrücken der Form \(m_0 + m_1 x + \dotsb + m_n x^n\) mit \(m_i \in M\).
	Zeige, daß \(M[x]\) mit der naheliegenden Definition der Multiplikation mit Elementen aus \(A[x]\) ein \(A[x]\)-Modul
	wird.
	
	Zeige weiter, daß \(M[x] \cong A[x] \otimes_A M\) als \(A\)-Moduln.
\end{exercise}

\begin{exercise}
	\label{exer:poly_over_prime}
	Sei \(A\) ein kommutativer Ring. Sei \(\ideal p\) ein Primideal in \(A\). Zeige, daß \(\ideal p[x]\) ein Primideal in
	\(A[x]\) ist.
	
	Ist \(\ideal m[x]\) im allgemeinen ein maximales Ideal in \(A[x]\), wenn \(\ideal m\) ein maximales Ideal in \(A\) ist?
\end{exercise}

\begin{exercise}
	Sei \(A\) ein kommutativer Ring. Zeige:
	\begin{enumerate}
	\item
		Sind \(M\) und \(N\) flache \(A\)-Moduln, so ist auch \(M \otimes_A N\) ein flacher \(A\)-Modul.
	\item
		Ist \(B\) eine flache \(A\)-Algebra und \(N\) ein flacher \(B\)-Modul, so ist \(N^A\) ein flacher \(A\)-Modul.
	\end{enumerate}
\end{exercise}

\begin{exercise}
	Sei \(A\) ein kommutativer Ring. Sei \(0 \to M' \to M \to M'' \to 0\) eine exakte Sequenz. Zeige: Sind \(M'\) und \(M''\)
	endlich erzeugt, so ist auch \(M\) endlich erzeugt.
\end{exercise}

\begin{exercise}
	Sei \(A\) ein kommutativer Ring. Sei \(\ideal a\) ein im Jacobsonschon Ideal von \(A\) enthaltenes Ideal. Seien \(M\) ein 
	\(A\)-Modul und \(N\) ein endlich erzeugter \(A\)-Modul. Sei \(\phi\colon M \to N\) ein Homomorphismus von \(A\)-Moduln.
	Zeige: Ist der induzierte Homomorphismus \(M/\ideal a M \to N/\ideal a N\) surjektiv, so ist auch \(\phi\) surjektiv.
\end{exercise}

\begin{exercise}
	Sei \(A\) ein kommutativer Ring mit \(A \neq 0\). Seien \(m, n \in \set N_0\).
	\begin{enumerate}
	\item
		Zeige:
		Sind \(A^m\) und \(A^n\) isomorphe \(A\)-Moduln, so folgt \(m = n\).
		
		(Tip: Sei \(\phi\colon A^m \isoto A^n\) ein Isomorphismus. Sei \(\ideal m\) ein maximales Ideal von \(A\). Sei
		\(k \coloneqq A/m\). Dann ist \(\id_k \otimes \phi\colon k \otimes A^m \to k \otimes A^n\) ein Isomorphismus
		von \(k\)-Vektorräumen der Dimension \(m\) beziehungsweis \(n\). Es folgt \(m = n\).)
	\item
		Zeige:
		Ist \(\phi\colon A^m \to A^n\) eine surjektive \(A\)-lineare Abbildung, so folgt \(m \ge n\).
	\item
		Sei \(\phi\colon A^m \to A^n\) eine \(A\)-lineare Abbildung. Folgt aus der Injektivität von \(\phi\), daß
		\(m \le n\)?
	\end{enumerate}
\end{exercise}

\begin{exercise}
	\label{exer:fg_ker_map_to_free}
	Sei \(A\) ein kommutativer Ring.
	Seien \(M\) ein endlich erzeugter \(A\)-Modul und \(\phi\colon M \to A^n\) eine surjektive \(A\)-lineare Abbildung.
	Zeige, daß der \(A\)-Modul \(\ker \phi\) endlich erzeugt ist.
	
	(Tip: Sei \((e_1, \dotsc, e_n)\) eine Basis von \(A^n\). Wähle \(u_i \in M\) mit \(\phi(u_i) = e_i\). Zeige, daß
	\(M\) die direkte Summe von \(\ker \phi\) und dem von den \(u_i\) erzeugten Untermodul ist.)
\end{exercise}

\begin{exercise}
	Sei \(\phi\colon A \to B\) ein Homomorphismus kommutativer Ringe. Sei \(N\) ein \(B\)-Modul. Betrachte den
	\(B\)-Modul \((N^A)_B = B \otimes_A N^A\). Zeige, daß die \(A\)-lineare Abildung \(\iota\colon
	N \to (N^A)_B, y \mapsto 1 \otimes y\) injektiv ist und daß \(\im \iota\) ein direkter Summand von \((N^A)_B\) ist.
	
	(Tip: Definiere \(\pi\colon (N^A)_B \to N, b \otimes y \mapsto by\) und zeige, daß \((N^A)_B = \im \iota + \ker \rho\).
\end{exercise}


\subsection{Lokalisierungen von Ringen und Moduln}

\begin{exercise}
	Sei \(S\) eine multiplikativ abgeschlossene Teilmenge eines kommutativen Ringes \(A\).
	Sei \(M\) ein endlich erzeugter \(A\)-Modul. Zeige, daß \(S^{-1} M = 0\) genau dann, wenn
	ein \(s \in S\) mit \(s M = 0\) existiert.
\end{exercise}

\begin{exercise}
	Sei \(\ideal a\) ein Ideal in einem kommutativen Ring \(A\). Sei \(S = 1 + \ideal a\). Zeige, daß
	\(S^{-1} \ideal a\) im Jacobsonschen Radikal von \(S^{-1} A\) enthalten ist.
\end{exercise}

\begin{exercise}
	Sei \(A\) ein kommutativer Ring. Seien \(S, T \subset A\) zwei multiplikativ abgeschlossene
	Teilmengen. Sei \(S^{-1} T\) das Bild von \(T\) in \(S^{-1} A\). Zeige, daß die Ringe
	\((ST)^{-1} A\) und \((S^{-1} T)^{-1} S^{-1} A\) isomorph sind.
\end{exercise}

\begin{exercise}
	Sei \(\phi\colon A \to B\) ein Homomorphismus kommutativer Ringe. Sei \(S \subset A\) multiplikativ
	abgeschlossen. Zeige, daß \(S^{-1} (B^A)\) und \(((\phi(S))^{-1} B)^{S^{-1} A}\) isomorph als
	\(S^{-1} A\)-Algebren sind.
\end{exercise}

\begin{exercise}
	Sei \(A\) ein kommutativer Ring. Für jedes Primideal \(\ideal p\) von \(A\) besitze \(A_{\ideal p}\) kein
	nilpotentes Element außer \(0\). Zeige, daß \(A\) außer \(0\) kein nilpotentes Element besitzt.
	
	Folgt aus der Tatsache, daß alle Halme \(A_{\ideal p}\) Integritätsbereiche sind, die Tatsache, daß \(A\)
	ein Integritätsbereich ist?
\end{exercise}

\begin{exercise}
	\label{exer:max_mult_closed}
	Sei \(A\) ein kommutativer Ring mit \(A \neq 0\). Sei \(\mathfrak S\) die Menge der multiplikativ abgeschlossenen
	Teilmengen \(S\) von \(A\) mit \(0 \notin S\). Zeige, daß \(\mathfrak S\) bezüglich der Inklusion maximale Elemente
	besitzt und daß \(S \in \mathfrak S\) genau dann maximal ist, falls \(A \setminus S\) ein minimales Primideal von \(A\)
	ist.
\end{exercise}

\begin{exercise}
	\label{exer:mult_sat}
	Wir nennen eine multiplikativ abgeschlossene Teilmenge \(S\) eines kommutativen Ringes \(A\) \emph{gesättigt},
	falls aus \(xy \in S\) schon \(x \in S\) und \(y \in S\) folgt. Zeige:
	\begin{enumerate}
	\item
		Es ist \(S\) genau dann gesättigt, wenn \(A \setminus S\) eine Vereinigung von Primidealen ist.
	\item
		Es gibt eine eindeutige, kleinste gesättigte multiplikativ abgeschlossene Teilmenge \(\bar S \subset A\) mit
		\(S \subset \bar S\), nämlich das Komplement der Vereinigung aller Primideale von \(A\), welche \(S\) nicht schneiden.
		
		Es heißt \(\bar S\) die \emph{Sättigung von \(S\)}.
	\end{enumerate}
	
	Berechne die Sättigung einer multiplikativ abgeschlossenen Teilmenge der Form \(1 + \ideal a\), wobei
	\(\ideal a\) ein Ideal von \(A\) ist.
\end{exercise}

\begin{exercise}
	Sei \(A\) ein kommutativer Ring. Seien \(S, T \subset A\) zwei multiplikativ abgeschlossene Teilmengen von \(A\) mit
	\(S \subset T\). Sei \(\phi\colon S^{-1} A \to T^{-1} A, \frac a s \mapsto \frac a s\). Zeige, daß die folgenden Aussagen
	äquivalent sind:
	\begin{enumerate}
	\item	
		Der Ringhomomorphismus \(\phi\) ist bijektiv.
	\item
		Für alle \(t \in T\) ist \(\frac t 1\) eine Einheit in \(S^{-1} A\).
	\item
		Für alle \(t \in T\) existiert ein \(x \in A\) mit \(x t \in S\).
	\item
		Es ist \(T\) in der Sättigung \(\bar S\) von \(S\) enthalten (\prettyref{exer:mult_sat}).
	\item
		Für jedes Primideal \(\ideal p\) von \(A\) mit \(\ideal p \cap T \neq \emptyset\) gilt auch
		\(\ideal p \cap S \neq \emptyset\).
	\end{enumerate}
\end{exercise}

\begin{exercise}
	Sei \(A\) ein kommutativer Ring. Die Menge \(S_0\) der regulären Elemente von \(A\) ist eine
	gesättigte, multiplikativ abgeschlossene Teilmenge. Damit ist die Menge \(D\) der Nullteiler von \(A\)
	eine Vereinigung von Primidealen nach~\prettyref{exer:ideals_of_zero_divs}. Zeige, daß jedes minimale Primideal
	von \(A\) in \(D\) enthalten ist.
	
	(Tip:~\prettyref{exer:max_mult_closed}.)
	
	Der Ring \(S_0^{-1} A\) heißt der \emph{vollständige Quotientenring von \(A\)}. Zeige:
	\begin{enumerate}
	\item
		Die Teilmenge \(S_0\) ist die größte multiplikativ abgeschlossene Teilmenge \(S\) von \(A\), 
		für die \(A \to S^{-1} A\) injektiv ist.
	\item
		Jedes Element in \(S_0^{-1} A\) ist entweder ein Nullteiler oder eine Einheit.
	\item	
		Ein kommutativer Ring \(A\), in dem jede Nichteinheit ein Nullteiler ist, ist gleich
		seinem vollständigen Quotientenring, das heißt \(A \to S^{-1}_0 A\) ist ein Isomorphismus.
	\end{enumerate}
\end{exercise}

\begin{exercise}
	\label{exer:trivial_stalks_in_closed_subset}
	Sei \(\ideal a\) ein Ideal in einem kommutativen Ring \(A\). Sei \(M\) ein \(A\)-Modul. Für die Halme an
	allen maximalen Idealen \(\ideal m\) von \(A\) mit \(\ideal m \supset \ideal a\) gelte \(M_{\ideal m} = 0\). Zeige,
	daß \(M = \ideal a M\).
	
	(Tip: Gehe auf den \(A/\ideal a\)-Modul \(M/\ideal a M\) über und nutze die Lokalität der Trivialität eines Moduls.)
\end{exercise}

\begin{exercise}
	Sei \(A\) ein kommutativer Ring. Sei \(F \coloneqq A^n\) als \(A\)-Modul. Zeige, daß jede Menge von \(n\) Erzeugern von
	\(F\) als \(A\)-Modul schon eine Basis von \(F\) als \(A\)-Modul ist.
	
	(Tip: Seien \(x_1, \dotsc, x_n\) Erzeuger von \(F\). Sei \((e_1, \dotsc, e_n)\) die kanonische Basis von \(F\). Definiere
	\(\phi\colon F \to F\) durch \(\phi(e_i) = x_i\). Dann ist \(\phi\) surjektiv. Es ist zu zeigen, daß \(\phi\) ein Isomorphismus
	ist. Da Injektivität eine lokale Eigenschaft ist, können wir annehmen, daß \(A\) ein lokaler Ring ist. Sei etwa \(k = A/\ideal m\)
	der Restklassenkörper von \(A\). Sei weiter \(N = \ker \phi\). Da \(F\) ein flacher \(A\)-Modul ist, führt die exakte
	Sequenz \(0 \to N \to F \xrightarrow{\phi} F \to 0\) zu einer exakten Sequenz \(0 \to k \otimes N \to k \otimes F \xrightarrow
	{\id_k \otimes \phi} k \otimes F \to 0\). Es ist \(k \otimes F = k^n\) ein \(n\)-dimensionaler \(k\)-Vektorraum. Aus der Surjektivität von
	\(\id_k \otimes \phi\) folgt damit auch die Injektivität, also \(k \otimes N = 0\).
	Nach~\prettyref{exer:fg_ker_map_to_free} ist \(N\) weiter endlich erzeugt. Nach dem Nakayamaschen Lemma ist damit \(N = 0\). Damit 
	ist \(\phi\) ein Isomorphismus.)
	
	Folgere, daß jede Erzeugermenge von \(F\) aus mindestens \(n\) Elementen bestehen muß.
\end{exercise}


\subsection{Primärzerlegung}

\begin{exercise}
	Sei \(\ideal a\) ein zerlegbares Ideal in einem kommutativen Ring
	\(A\) mit \(\ideal a = \sqrt{\ideal a}\). Zeige, daß \(\ideal a\) keine
	assoziierten eingebetteten Primideale besitzt.
\end{exercise}

\begin{exercise}
	Zeige, daß in dem Polynomring \(\set Z[t]\) das Ideal \(\ideal m = (2, t)\)
	maximal ist. Zeige weiter, daß das Ideal \(\ideal q = (4, t)\) ein
	\(\ideal m\)-primäres Ideal ist, aber keine Potenz von \(\ideal m\).
\end{exercise}

\begin{exercise}
	Sei \(K\) ein Körper. Seien \(\ideal p_1 \coloneqq (x, y),
	\ideal p_2 \coloneqq (x, z),
	\ideal m \coloneqq (x, y, z)\) drei Ideale im Polynomring \(K[x, y, z]\).
	Zeige, daß \(\ideal p_1, \ideal p_2\) Primideale sind und daß \(\ideal m\)
	ein maximales Ideal ist.
	
	Sei \(\ideal a = \ideal p_1 \ideal p_2\). Zeige, daß \(\ideal a = \ideal p_1
	\cap \ideal p_2 \cap \ideal m^2\) eine minimale Primärzerlegung von
	\(\ideal a\) ist. Welche Komponenten sind isoliert und welche eingebettet?
\end{exercise}

\begin{exercise}
	\label{exer:primary_in_poly}
	Sei \(A\) ein kommutativer Ring. Zeige:
	\begin{enumerate}
	\item
		Ist \(\ideal a\) ein Ideal in \(A\), so ist \(\ideal a[x]\)
		(vergleiche~\prettyref{exer:poly_over_mod})
		die Erweiterung von \(\ideal a\) nach \(A[x]\).
	\item
		Seien \(\ideal p\) ein Primideal in \(A\) und \(\ideal q\) ein
		\(\ideal p\)-primäres Ideal. Dann ist \(\ideal q[x]\) ein
		\(\ideal p[x]\)-primäres Ideal.
		
		(Tip: Nach~\prettyref{exer:poly_over_prime} ist \(\ideal p[x]\) ein
		Primideal. Nutze~\prettyref{exer:polys}.)
	\item
		Ist \(\ideal a = \bigcap\limits_{i = 1}^n \ideal q_i\) eine minimale
		Primärzerlegung eines Ideals \(\ideal a\) in \(A\), so ist \(\ideal a[x]
		= \bigcap\limits_{i = 1}^n \ideal q_i[x]\) eine minimale Primärzerlegung
		von \(\ideal a[x]\).
	\item
		Ist \(\ideal p\) ein zu einem zerlegbaren Ideal \(\ideal a\)
		assoziiertes isoliertes Primideal, so ist \(\ideal p[x]\) ein zum Ideal
		\(\ideal a[x]\) assoziiertes isoliertes Primideal.
	\end{enumerate}
\end{exercise}

\begin{exercise}
	Sei \(K\) ein Körper. Zeige, daß die Ideale \(\ideal p_i \coloneqq
	(x_1, \dotsc, x_i)\) von \(K[x_1,\ldots,x_n]\) für \(1 \leq i \leq n\) alle Primideale sind und
	daß ihre Potenzen alle Primärideale sind.
	
	(Tip:~\prettyref{exer:primary_in_poly}.)
\end{exercise}

\begin{exercise}
	\label{exer:assoc_primes_to_zero}
	Sei \(A\) ein kommutativer Ring. Sei \(D(A)\) die Menge der Primideale
	\(\ideal p\) von \(A\), für die ein \(a \in A\) existiert, so daß
	\(\ideal p\) ein minimales Element in der Menge Primideale ist, die
	\((0 : a)\) umfassen. Zeige:
	\begin{enumerate}
	\item
		Ein Element \(x \in A\) ist genau dann ein Nullteiler,
		wenn \(x \in \ideal p\) für ein \(\ideal p \in D(A)\).
	\item
		Sei \(S \subset A\) multiplikativ abgeschlossen. Dann ist
		\(D(S^{-1} A) = \{S^{-1} \ideal p \mid \ideal p \cap S = \emptyset,
			\ideal p \in D(A)\}\).
	\item
		Ist das Nullideal zerlegbar, so ist \(D(A)\) die Menge der assoziierten
		Primideale von \(0\).
	\end{enumerate}
\end{exercise}

\begin{exercise}
	\label{exer:zero_germ}
	Sei \(A\) ein kommutativer Ring. Für jedes Primideal \(\ideal p\) in \(A\)
	sei \(S_{\ideal p}(0)\) der Kern des Strukturmorphismus' \(A \to A_\ideal p\).
	Zeige:
	\begin{enumerate}
	\item
		\(S_{\ideal p}(0) \subset \ideal p\).
	\item
		Es ist \(\sqrt{S_{\ideal p}(0)} = \ideal p\) genau dann, wenn
		\(\ideal p\) ein minimales Primideal von \(A\) ist.
	\item
		Ist \(\ideal p'\) ein Primideal in \(A\) mit \(\ideal p' \subset
		\ideal p\), so folgt \(S_{\ideal p}(0) \subset S_{\ideal p'}(0)\).
	\item
		Sei \(D(A)\) wie in~\prettyref{exer:assoc_primes_to_zero} definiert.
		Dann ist \(\bigcap\limits_{\ideal p \in D(A)} S_{\ideal p}(0) = (0)\).
	\end{enumerate}
\end{exercise}

\begin{exercise}
	\label{exer:everywhere_zero_germ}
	Sei \(\ideal p\) ein Primideal in einem kommutativen Ring \(A\).
	Sei \(S_{\ideal p}(0)\) wie in~\prettyref{exer:zero_germ} definiert. Zeige:
	\begin{enumerate}
	\item
		Ist \(\ideal p\) ein minimales Primideal, so ist
		\(S_{\ideal p}(0)\) das kleinste \(\ideal p\)-primäre Ideal.
	\item
		Sei \(\ideal a\) der Schnitt aller \(S_{\ideal p}(0)\), wobei
		\(\ideal p\) über alle minimalen Primideale von \(A\) läuft. Dann
		ist \(\ideal a\) im Nilradikal von \(A\) enthalten.
	\item
		Das Nullideal von \(A\) sei zerlegbar. Dann ist \(\ideal a = 0\) genau
		dann, falls jedes zu \((0)\) assoziierte Primideal \(\ideal a\) isoliert
		ist.
	\end{enumerate}
\end{exercise}

\begin{exercise}
	\label{exer:saturation}
	Sei \(S\) eine multiplikativ abgeschlossene Teilmenge eines kommutativen
	Ringes \(A\). Mit \(S(\ideal a)\) bezeichnen wir wie üblich die Sättigung
	eines Ideales \(\ideal a\) nach \(S\).
	Zeige:
	\begin{enumerate}
	\item
		Für je zwei Ideale \(\ideal a, \ideal b\) von \(A\) gilt
		\(S(\ideal a) \cap S(\ideal b) = S(\ideal a \cap \ideal b)\).
	\item
		Für ein Ideal \(\ideal a\) von \(A\) gilt \(S(\sqrt{\ideal a})
		= \sqrt{S(\ideal a)}\).
	\item
		Für ein Ideal \(\ideal a\) von \(A\) gilt \(S(\ideal a) = (1)\) genau
		dann, wenn \(\ideal a \cap S \neq \emptyset\).
	\item
		Seien \(S_1, S_2 \subset A\) multiplikativ abgeschlossen. Für ein
		Ideal \(\ideal a\) gilt dann \(S_1(S_2(\ideal a))
		= (S_1 S_2)(\ideal a)\).
	\end{enumerate}
	
	Sei \(\ideal a\) zerlegbar. Zeige, daß die Menge aller \(S(\ideal a)\),
	wobei \(S\) alle multiplikativ abgeschlossenen Teilmengen von \(A\)
	durchläuft, endlich ist.
\end{exercise}

\begin{exercise}
	\label{exer:symbolic_power}
	Sei \(\ideal p\) ein Primideal eines kommutativen Ringes \(A\). Sei
	\(n \in \set N_0\). Die \emph{\(n\)-te symbolische Potenz von \(\ideal p\)}
	ist die Sättigung \(\ideal p^{(n)} \coloneqq S_{\ideal p}(\ideal p^n)\),
	wobei \(S_{\ideal p} \coloneqq A \setminus \ideal p\).
	Zeige:
	\begin{enumerate}
	\item
		Es ist \(\ideal p^{(n)}\) ein \(\ideal p\)-primäres Ideal.
	\item
		Ist \(\ideal p^n\) zerlegbar, so ist \(\ideal p^{(n)}\) seine
		\(\ideal p\)-primäre Komponente.
	\item
		Ist \(\ideal p^{(m)} \ideal p^{(n)}\) zerlegbar, so ist
		\(\ideal p^{(m + n)}\) seine \(\ideal p\)-primäre Komponente.
	\item
		Es ist \(\ideal p^{(n)} = \ideal p^n\) genau dann, wenn \(\ideal p^n\)
		ein \(\ideal p\)-primäres Ideal ist.
	\end{enumerate}
\end{exercise}

\begin{exercise}
	Sei \(\ideal a\) ein zerlegbares Ideal in einem kommutativen Ring \(A\).
	Sei \(\ideal p\) ein maximales Element unter allen Idealen der Form
	\((\ideal a : x)\) mit \(x \in A\) und \(x \notin \ideal a\). Zeige, daß
	\(\ideal p\) ein zu \(\ideal a\) assoziiertes Primideal ist.
\end{exercise}

\begin{exercise}
	\label{exer:component_to_isolated_set}
	Sei \(\ideal a\) ein zerlegbares Ideal in einem kommutativen Ring \(A\).
	Sei \(\mathfrak S\) eine isolierte Menge von zu \(\ideal a\)
	assoziierten Primidealen. Sei \(\ideal q_{\mathfrak S}\) der Schnitt aller
	\(\ideal p\)-primären Komponenten von \(\ideal a\) mit \(\ideal p \in
	\mathfrak S\). Sei \(f \in A\), so daß für alle zu \(\ideal a\) assoziierten
	Primideale gilt, daß \(f \in \ideal p \iff \ideal p \notin \mathfrak S\).
	Wir setzen \(S_f \coloneqq \{f^n \mid n \in \set N_0\}\). Zeige, daß
	\(\ideal q_{\mathfrak S} = S_f(\ideal a) = (\ideal a \colon f^n)\)
	für alle \(n \gg 0\).
\end{exercise}

\begin{exercise}
	Sei \(A\) ein kommutativer Ring, in dem jedes Ideal zerlegbar ist. Zeige,
	daß \(S^{-1} A\) dieselbe Eigenschaft hat, wenn \(S \subset A\)
	multiplikativ abgeschlossen ist.
\end{exercise}

\begin{exercise}
	\label{exer:prop_l1}
	Sei \(A\) ein kommutativer Ring mit der folgenden Eigenschaft (L1):
	Zu jedem Ideal \(\ideal a \neq (1)\) und zu jedem Primideal \(\ideal p\)
	in \(A\) existiert ein \(x \notin \ideal p\) mit \(S_{\ideal p}(\ideal a)
	= (\ideal a : x)\), wobei \(S_{\ideal p} \coloneqq A \setminus \ideal p\).
	
	Zeige, daß dann jedes Ideal in \(A\) der Schnitt (möglicherweise unendlich
	vieler) primärer Ideale ist.
	
	(Tip: Sei \(\ideal p_1\) ein minimales Element der Menge aller Primideale,
	welche \(\ideal a\) umfassen. Nach~\prettyref{exer:everywhere_zero_germ}
	ist \(\ideal q_1 = S_{\ideal p_1}\). Weiter ist \(\ideal q_1 = (\ideal a :
	x)\) für ein \(x \notin \ideal p_1\). Folgere, daß \(\ideal a = \ideal q_1
	\cap (\ideal a + (x))\).
	
	Sei sodann \(\ideal a_1\) ein maximales Element unter allen Idealen
	\(\ideal b \supset \ideal a\) mit \(\ideal q_1 \cap \ideal b = \ideal a\)
	und \(x \in \ideal a_1\), so daß \(\ideal a_1 \not\subset \ideal p_1\).
	Wiederhole die Konstruktion diesmal mit \(\ideal a_1\) und so weiter.
	
	Im \(n\)-ten Schritt haben wir \(\ideal a = \ideal q_1 \cap \dotsb \cap
	\ideal q_n \cap \ideal a_n\), wobei die \(\ideal q_i\) Primärideale sind
	und \(\ideal a_n\) maximal unter allen Idealen \(\ideal b\) mit
	\(\ideal b \supset \ideal a_{n - 1} = \ideal a_n \cap \ideal q_n\), so
	daß \(\ideal a = \ideal q_1 \cap \dotsb \cap q_n \cap \ideal b\) und
	\(\ideal a_n \not\subset \ideal p_n\).
	
	Sollte in irgendeinem Schnitt \(\ideal a_n = (1)\) gelten, hört die
	Konstruktion auf und \(\ideal a\) ist folglich ein endlicher Schnitt
	primärer Ideale. Im anderen Fall fahre mit transfiniter Induktion fort, und
	zwar unter Beachtung der Tatsache, daß jedes \(\ideal a_n\) das Ideal
	\(\ideal a_{n - 1}\) echt enthält.)
\end{exercise}

\begin{exercise}
	Betrachte folgende Eigenschaft (L2) kommutativer Ringe \(A\): Ist
	\(\ideal a\) ein Ideal von \(A\) und \(S_1 \supset S_2 \supset \dotsb\)
	eine absteigende Kette multiplikativ abgeschlossener Teilmengen von \(A\),
	so existiert ein \(n \in \set N_0\) mit
	\(S_n(\ideal a) = S_{n + 1}(\ideal a) = \dotsb\).
	
	Zeige, daß folgende Aussagen für einen kommutativen Ring \(A\)
	äquivalent sind:
	\begin{enumerate}
	\item
		Jedes Ideal in \(A\) ist zerlegbar.
	\item
		Der Ring \(A\) erfüllt die Eigenschaften (L1)
		(siehe~\prettyref{exer:prop_l1}) und (L2).
	\end{enumerate}
	
	(Tip: Um aus der ersten die zweite Aussage zu folgern,
	benutze~\prettyref{exer:saturation}
	und~\prettyref{exer:component_to_isolated_set}.
	
	Um aus der zweiten die erste Aussage zu folgern, gehe folgendermaßen vor:
	Gilt \(S_n = S_{\ideal p_1} \cap \dotsb \cap S_{\ideal p_n}\) in der
	Notation von~\prettyref{exer:prop_l1}, so ist \(S_n \cap \ideal a_n \neq
	\emptyset\), also \(S_n(\ideal a_n) = (1)\), also
	\(S_n(\ideal a) = \ideal q_1 \cap \dotsb \cap \ideal q_n\).
	Folgere mit (L2), daß die Kontruktion nach einer endlichen Anzahl von
	Schritten aufhören muß.
\end{exercise}

\begin{exercise}
	Sei \(\ideal p\) ein Primideal eines kommutativen Ringes \(A\). Zeige, daß
	jedes \(\ideal p\)-primäres Ideal das Ideal \(S_{\ideal p}(0) = \ker
	(A \to A_{\ideal p})\) umfaßt.
	
	Erfülle \(A\) die folgende Bedingung: Für jedes Primideal sei der Schnitt
	aller \(\ideal p\)-primären Ideale gerade \(S_{\ideal p}(0)\). (Wir
	werden später sehen, daß alle sogenannten noetherschen Ringe diese Bedingung
	erfüllen.) % FIXME Reference missing.
	Seien weiter \(\ideal p_1, \dotsc, \ideal p_n\) paarweise verschiedene
	Primideale von \(A\), welche nicht minimal sind. Zeige, daß dann ein Ideal
	\(\ideal a\) existiert, dessen assoziierte Primideale gerade \(\ideal p_1,
	\dotsc, \ideal p_n\) sind.
	
	(Tip: Beweis per Induktion über \(n\). Der Fall \(n = 1\) ist trivial, denn
	wir können \(\ideal a = \ideal p_1\) setzen. Sei also \(n > 1\), und sei
	\(\ideal p_n\) maximales Element der Menge \(\ideal p_1, \dotsc,
	\ideal p_n\). Nach Induktionsvoraussetzung existiert ein Ideal \(\ideal b\)
	und eine Primärzerlegung \(\ideal b = \ideal q_1 \cap \dotsb
	\ideal q_{n - 1}\), wobei \(\ideal q_i\) ein \(\ideal p_i\)-primäres Ideal
	ist.
	
	Angenommen, \(\ideal b \subset S_{\ideal p_n}(0)\). Sei dann \(\ideal p\)
	ein minimales Primideal von \(A\) mit \(\ideal p \subset \ideal p_n\).
	Dann ist \(S_{\ideal p_n}(0) \subset S_{\ideal p}(0)\), also
	\(\ideal b \subset S_{\ideal p}(0)\). Ziehen der Wurzel liefert
	nach~\prettyref{exer:zero_germ}, daß \(\ideal p_1 \cap \dotsb \cap \ideal
	p_{n - 1} \subset \ideal p\), also \(\ideal p_i \subset \ideal p\) für ein
	\(i < n\), also \(\ideal p_i = \ideal p\), da \(\ideal p\) minimal ist.
	Widerspruch, da kein \(\ideal p_i\) minimal ist.
	
	Also ist \(\ideal b \not\subset S_{\ideal p_n}(0)\). Damit existiert ein
	\(\ideal p_n\)-primäres Ideal \(\ideal q_n\) mit \(\ideal b
	\not\subset \ideal q_n\). Zeige, daß \(\ideal a = \ideal q_1 \cap \dotsb
	\cap \ideal q_n\) die gewünschten Eigenschaften hat.)
\end{exercise}

\begin{exercise}
	Sei \(A\) ein kommutativer Ring. Seien \(M\) ein \(A\)-Modul und
	\(N \subset M\) ein Untermodul. Die \emph{Wurzel von \(N\) in \(M\)} ist
	\[
		\sqrt{N}_M \coloneqq \{x \in A \mid \exists q \in \set N_0\colon
		x^q M \subset N\}.
	\]
	Zeige, daß \(\sqrt{N}_M = \sqrt{(N : M)} = \sqrt{\ann(M/N)}\). Insbesondere
	ist \(\sqrt{N}_M\) ein Ideal von \(A\).
	
	Formuliere und beweise entsprechende Aussagen für \(\sqrt{}_M\) wie
	in~\prettyref{prop:radical}.
\end{exercise}

\begin{exercise}
	Sei \(A\) ein kommutativer Ring. Sei \(M\) ein \(A\)-Modul. Jedes Element
	\(x \in A\) definiert einen Endomorphismus \(\phi_x\colon M \to M,
	m \mapsto x m\) von \(M\). Das Element \(x\) heißt
	\emph{regulär in \(M\)}, falls \(\phi_x\) injektiv ist, und andernfalls
	\emph{Nullteiler in \(M\)}. Weiter heißt \(x\) \emph{nilpotent in \(M\)},
	falls \(\phi_x\) nilpotent ist, falls also \(\phi_x^n = 0\) für ein
	\(n \in \set N_0\).
	
	Ein Untermodul \(Q\) von \(M\) heißt \emph{primär}, falls die Nullteiler
	in \(M/Q\) genau die nilpotenten Elemente in \(M/Q\) sind.
	
	Zeige, daß für einen primären Untermodul \(Q\) von \(M\) das Ideal
	\((Q : M)\) ein primäres Ideal ist und damit \(\ideal p \coloneqq
	\sqrt{Q}_M\) ein Primideal. Wir sagen in diesem Falle, daß
	\(Q\) ein \emph{\(\ideal p\)-primärer Untermodul von \(M\)} ist.
	
	Formuliere und beweise entsprechende Aussagen für \(Q\) wie
	in~\prettyref{lem:primary1} und~\prettyref{lem:primary2}.
\end{exercise}

\begin{exercise}
	\label{exer:primary_decomp_for_mod}
	Sei \(A\) ein kommutativer Ring. Seien \(M\) ein \(A\)-Modul und
	\(N\) ein Untermodul in \(M\). Eine \emph{Primärzerlegung von \(N\) in
	\(M\)} ist eine Darstellung \(N = Q_1 \cap \dotsb \cap Q_n\) von \(N\)
	als Schnitt primärer Untermoduln in \(M\). Sie heißt \emph{minimal},
	falls alle \(\ideal p_i \coloneqq \sqrt{Q_i}_M\) paarweise verschieden sind
	und falls \(Q_i \not\supset \bigcap\limits_{j \neq i} Q_j\) für alle \(i\).
	
	Beweise die Entsprechung des ersten
	Eindeutigkeitssatzes~\ref{thm:first_uniqueness}, nämlich daß die
	Primideale \(\ideal p_i\) nur von \(N\) und \(M\) abhängen. Diese heißen
	die \emph{zu \(N\) in \(M\) assoziierten Primideale}. Zeige weiter, daß
	diese auch die Primideale sind, welche zu \(0\) in \(M/N\) assoziiert sind.
\end{exercise}

\begin{exercise}
	Formuliere und beweise entsprechende Aussagen für die Primärzerlegung von
	Moduln wie
	in~\prettyref{prop:isolated_prime},
	\prettyref{prop:union_of_assoc_primes},
	\prettyref{prop:primaries_in_localisation},
	\prettyref{cor:correspondence_for_primaries},
	\prettyref{prop:sat_decomp},
	\prettyref{thm:second_uniqueness}
	und~\prettyref{cor:second_uniqueness}.
	
	(Tip: Ohne Einschränkung kann davon ausgegangen werden, daß \(N = 0\).)
\end{exercise}


\subsection{Ganzheit und Bewertungen}

\begin{exercise}
	\label{exer:extension_to_integral_extension}
	Sei \(A \subset B\) eine ganze Erweiterung kommutativer Ringe.
	Sei \(\phi\colon A \to L\) ein Ringhomomorphismus in einen algebraisch
	abgeschlossenen Körper \(L\). Zeige, daß \(\phi\) zu einem
	Ringhomomorphismus \(\psi\colon B \to L\) fortgesetzt werden kann.
	
	(Tip:~\prettyref{thm:existence_of_primes_in_integral_extensions}.)
\end{exercise}

\begin{exercise}
	Sei \(A\) ein kommutativer Ring.
	Sei \(\phi\colon B \to B'\) ein Homomorphismus kommutativer \(A\)-Algebren.
	Sei \(C\) eine weitere \(A\)-Algebra. Zeige: Ist \(\phi\) ganz, so ist
	auch \(\phi \otimes \id_C\colon B \otimes_A C \to B' \otimes_A C\)
	ganz.
\end{exercise}

\begin{exercise}
	Sei \(A \subset B\) eine ganze Erweiterung kommutativer Ringe. Sei \(\ideal
	n\) ein maximales Ideal von \(B\) und \(\ideal m \coloneqq A \cap
	\ideal n\) das entsprechende maximale Ideal von \(A\). Ist \(B_{\ideal n}\)
	in jedem Falle ganz über \(A_{\ideal m}\)?
	
	(Tip: Betrachte die Ringerweiterung \(K[x^2 - 1] \subset K[x]\) für einen
	Körper \(K\), und sei \(\ideal n = (x - 1)\). Kann das Element \(\frac
	1 {x + 1}\) ganz sein?)
\end{exercise}

\begin{exercise}
	Sei \(A \subset B\) eine ganze Erweiterung kommutativer Ringe. Zeige:
	\begin{enumerate}
	\item
		Ist \(x \in A\) eine Einheit in \(B\), so ist \(x\) auch eine Einheit
		in \(A\).
	\item
		Ist \(\ideal k\) das Jacobsonsche Radikal von \(B\), so ist die
		Kontraktion
		\(\ideal j \coloneqq A \cap \ideal k\) das Jacobsonsche Radikal von
		\(A\).
	\end{enumerate}
\end{exercise}

\begin{exercise}
	Sei \(A\) ein kommutativer Ring. Seien \(B_1, \dotsc, B_n\) ganze
	kommutative \(A\)-Algebren. Zeige, daß \(\prod\limits_{i = 1}^n B_i\)
	eine ganze \(A\)-Algebra ist.
\end{exercise}

\begin{exercise}
	Sei \(A \subset B\) eine Erweiterung kommutativer Ringe, so daß
	\(S \coloneqq B \setminus A\) in \(B\) multiplikativ abgeschlossen ist.
	Zeige, daß dann \(A\) ganz abgeschlossen in \(B\) ist.
\end{exercise}

\begin{exercise}
	\label{exer:product_poly_in_closure}
	Sei \(A \subset B\) eine Erweiterung kommutativer Ringe. Sei \(C\) der
	ganze Abschluß von \(A\) in \(B\). Seien \(f, g \in B[x]\) normierte
	Polynome mit \(fg \in C[x]\). Zeige, daß dann auch \(f, g \in C[x]\).
	
	(Tip: Sei \(B \subset D\) eine Ringerweiterung, in dem \(f\) und \(g\)
	in Linearfaktoren zerfallen, etwa \(f = \prod (x - a_i)\) und \(g = \prod
	(x - b_j)\). Die \(a_i, b_j\) sind Wurzeln von \(fg\) und damit ganz über
	\(C\). Damit sind die Koeffizienten von \(f, g\) ganz über \(C\).)
\end{exercise}

\begin{exercise}
	Sei \(A \subset B\) eine Erweiterung kommutativer Ringe. Sei \(C\) der
	ganze Abschluß von \(A\) in \(B\). Zeige, daß dann \(C[x]\) der ganze
	Abschluß von \(A[x]\) in \(B[x]\) ist.

	(Tip: Ist \(f \in B[x]\) ganz über \(A[x]\), so ist \[f^m + g_1 f^{m - 1}
	+ \dotsb + g_m = 0\] für gewisse \(g_i \in A[x]\). Sei \(r \gg 0\) eine ganze
	Zahl. Sei \(f_1 \coloneqq f - x^r\), also
	\[(f_1 + x^r)^m + g_1 (f + x^r)^{m - 1} + \dotsb + g_m = 0,\]
	das heißt \[f_1^m + h_1 f_1^{m - 1} + \dotsb + h_m = 0\]
	für gewisse \(h_i \in A[x]\), wobei \(h_m = (x^r)^m + g_1(x^r)^{m - 1}
	+ \dotsb + g_m\). Wende jetzt~\prettyref{exer:product_poly_in_closure} auf
	die Polynome \(-f_1\) und \(f_1^{m - 1} + h_1 f_1^{m - 2} + \dotsb +
	h_{m - 1}\) an.
\end{exercise}

\begin{exercise}
	Sei \(G\) eine endliche Gruppe von Automorphismen eines kommutativen Ringes
	\(A\). Sei \(A^G\) der Unterring der \(G\)-Invarianten, das heißt derjenigen
	Elemente \(x \in A\) für die \(g(x) = x\) für alle \(g \in G\).
	\begin{enumerate}
	\item
		Zeige, daß \(A\) ganz über \(A^G\) ist.
	
		(Tip: Sei \(x \in A\). Überlege Dir, daß \(x\) Wurzel des Polynoms
		\(\prod\limits_{g \in G} (t - g(x))\) in \(t\) ist.)
	\item
		Sei \(S\) eine \(G\)-invariante multiplikativ abgeschlossene Teilmenge
		in \(A\), das heißt \(g(S) \subset S\) für alle \(g \in G\). Sei
		\(S^G \coloneqq S \cap A^G\). Zeige, daß sich die Wirkung von \(G\)
		auf \(A\) zu einer Wirkung auf \(S^{-1} A\) fortsetzen läßt und daß
		\((S^G)^{-1} A^G \cong (S^{-1} A)^G\).
	\end{enumerate}
\end{exercise}

\begin{exercise}
	Sei \(G\) eine endliche Gruppe von Automorphismen eines kommutativen Ringes
	\(A\). Sei \(\ideal p\) ein Primideal in \(A^G\). Sei \(\mathfrak Q\) die
	Menge aller Primideale \(\ideal q\) von \(A\) mit \(\ideal p = A^G \cap
	\ideal q\). Zeige, daß \(G\) auf \(\mathfrak Q\) transitiv operiert, und
	folgere, daß \(\mathfrak Q\) endlich ist.

	(Tip: Seien \(\ideal q_1, \ideal q_2 \in \mathfrak Q\) und \(x \in \ideal
	q_1\). Dann ist \(\prod\limits_{g \in G} g(x) \in \ideal A^G \cap \ideal q_1
	= \ideal p \subset \ideal q_2\), also \(g(x) \in \ideal q_2\) für ein
	\(g \in G\). Folgere, daß \(\ideal q_1 \subset \bigcup\limits_{g \in G}
	g(\ideal p_2)\), und wende dann~\prettyref{prop:ideal_in_union_of_primes}
	und~\prettyref{cor:equality_of_primes_in_integral_extension} an.)
\end{exercise}

\begin{exercise}
	\label{exer:global_noether_norm}
	Sei \(A \subset B\) eine endlich erzeugte Erweiterung von
	Integritätsbereichen. Zeige, daß ein \(s \in A \setminus \{0\}\) und eine
	\(A\)-Algebra \(B' \subset B\) mit \(B' \cong A[y_1, \dotsc, y_n]\)
	existieren, so daß \(B[s^{-1}]\) ganz über \(B'[s^{-1}]\) ist.
	
	(Tip: Sei \(S \coloneqq A \setminus \{0\}\), das heißt \(K = S^{-1} A\) ist
	der Quotientenkörper von \(A\). Dann ist \(S^{-1} B\) eine endlich erzeugte
	\(K\)-Algebra, und nach~\prettyref{prop:noether_norm} existieren damit
	\(x_1, \dotsc, x_n \in S^{-1} B\), so daß \(K[x_1, \dotsc, x_n] \subset
	S^{-1} B\) der Polynomring über \(K\) in \(n\) Variablen ist und daß
	\(S^{-1} B\) ganz über \(K[x_1, \dotsc, x_n]\) ist. Seien
	\(z_1, \dotsc, z_m\) Erzeuger von \(B\) als kommutative \(A\)-Algebra. Dann
	sind die \(z_i\), aufgefaßt als Element in \(S^{-1} B\) ganz über
	\(K[x_1, \dotsc, x_n]\). Schreibe Ganzheitsbedingungen für die \(z_j\) hin,
	und zeige, daß ein \(s \in S\) existiert, so daß \(x_i = \frac{y_i} s\)
	mit
	\(y_i \in B\) und so daß \(s z_j\) ganz über \(B'\) ist. Folgere, daß dieses
	\(s\) die geforderte Eigenschaft hat.)
\end{exercise}

\begin{exercise}
	\label{exer:global_extension_to_ac}
	Seien \(A \subset B\) eine endlich erzeugte Erweiterung von
	Integritätsbereichen. Zeige, daß ein \(s \in A \setminus \{0\}\) existiert,
	so daß jeder Ringhomomorphismen \(\phi\colon A \to L\) in einen algebraisch
	abgeschlossenen Körper \(L\) mit \(\phi(s) \neq 0\) zu einem
	Ringhomomorphismus \(B \to L\) fortgesetzt werden kann.
	
	(Tip: Mit den Bezeichnungen des
	von~\prettyref{exer:global_noether_norm} kann \(\phi\) zunächst auf
	\(B'\) fortgesetzt werden, etwa indem alle \(y_i\) auf \(0\) geschickt
	werden. Sodann kann \(\phi\) weiter auf \(B'[s^{-1}]\) fortgesetzt werden,
	da \(\phi(s) \in L^\units\), schließlich auf \(B[s^{-1}]\)
	nach~\prettyref{exer:extension_to_integral_extension},
	da \(B[s^{-1}]\) ganz über \(B'[s^{-1}]\) ist.)
\end{exercise}

\begin{exercise}
	\label{exer:trivial_jacobson_ideal}
	Sei \(A \subset B\) eine endlich erzeugte Erweiterung von
	Integritätsbereichen. Zeige: Ist das Jacobsonsche Radikal von \(A\) das
	Nullideal, so ist auch das Jacobsonsche Radikal von \(B\) das Nullideal.
	
	(Tip: Sei \(v \in B \setminus \{0\}\). Wir müssen zeigen, daß ein
	maximales Ideal \(\ideal n\) von \(B\) mit \(v \notin \ideal n\) existiert.
	Anwenden von~\prettyref{exer:global_extension_to_ac} auf die Ringerweiterung
	\(A \subset B[v^{-1}]\)
	liefert ein Element \(s \in A \setminus \{0\}\). Sei \(\ideal m\) ein
	maximales Ideal von \(A\) mit \(s \notin \ideal m\). Seien \(k \coloneqq
	A/\ideal m\) und \(L\) ein algebraischer Abschluß von \(k\). Die Projektion
	\(A \surjto k\) setzt sich zu einem Ringhomomorphismus \(\psi\colon B[v^{-1}]
	\to L\) fort. Zeige, daß \(\psi(v) \neq 0\) und daß \(B \cap \ker \psi\)
	ein maximales Ideal von \(B\) ist.)
\end{exercise}

\begin{exercise}
	\label{exer:jacobson_ring}
	Zeige, daß folgende Aussagen über einen kommutativen Ring \(A\) äquivalent
	sind:
	\begin{enumerate}
	\item
		Jedes Primideal in \(A\) ist Schnitt maximaler Ideale in \(A\).
	\item
		Ist \(\phi\colon A \to B\) ein Homomorphismus kommutativer Ringe,
		so ist das Nilradikal in \(\phi(A)\) gleich dem Jacobsonschen
		Radikal.
	\item
		Jedes Primideal \(\ideal p\) in \(A\), welches nicht maximal ist, ist
		Schnitt aller Primideale, welche \(\ideal p\) echt enthalten.
	\end{enumerate}
	
	(Tip: Der schwierige Teil ist es, die erste Aussage aus der dritten zu
	folgern: Angenommen, die dritte Aussage sei wahr, allerdings gebe es ein
	Primideal \(\ideal p\), welches nicht Schnitt maximaler Ideale ist.
	In dem wir von \(A\) nach \(A/\ideal p\) übergehen, können wir annehmen,
	daß \(A\) ein Integritätsbereich mit Jacobsonschen Radikal \(\ideal j
	\neq (0)\) ist. Sei \(f \in \ideal j\) mit \(f \neq 0\). Dann ist \(A_f\)
	nicht der Nullring, besitzt also ein maximales Ideal \(\ideal m\). Für
	seine Kontraktion \(\ideal p \coloneqq A \cap \ideal m\) in \(A\) gilt dann,
	daß \(f \notin \ideal p\) und daß \(\ideal p\) maximal mit dieser
	Eigenschaft ist. Es ist \(\ideal p\) kein maximales Ideal in \(A\), ist aber
	auch nicht gleich dem Schnitt aller Primideale, welche \(\ideal p\) echt
	enthalten.)
	
	Ein kommutativer Ring \(A\), welcher diese drei äquivalenten Aussagen
	erfüllt heißt ein \emph{Jacobsonscher Ring}.
\end{exercise}

\begin{exercise}
	Sei \(A\) ein Jacobsonscher Ring (siehe~\prettyref{exer:jacobson_ring}).
	Sei \(B\) eine kommutative \(A\)-Algebra. Zeige:
	\begin{enumerate}
	\item
		Ist \(B\) ganz über \(A\), so ist \(B\) ein Jacobsonscher Ring.
	\item
		Ist \(B\) endlich erzeugt über \(A\), so ist \(B\) ebenfalls ein
		Jacobsonscher Ring.
		
		(Tip:~\prettyref{exer:trivial_jacobson_ideal}.)
	\end{enumerate}
\end{exercise}

\begin{exercise}
	Sei \(A\) ein kommutativer Ring. Zeige, daß die folgenden beiden
	Aussagen äquivalent sind:
	\begin{enumerate}
	\item
		Es ist \(A\) ein Jacobsonscher Ring.
	\item
		Jede endlich erzeugte kommutative \(A\)-Algebra \(B\), welche ein
		Körper ist, ist endlich über \(A\).
	\end{enumerate}
	
	(Tip: Um aus der ersten die zweite Aussage zu folgern: Reduziere auf den
	Fall, daß \(A\) ein Unterring von \(B\) ist. Nutze
	dann~\prettyref{exer:global_extension_to_ac}. Ist
	\(s \in A \setminus \{0\}\) wie dort, so existiert ein maximales Ideal
	\(\ideal m\) von \(A\) mit \(s \notin \ideal m\), und die
	Projektion \(A \surjto A/\ideal m \eqqcolon K\) setzt sich zu einem
	Ringhomomorphismus \(\psi\colon B \to L\) in einen algebraischen Abschluß
	\(L\) von \(K\) fort. Da \(B\) ein Körper ist, ist \(\psi\) injektiv
	und \(\phi(B)\) ist algebraisch über \(K\) und damit endlich algebraisch
	über \(K\).
	
	Um aus der zweiten die erste Aussage zu folgern: Benutze die dritte
	Charakterisierung aus~\prettyref{exer:jacobson_ring}. Sei \(\ideal p\)
	ein Primideal von \(A\), welches nicht maximal ist, und sei
	\(B \coloneqq A/\ideal p\). Sei \(f \in B \setminus \{0\}\). Dann ist
	\(B[f^{-1}]\) eine endlich erzeugte \(A\)-Algebra. Wäre \(B[f^{-1}]\) ein
	Körper, wäre es endlich über \(B\), also ganz über \(B\), also wäre
	\(B\) ein Körper nach~\prettyref{prop:fields_and_integral_extensions} im
	Widerspruch dazu, daß \(\ideal p\) kein maximales Ideal ist. Also ist
	\(B[f^{-1}]\) kein Körper, besitzt also ein nicht triviales Primideal
	\(\ideal q\), so daß \(\ideal p' \coloneqq B \cap \ideal q\) ein nicht
	triviales Primideal mit \(f \notin \ideal p'\) ist.)
\end{exercise}

\begin{exercise}
	Sei \(A \subset B\) eine Erweiterung lokaler Ringe. Wir sagen, \emph{\(B\)
	dominiere \(A\)}, falls das maximale Ideal \(\ideal m\) von \(A\) im
	maximalen Ideal \(\ideal n\) von \(B\) enthalten ist. (Dies ist äquivalent
	zu \(\ideal m = A \cap \ideal n\).)
	
	Sei \(K\) ein Körper. Sei \(\mathfrak S\) die Menge aller Unterringe von
	\(K\), welche lokal sind. Die Menge \(\mathfrak S\) wird durch die
	Dominanzbeziehung teilweise geordnet. Zeige, daß \(\mathfrak S\) maximale
	Elemente besitzt und daß ein Element \(A \in \mathfrak S\) genau dann
	maximal ist, wenn \(A\) ein Bewertungsring von \(K\) ist.
	
	(Tip:~\prettyref{thm:existence_of_valuation_rings}.)
\end{exercise}

\begin{exercise}
	Sei \(A\) ein Integritätsbereich mit Quotientenkörper \(K\). Zeige, daß
	die folgenden beiden Aussagen äquivalent sind:
	\begin{enumerate}
	\item
		Es ist \(A\) ein Bewertungsring für \(K\).
	\item
		Für je zwei Ideale \(\ideal a, \ideal b\) von \(A\) gilt
		\(\ideal a \subset \ideal b\) oder \(\ideal b \subset \ideal a\).
	\end{enumerate}
	
	Folgere dann: Ist \(A\) ein Bewertungsring und \(\ideal p\) ein Primideal
	von \(A\), so sind auch \(A_{\ideal p}\) und \(A/\ideal p\) Bewertungsringe
	(ihrer jeweiligen Quotientenkörper).
\end{exercise}

\begin{exercise}
	Sei \(A\) ein Bewertungsring mit Quotientenkörper \(K\). Zeige, daß jeder
	Unterring \(B\) von \(K\) mit \(A \subset B\) ein lokaler Ring von \(A\)
	ist, das heißt eine Lokalisierung von \(A\) an einem Primideal \(\ideal p\).
\end{exercise}

\begin{exercise}
	Sei \(A\) ein Bewertungsring mit Quotientenkörper \(K\). Die Gruppe
	\(A^\units\) der Einheiten von \(A\) bildet eine Untergruppe von
	\(K^\units\). Sei \(G \coloneqq \log K^\units/A^\units\) die Faktorgruppe,
	additiv geschreiben.
	
	Zeige, daß durch die Setzung \(\log [x]_{A^\units} \ge
	\log [y]_{A^\units} \iff xy^{-1} \in A\) für \(x, y \in K^\units\) eine
	Ordnung von \(G\) definiert wird, welche mit der Gruppenstruktur
	verträglich ist, das heißt \(\xi \ge \eta \implies \xi + \omega \ge
	\eta + \omega\) für alle \(\xi, \eta, \omega \in G\).
	
	Sei \(\nu\colon K \to G \cup \{\infty\}\) durch \(\nu(x) = \log [x]_{A^\units}\)
	für \(x \in K^\times\) und \(\nu(0) = \infty\) definiert. Zeige, daß
	\(\nu(x + y) \ge \min(\nu(x), \nu(y))\) für alle \(x, y \in K\).
\end{exercise}

\begin{exercise}
	\label{exer:valuation}
	Sei umgekehrt \(G\) eine vollständig geordnete abelsche Gruppe (additiv
	geschrieben).
	Sei \(K\) ein Körper. Eine \emph{Bewertung auf \(K\) mit Werten in \(G\)}
	ist eine Abbildung \(\nu\colon K \to G \cup \{\infty\}\) mit
	\(\nu(xy) = \nu(x) + \nu(y)\) und \(\nu(x + y) \ge \min(\nu(x), \nu(y))\)
	und \(\nu(x) = \infty \iff x = 0\)
	für alle \(x, y \in K\). Zeige, daß die Menge der Elemente \(x \in K\) mit
	\(\nu(x) \ge 0\) ein Bewertungsring von \(K\) ist.
	
	Dieser Ring ist der \emph{Bewertungsring von \(\nu\)} und die Untergruppe
	\(\nu(K^\units)\) von \(G\) ist die \emph{Bewertungsgruppe von \(\nu\)}.
	Im wesentlichen sind also die Konzepte "`Bewertungsring"' und
	"`Bewertung"' äquivalent.
\end{exercise}

\begin{exercise}
	Sei \(G\) eine vollständig geordnete abelsche Gruppe. Eine Untergruppe
	\(H\) von \(G\) heißt \emph{isoliert}, falls aus \(0 \le \beta \le \alpha\)
	mit \(\beta \in G, \alpha \in H\) auch \(\beta \in H\) folgt.
	
	Sei \(A\) ein Bewertungsring mit Bewertungsgruppe \(G\) von \(K\)
	(siehe~\prettyref{exer:valuation}). Sei \(\nu\colon K \to G \cup
	\{\infty\}\) die Bewertung.
	
	Sei \(\ideal p\) ein Primideal von \(A\).
	Zeige, daß \(\nu(A \setminus \ideal p)\) alle Elemente \(x\) mit \(x \ge 0\)
	einer isolierten Untergruppe \(H(\ideal p)\) durchläuft. Zeige weiter, daß
	so eine injektive Zuordnung von der Menge der Primideale von \(A\) in die
	Menge der isolierten Untergruppen von \(G\) definiert wird.
	
	Was sind die Bewertungsgruppen der Bewertungsringe \(A/\ideal p\) und
	\(A_{\ideal p}\) für ein Primideal \(\ideal p\)?
\end{exercise}

\begin{exercise}
	Sei \(G\) eine vollständig geordnete abgeschlossene Gruppe. Sei \(F\) ein
	beliebiger Körper. Mit \(A \coloneqq F[G]\) bezeichnen wir die
	\emph{Gruppenalgebra von \(G\) über \(F\)}: Es besitzt \(A\) als
	\(F\)-Vektorraum eine Basis \((x_\alpha)_{\alpha \in G}\) mit
	\(x_\alpha x_\beta =
	x_{\alpha \beta}\). Zeige, daß \(A\) ein Integritätsbereich ist.
	
	Ist \(u = a_1 x_{\alpha_1} + \dotsb + a_n x_{\alpha_n} \in A\) mit
	\(a_i \in F^\units\) und \(\alpha_1 < \dotsb < \alpha_n\), so definieren wir
	\(\nu_0(u) \coloneqq \alpha_1\). Zeige, daß die Abbildung
	\(\nu_0\colon A \setminus \{0\} \to G\) die Bedingungen
	\(\nu_0(xy) \nu_0(x) + \nu_0(y)\) und \(\nu_0(x + y) \ge \min(\nu(x), \nu(y))\)
	für \(x, y \in A \setminus \{0\}\) mit \(x + y \neq 0\) erfüllt.
	
	Sei \(K\) der Quotientenkörper \(A\). Zeigen Sie, daß \(\nu_0\) eindeutig zu
	einer Bewertung \(\nu\) auf \(K\) mit Bewertungsgruppe \(G\) fortgesetzt
	werden kann.
\end{exercise}


\subsection{Kettenbedingungen}

\begin{exercise}
	Sei \(A\) ein Ring. Sei \(\phi\colon M \to M\) ein Endomorphismus eines \(A\)-Moduls.
	Zeige:
	\begin{enumerate}
	\item
		Ist \(\phi\) surjektiv und \(M\) noethersch, so ist \(\phi\) ein Isomorphismus.
		
		(Tip: Betrachte die Kette der Untermoduln \(\ker(\phi^n)\).)
	\item
		Ist \(\phi\) injektiv und \(M\) artinsch, so ist \(\phi\) ein Isomorphismus.
		
		(Tip: Betrachte die Kette der Untermoduln \(\im(\phi^n)\).)
	\end{enumerate}
\end{exercise}

\begin{exercise}
	Sei \(A\) ein Ring. Sei \(M\) ein \(A\)-Modul. Jede nicht leere Menge endlich erzeugter
	Untermoduln von \(M\) besitze ein maximales Element. Zeige, daß \(M\) noethersch ist.
\end{exercise}

\begin{exercise}
	Sei \(A\) ein Ring. Sei \(M\) ein \(A\)-Modul. Seien \(N_1, N_2\) Untermoduln von \(M\).
	Zeige, daß \(M/(N_1 \cap N_2)\) noethersch ist, wenn \(M/N_1\) und \(M/N_2\) noethersch
	sind.
	
	Formuliere und beweise die entsprechende Aussage für artinsche Moduln.
\end{exercise}

\begin{exercise}
	Sei \(A\) ein Ring. Zeige: Ist \(M\) ein noetherscher \(A\)-Modul, so ist \(A/\ann M\)
	ein noetherscher Ring.
	
	Ist die entsprechende Aussage für artinsche Moduln wahr?
\end{exercise}

\begin{exercise}
	Sei \(A\) ein noetherscher Ring. Zeige, daß \(A\) nur endlich viele minimale Primideale
	besitzt.
	
	(Tip: Angenommen, daß Nilradikal läßt sich nicht als Schnitt endlich vieler Primideale schreiben.
	Dann gibt es ein maximales Wurzelideal \(\ideal a\), welches nicht Schnitt endlich vieler Primideale ist.
	Ein Wurzelideal läßt sich nach~\prettyref{exer:radicals} aber immer als Schnitt (eventuell unendlich vieler) Primideale 
	schreiben. Führe dies zu einem Widerspruch. Folglich existieren endlich viele Primideale \(\ideal p_1, \dotsc,
	\ideal p_n\) mit \(\sqrt{(0)} = \ideal p_1 \cap \dotsb \cap \ideal p_n\). Ist dann \(\ideal q\) ein minimales Primideal, so 
	ist damit \(\ideal p_1 \cap \dotsb \cap \ideal p_n \subset \ideal q\), also \(\ideal p_i \subset \ideal q\) für
	ein \(i\), also \(\ideal p_i = \ideal q\).)
\end{exercise}


\subsection{Noethersche Ringe}

\begin{exercise}
	Sei \(A\) ein nicht noetherscher kommutativer Ring. Zeige, daß die
	Menge \(\mathfrak S\) der nicht endlich erzeugten Ideale von \(A\)
	ein maximales Element besitzt und daß maximale Elemente Primideale sind.
	
	(Tip: Sei \(\ideal a\) ein maximales Element von \(\mathfrak S\). Seien
	\(x, y \in A\) mit \(xy \in \ideal a\), aber \(x, y \notin \ideal a\).
	Zeige, daß ein endlich erzeugtes Ideal \(\ideal a_0 \subset \ideal a\) mit
	\(\ideal a_0 + (x) = \ideal a + (x)\) und \(\ideal a = \ideal a_0 +
	x \cdot (a : x)\) existiert. Da \(\ideal a \subsetneq (\ideal a : x)\), kann
	\(\ideal a\) nicht maximal sein. Widerspruch.)
	
	Folgere, daß ein kommutativer Ring, in dem jedes Primideal endlich erzeugt
	ist, ein noetherscher Ring ist.
\end{exercise}

\begin{exercise}
	\label{exer:nilp_powerseries}
	Sei \(A\) ein noetherscher kommutativer Ring. Zeige, daß eine Potenzreihe
	\(f = \sum\limits_{n = 0}^\infty a_n x^n \in \ps A x\) genau dann nilpotent ist,
	wenn alle \(a_n\) nilpotent sind.
\end{exercise}

\begin{exercise}
	Sei \(\ideal a\) ein irreduzibles Ideal in einem kommutativen Ring \(A\).
	Zeige, daß die folgenden Aussagen äquivalent sind:
	\begin{enumerate}
	\item
		Das Ideal \(\ideal a\) ist ein Primärideal.
	\item
		Für jede mutliplikativ abgeschlossene Teilmenge \(S\) von \(A\)
		existiert ein \(x \in S\) mit \(A \cap S^{-1} \ideal a = (\ideal a : x)\).
	\item
		Für alle \(x \in A\) ist die Kette
		\((\ideal a : x) \subset (\ideal a : x^2) \subset \dotsb\)
		stationär.
	\end{enumerate}
\end{exercise}

\begin{exercise}
	Sei \(A\) ein noetherscher kommutativer Ring. Seien weiter \(B\) eine
	endlich erzeugte kommutative \(A\)-Algebra und \(G\) eine endliche Gruppe
	von \(A\)-Algebrenautomorphismen von \(B\). Zeige, daß
	\(B^G \coloneqq \{y \in B \mid \forall g \in G\colon g(y) = y\}\) eine
	endlich erzeugte \(A\)-Algebra ist.
\end{exercise}

\begin{exercise}
	Sei \(K\) ein endlich erzeugter kommutativer Ring, das heißt \(K\) ist
	endlich erzeugt als \(\set Z\)-Algebra. Zeige: Ist \(K\) ein Körper, so
	ist \(K\) ein endlicher Körper.
	
	(Tip: Ist \(K\) von Charakteristik \(0\), so haben wir \(\set Z \subset
	\set Q \subset K\). Da \(K\) endlich über \(\set Z\) erzeugt ist, ist
	\(K\) dann auch endlich über \(\set Q\) erzeugt, und ist
	nach~\prettyref{cor:weak_hilbert1} damit ein endlich erzeugter
	\(\set Q\)-Vektorraum. Leite aus~\prettyref{prop:intermediate_ring} dann
	einen Widerspruch her.
	
	Also ist \(K\) von positiver Charakteristik \(p\), also endlich erzeugt
	als \(\set Z/(p)\)-Algebra. Nutze wieder~\prettyref{cor:weak_hilbert1}.)
\end{exercise}

\begin{exercise}
	Sei \(A\) ein kommutativer Ring, so daß \(A[x]\) noethersch ist. Ist
	dann auch \(A\) noethersch?
\end{exercise}

\begin{exercise}
	\label{exer:stalks_are_noetherian}
	Sei \(A\) ein kommutativer Ring. Seien die Halme \(A_{\ideal m}\) für
	alle maximalen Ideale \(\ideal m\) von \(A\) noethersch. Sei weiter für
	alle \(x \in A \setminus \{0\}\) die Menge der maximalen Ideale mit
	\(x \in \ideal m\) endlich. Zeige, daß \(A\) noethersch ist.
	
	(Tip: Sei \(\ideal a \neq (0)\) ein Ideal in \(A\). Seien \(\ideal m_1,
	\dotsc, \ideal m_r\) die maximalen Ideale, welche \(\ideal a\) umfassen.
	Wähle \(x_0 \in \ideal a \neq (0)\). Seien \(\ideal m_1, \dotsc,
	\ideal m_{r + s}\) diejenigen maximalen Ideale, welche \(x_0\) enthalten.
	Da \(\ideal m_{r + 1}, \dotsc, \ideal m_{r + s}\) das Ideal \(\ideal a\)
	nicht umfassen, existieren \(x_j \in \ideal a\) mit \(x_j \notin
	\ideal m_{r + j}\) für \(1 \leq j \leq s\). Da die \(A_{\ideal m_i}\)
	noethersch sind, sind die \(A_{\ideal m_i} \ideal a\) endlich erzeugt.
	Damit existieren \(x_{s + 1}, \dotsc, x_t \in \ideal a\), deren Bilder
	in \(A_{\ideal m_i}\) die \(A_{\ideal m_i} \ideal a\) für
	\(1 \leq i \leq r\) erzeugen. Sei \(\ideal a_0 = (x_0, \dotsc, x_t)\).
	Zeige, daß \(A_{\ideal m} \ideal a_0 = A_{\ideal m} \ideal a\) für alle
	maximalen Ideale \(\ideal m\) von \(A\) und folgere
	mit~\prettyref{prop:inj_is_local}, daß \(\ideal a_0 = \ideal a\).)
\end{exercise}

\begin{exercise}
	Sei \(A\) ein kommutativer Ring.
	Sei \(M\) ein noetherscher \(A\)-Modul. Zeige, daß \(M[x]\) (siehe
	\prettyref{exer:poly_over_mod}) ein noetherscher \(A[x]\)-Modul ist.
\end{exercise}

\begin{exercise}
	Sei \(A\) ein kommutativer Ring, so daß der Halm \(A_{\ideal p}\)
	für jedes Primideal ein noetherscher Ring ist. Ist dann \(A\)
	notwendigerweise auch noethersch?
\end{exercise}

\begin{exercise}
	Sei \(A\) ein kommutativer Ring. Sei \(M\) ein noetherscher \(A\)-Modul.
	Zeige, daß jeder Untermodul \(N\) von \(M\) eine Primärzerlegung in
	\(M\) besitzt (siehe~\prettyref{exer:primary_decomp_for_mod}).
	
	(Tip: Imitiere die Beweise von \prettyref{lem:lasker1} und
	\prettyref{lem:lasker2}.)
\end{exercise}

\begin{exercise}
	\label{exer:primes_to_mod}
	Sei \(A\) ein noetherscher kommutativer Ring. Sei \(M\) ein endlich
	erzeugter \(A\)-Modul. Zeige, daß für ein Primideal \(\ideal p\) von \(A\)
	die folgenden Aussagen äquivalent sind:
	\begin{enumerate}
	\item
		Das Primideal \(\ideal p\) ist zu \(0\) in \(M\) assoziiert.
	\item
		Es existiert ein \(x \in M\) mit \(\ann(x) = \ideal p\).
	\item
		Es existiert ein Untermodul \(N\) von \(M\) mit \(N \cong A/\ideal p\).
	\end{enumerate}
	
	Folgere, daß eine Kette von Untermoduln \(0 = M_0 \subset M_1 \dotsb
	M_r = M\) existiert, so daß jeder Quotient \(M_{i + 1}/M_i\) von der
	Form \(A/\ideal p_i\) mit einem Primideal \(\ideal p_i\) in \(A\) ist.
\end{exercise}

\begin{exercise}
	Sei \(A\) ein noetherscher kommutativer Ring. Seien
	\(\ideal a = \bigcap\limits_{i = 1}^r \ideal b_i
	= \bigcap\limits_{j = 1}^s \ideal c_j\) zwei minimale Zerlegungen
	eines Ideals \(\ideal a\) in irreduzible Ideale. Zeige, daß
	\(r = s\) und daß eine Permutation \(\sigma \in \SG_n\) mit
	\(\sqrt{\ideal b_i} = \sqrt{\ideal c_{\sigma(i)}}\) existiert.
	
	(Tip: Zeige, daß für alle \(1 \leq i \leq r\) ein \(j\) mit
	\(\ideal a = \ideal b_1 \cap \dotsb \cap \ideal b_{i - 1} \cap
	\ideal c_j \cap \ideal b_{i + 1} \cap \dotsb \ideal b_r\) existiert.)
	
	Formuliere und beweise eine entsprechende Aussage für Moduln.
\end{exercise}

\begin{exercise}
	\label{exer:groth_group}
	Sei \(A\) ein noetherscher kommutativer Ring. Mit \(\mathfrak F(A)\)
	bezeichnen wir die Menge der Isomorphieklassen \([M]\) endlich erzeugter
	\(A\)-Moduln \(M\). Sei \(C\) die durch \(\mathfrak F(A)\) frei erzeugte
	abelsche Gruppe, das heißt Elemente in \(C\) sind formale (endliche)
	\(\set Z\)-Linearkombinationen von Isomorphieklassen endlich erzeugter
	\(A\)-Moduln. Jeder exakten Sequenz \(0 \to M' \to M \to M'' \to 0\)
	endlich erzeugter \(A\)-Moduln ordnen wir das Element
	\([M] - [M'] - [M''] \in C\) zu. Sei \(D\) die von diesen Elementen für alle
	exakten Sequenzen erzeugte Untergruppe. Die Quotientengruppe
	\(\GrothK(A) \coloneqq C/D\) heißt die \emph{Grothendiecksche Gruppe
	von \(A\)}. Für jeden endlich erzeugten \(A\)-Modul \(A\) bezeichnen wir mit
	\(\gamma(M) = \gamma_A(M)\) das Bild von \([M]\) in \(\GrothK(A)\).
	Zeige:
	\begin{enumerate}
	\item
		Für jede additive Funktion \(\lambda\) mit Werten in einer abelschen
		Gruppe \(G\), welche auf der Klasse der endlich erzeugten \(A\)-Moduln
		definiert ist, existiert genau ein Gruppenhomorphismus
		\(\lambda_0\colon \GrothK(A) \to G\) mit \(\lambda(M) = 
		\lambda_0(\gamma(M))\) für alle endlich erzeugten \(A\)-Moduln \(M\).
	\item
		Zeige, daß \(\GrothK(A)\) von Elementen der Form \(\gamma(A/\ideal p)\)
		erzeugt wird, wobei \(\ideal p\) ein Primideal in \(a\) ist.
		
		(Tip: \prettyref{exer:primes_to_mod}.)
	\item
		Zeige, daß \(\GrothK(A) \cong \set Z\), wenn \(A\) ein Körper oder
		allgemeiner ein Hauptidealbereich ist.
	\item
		Sei \(f^*\colon A \to B\) ein endlicher Homomorphismus
		noetherscher kommutativer Ringe. Zeige, daß ein Gruppenhomomorphismus
		\(f_!\colon \GrothK(B) \to \GrothK(A)\) mit \(f_!(\gamma_B(N))
		= \gamma_A(N^A)\) für alle endlich erzeugten \(B\)-Moduln
		\(N\) existiert.
	
		Sei \(g^*\colon B \to C\) ein weiterer Homomorphismus noetherscher
		kommutativer Ringe. Zeige, daß \((f \circ g)_! = f_! \circ g_!\colon
		\GrothK(C) \to \GrothK(A).\).
	\end{enumerate}
\end{exercise}


\subsection{Artinsche Ringe}

\begin{exercise}
	Sei \(A\) ein noetherscher kommutativer Ring. Sei \((0) = \ideal q_1 \cap \dotsb \cap \ideal q_n\) eine
	minimale Primärzerlegung des Nullideals. Sei \(\ideal p_i \coloneqq \sqrt{\ideal q_i}\). 
	Mit \(\ideal p_i^{(r)}\) bezeichnen wir die \(r\)-te symbolische Potenz von \(\ideal p_i\) wie in
	\prettyref{exer:symbolic_power}. Zeige, daß für alle \(i\) ein \(r_i \in \set N_0\) mit
	\(\ideal p_i^{(r_i)} \subset \ideal q_i\) existiert.
	
	Im Falle, daß \(\ideal q_i\) eine isolierte Primärkomponente ist, ist \(A_{\ideal p_i}\) ein artinscher
	lokaler Ring mit maximalem Ideal \(\ideal m_i\). Wir haben daher \(\ideal m_i^r = 0\) für \(r \gg 0\).
	Daraus folgt \(\ideal q_i = \ideal p_i^{(r)}\) für \(r \gg 0\).
	
	Ist umgekehrt \(\ideal q_i\) eine eingebettete Primärkomponente, so ist \(A_{\ideal p_i}\) nicht artinsch,
	die Potenzen \(\ideal m_i^r\) sind also alle unterschiedlich, so daß ebenfalls die \(\ideal p_i^{(r)}\) unterschiedlich
	sind. Damit kann in der gegebenen Primärzerlegung \(\ideal q_i\) durch irgendeines der \(\ideal p_i\)-primären Ideale
	\(\ideal p_i^{(r)}\) mit \(r \ge r_i\) ersetzt werden, so daß es unendlich viele verschiedene
	minimale Primärzerlegungen des Nullideals gibt, welche sich nur in der \(\ideal p_i\)-Komponente
	unterscheiden.	
\end{exercise}

\begin{exercise}
	Seien \(F\) ein Körper und \(A\) eine endlich erzeugte kommutative \(F\)-Algebra. Zeige, daß die beiden
	folgenden Aussagen äquivalent sind:
	\begin{enumerate}
	\item
		Es ist \(A\) ein artinscher Ring.
	\item
		Es ist \(A\) eine endliche \(F\)-Algebra.
	\end{enumerate}
	
	(Tip: Um aus der ersten die zweite Aussage zu folgern, benutze \prettyref{thm:artin_structure}, um sich
	auf den Fall eines artinschen lokalen Ringes einschränken zu können. Nach \prettyref{cor:weak_hilbert1} ist
	der Restklassenkörper von \(A\) eine endliche Erweiterung von \(F\). Dann nutze aus, daß \(A\) als \(A\)-Modul
	endliche Länge hat.
	
	Um aus der zweiten die erste Aussage zu folgern, nutze aus, daß die Ideale in \(A\) unter anderem \(F\)-Vektorräume
	sind und daher die absteigende Kettenbedingung erfüllen.)
\end{exercise}

\begin{exercise}
	Sei \(A\) ein noetherscher kommutativer Ring. Seien \(\ideal p\) ein Primideal und \(\ideal q\) ein
	\(\ideal p\)-primäres Ideal in \(A\). Betrachte Ketten von Primäridealen von \(\ideal q\) nach \(\ideal p\).
	Zeige, daß die Länge dieser Ketten nach oben beschränkt ist und daß alle maximalen Ketten dieselbe Länge besitzen.
\end{exercise}


\subsection{Diskrete Bewertungsringe und Dedekindsche Bereiche}

\begin{exercise}
	Sei \(A\) ein Dedekindscher Bereich. Sei \(S \subset A\) multiplikativ
	abgeschlossen. Zeige, daß \(S^{-1} A\) entweder ein Dedekindscher Bereich
	oder der Quotientenkörper von \(A\) ist.
	
	Sei jetzt \(S \neq A \setminus \{0\}\). Ist \(\ideal r\) ein nicht
	verschwindendes gebrochenes Ideal in \(A\) bzw.~\(S^{-1} A\) bezeichnen wir
	mit \([\ideal r]\) sein Bild in der Idealklassengruppe von
	\(A\) bzw.\ \(S^{-1} A\). Zeige, daß
	\(\ClassG(A) \to \ClassG(S^{-1} A), [\ideal r] \mapsto [S^{-1} \ideal r]\)
	ein surjektiver Gruppenhomomorphismus ist.
\end{exercise}

\begin{exercise}
	Sei \(A\) ein Dedekindscher Bereich. Ist \(f = a_0 + a_1 x + \dotsb + a_n
	x^n \in A[x]\) ein Polynom über \(A\), so heißt das Ideal \(\content(f)
	\coloneqq (a_0, \dotsb, a_n)\) der \emph{Inhalt von \(f\)}. Zeige das
	Gaußsche Lemma, nämlich daß \(\content(fg) = \content(f) \content(g)\) für
	\(f, g \in A[x]\).
	
	(Tip: Lokalisiere an jedem maximalen Ideal.)
\end{exercise}

\begin{exercise}
	Sei \(A\) ein Bewertungsring, welcher kein Körper ist. Zeige, daß \(A\)
	genau dann noethersch ist, wenn \(A\) ein diskreter Bewertungsring ist.
\end{exercise}

\begin{exercise}
	Sei \((A, \ideal m)\) ein lokaler Integritätsbereich, welcher kein
	Körper ist. Sei \(\ideal m\) ein Hauptideal, und sei \(\bigcap\limits_{n = 0}^\infty
	\ideal m^n = 0\). Zeige, daß \(A\) ein diskreter Bewertungsring ist.
\end{exercise}

\begin{exercise}
	Sei \(A\) ein Dedekindscher Bereich. Sei \(\ideal a\) ein Ideal in \(A\).
	Zeige, jedes Ideal im Ring \(A/\ideal a\) ein Hauptideal ist.
	
	Folgere, daß jedes Ideal in \(A\) von höchstens zwei Elementen erzeugt werden
	kann.
\end{exercise}

\begin{exercise}
	Sei \(A\) ein Dedekindscher Bereich. Seien \(\ideal a, \ideal b, \ideal c\)
	drei Ideale von \(A\). Zeige:
	\begin{enumerate}
	\item
		\(\ideal a \cap (\ideal b + \ideal c) = (\ideal a \cap \ideal b)
		+ (\ideal a \cap \ideal c)\).
	\item
		\(\ideal a + (\ideal b \cap \ideal c) = (\ideal a + \ideal b)
		\cap (\ideal a + \ideal c)\).
	\end{enumerate}
	
	(Tip: Lokalisiere.)
\end{exercise}

\begin{exercise}[Chinesischer Restsatz]
	Sei \(A\) ein Dedekindscher Bereich. Seien \(\ideal a_1, \dotsc, \ideal a_n\)
	Ideale in \(A\) und \(x_1, \dotsc, x_n \in A\). Zeige dann, daß das
	System \(x = x_i \pmod{\ideal a_i}\) von Kongruenzen genau dann eine Lösung \(x \in A\)
	in \(A\) besitzt, wenn \(x_i = x_j \pmod{\ideal a_i + \ideal a_j}\) für alle
	\(i \neq j\).
	
	(Tip: Die Aussage ist äquivalent zur Exaktheit der \(A\)-Modulsequenz
	\[
		A \xrightarrow{\phi} \bigoplus\limits_i A/\ideal a_i
		\xrightarrow{\psi} \bigoplus\limits_{i < j} A/(\ideal a_i + \ideal a_j),
	\]
	wobei die \(i\)-te Komponente von \(\phi(a)\) durch \(x + \ideal a_i\) und
	die \((i, j)\)-te Komponente von \(\psi(x_1 + \ideal a_1, \dotsc,
	x_n + \ideal a_n)\) durch \(x_i - x_j + \ideal a_i + \ideal a_j\) gegeben ist.
	
	Um zu zeigen, daß diese Sequenz exakt ist, reicht es zu zeigen, daß ihre
	Lokalisierung an jedem Primideal \(\ideal p \neq (0)\) exakt ist. Mit anderen
	Worten können wir also annehmen, daß \(A\) ein diskreter Bewertungsring ist.
	Dann ist die Aussage einfach.)
\end{exercise}


\subsection{Vervollständigungen}

\begin{exercise}
	\label{exer:compl_not_right_exact}
	Sei \(p\) eine Primzahl. Sei \(\alpha_n\colon \set Z/(p) \injto \set Z/(p^n), [x]_{(p)} \mapsto [p^{n - 1} x]_{(p^n)}\).
	Seien \(A \coloneqq \bigoplus\limits_{n = 1}^\infty \set Z/(p)\) und \(B \coloneqq \bigoplus\limits_{n = 1}^\infty \set Z/(p^n)\).
	Sei \(\alpha\colon A \injto B, (\xi_1, \xi_2, \dotsc) \mapsto (\alpha_1(\xi_1), \alpha_2(\xi_2), \dotsc)\).
	\\
	Zeige, daß die \(p\)-adische Vervollständigung von \(A\) wieder \(A\) ist.
	Zeige weiter, daß die Vervollständigung von \(A\) bezüglich der von der \(p\)-adischen Topologie auf \(B\) induzierten Topologie
	durch \(\prod\limits_{n = 1}^\infty \set Z/(p)\) gegeben ist.
	
	Folgere, daß die \(p\)-adische Vervollständigung kein rechtsexakter Funktor auf der Kategorie aller \(\set Z\)-Moduln ist.
\end{exercise}

\begin{exercise}
	In den Bezeichnungen von \prettyref{exer:compl_not_right_exact} sei \(A_n \coloneqq \alpha^{-1} (B p^n)\). Betrachte die
	exakten Sequenzen
	\[
		0 \to A_n \to A \to A/A_n \to 0.
	\]
	Zeige, daß \(\varprojlim\) kein rechtsexakter Funktor ist, und berechne \({\varprojlim\limits_n}^1 A_n\).
\end{exercise}

\begin{exercise}
	Sei \(\ideal a\) ein Ideal in einem noetherschen kommutativen Ring. Sei \(M\) ein endlich erzeugter \(A\)-Modul.
	Zeige mit \prettyref{thm:krull} und \prettyref{exer:trivial_stalks_in_closed_subset}, daß
	\[
		\bigcap\limits_{n = 1}^\infty \ideal a^n M = \bigcap\limits_{\ideal m \supset \ideal a} \ker(M \to M_{\ideal m}),
	\]
	wobei \(\ideal m\) über alle maximalen Ideale läuft, welche \(\ideal a\) enthalten.
\end{exercise}

\begin{exercise}
	Sei \(\ideal a\) ein Ideal in einem noetherschen kommutativen Ring \(A\). Für jedes \(x \in A\) bezeichnen wir mit
	\(\hat x\) das Bild unter dem kanonischen Homomorphismus \(A \to \hat A = \hat A_{\ideal a}\). Zeige, daß \(\hat x\) ein
	reguläres Element in \(\hat A\) ist, wenn \(x\) ein reguläres Element in \(A\) ist.
	
	Folgt daraus, daß \(\hat A\) ein Integritätsbereich ist, wenn \(A\) ein Integritätsbereich ist?
	
	(Tip: Nutze die Exaktheit der Vervollständigung der Sequenz \(0 \to A \xrightarrow{x \cdot} A\) aus.)
\end{exercise}

\begin{exercise}
	Seien \(\ideal a, \ideal b\) zwei Ideale in einem noetherschen kommutativen Ring \(A\). Zeige, daß für einen endlich erzeugten
	\(A\)-Modul ein Isomorphismus
	\[
		\widehat{(\hat M_{\ideal a})}_{\hat{\ideal b}_{\ideal a}} \cong \hat M_{\ideal a + \ideal b}
	\]
	existiert.
	
	(Tip: Betrachte die \(\ideal a\)-adische Vervollständigung der kurzen exakten Sequenz
	\[
		0 \to \ideal b^m M \to M \to M/\ideal b^m M \to 0
	\]
	und verwende \prettyref{prop:compl_as_scalar_ext}. Dann benutze die Isomorphismen
	\[
		\varprojlim\limits_m (\varprojlim\limits_n M/(\ideal a^n M + \ideal b^m M)) \cong
		\varprojlim\limits_n M/(\ideal a^n M + \ideal b^n M)
	\]
	und die Inklusionen \((\ideal a + \ideal b)^{2n} \subset \ideal a^n + \ideal b^n
	\subset (\ideal a + \ideal b)^n\).
\end{exercise}

\begin{exercise}
	Sei \(\ideal a\) ein Ideal in einem Ring \(A\). Zeige, daß \(\ideal a\) genau dann
	im Jacobsonschen Radikal enthalten ist, wenn jedes maximale Ideal von \(A\) abgeschlossen
	bezüglich der \(\ideal a\)-adischen Topologie ist.
\end{exercise}

\begin{exercise}
	\label{exer:hensel}
	Sei \((A, \ideal m, F)\) ein lokaler Ring, welcher \(\ideal m\)-adisch vollständig ist. Für ein Polynom
	\(f \in A[x]\) bezeichne \(\bar f \in F[x]\) die Reduktion modulo \(\ideal m\).
	
	Zeige das "`Henselsche Lemma"':
	Ist \(f \in A[x]\) ein normiertes Polynom mit \(\bar f = \tilde g \tilde h\) für teilerfremde normierte Polynome \(\tilde g,
	\tilde h \in F[x]\), so existieren normierte Polynome \(g, h \in A[x]\) mit \(f = g h\) und \(\bar g = \tilde g\) und
	\(\bar h = \tilde h\).
	
	(Tip: Nimm an, daß induktiv Polynome \(g_k, h_k \in A[x]\) mit \(g_k h_k - f \in A[x] \ideal m^k\) konstruiert worden sind.
	Dann folgere aus der Teilerfremdheit von \(\tilde g, \tilde h\), daß für jede Zahl \(1 \leq p \leq n\) Polynome
	\(\tilde a_p, \tilde b_p \in F[x]\) mit \(x^p = \tilde a_p \bar g_k + \tilde b_p \bar h_k\) existieren.
	Schließlich folgere aus der Vollständigkeit von \(A\), daß die Folgen \(g_k, h_k\) gegen die gewünschten Polynome \(g, h \in A[x]\)
	konvergieren.)
\end{exercise}

\begin{exercise}
	\begin{enumerate}
	\item
		Sei \((A, \ideal m, F)\) ein lokaler Ring, welcher \(\ideal m\)-adisch vollständig ist. Sei \(f \in A[x]\).
		Sei weiter \(\tilde a \in F\) eine einfache Nullstelle der Reduktion \(\bar f \in F[x]\) von \(f\) modulo \(\ideal m\).
		Zeige, daß \(f\) eine einfache Nullstelle \(a \in A\) mit \(\bar a = \tilde a\) besitzt.
	\item
		Zeige, daß \(2\) eine Quadratwurzel in den \(7\)-adischen ganzen Zahlen \(\set Z_7\) besitzt.
	\item
		Seien \(K\) ein Körper und \(f \in K[x, y]\). Besitze \(f(0, y) \in K[y]\) eine einfache Nullstelle in \(a_0 \in K\).
		Zeige, daß eine formale Potenzreihe \(g = \sum\limits_{n = 0}^\infty a_n x^n \in \ps K x\) mit \(f(x, g(x)) = 0\) existiert.
	\end{enumerate}
\end{exercise}

\begin{exercise}
	Zeige, daß die Umkehrung von \prettyref{thm:compl_is_noeth} falsch ist, selbst unter der Annahme, daß \(A\) lokal ist und
	\(\hat A\) ein endlich erzeugter \(A\)-Modul.
	
	(Tip: Nimm als \(A\) die Lokalisierung des Ringes aller \(\Cont^\infty\)-Funktionen auf \(\set R\) nach dem maximalen Ideal
	der bei \(0\) verschwindenden Funktionen. Benutze den Borelschen Satz, daß jede Potenzreihe über \(\set R\) als Taylorreihe einer
	\(\Cont^\infty\)-Funktion auf \(\set R\) realisiert werden kann.
\end{exercise}


\subsection{Dimensionstheorie}

\begin{exercise}
	Sei \(K\) ein algebraisch abgeschlossener Körper. Sei \(f \in K[x_1, \dotsc, x_n]\) ein irreduzibles Polynom.
	Wir sagen \(P = (a_1, \dotsc, a_n) \in K^n\) sei \emph{nicht singulär}, falls \((\frac{\partial f}{\partial x_1}(a_1),
	\dotsc, \frac{\partial f}{\partial x_n}(a_n)) \in K^n\) nicht der Nullvektor ist.
	
	Sei \(A = K[x_1, \dotsc, x_n]/(f)\). 
	Sei \(\ideal m\) das Bild des Ideals \((x_1 - a_1, \dotsc, x_n - a_n)\) in \(A\). Zeige, daß \(P\)
	genau dann nicht singulär ist, wenn \(A_{\ideal m}\) ein regulärer lokaler Ring ist.
	
	(Tip: Nach \prettyref{cor:dim_of_reg_quot} ist \(\dim A_{\ideal m} = n - 1\). Weiter ist
	\(\ideal m/\ideal m^2 = (x_1, \dotsc, x_n)/(x_1, \dotsc, x_n)^2 + (f)\). Dieser \(A/\ideal m\)-Vektorraum
	hat Dimension \(n - 1\) genau dann, wenn \(f \notin (x_1, \dotsc, x_n)^2\).)
\end{exercise}

\begin{exercise}
	Sei \((A, \ideal m)\) ein vollständiger lokaler Ring. Sei \(K \subset A\) ein Körper, welcher isomorph auf
	\(A/\ideal m\) abgebildet wird. Sei \((x_1, \dotsc, x_d)\) ein Parametersystem für \(A\). Zeige, daß der
	Homomorphismus
	\(\ps K{t_1, \dotsc, t_d} \to A, t_i \mapsto x_i\) injektiv ist und daß \(A\) ein endlich erzeugter Modul über
	\(\ps K{t_1, \dotsc, t_d}\) wird.
	
	(Tip: \prettyref{prop:weighted_mod_is_ft}.)
\end{exercise}

\begin{exercise}
	Sei \(K\) ein Körper. Sei \(A \coloneqq K[x_1, x_2, \dotsc]\) der Polynomring über \(K\) in unendlich
	vielen Variablen. Sei \(0 < m_1, m_2, \dotsc\) eine aufsteigende Folge natürlicher Zahlen mit \(m_{i + 1} - m_i
	> m_{i} - m_{i - 1}\) für alle \(i > 1\). Sei \(\ideal p_i \coloneqq (x_{m_i + 1}, \dotsc, x_{m_{i + 1}})\) für
	alle \(i\). Sei schließlich \(S \coloneqq A \setminus \bigcup\limits_{i = 1}^\infty \ideal p_i\).
	Zeige dann folgende Behauptungen:
	\begin{enumerate}
	\item
		Die Ideale \(\ideal p_i\) sind Primideale, und \(S\) ist damit multiplikativ abgeschlossen in \(A\).
	\item
		Der Ring \(S^{-1} A\) ist noethersch. (\prettyref{exer:stalks_are_noetherian}.)
	\item
		Die Höhe von \(S^{-1} \ideal p_i\) ist \(m_{i + 1} - m_i\).
	\item
		Es ist \(\dim S^{-1} A = \infty\).
	\end{enumerate}
	Es existieren damit Noethersche Integritätsbereiche unendlicher Dimension.
\end{exercise}

\begin{exercise}
	Formuliere \prettyref{thm:hilbert_serre} in Termen der Grothendieckschen Gruppe \(\GrothK(A_0)\) (\prettyref{exer:groth_group}). 
\end{exercise}

\begin{exercise}
	\label{exer:poly_dim}
	Sei \(A\) ein kommutativer Ring. Zeige, daß
	\[
		1 + \dim A \le \dim A[x] \le 1 + 2 \dim A.
	\]
	
	(Tip: Sei \(\phi\colon A \to A[x]\) die Einbettung. Sei \(\ideal p\) ein Primideal in \(A\). Die Menge der Primideale
	\(\ideal q\) in \(A[x]\) mit \(A \cap \ideal q = \ideal p\) steht in kanonischer bijektiver Korrespondenz zur Menge der
	Primideale von \(F[X]\), wobei \(F = A_{\ideal p}/A_{\ideal p} \ideal p\). Weiter ist \(\dim F[X] = 1\). Dann
	\prettyref{exer:primary_in_poly}.)
\end{exercise}

\begin{exercise}
	Sei \(A\) ein noetherscher kommutativer Ring. Zeige, daß
	\[
		\dim A[x] = 1 + \dim A
	\]
	und damit \(\dim A[x_1, \dotsc, x_n] = n + \dim A\).
	
	(Tip: Sei \(\ideal p\) ein Primideal der Höhe \(m\) in \(A\). Dann existieren \(a_1, \dotsc, a_m \in \ideal p\), so daß
	\(\ideal p\) minimales Primideal zu \(\ideal a \coloneqq (a_1, \dotsc, a_m)\) ist. Nach \prettyref{exer:primary_in_poly}
	ist \(\ideal p[x]\)	ein minimales Primideal zu \(\ideal a[x]\) und damit \(\height \ideal p[x] \leq m\).
	
	Auf der anderen Seite induziert jede Primidealkette \(\ideal p_0 \subsetneq \ideal p_1 \subsetneq \dotsb \subsetneq
	\ideal p_m = \ideal p\) eine Primidealkette \(\ideal p_0[x] \subsetneq \ideal p_1[x] \subsetneq \dotsb \subsetneq
	\ideal p_m[x] = \ideal p[x]\). Damit ist also \(\height \ideal p[x] \ge m\).
	
	Schließlich nutze das Argument von \prettyref{exer:poly_dim}.)
\end{exercise}


%\subsection{Tor, Ext und Moduln über Dedekindschen Bereichen}



\section{Grundlagen aus den Anfängervorlesungen}

\subsection{Mengen}

\begin{definition}
	Sei \(X\) eine Menge. Eine Relation \(\sim\) auf \(X\) heißt
	\begin{enumerate}
	\item
		\emph{transitiv}, falls aus \(x \sim y\) und \(y \sim z\) auch \(x \sim z\) für \(x, y, z \in X\) folgt,
	\item
		\emph{reflexiv}, falls \(x \sim x\) für \(x \in X\) gilt und
	\item
		\emph{antisymmetrisch}, falls auch \(x \sim y\) und \(y \sim x\) schon \(x = y\) für \(x, y \in X\) folgt.
	\end{enumerate}
\end{definition}

\begin{definition}
	Eine \emph{Halbordnung \(\le\) auf einer Menge \(X\)} ist eine transitive, reflexive und antisymmetrische Relation auf \(X\).
	Eine Menge \((X, \le)\) zusammen mit einer Halbordnung heißt \emph{halbgeordnete Menge}.
\end{definition}

Anstelle von einer Halbordnung wird auch der Ausdruck "`teilweise Ordnung"' benutzt oder auch einfach nur "`Ordnung"'.

\begin{example}
	Sei \((X, \le)\) eine halbgeordnete Menge. Ist \(Z \subset X\) eine Teilmenge, so definiert die Einschränkung von
	\(\le\) auf \(Z\) in kanonischer Weise eine Halbordnung auf \(Z\).
\end{example}

\begin{example}
	Sei \(Y\) eine Menge. Sei \(\mathfrak X\) ein System von Teilmengen von \(Y\), also eine Teilmenge der Potenzmenge von
	\(Y\). Dann ist die Inklusionsrelation \(\subset\) eine Halbordnung auf \(\mathfrak X\). Damit ist jedes System \(\mathfrak X\)
	von Teilmengen in natürlicher Weise eine geordnete Menge.
\end{example}

\begin{definition}
	Sei \((X, \le)\) eine halbgeordnete Menge.
	\begin{enumerate}
	\item
		Ein \emph{größtes Element \(x \in X\)} ist ein Element, so daß \(y \le x\) für alle \(y \in X\).
	\item
		Ein \emph{maximales Element \(x \in X\)} ist ein Element, so daß aus \(x \le y\) schon \(x = y\) für alle \(y \in X\) folgt.
	\item
		Eine \emph{obere Schranke \(x \in X\) einer Teilmenge \(Z \subset X\)} ist ein Element mit \(z \le x\) für alle \(z \in Z\).
	\end{enumerate}
\end{definition}

\begin{remark}
	Größte Elemente sind immer eindeutig und auch maximal. Existiert ein größtes Element, so gibt es keine weiteren maximalen
	Elemente.
	
	Ein größtes Element ist dasselbe wie eine obere Schranke der gesamten Menge.
\end{remark}

\begin{definition}
	Sei \((X, \le)\) eine halbgeordnete Menge.
	\begin{enumerate}
	\item
		Die Menge \(X\) heißt \emph{vollständig geordnet}, falls \(x \le y\) oder \(y \le x\) für alle \(x, y \in X\) gilt.
	\item
		Eine \emph{Kette \(Z\) in \(X\)} ist eine Teilmenge \(Z \subset X\), welche mit der induzierten Halbordnung vollständig
		geordnet ist, das heißt \(x \le y\) oder \(y \le x\) für alle \(x, y \in Z\).
	\end{enumerate}
\end{definition}

\begin{theorem}[Zornsches Lemma]
	Sei \(X\) eine halbgeordnete Menge. Jede Kette in \(X\) besitze eine obere Schranke in \(X\). Dann besitzt \(X\) ein
	maximales Element.
\end{theorem}

\subsection{Topologie}

\begin{definition}
		Sei \(X\) eine Menge. Eine \emph{Topologie auf \(X\)} ist eine Menge von
		Teilmengen von \(X\), den \emph{offenen Mengen} der Topologie, so daß
		\begin{enumerate}
		\item
			endliche Schnitte offener Mengen in \(X\) wieder offen sind (damit ist insbesondere die ganze Menge \(X\) als
			leerer Schnitt offen) und
		\item
			beliebige Vereinigungen offener Mengen in \(X\) wieder offen sind (damit ist insbesondere die leere Menge
			\(\emptyset\) als leere Vereinigung offen).
		\end{enumerate}
		Eine Teilmenge heißt \emph{abgeschlossene Menge} der Topologie, falls sie Komplement einer 
		offenen Menge in \(X\) ist.
\end{definition}

\begin{definition}
		Ein \emph{topologischer Raum} ist eine Menge \(X\) zusammen mit einer Topologie.
		
		Ist \(X\) ein topologischer Raum und ist \(x \in X\), so heißt eine offene Menge \(U\) von \(X\) mit \(x \in U\)
		eine \emph{offene Umgebung von \(x\) in \(X\)}.
		
		Eine \emph{Umgebung von \(x\) in \(X\)} ist eine Teilmenge \(U\) von \(X\), so daß eine offene Umgebung
		\(U'\) von \(x\) in \(X\) mit \(U' \subset U\) existiert. 
\end{definition}

\begin{example}
	Sei \(X\) eine Menge. Die Potenzmenge von \(X\) ist eine Topologie auf \(X\), die \emph{diskrete Topologie
	auf \(X\)}.
\end{example}

\begin{example}
	Sei \(X\) ein topologischer Raum. Sei \(Y\) eine Teilmenge von \(X\). Die Menge der Schnitte von \(Y\) mit den
	offenen Teilmengen von \(X\) ist eine Topologie auf \(Y\). Diese Topologie heißt die \emph{Teilraumtopologie
	von \(Y\) in \(X\)}.
	
	In Zukunft versehen wir \(Y\) immer mit dieser Topologie.
\end{example}

\begin{definition}
	Sei \(X\) eine Menge. Eine Menge \(\mathfrak U\) von Teilmengen von \(X\) heißt
	\emph{Basis einer Topologie auf \(X\)}, falls \(\mathfrak U\) abgeschlossen unter endlichen
	Schnitten ist.
\end{definition}

\begin{proposition}
	Sei \(X\) eine Menge. Ist \(\mathfrak U\) die Basis einer Topologie auf \(X\), so
	ist die Menge aller beliebigen Vereinigungen von Teilmengen in \(\mathfrak U\) in \(X\)
	eine Topologie auf \(X\). Diese Topologie heißt die durch \(\mathfrak U\) erzeugte Topologie.
    \qed
\end{proposition}

\begin{example}
	Seien \(X, Y\) zwei topologische Räumen. Dann ist die Menge aller Teilmengen der Form
	\(U \times V\) von \(X \times Y\), wobei \(U\) offen in \(X\) und \(V\) offen in \(Y\) ist,
	eine Basis einer Topologie auf \(X \times Y\). Die davon erzeugte Topologie auf \(X \times Y\)
	heißt die \emph{Produkttopologie}.
	
	In Zukunft versehen wir \(X \times Y\) immer mit dieser Topologie.
\end{example}

In Verallgemeinerung des vorherigen Beispiels wird definiert:
\begin{example}
	Sei \((X_i)_{i \in I}\) eine Familie topologischer. Dann ist die Menge aller Teilmengen von \(X \coloneqq
	\prod_{i \in I} X_i\) der Form
	\(\prod_{i \in I} U_i\) mit \(U_i \subset X_i\) offen für alle \(i \in I\) und \(U_i = X_i\) für fast alle
	\(i \in I\) die Basis einer Topologie auf \(X \times Y\). Die davon erzeugte Topologie auf \(X\) heißt die
	\emph{Produkttopologie}.
	
	In Zukunft versehen wir \(X\) immer mit dieser Topologie.
\end{example}

\begin{definition}
	Sei \(X\) ein topologischer Raum. Sei \(x \in X\). Eine Familie \(\mathfrak U\) von Teilmengen \(U\) von \(X\)
	mit \(U \ni x\) heißt \emph{Umgebungsbasis von \(x\)}, falls für alle Umgebungen \(V\) von \(x\) in \(X\)
	ein \(U \in \mathfrak U\) mit \(U \subset V\) existiert.
	
	Wir sagen, \(X\) \emph{erfülle das erste Abzählbarkeitsaxiom}, falls jeder Punkt von \(X\) eine abzählbare
	Umgebungsbasis besitzt.
\end{definition}

\begin{definition}
	Ein topologischer Raum \(X\) heißt \emph{hausdorffsch}, falls die Diagonale
	\(\{(x, x) \in X \times X \mid x \in X\}\) in \(X \times X\) abgeschlossen ist.
\end{definition}

\begin{definition}
	Sei \(X\) ein topologischer Raum. Sei \((x_n)_{n \in \set N_0}\) eine Folge in \(X\). Wir sagen, daß
	ein Element \(x \in X\) ein \emph{Grenzwert von \((x_n)\) ist}, geschrieben \(\varinjlim_{n \to \infty}\limits x_n = x\),
	falls für jede Umgebung \(U\) von \(x\) in \(X\) gilt, daß \(x_n \in U\) für \(n \gg 0\).
\end{definition}
Offensichtlich reicht es aus, sich auf Umgebungen einer Umgebungsbasis von \(x\) zu beschränken.

\begin{proposition}
	Sei \(X\) ein Hausdorffraum. Dann besitzt eine Folge höchstens einen Grenzwert.
	\qed
\end{proposition}

\begin{definition}
	Seien \(X, Y\) zwei topologische Räume. Eine Abbildung \(f\colon X \to Y\) heißt \emph{stetig}, falls
	für alle offenen Teilmengen \(V\) von \(Y\) das Urbild \(f^{-1}(U)\) in \(X\) offen ist.
	
	Eine bijektive stetige Abbildung zwischen topologischen Räumen, deren Umkehrung auch stetig ist, heißt
	\emph{Homöomorphismus}.
\end{definition}

\begin{example}
	Sei \(X\) eine Menge. Sei \(Y\) ein topologischer Raum. Versehen wir \(X\) mit der
	diskreten Topologie, so ist jede Abbildung \(f\colon X \to Y\) stetig.
\end{example}

\begin{example}
	Sei \(Y\) ein topologischer Raum. Sei \(Z\) eine Teilmenge von \(Y\). Dann ist eine Abbildung \(f\colon X \to Z\)
	von einem weiteren topologischen Raum \(X\) genau dann stetig, wenn \(f\) als Abbildung nach \(Y\) stetig ist.
\end{example}

\begin{definition}
	Sei \(X\) ein topologischer Raum. Sei \(p\colon X \to Y\) eine surjektive Abbildung in eine Menge \(Y\). Die Menge
	derjenigen Teilmengen \(V\) von \(Y\), so daß \(f^{-1}(V)\) offen in \(X\) ist, ist eine Topologie auf \(Y\), die
	\emph{Quotiententopologie bezüglich \(p\)}.
\end{definition}

\begin{example}
	Sei \(X\) ein topologischer Raum. Sei \(p\colon X \to Y\) eine surjektive Abbildung in eine Menge \(Y\), die wir
	diesbezüglich mit der Quotiententopologie versehen. Dann ist eine Abbildung \(f\colon Y \to Z\) in
	einen weiteren topologischen Raum \(Z\) genau dann stetig, wenn \(f \circ p\colon X \to Z\) stetig ist.
\end{example}

\begin{proposition}
	Ist \(A\) eine beliebige Teilmenge eines topologischen Raumes \(X\), so existiert eine kleinste Teilmenge
	\(\overline A \supset A\) von \(X\), welche abgeschlossen ist, der \emph{topologische Abschluß von \(A\)}.
	\qed
\end{proposition}

\begin{definition}
	Eine Teilmenge \(A\) eines topologischen Raumes \(X\) heißt \emph{dicht in \(X\)}, falls für den topologischen
	Abschluß \(\overline A = X\) gilt.
\end{definition}


\section{Grundlagen aus der Einführung in die Algebra}

%\subsection{Gruppen}

\subsection{Körper}

\begin{definition}
	Die \emph{Charakteristik \(\chr K\)} eines Körpers \(K\) ist die
	kleinste positive Zahl \(p \in \set N\) mit \(p \cdot 1 = 0 \in K\) oder
	\(0\), wenn kein solches \(p\) existiert.
\end{definition}

\begin{example}
	Der Körper \(\set Q\) der rationalen Zahlen ist ein Körper der
	Charakteristik \(0\).
\end{example}

\begin{proposition}
	Die Charakteristik eines Körpers ist entweder \(0\) oder eine Primzahl.
\end{proposition}

\subsection{Algebraische Erweiterungen}

\begin{definition}
	Ein Körper \(K\) heißt \emph{algebraisch abgeschlossen}, falls
	jedes normierte Polynom \(f \in K[x]\) vollständig in Linearfaktoren zerfällt,
	falls also \(x_1, \dotsc, x_n \in K\) mit \(f = (x - x_1) \dotsm (x - x_n)\) existieren.
\end{definition}

Diese Definition ist gleichbedeutend damit, daß jedes Polynom über \(K\) mindestens eine Nullstelle besitzt.

\begin{example}<+->
	Der Körper \(\set C\) der komplexen Zahlen ist algebraisch abgeschlossen.
\end{example}

\begin{definition}
	Sei \(K \subset L\) eine Körpererweiterung.
	\begin{enumerate}
	\item
		Ein Element \(x \in L\) heißt \emph{algebraisch über \(K\)}, falls es
		einer Gleichung der Form \(x^n + a_1 x^{n + 1} + \dotsb + a_n\) mit \(a_i \in K\)
		genügt.
	\item
		Ein Element \(x \in L\) heißt \emph{transzendent über \(K\)}, falls es
		nicht algebraisch ist.
	\item
		Die Körpererweiterung \(K \subset L\) heißt \emph{algebraisch}, falls jedes
		Element von \(L\) algebraisch über \(K\) ist.
	\end{enumerate}
\end{definition}

\begin{definition}
	Eine Körpererweiterung \(K \subset L\) heißt \emph{endlich}, falls
	\(L\) als \(K\)-Vektorraum endlich-dimensional ist. Die Zahl
	\([L : K] \coloneqq \dim_K L\) heißt der \emph{Grad der Körpererweiterung}.
\end{definition}

\begin{example}
	Jede endliche Körpererweiterung ist algebraisch.
\end{example}

\begin{example}
	Sei \(K \subset L\) eine Körpererweiterung. Für ein \(x \in L\) bezeichne
	\(K(x)\) den kleinsten Zwischenkörper von \(K \subset L\), welcher \(x\) enthält.
	Dann ist \(K \subset K(x)\) genau dann eine endliche Körpererweiterung, wenn
	\(x\) algebraisch über \(K\) ist.
\end{example}

\begin{definition}
	Sei \(K \subset L\) eine Körpererweiterung. Ist \(x \in L\) algebraisch
	über \(K\), so heißt das normierte Polynom \(m \in K[x]\) minimalen Grades
	mit \(m(x) = 0 \in L\) das \emph{Minimalpolynom von \(x\) über \(K\)}.
\end{definition}

\begin{definition}
	Sei \(K \subset L\) eine Körpererweiterung. Zerfällt ein Polynom \(f \in K[x]\)
	über \(L\) vollständig in Linearfaktoren, so heißt \(L\) ein \emph{Zerfällungskörper von \(K\)}.
\end{definition}

\begin{theorem}
	Sei \(K\) ein Körper. Dann existiert eine
	algebraische Körpererweiterung \(K \subset L\), so
	daß \(L\) ein algebraisch abgeschlossener Körper ist. Insbesondere ist
	\(L\) ein Zerfällungskörper für jedes Polynom über \(K\).
\end{theorem}

Der Körper \(L\) heißt ein \emph{algebraischer Abschluß von \(K\)}.

\begin{theorem}
	Sei \(K \subset K'\) eine algebraische Körpererweiterung. Sei \(\phi\colon K \to L\)
	ein Körperhomomorphismus in einen algebraisch abgeschlossenen Körper \(L\).
	\begin{enumerate}
	\item
		Der Homomorphismus \(\phi\) läßt sich zu einem Körperhomomorphismus \(K' \to L\) fortsetzen.
	\item
		Die Anzahl \([K' : K]_\sep\) der möglichen Fortsetzungen hängt nur von der Körpererweiterung
		\(K \subset K'\) ab und heißt ihr \emph{Separabilitätsgrad}.
	\item
		Es gilt \([K' : K]_\sep \le [K' : K]\). Insbesondere ist der Separabilitsgrad einer endlichen
		Körpererweiterung endlich.
	\end{enumerate}
\end{theorem}

\begin{definition}
	Eine endliche Körpererweiterung \(K \subset K'\) heißt \emph{separabel}, falls
	\([K' : K]_\sep = [K' : K]\).
\end{definition}

\begin{proposition}
	Ist \(K\) ein Körper der Charakteristik Null, so ist jede endliche Körpererweiterung \(K \subset K'\)
	separabel.
\end{proposition}

\subsection{Galoissche Theorie}

\begin{definition}
	Sei \(K \subset L\) eine endliche Körpererweiterung. Ist \(x \in L\), so bezeichnen wir
	mit \(\phi_x\colon L \to L, y \mapsto x y\) den durch Multiplikation mit \(x\) induzierten
	Endomorphismus des \(K\)-Vektorraumes \(L\).
	
	Seine Spur \(\tr_K \phi_x\) heißt die \emph{Spur \(\tr_{L/K}(x)\) von \(x\) in der Körpererweiterung
	\(K \subset L\)}.
\end{definition}

\begin{proposition}
	Sei \(K \subset L\) eine endliche Körpererweiterung.
	\begin{enumerate}
	\item
		Die Spur \(\tr_{L/K}\colon L \to K\) ist eine \(K\)-lineare Abbildung.
	\item
		Sei \(x \in L\) mit Minimalpolynom \(x^n + a_1 x^{n - 1} + \dotsb + a_n\) über \(K\).
		Dann ist \(\tr_{L/K} (x) = - [L : K(x)] \cdot a_1\), insbesondere also ein Vielfaches eines
		Koeffizienten des Minimalpolynomes.
	\end{enumerate}
\end{proposition}

\begin{theorem}
	Sei \(K \subset L\) eine separable endliche Körpererweiterung. Dann ist die Bilinearform
	\[
		L \times L \to K, (x, y) \mapsto \tr_{L/K} (xy)
	\]
	auf \(L\) über \(K\) nicht ausgeartet.
\end{theorem}

\begin{corollary}
	Ist \(K \subset L\) eine separable endliche Körpererweiterung, und ist
	\((x_1, \dotsc, x_n)\) eine Basis von \(L\) über \(K\), so existiert genau eine Basis
	\((y_1, \dotsc, y_n)\) von \(L\) über \(K\) mit \(\tr_{L/K} (x_i y_j) = \kron_{ij}\).
\end{corollary}

\subsection{Transzendente Erweiterungen}

\begin{definition}
	Sei \(K \subset L\) eine Körpererweiterung. Eine Familie \((x_i)_{i \in I}\)
	von	Elementen in \(L\) heißt \emph{algebraisch unabhängig}, wenn 
	für je endlich
	viele Elemente \(x_{i_1}, \dotsc, x_{i_n}\) der Folge und \(f \in K[t_1, \dotsc, t_n]\)
	mit \(f(x_{i_1}, \dotsc, x_{i_n}) = 0\) schon \(f = 0\) folgt.
\end{definition}

\begin{remark}
	Sei \(K \subset L\) eine Körpererweiterung. Eine Familie \((x_i)_{i \in I}\)
	von Elementen in \(L\) ist also genau dann algebraisch unabhängig, wenn
	der Ringhomomorphismus \(K[(t_i)] \to L, t_i \mapsto x_i\) injektiv ist.
\end{remark}

\begin{definition}
	Sei \(K \subset L\) eine Körpererweiterung. Eine \emph{Transzendenzbasis
	von \(L\) über \(K\)} ist eine maximale Familie algebraisch unabhängiger
	Elemente.
\end{definition}

\begin{remark}
	Sei \(K \subset L\) eine Körpererweiterung. Dann ist \((x_i)_{i \in I}\)
	genau dann eine Transzendenzbasis von \(L\) über \(K\), wenn
	\(K((t_i)) \to L, t_i \mapsto x_i\) eine wohldefinierte algebraische
	Körpererweiterung ist.
\end{remark}

\begin{proposition}
	Sei \(K \subset L\) eine Körpererweiterung. Dann besitzt \(L\) eine
	Transzendenzbasis über \(K\) und je zwei Transzendenzbasen haben dieselbe
	Mächtigkeit.
\end{proposition}

\begin{definition}
	Sei \(K \subset L\) eine Körpererweiterung. Der
	\emph{Transzendenzgrad \(\trdeg_K L\) von
	\(L\) über \(K\)} ist die Mächtigkeit einer Transzendenzbasis von \(L\)
	über \(K\).
\end{definition}

\begin{example}
	Eine Körpererweiterung \(K \subset L\) ist genau dann algebraisch, wenn
	der Transzendenzgrad von \(L\) über \(K\) gleich \(0\) ist.
\end{example}



\end{document}

