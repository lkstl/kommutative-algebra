\section{Tensorprodukte von Algebren}

\subsection{Definition des Tensorproduktes zweier Algebren}

\begin{frame}{Ein Produkt auf dem Tensorprodukt zweier Algebren}
	Sei \(A\) ein kommutativer Ring, und seien \(B\) und \(C\) zwei kommutative \(A\)-Algebren.
	\\
	Da \(B\) und \(C\) insbesondere \(A\)-Moduln sind, können wir den \(A\)-Modul \(D \coloneqq B^A \otimes_A C^A\)
	betrachten.
	\\
	Die Abbildung \(B \times C \times B \times C \mapsto D, (b, c, b', c') \mapsto bb' \otimes cc'\)
	ist \(A\)-linear in jedem Faktor, so daß sie einen Homomorphismus
	\(B \otimes C \otimes B \otimes C \to D, b \otimes c \otimes b' \otimes c' \mapsto bb' \otimes cc'\)
	von \(A\)-Moduln induziert.
	\\
	Durch Klammersetzung erhalten wir also einen Homomorphismus \(D \otimes D \to D\) von \(A\)-Moduln,
	welcher wiederum zu einer \(A\)-bilinearen Abbildung
	\[
		\mu\colon D \times D \to D, (b \otimes c, b' \otimes c') \mapsto bb' \otimes cc'
	\]
	korrespondiert.
\end{frame}

\begin{frame}{Das Tensorprodukt zweier Algebren als Algebra}
	Sei \(A\) ein kommutativer Ring und seien \(\phi\colon A \to B\) und \(\psi\colon A \to C\) zwei
	kommutative \(A\)-Algebren.
	\\
	Wir definieren eine kommutative \(A\)-Algebra \(B \otimes_A C\) wie folgt: Als abelsche Gruppe sei
	\(B \otimes_A C\) die abelsche Gruppe des \(A\)-Moduls \(D = B^A \otimes_A C^A\).
	\\
	Die Multiplikation ist \(B \otimes_A C\) wird durch die eben definierte Abbildung
	\[
		\mu\colon (B \otimes_A C) \times (B \otimes_A C) \to B \otimes_A C, (b \otimes c, b' \otimes c') \mapsto bb' \otimes cc'
	\]
	gegeben.
	\\
	Die Eins ist durch \(1_B \otimes 1_C \in B \otimes_A C\) gegeben.
	\\
	Schließlich ist der Strukturhomomorphismus der \(A\)-Algebra durch \(A \to B \otimes_A C, a \mapsto \phi(a) \otimes
	1 = 1 \otimes \psi(a)\) gegeben.
	\begin{definition}<+->
		Die kommutative \(A\)-Algebra \(B \otimes_A C\) heißt das \emph{Tensorprodukt der kommutativen \(A\)-Algebren \(B\) und \(C\)}.
	\end{definition}
\end{frame}

\begin{frame}{Bemerkung zur Modulstruktur des Tensorproduktes von Algebren}
	\begin{remark}<+->
		Sei \(A\) ein kommutativer Ring. Seien \(B, C\) zwei kommutative \(A\)-Algebren. Als \(A\)-Algebra trägt \(B \otimes_A C\)
		insbesondere die Struktur eines \(A\)-Moduls, nämlich \((B \otimes_A C)^A\). Auf der anderen Seite ist die \(B \otimes_A C\)
		zugrundeliegende abelsche Gruppe in natürlicher Weise ein \(A\)-Modul, nämlich \(B^A \otimes_A C^A\). Beide \(A\)-Modulstrukturen
		stimmen überein, das heißt die identische Abbildung \(\id\colon B^A \otimes_A C^A \to (B \otimes_A C)^A\) ist ein
		Isomorphismus von \(A\)-Moduln.
	\end{remark}
\end{frame}

