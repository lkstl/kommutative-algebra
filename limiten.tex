\section{Gerichtete Limiten}

\subsection{Definition des gerichteten Limes}

\begin{frame}{Gerichtete Mengen}
	\begin{definition}<+->
		Eine \emph{gerichtete Menge} ist eine nicht leere teilweise geordnete Menge
		\(I = (I, \le)\), so daß für jedes Paar von Elementen \(i, j \in I\) ein
		\(k \in I\) mit \(i \le k\) und \(j \le k\) existiert.
	\end{definition}
	\begin{visibleenv}<+->
		Eine teilweise geordnete Menge ist also genau gerichtet, wenn jede endliche
		Teilmenge eine obere Schranke besitzt.
	\end{visibleenv}
\end{frame}

\begin{frame}{Gerichtete Systeme von Moduln}
	Sei \(A\) ein Ring. Sei \(I\) eine gerichtete Menge. Sei weiter \((M_i)_{i \in I}\)
	eine Familie von \(A\)-Moduln. Schließlich sei für jedes Paar \(i, j \in I\) mit
	\(i \le j\) ein Homomorphismus \(\mu^i_j\colon M_i \to M_j\) von \(A\)-Moduln
	gegeben.
	\begin{definition}<+->
		Das Datum \(M_\bullet = (M_i, \mu^i_j)\) heißt ein \emph{gerichtetes System
		von \(A\)-Moduln über \(I\)}, falls folgende Axiome erfüllt sind:
		\begin{enumerate}[<+->]
		\item<.->
			Für alle \(i \in I\) ist \(\mu^i_i = \id_{M_i}\colon M_i \to M_i\).
		\item
			Für alle \(i \le j \le k\) ist \(\mu^i_k = \mu^j_k \circ \mu^i_j\colon M_i \to M_k\).
		\end{enumerate}
	\end{definition}
\end{frame}

\begin{frame}{Der gerichtete Limes}
	Seien \(A\) ein Ring und \(I\) eine gerichtete Menge. Sei \(M_\bullet
	= (M_i, \mu^i_j)\) ein gerichtetes System von \(A\)-Moduln über \(I\).
	\\
	Sei \(C \coloneqq \bigoplus\limits_{i \in I} M_i\). Wir identifizieren die
	\(M_i\) mit ihren kanonischen Bildern in \(C\).
	Sei \(D\) der Untermodul von \(C\), welcher von allen Elementen der Form
	\(x_i - \mu^i_j(x_i)\) mit \(i \le j\) und \(x_i \in M_i\) erzeugt wird.
	\\
	Sei \(\mu\colon C \surjto M \coloneqq C/D\) die kanonische Projektion.
	\\
	Sei schließlich \(\mu^i \coloneqq \mu|M_i\colon M_i \to M\).
	\begin{definition}<+->
		Der Modul \(M\) heißt der \emph{gerichtete Limes von \(M_\bullet\)}. Die
		kanonischen \(A\)-linearen Abbildungen \(\mu^i\colon M_i \to M\) heißen
		die \emph{Strukturhomomorphismen von \(M\)}.
	\end{definition}
	\begin{notation}<+->
		Wir schreiben \(\varinjlim\limits_{i \in I} M_i\) für den
		gerichteten Limes des gerichteten Systems \(M_\bullet\).
	\end{notation}
\end{frame}

\subsection{Universelle Eigenschaft des gerichteten Limes}

\begin{frame}{Elemente im gerichteten Limes}
	Seien \(A\) ein Ring und \(I\) eine gerichtete Menge. Sei
	\(M_\bullet = (M_i, \mu^i_j)\) ein gerichtetes System von \(A\)-Moduln über
	\(I\).
	\\
	Seien die \(\mu^i\colon M_i \to M \coloneqq \varinjlim\limits_{i \in I} M_i\) die
	Strukturhomomorphismen.
	\begin{proposition}<+->
		Sei \(x \in M\).
		Dann existiert ein \(i \in I\) und ein \(x_i \in M_i\)
		mit \(\mu^i(x_i) = x\).
		\qed
	\end{proposition}
	\begin{proposition}<+->
		Seien \(i \in I\) und \(x_i \in M_i\) mit \(\mu^i(x_i) = 0 \in M\).
		Dann existiert ein \(j \ge i\) mit \(\mu^i_j(x_i) = 0 \in M_j\).
	\end{proposition}
\end{frame}

\begin{frame}{Universelle Eigenschaft des gerichteten Limes}
	Seien \(A\) ein Ring und \(I\) eine gerichtete Menge. Sei
	\(M_\bullet = (M_i, \mu^i_j)\) ein gerichtetes System von \(A\)-Moduln über
	\(I\).
	\\
	Seien die \(\mu^i\colon M_i \to M \coloneqq \varinjlim\limits_{i \in I} M_i\) die
	Strukturhomomorphismen.
	\begin{proposition}<+->
		Sei \(N\) ein \(A\)-Modul. Sei weiter für alle \(i \in I\) eine \(A\)-lineare
		Abbildung \(\alpha^i\colon M_i \to N\) gegeben, so daß \(\alpha^i = \alpha^j \circ \mu^i_j\)
		für alle Paare \(i \le j\). Dann existiert genau eine \(A\)-lineare
		Abbildung \(\alpha\colon M \to N\) mit \(\alpha^i = \alpha \circ \mu^i\)
		für alle \(i \in I\).
		\qed
	\end{proposition}
\end{frame}

\begin{frame}{Moduln als gerichtete Limiten endlich erzeugter}
	\begin{example}<+->
		Sei \(A\) ein Ring. 
		Sei \(M\) ein \(A\)-Modul.
		Sei \((M_i)_{i \in I}\) eine Familie von Untermoduln von \(M\).
		\\
		Für je zwei Elemente \(i, j \in I\) existieren ein \(k \in I\) mit
		\(M_i + M_j \subset M_k\). Durch die Setzung \(i \le j \iff
		M_i \subset M_j\) wird \(I\) zu einer gerichteten Menge.
		\\
		Weiter sei im Falle \(i \le j\) die Abbildung \(\mu^i_j\colon M_i \to M_j\)
		die Inklusionsabbildung.
		\\
		Dann ist
		\[
			\varinjlim\limits_{i \in I} M_i \cong \sum\limits_{i \in I} M_i
			= \bigcup\limits_{i \in I} M_i.
		\]
	\end{example}
	\begin{visibleenv}<+->
		Damit ist insbesondere jeder \(A\)-Modul der gerichtete Limes seiner
		endlich erzeugten Untermoduln.
	\end{visibleenv}
\end{frame}

\subsection{Exakte Sequenzen gerichteter Systeme}

\begin{frame}{Homomorphismen gerichteteter Systeme}
	Seien \(A\) ein Ring und \(I\) eine gerichtete Menge. Seien \(M_\bullet
	= (M_i, \mu^i_j)\) und \(N_\bullet = (N_i, \nu^i_j)\) zwei gerichtete Systeme
	von \(A\)-Moduln über \(I\).
	\\
	Mit \(\mu^i\colon M^i \to M \coloneqq \varinjlim\limits_{i \in I} M_i\)
	und \(\nu^i\colon N^i \to N \coloneqq \varinjlim\limits_{i \in I} N_i\)
	bezeichnen wir die Strukturhomomorphismen der gerichteten Limiten.
	\\
	Sei \(\phi_\bullet = (\phi_i)\) eine Familie von \(A\)-linearen Abbildungen
	\(\phi_i\colon M_i \to N_i\).
	\begin{definition}<+->
		Die Familie \(\phi_\bullet\) heißt ein \emph{Homomorphismus
		\(\phi_\bullet\colon M_\bullet \to N_\bullet\) gerichteter Systeme}, falls
		\(\phi_j \circ \mu_j^i = \nu_j^i \circ \phi_i\colon M_i \to N_j\) für
		alle \(i \le j\).
	\end{definition}
\end{frame}

\begin{frame}{Der gerichtete Limes als Funktor}
	Seien \(A\) ein Ring und \(I\) eine gerichtete Menge. Seien \(M_\bullet
	= (M_i, \mu^i_j)\) und \(N_\bullet = (N_i, \nu^i_j)\) zwei gerichtete Systeme
	von \(A\)-Moduln über \(I\).
	\\
	Mit \(\mu^i\colon M^i \to M \coloneqq \varinjlim\limits_{i \in I} M_i\)
	und \(\nu^i\colon N^i \to N \coloneqq \varinjlim\limits_{i \in I} N_i\)
	bezeichnen wir die Strukturhomomorphismen der gerichteten Limiten.
	\begin{proposition}<+->
		Es existiert genau ein Homomorphismus \(\phi \coloneqq \varinjlim\limits_{i \in I}
		\phi_i\colon M \to N\) von \(A\)-Moduln, so daß
		\(\phi \circ \mu^i = \nu^i \circ \phi_i\colon M_i \to N\) für alle \(i \in I\).
		\qedhere
	\end{proposition}
\end{frame}

\begin{frame}{Exakte Sequenzen gerichteter Systeme}
	\begin{definition}<+->
		Seien \(A\) ein Ring und \(I\) eine gerichtete Menge. Eine Sequenz
		\(M_\bullet \xrightarrow{\phi_\bullet} N_\bullet \xrightarrow{\psi_\bullet}
		P_\bullet\) von gerichteten Systemen von \(A\)-Moduln über \(I\) heißt
		\emph{exakt}, falls die induzierten Sequenzen
		\(M_i \xrightarrow{\phi_i} N_i \xrightarrow{\psi_i} P_i\) für alle
		\(i \in I\) exakt sind.
	\end{definition}
\end{frame}

\begin{frame}{Exaktheit des gerichteten Limes}
	\begin{proposition}<+->
		Seien \(A\) ein Ring und \(I\) eine gerichtete Menge. Sei
		\(M_\bullet \xrightarrow{\phi_\bullet} N_\bullet \xrightarrow{\psi_\bullet}
		P_\bullet\) eine exakte Sequenz von gerichteten Systemen von \(A\)-Moduln über \(I\).
		Dann ist die induzierte Sequenz
		\(\varinjlim\limits_{i \in I} M_i \xrightarrow{\varinjlim\limits_{i \in I} \phi_i}
		\varinjlim\limits_{i \in I} N_i \xrightarrow{\varinjlim\limits_{i \in I} \psi_i}
		\varinjlim\limits_{i \in I} P_i\)
		exakt.
		\qed
	\end{proposition}
\end{frame}

\subsection{Tensorprodukte und gerichtete Limiten}

\begin{frame}{Tensorprodukte und gerichtete Limiten}
	Seien \(A\) ein Ring und \(I\) eine gerichtete Menge. Sei \(N\) ein \(A\)-Modul.
	Sei weiter \(M_\bullet = (M_i, \mu^i_j)\) ein gerichtetes System von \(A\)-Moduln über \(I\).
	\begin{example}<+->
		Zusammen mit den Abbildungen \(\mu^i_j \otimes \id_N\colon M_i \otimes N \to M_j \otimes N\) für
		alle \(i \le j\) wird \(M_\bullet \otimes N = (M_i \otimes N)\) zu einem gerichteten System über \(I\).
	\end{example}
	\begin{visibleenv}<+->
		Seien \(\mu^i\colon M_i \to M \coloneqq \varinjlim\limits_{i \in I} M_i\) 
		und \(\iota^i\colon M_i \otimes N \to P \coloneqq \varinjlim\limits_{i \in I} (M_i \otimes N)\)
		die Strukturhomomorphismen für \(i \in I\).
		\\
		Nach der universellen Eigenschaft des gerichteten Limes induzieren die \(\mu^i \otimes \id_N\colon M_i \otimes N \to
		M \otimes N\) eine eindeutige \(A\)-lineare Abbildung \(\psi\colon P
		\to M \otimes N\) mit \(\mu^i \otimes \id_N = \psi \circ \iota^i\) für alle \(i\).
	\end{visibleenv}
	\begin{proposition}<+->
		Die \(A\)-lineare Abbildung \(\psi\) ist ein Isomorphismus
		\(\varinjlim\limits_{i \in I} (M_i \otimes N) \isoto (\varinjlim\limits_{i \in I} M_i) \otimes N\).
		\qed
	\end{proposition}
\end{frame}

\subsection{Gerichtete Limiten von Ringen}

\begin{frame}{Gerichtete Limiten von Ringen}
	Sei \(I\) eine gerichtete Menge. Sei \((A_i)_{i \in I}\) eine Familie von Ringen über \(I\).
	Für \(i \le j\) sei \(\alpha^i_j\colon A_i \to A_j\) ein Ringhomomorphismus. 
	Die additiven Gruppen der Ringe \(A_i\) mögen ein gerichtetes System von \(\set Z\)-Moduln	
	\(A_\bullet = (A_i, \alpha^i_j)\) bilden.
	Seien die \(\alpha^i\colon A_i \to \varinjlim\limits_{i \in I} A_i\) die Strukturhomomorphismen,
	welche Homomorphismen abelscher Gruppen sind.
	\begin{proposition}<+->
		Auf \(\varinjlim\limits_{i \in I} A_i\) existiert genau eine Struktur eines Ringes, so daß die \(\alpha^i\colon A^i \to A\)
		Ringhomomorphismen werden.
		\qed
	\end{proposition}
	\begin{proposition}<+->
		Ist \(\varinjlim\limits_{i \in I} A_i = 0\), so existiert ein \(i \in I\) mit \(A_i = 0\).
		\qed
	\end{proposition}
\end{frame}

\begin{frame}{Tensorprodukte über beliebige Familien}
	Sei \(A\) ein kommutativer Ring. Sei \((B_i)_{i \in I}\) eine Familie kommutativer \(A\)-Algebren. Ist \(J \subset I\) eine
	endliche Teilmenge, so sei \(B_J = \bigotimes_{j \in J} B_j\).
	\\
	Ist \(K \subset I\) eine weitere endliche Teilmenge mit \(K \subset J\), so sei \(\beta^K_J\colon B_K \to B_J\) der kanonische
	Homomorphismus von \(A\)-Algebren, welcher \(\bigotimes\limits_{k \in K} b_k\) auf \(\bigotimes\limits_{j \in J} b_j\) abbildet,
	wobei wir \(b_j = 1\) für \(j \notin K\) setzen.
	\\
	Der gerichtete Limes \(B \coloneqq \bigotimes\limits_{i \in I} B_i \coloneqq \varinjlim_{J \subset I} B_J\) ist in kanonischer Weise
	eine \(A\)-
	Algebra, so daß die
	Strukturmorphismen \(B_J \to B\) Morphismen von \(A\)-Algebren werden.
	\begin{definition}<+->
		Die kommutative \(A\)-Algebra \(\bigotimes\limits_{i \in I} B_i\) heißt das \emph{Tensorprodukt über die Familie \((B_i)\)}.
	\end{definition}
\end{frame}

