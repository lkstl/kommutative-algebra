\section{Primärzerlegung I}

\subsection{Primäre Ideale}

\mode<article>{Ein Primideal in einem kommutativen Ring ist in gewisser Weise eine Verallgemeinerung einer Primzahl.
Die entsprechende Verallgemeinerung einer Potenz einer Primzahl ist ein Primärideal.}

\begin{frame}{Definition primärer Ideale}
	Sei \(A\) ein kommutativer Ring.
	\begin{definition}
		Ein Ideal \(\ideal q\) in \(A\) heißt \emph{primär}, falls \(1 \notin \ideal q\) und falls aus \(x y \in
		\ideal q\) schon \(x \in \ideal q\) oder \(y^n \in \ideal q\) für ein \(n \in \set N_0\) folgt.
	\end{definition}
	\begin{visibleenv}<+->
		Das Ideal \(\ideal q\) ist also genau dann primär, falls die nilpotenten Elemente in \(A/\ideal q\) gerade die
		Nullteiler in \(A/\ideal q\) sind.
	\end{visibleenv}
	\begin{example}<+->
		Jedes Primideal in \(A\) ist auch primär.
	\end{example}
	\begin{example}<+->
		Sei \(\phi\colon A \to B\) ein Homomorphismus kommutativer Ringe. Ist dann \(\ideal q\) ein primäres Ideal in
		\(B\), ist die Kontraktion \(A \cap \ideal q\) primär in \(A\), denn \(A/(A \cap \ideal q)\) ist ein Unterring
		von \(B/\ideal q\).
	\end{example}
\end{frame}

\begin{frame}{Wurzeln primärer Ideale}
	Sei \(\ideal q\) ein primäres Ideal in einem kommutativen Ring \(A\). 
	\begin{proposition}<+->
		Es ist \(\sqrt{\ideal q}\) das kleinste Primideal \(\ideal p\) mit \(\ideal p \supset \ideal q\).
	\end{proposition}
	\begin{proof}<+->
		\begin{enumerate}[<+->]
		\item<.->
			Da \(\sqrt{\ideal q} = \bigcap\limits_{\ideal p' \supset \ideal q} \ideal p'\), wobei \(\ideal p'\) für ein
			Primideal von \(A\) steht, reicht es zu zeigen, daß \(\sqrt{\ideal q}\) ein Primideal ist.
		\item
			Sei dazu \(x y \in \sqrt{\ideal q}\). Dann ist \(x^m y^m = (x y)^m \in \ideal q\) für ein \(m \in \set N_0\).
		\item
			Da \(\ideal q\) primär ist, folgt \(x^m \in \ideal q\) oder \(y^{nm} \in \ideal q\) für ein \(n \in \set N_0\).
		\item
			Daraus folgt \(x \in \sqrt{\ideal q}\) oder \(y \in \sqrt{\ideal q}\).
			\qedhere
		\end{enumerate}
	\end{proof}
	\begin{definition}<+->
		Ist \(\ideal p = \sqrt{\ideal q}\), so heißt \(\ideal q\) ein \emph{\(\ideal p\)-primäres Ideal von \(A\)}.
	\end{definition}
\end{frame}

\begin{frame}{Beispiele primärer Ideale}
	\begin{example}<+->
		Die primären Ideale in \(\set Z\) sind die Ideale der Form \((0)\) und \((p^n)\), wobei \(p\) eine Primzahl ist, denn
		diese sind die einzigen Ideale, deren Wurzel ein Primideal ist, und es ist klar, daß diese Ideale primär sind.
	\end{example}
	\begin{example}<+->
		Sei \(A \coloneqq K[x, y]\) der Polynomring in zwei Variablen über einem Körper. Sei \(\ideal q \coloneqq (x, y^2)\). Dann ist
		\(A/\ideal q \cong k[y]/(y^2)\), ein Ring, in dem alle Nullteiler Vielfache von \(y\) sind, also nilpotent sind. Damit ist
		\(\ideal q\) ein primäres Ideal, und zwar mit Wurzel \(\ideal p \coloneqq (x, y)\).
		\\
		Da \(\ideal p^2 \subsetneq \ideal q \subsetneq \ideal p\) sehen wir, daß ein primäres Ideal im allgemeinen keine Potenz
		eines Primideals sein muß.
	\end{example}
\end{frame}

\begin{frame}{Nicht primäre Potenz eines Primideals}
	\begin{example}<+->
		Seien \(K\) ein Körper und \(A \coloneqq K[x, y, z]/(xy - z^2)\). Mit \(\bar x, \bar y, \bar z\) bezeichnen wir die
		Bilder von \(x, y, z\) in \(A\).
		\\
		Es ist \(\ideal p \coloneqq (\bar x, \bar z)\) ein Primideal in \(A\), denn \(A/\ideal p \cong K[y]\) ist ein Integritätsbereich.
		\\
		Wir haben \(\bar x \bar y = \bar z^2 \in \ideal p^2\), aber \(\bar x \notin \ideal p^2\) und \(\bar y \notin \sqrt{\ideal p^2}
		= \ideal p\), also ist \(\ideal p^2\) nicht primär.
	\end{example}
	\begin{visibleenv}<+->
		Wir sehen also, daß Potenzen von Primidealen nicht notwendigerweise primär sind.
	\end{visibleenv}
\end{frame}

\begin{frame}{Ideale, deren Wurzel ein maximales Ideal ist}
	Sei \(A\) ein kommutativer Ring.
	\begin{proposition}<+->
		Sei \(\ideal a\) ein Ideal in \(A\). Ist \(\ideal m \coloneqq \sqrt{\ideal a}\) ein maximales Ideal, so ist \(\ideal a\) ein
		\(\ideal m\)-primäres
		Ideal.
	\end{proposition}
	\begin{proof}<+->
		\begin{enumerate}[<+->]
		\item<.->
			Das Bild von \(\ideal m\) in \(A/\ideal a\) ist das Nilradikal von \(A/\ideal a\). Da jedes Primideal das Nilradikal enthält,
			besitzt \(A/\ideal a\) damit genau ein Primideal.
		\item
			Damit ist jedes Element in \(A/\ideal a\) entweder eine Einheit oder nilpotent, und damit ist jeder Nullteiler in
			\(A/\ideal a\) auch nilpotent.
			\qedhere
		\end{enumerate}
	\end{proof}
	\begin{example}<+->
		Ist \(\ideal m\) ein maximales Ideal in \(A\), so sind die Potenzen \(\ideal m^n\) mit \(n > 0\) alle \(\ideal m\)-primär.
	\end{example}
\end{frame}

\subsection{Schnitte und Idealquotienten primärer Ideale}

\begin{frame}{Schnitte primärer Ideale mit dem gleichen Wurzelideal}
	\begin{lemma}<+->
		\label{lem:primary1}
		Sei \(\ideal p\) ein Primideal in einem kommutativen Ring \(A\). Dann ist der Schnitt
		\(\ideal q \coloneqq \bigcap\limits_{i = 1}^n \ideal q_i\) endlich vieler \(\ideal p\)-primärer Ideale
		\(\ideal q_1, \dotsc, \ideal q_n\) wieder \(\ideal p\)-primär.
                % Wenn man sehr genau sein möchte, sollte man noch anmerken,
                % dass hier n >= 1 sein muss. Der leere Schnitt, das
                % Einsideal, ist nämlich nicht primär.
	\end{lemma}
	\begin{proof}<+->
		\begin{enumerate}[<+->]
		\item<.->
			\(\sqrt{\ideal q} = \sqrt{\bigcap\limits_i \ideal q_i} = \bigcap\limits_i \sqrt{\ideal q_i} = \ideal p\).
		\item
			Sei \(xy \in \ideal q\) mit \(x \notin \ideal q\). Dann ist \(xy \in \ideal q_i\) mit \(x \notin \ideal q_i\) für ein
			\(i\). Es folgt \(y \in \ideal p\), da \(\ideal q_i\) ein \(\ideal p\)-primäres Ideal ist.			
			\qedhere
		\end{enumerate}
	\end{proof}
\end{frame}

\begin{frame}{Idealquotienten primärer Ideale}
	\begin{lemma}<+->
		\label{lem:primary2}
		Sei \(\ideal p\) ein Primideal eines kommutativen Ringes \(A\). Sei \(\ideal q\) ein \(\ideal p\)-primäres
		Ideal und \(x \in A\). Dann gilt:
		\begin{enumerate}[<+->]
		\item<.->
			Ist \(x \in \ideal q\), so \((\ideal q : x) = (1)\).
		\item
			Ist \(x \notin \ideal q\), so ist \((\ideal q : x)\) ein \(\ideal p\)-primäres Ideal.
		\item
			Ist \(x \notin \ideal p\), so ist \((\ideal q : x) = \ideal q\).
		\end{enumerate}
	\end{lemma}
	\begin{proof}<+->
		\begin{enumerate}[<+->]
		\item<.->
			Die erste und die dritte Aussage folgt sofort aus den Definitionen.
		\item
			Sei \(x \notin \ideal q\). Für \(y \in (\ideal q : x)\) folgt dann \(x y \in \ideal q\), also \(y \in \ideal p\).
			Damit ist \(\ideal q \subset (\ideal q : x) \subset \ideal p\). Durch Wurzelziehen folgt \(\sqrt{(\ideal q : x)}
			= \ideal p\).
		\item
			Sei weiter \(yz \in (\ideal q : x)\) mit \(x \notin \ideal q\). Sei \(y \notin \ideal p\). Aus \(x y z \in \ideal q\)
			folgt dann \(xz \in \ideal q\), also \(z \in (\ideal q : x)\).
			\qedhere
		\end{enumerate}
	\end{proof}
\end{frame}

\subsection{Primärzerlegungen}

\begin{frame}{Definition der Primärzerlegung}
	\begin{definition}<+->
		Sei \(\ideal a\) ein Ideal in einem kommutativen Ring \(A\). Ein Ausdruck von \(\ideal a\) als endlicher Schnitt
		\(\ideal a = \bigcap\limits_{i = 1}^n \ideal q_i\) primärer Ideale \(\ideal q_i\) heißt \emph{Primärzerlegung von \(\ideal a\)}.
		\\
		Sind die \(\sqrt{\ideal q_i}\) paarweise verschieden und gilt \(\ideal q_i \not\supset \bigcap\limits_{j \neq i}
		\ideal q_j\) für alle \(i\), so heißt die Primärzerlegung \emph{minimal}.
	\end{definition}
	\begin{remark}<+->
		Nach dem vorletzten Hilfssatz können wir jede Primärzerlegung eines Ideals zu einer minimalen reduzieren.
	\end{remark}
	\begin{remark}<+->
		Im allgemeinen muß nicht jedes Ideal eine Primärzerlegung besitzen. Im Falle, daß es eine hat, heißt es \emph{zerlegbar}.
	\end{remark}
\end{frame}

\begin{frame}{Erster Eindeutigkeitssatz}
	\begin{theorem}[Der erste Eindeutigkeitssatz]<+->
		\label{thm:first_uniqueness}
		Sei \(\ideal a\) ein zerlegbares Ideal in einem kommutativen Ring \(A\). Sei \(\ideal a = \bigcap\limits_{i = 1}^n \ideal
		q_i\) eine minimale Primärzerlegung von \(\ideal a\). Sei \(\ideal p_i \coloneqq \sqrt{\ideal q_i}\). Dann sind die Ideale
		\(\ideal p_i\) genau diejenigen Primideale von \(A\), welche von der Form \(\sqrt{(\ideal a : x)}\) mit \(x \in A\) sind.
		\\
		Insbesondere sind die \(\ideal p_i\) unabhängig von der Primärzerlegung von \(A\). Sie heißen die 			\alert{zu \(\ideal a\)
		assoziierten Primideale}.
	\end{theorem}
	\begin{proof}<+->
		\begin{enumerate}[<+->]
		\item<.->
			\(\sqrt{(\ideal a : x)} = \sqrt{(\bigcap\limits_i \ideal q_i : x)} = \bigcap\limits_i
			\sqrt{(\ideal q_i : x)} = \bigcap\limits_{\ideal q_j \not\ni x} \ideal p_j\).
		\item
			Ist \(\ideal p \coloneqq \sqrt{(\ideal a : x)}\) für ein \(x \in A\) prim, folgt damit \(\ideal p = \ideal p_j\) für ein
			\(j\).
		\item
			Umgekehrt folgt aus der Minimalität der Zerlegung, daß für jedes \(i\) ein \(x \notin \ideal q_i\) mit
			\(x \in \bigcap\limits_{j \neq i} \ideal q_j\) existiert. Dafür gilt \(\sqrt{(\ideal a : x)} = \ideal p_i\).
			\qedhere
		\end{enumerate}
	\end{proof}
\end{frame}

\begin{frame}{Bemerkungen zum ersten Eindeutigkeitssatz}
	Sei \(\ideal a\) ein zerlegbares Ideal in einem kommutativen Ring \(A\).
	\begin{remark}<+->
		Sei \(\ideal p\) ein assoziiertes Primideal zu \(\ideal a\).
		Aus dem letzten Beweis und dem letzten Hilfssatz folgt, daß ein \(x \in A\) existiert, so daß
		\((\ideal a : x)\) ein \(\ideal p\)-primäres Ideal ist.
	\end{remark}
	\begin{remark}<+->
		Sehen wir \(A/\ideal a\) als \(A\)-Modul an, ist der Eindeutigkeitssatz äquivalent dazu zu sagen, daß die zu
		\(\ideal a\) assoziierten Primideale genau diejenigen Primideale sind, welche als Wurzeln von Annulatoren von Elementen von
		\(A/\ideal a\) auftauchen.
	\end{remark}
\end{frame}

\subsection{Isolierte und minimale Primideale}

\begin{frame}{Definition isolierter und eingebetteter Primideale}
	\begin{definition}<+->
		Sei \(\ideal a\) ein zerlegbares Ideal eines kommutativen Ringes \(A\). Die minimalen Elemente der Menge der zu
		\(\ideal a\) assoziierten Primideale heißen die \emph{isolierten Primideale zu \(\ideal a\)}. Die übrigen zu
		\(\ideal a\) assoziierten Primideale heißen die \emph{eingebetteten Primideale zu \(\ideal a\)}.
	\end{definition}
	\begin{example}<+->
		Sei \(K\) ein Körper. Sei \(A = K[x, y]\). Sei \(\ideal a = (x^2, xy)\). Dann ist \(\ideal a = \ideal p_1 \cap \ideal p_2^2\)
		mit \(\ideal p_1 = (x)\) und \(\ideal p_2 = (x, y)\). Da \(\ideal p_2\) maximal ist, ist \(\ideal p_2^2\) ein primäres
		Ideal. Damit sind \(\ideal p_1\) und \(\ideal p_2\) die zu \(\ideal a\) assoziierten Ideale.
		\\
		In diesem Beispiel ist \(\ideal p_1 \subset \ideal p_2\), weiter haben wir \(\sqrt{\ideal a} = \ideal p_1 \cap \ideal p_2 =
		\ideal p_1\), allerdings ist \(\ideal a\) selbst kein primäres Ideal.
		\\
		Es ist \(\ideal p_2\) ein eingebettetes Primideal zu \(\ideal a\).
	\end{example}
	\begin{visibleenv}<+->
		Die Begriffe "`isoliert"' und "`eingebettet"' kommen aus der algebraischen Geometrie.
	\end{visibleenv}
\end{frame}

\begin{frame}{Isolierte Primideale}
	\begin{proposition}<+->
		\label{prop:isolated_prime}
		Sei \(\ideal a\) ein zerlegbares Ideal eines kommutativen Ringes. Jedes Primideal \(\ideal p\) mit \(\ideal p \supset \ideal a\)
		enthält ein isoliertes Primideal zu \(\ideal a\).
	\end{proposition}
	Die isolierten Primideale zu \(\ideal a\) sind damit genau die minimalen Elemente der Menge aller Primideale, welche \(\ideal a\)
	umfassen.
	\begin{proof}<+->
		Sei \(\ideal a = \bigcap\limits_{i = 1}^n \ideal q_i\) eine Primärzerlegung von \(\ideal a\). Aus
		\(\ideal p \supset \bigcap\limits_i \ideal q_i\) folgt \(\ideal p = \sqrt{\ideal p}
		\supset \bigcap\limits_i \sqrt{\ideal q_i}\). Damit muß \(\ideal p \supset \sqrt{\ideal q_i}\) für ein \(i\) gelten.
		Damit umfaßt \(\ideal p\) ein isoliertes Primideal zu \(\ideal a\).
	\end{proof}
\end{frame}

\begin{frame}{Die primären Komponenten sind nicht eindeutig}
	\begin{remark}<+->
		Sei \(K\) ein Körper. Im Ring \(K[x, y]\) sind \((x^2, xy) = (x) \cap (x, y)^2\) und
		\((x) \cap(x^2, y)\) zwei verschiedene Primärzerlegungen von \((x^2, xy)\). Damit sind primären Komponenten
		eines zerlegbaren Ideals im allgemeinen nicht eindeutig.
		\\
		In Kürze werden wir allerdings gewisse Eindeutigkeitseigenschaften kennenlernen.
	\end{remark}
\end{frame}

