\section{Erweiterungen und Kontraktionen von Idealen}

\subsection{Kontraktionen}

\begin{frame}{Definition der Kontraktion eines Ideals}
	Sei \(\phi\colon A \to B\) ein Homomorphismus kommutativer Ringe.
	Sei \(\ideal b\) ein Ideal von \(B\).
	\\
	Wir haben gesehen, daß das Urbild \(\phi^{-1}(\ideal b)\)
	ein Ideal von
	\(A\) ist.
	\begin{definition}<+->
		Das Ideal \(A \cap \ideal b := \phi^{-1}(\ideal b)\) von \(A\) heißt die
		\emph{Kontraktion von \(\ideal b\) (bezüglich \(\phi\))}.
	\end{definition}
	\begin{remark}<+->
		Ist \(\phi\) die Inklusion eines Unterringes \(A\) in \(B\), ist die 
		Kontraktion von \(\ideal b\) in der Tat der mengentheoretische Schnitt
		von \(A\) mit \(\ideal b\).
	\end{remark}
	\begin{visibleenv}<+->
		Es ist \(A \cap \ideal b\) der Kern des Homomorphismus
		\(A \to B/\ideal b, a \mapsto [\phi(a)]\).
		\\
		Nach dem Homomorphiesatz existiert damit ein injektiver Homomorphismus
		\(A/(A \cap \ideal b) \to B/\ideal b\).
	\end{visibleenv}
\end{frame}

\begin{frame}{Kontraktionen von Primidealen}
	\begin{proposition}<+->
		Ist \(\ideal q\) ein Primideal von \(B\), so ist \(A \cap \ideal q\) ein
		Primideal von \(A\).
	\end{proposition}
	\begin{proof}<+->
		Da ein injektiver Ringhomorphismus \(A/(A \cap \ideal q) \injto B/\ideal q\) existiert,
		muß \(A/(A \cap \ideal q)\) mit \(B/\ideal q\) auch ein Integritätsbereich sein.
	\end{proof}
	\begin{example}<+->
		Die Kontraktion eines maximalen Ideals ist im allgemeinen nicht mehr maximal.
		Sei etwa \(\phi\colon \set Z \injto \set Q\) die Inklusion der ganzen in die
		rationalen Zahlen. Dann ist die Kontraktion des maximalen Ideals \((0)\) in
		\(\set Q\) das Primideal \((0)\) in \(\set Z\), welches aber nicht maximal 
		ist. 
	\end{example}
\end{frame}

\subsection{Erweiterungen}

\begin{frame}{Definition der Erweiterung eines Ideals}
	Sei \(\phi\colon A \to B\) ein Homomorphismus kommutativer Ringe.
	Sei \(\ideal a\) ein Ideal von \(A\).
	\\
	Im allgemeinen ist das Bild \(\phi(\ideal a)\) kein Ideal in \(B\), aber wir 
	können das von \(\phi(\ideal a)\) erzeugte Ideal \((\phi(\ideal a))\) in
	\(B\) betrachten.
	\begin{definition}<+->
		Das Ideal \(B \ideal a := (\phi(\ideal a))\) von \(B\) heißt die
		\emph{Erweiterung von \(\ideal a\) (bezüglich \(\phi\))}.
	\end{definition}
	\begin{remark}<+->
		Ist \(\phi\) die Inklusion eines Unterringes \(A\) in \(B\), ist die 
		Erweiterung von \(\ideal a\) in der Tat die Menge der
		\(B\)\nobreakdash-Linearkombinationen von Elementen in \(\ideal a\).
	\end{remark}
	\begin{example}<+->
		Die Erweiterung eines Primideals ist im allgemeinen nicht mehr prim.
		Sei etwa \(\phi\colon \set Z \injto \set Q\) die Inklusion der ganzen in die
		rationalen Zahlen. Ist dann \(\ideal a \neq (0)\) ein nicht triviales
		Ideal von \(\set Z\), ist \(\set Q \ideal a = (1)\).
	\end{example}
\end{frame}

\begin{frame}{Ein Beispiel aus der algebraischen Zahlentheorie}
	\begin{example}<+->
		Sei die kanonische Injektion \(\set Z \to \set Z[\iu]\) gegeben, wobei \(\iu^2 = -1\).
		Der Ring \(\set Z[\iu]\) ist wie \(\set Z\) ein euklidischer Ring
		und damit ebenso ein Hauptidealbereich. Die Erweiterung eines
		Primideals \((p)\) in \(\set Z\) ist dann wie folgt gegeben:
		\begin{enumerate}[<+->]
		\item
			Ist \(p = 2\), ist \(\set Z[\iu](p) = (1 + \iu)(1 + \iu)\), das Quadrat eines
			Primideals in \(\set Z[\iu]\).
		\item
			Ist \(p = 1\) modulo \(4\), ist \(\set Z[\iu](p)\) das Produkt zweier
			verschiedener Primideale, also etwa \(\set Z[\iu](5) = (2 + \iu)(2 - \iu)\).
			\\
			Diese nicht triviale Tatsache ist effektiv der Fermatsche
			Zwei-Quadrate-Satz, der besagt, daß eine Primzahl \(p\) mit \(p = 1\)
			modulo \(4\) als Summe zweier Quadratzahlen dargestellt werden 
			kann. (Also etwa \(5 = 2^2 + 1^2\).)
		\item
			Ist \(p = 3\) modulo \(4\), ist \(\set Z[\iu](p)\) ein Primideal.
		\end{enumerate}
	\end{example}
\end{frame}

\subsection{Operationen mit Erweiterungen und Kontraktionen}

\begin{frame}{Erweiterungen von Kontraktionen und Kontraktionen von Erweiterungen}
	\begin{proposition}<+->
		Sei \(\phi\colon A \to B\) ein Homomorphismus kommutativer Ringe.
		Sei \(\ideal a\) ein
		Ideal von \(A\) und \(\ideal b\) ein Ideal von \(B\). Dann gilt:
		\begin{enumerate}[<+->]
		\item<.->
			\(\ideal a \subset A \cap (B \ideal a)\) und
			\(\ideal b \supset B (A \cap \ideal b)\).
		\item
			\(A \cap \ideal b = A \cap (B (A \cap \ideal b))\) und
			\(B \ideal a = B (A \cap (B \ideal a))\).
		\end{enumerate}
	\end{proposition}
	\begin{proof}<+->
		\begin{enumerate}[<+->]
		\item<.->
			\(A \cap (B \ideal a) \supset \phi^{-1}(\phi(\ideal a)) \supset
			\ideal a\).
		\item
			\(\phi(\phi^{-1}(\ideal b)) \subset \ideal b\). Damit auch
			\((\phi(\phi^{-1}(\ideal b))) \subset (\ideal b) = \ideal b\).
		\item
			Aus \(B (A \cap \ideal b) \subset \ideal b\) 
			folgt \(A \cap (B (A \cap \ideal b)) \subset
			A \cap \ideal b \subset A \cap (B (A \cap \ideal b))\). 
		\item
			Aus \(\ideal a \subset A \cap (B \ideal a)\)
			folgt \(B \ideal a \subset B (A \cap (B \ideal a)) \subset
			B \ideal a\).
			\qedhere
		\end{enumerate}
	\end{proof}
\end{frame}

\begin{frame}{Erweiterte und kontrahierte Ideale}
	\begin{proposition}<+->
		Sei \(\phi\colon A \to B\) ein Homomorphismus kommutativer Ringe.
		Durch \(\ideal a = A \cap \ideal b\) und \(\ideal b = B \ideal a\) wird
		eine bijektive ordnungserhaltende Korrespondenz zwischen den kontrahierten
		Idealen \(\ideal a\) von \(A\) und den erweiterten Idealen \(\ideal b\)
		von \(B\) gegeben.
	\end{proposition}
	\begin{proof}<+->
		\begin{enumerate}[<+->]
		\item<.->
			Ist \(\ideal a\) ein kontrahiertes Ideal von \(A\), also etwa
			\(\ideal a = A \cap \ideal b\), so ist \(\ideal a = 
			A \cap \ideal b = A \cap (B (A \cap \ideal b)) = A \cap
			(B \ideal a)\), also die Kontraktion
			eines erweiterten Ideals von \(B\).
		\item
			Ist \(\ideal b\) ein erweitertes Ideal von \(B\), also etwa
			\(\ideal b = B \ideal a\), so ist \(\ideal b = B \ideal a
			= B (A \cap (B \ideal a)) = B (A \cap \ideal b)\),
			also die Erweiterung eines kontrahierten
			Ideals von \(A\).
			\qedhere
		\end{enumerate}
	\end{proof}
\end{frame}

\begin{frame}{Rechenregeln für Erweiterungen}
	\begin{proposition}<+->
		Sei \(\phi\colon A \to B\) ein Homomorphismus kommutativer Ringe.
		Seien \(\ideal a, \ideal a_1,
		\ideal a_2\) Ideale von \(A\). Dann gilt:
		\begin{enumerate}[<+->]
		\item<.->
			\(B(\ideal a_1 + \ideal a_2) = B \ideal a_1 + B \ideal a_2\).
		\item
		    \(B(\ideal a_1 \cap \ideal a_2) \subset B \ideal a_1 \cap
		    B \ideal a_2\).
		\item
		    \(B(\ideal a_1 \ideal a_2) = (B \ideal a_1) (B \ideal a_2)\).
		\item
		    \(B(\ideal a_1 : \ideal a_2) \subset (B \ideal a_1 : B \ideal a_2)\).
		\item
		    \(B \sqrt{\ideal a} \subset \sqrt{B \ideal a}\).
		    \qed
		\end{enumerate}
	\end{proposition}
\end{frame}

\begin{frame}{Rechenregeln für Kontraktionen}
	\begin{proposition}<+->
		Sei \(\phi\colon A \to B\) ein Ringhomomorphismus. Seien \(\ideal b,
		\ideal b_1, \ideal b_2\) Ideale von \(B\). Dann gilt:
		\begin{enumerate}[<+->]
		\item<.->
		    \(A \cap (\ideal b_1 + \ideal b_2) \supset (A \cap \ideal b_1)
		    + (A \cap \ideal b_2)\).
		\item
		    \(A \cap (\ideal b_1 \cap \ideal b_2) = (A \cap \ideal b_1) \cap
		    (A \cap \ideal b_2)\).
		\item   
		    \(A \cap (\ideal b_1 \ideal b_2) \supset (A \cap \ideal b_1)
		    (A \cap \ideal b_2)\).
		\item
		    \(A \cap (\ideal b_1 : \ideal b_2) \subset (A \cap \ideal b_1 :
		    A \cap \ideal b_2)\).
		\item
		    \(A \cap \sqrt{\ideal b} = \sqrt{A \cap \ideal b}\).
		    \qed
		\end{enumerate}
	\end{proposition}
\end{frame}

