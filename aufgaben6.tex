\subsection{Kettenbedingungen}

\begin{exercise}
	Sei \(A\) ein Ring. Sei \(\phi\colon M \to M\) ein Endomorphismus eines \(A\)-Moduls.
	Zeige:
	\begin{enumerate}
	\item
		Ist \(\phi\) surjektiv und \(M\) noethersch, so ist \(\phi\) ein Isomorphismus.
		
		(Tip: Betrachte die Kette der Untermoduln \(\ker(\phi^n)\).)
	\item
		Ist \(\phi\) injektiv und \(M\) artinsch, so ist \(\phi\) ein Isomorphismus.
		
		(Tip: Betrachte die Kette der Untermoduln \(\im(\phi^n)\).)
	\end{enumerate}
\end{exercise}

\begin{exercise}
	Sei \(A\) ein Ring. Sei \(M\) ein \(A\)-Modul. Jede nicht leere Menge endlich erzeugter
	Untermoduln von \(M\) besitze ein maximales Element. Zeige, daß \(M\) noethersch ist.
\end{exercise}

\begin{exercise}
	Sei \(A\) ein Ring. Sei \(M\) ein \(A\)-Modul. Seien \(N_1, N_2\) Untermoduln von \(M\).
	Zeige, daß \(M/(N_1 \cap N_2)\) noethersch ist, wenn \(M/N_1\) und \(M/N_2\) noethersch
	sind.
	
	Formuliere und beweise die entsprechende Aussage für artinsche Moduln.
\end{exercise}

\begin{exercise}
	Sei \(A\) ein Ring. Zeige: Ist \(M\) ein noetherscher \(A\)-Modul, so ist \(A/\ann M\)
	ein noetherscher Ring.
	
	Ist die entsprechende Aussage für artinsche Moduln wahr?
\end{exercise}

\begin{exercise}
	Sei \(A\) ein noetherscher Ring. Zeige, daß \(A\) nur endlich viele minimale Primideale
	besitzt.
	
	(Tip: Angenommen, daß Nilradikal läßt sich nicht als Schnitt endlich vieler Primideale schreiben.
	Dann gibt es ein maximales Wurzelideal \(\ideal a\), welches nicht Schnitt endlich vieler Primideale ist.
	Ein Wurzelideal läßt sich nach~\prettyref{exer:radicals} aber immer als Schnitt (eventuell unendlich vieler) Primideale 
	schreiben. Führe dies zu einem Widerspruch. Folglich existieren endlich viele Primideale \(\ideal p_1, \dotsc,
	\ideal p_n\) mit \(\sqrt{(0)} = \ideal p_1 \cap \dotsb \cap \ideal p_n\). Ist dann \(\ideal q\) ein minimales Primideal, so 
	ist damit \(\ideal p_1 \cap \dotsb \cap \ideal p_n \subset \ideal q\), also \(\ideal p_i \subset \ideal q\) für
	ein \(i\), also \(\ideal p_i = \ideal q\).)
\end{exercise}

