\section{Dimensionstheorie noetherscher lokaler Ringe}

\subsection{Die Größe regulärer Quotienten}

\begin{frame}{Größen eines noetherschen lokalen Ringes}
	\begin{visibleenv}<+->
		Einem noetherschen lokalen Ring \((A, \ideal m)\) können wir folgende Größen zuordnen:
		\begin{enumerate}[<+->]
		\item<.->
			Die minimale Anzahl \(\updelta(A)\) von Elementen \(x_1, \dotsc, x_{\updelta(A)} \in A\), so daß
			\((x_1, \dotsc, x_{\updelta(A)})\) ein \(\ideal m\)-primäres Ideal ist.
		\item
			Der Grad \(\size(A)\) des charakteristischen Polynoms \(\chi_{\ideal m}\) von \(A\), also
			die Ordnung plus \(1\), mit der \(\ell(A/\ideal m^n)\) für
			\(n \gg 0\) wächst.
		\item
			Die Dimension \(\dim A\), also das Supremum der Längen aller Primidealketten in \(A\).
		\end{enumerate}
	\end{visibleenv}
	\begin{visibleenv}<+->
		Im folgenden zeigen wir, daß \(\updelta(A) \ge \size(A) \ge \dim A \ge \updelta(A)\), daß also alle drei Größen
		übereinstimmen.
	\end{visibleenv}
	\begin{proposition}<+->
		Es ist \(\delta(A) \ge \size(A)\).
		\qed
	\end{proposition}
\end{frame}

\begin{frame}{Die Größe eines regulären Quotienten eines Moduls}
	\begin{proposition}<+->
		Sei \((A, \ideal m)\) ein noetherscher lokaler Ring. Sei \(\ideal q\) ein \(\ideal m\)-primäres
		Ideal. Sei \(M\) ein endlich erzeugter \(A\)-Modul. Seien \(x \in A\) regulär in \(M\) und
		\(M'' \coloneqq M/xM\). Dann gilt: \(\deg \chi_{\ideal q}^{M''} \leq  \deg \chi_{\ideal q}^M - 1\).
	\end{proposition}
	\begin{proof}<+->
		\begin{enumerate}[<+->]
		\item<.->
			Da \(x\) regulär in \(M\) ist, ist \(M \to N \coloneqq x M, m \mapsto x m\) ein Isomorphismus von
			\(A\)-Moduln. Sei \(N_n \coloneqq N \cap \ideal q^n M\) für \(n \in \set N_0\). Es existieren exakte Sequenzen
			\(0 \to N/N_n \to M/\ideal q^n M \to M''/\ideal q^n M'' \to 0\).
		\item
			Sei \(g(n) = \ell(N/N_n)\) für \(n \gg 0\), so \(g(n) - \chi_{\ideal q}^M(n) + \chi_{\ideal q}^{M''}(n) = 0\).
			Nach Artin--Rees ist \((N_n)\) eine stabile \(\ideal q\)-Filtration von \(N\), wir können daher
			\(g \in \set Q[n]\) annehmen.
			Da \(N \cong M\) müssen Grad und Leitkoeffizient von \(g\) und \(\chi_{\ideal q}^M\) übereinstimmen.
			\qedhere
		\end{enumerate}
	\end{proof}
\end{frame}

\begin{frame}{Die Größe eines regulären Quotienten}
	\begin{corollary}<+->
		Sei \(A\) ein noetherscher lokaler Ring. Sei \(x \in A\) ein reguläres Element. Dann ist
		\(\size(A/(x)) \le \size(A) - 1\).
	\end{corollary}
	\begin{proof}<+->
		Wir wenden die Proposition auf den \(A\)-Modul \(M = A\) an.
	\end{proof}
\end{frame}

\subsection{Die Dimension noetherscher lokaler Ringe}

\begin{frame}{Größe und Dimension}
	\begin{proposition}<+->
		Sei \((A, \ideal m)\) ein noetherscher lokaler Ring. Dann ist \(\size(A) \ge \dim A\).
	\end{proposition}
	\begin{proof}[Induktionsanfang]<+->
		Der Beweis erfolge mit Induktion über \(d \coloneqq \size(A)\). Ist \(d = 0\), so ist
		\(\ell(A/\ideal m^n)\) für \(n \gg 0\) konstant, also ist \(\ideal m^n = \ideal m^{n + 1}\) für
		\(n \gg 0\), also ist \(A\) artinsch, also \(\dim A = 0\).
		\renewcommand{\qedsymbol}{}
	\end{proof}
\end{frame}

\begin{frame}{Fortsetzung des Beweises}
	\begin{proof}[Induktionsschritt]<+->
		\begin{enumerate}[<+->]
		\item<.->
			Sei \(d > 0\). Sei \(\ideal p_0 \subsetneq \ideal p_1 \subsetneq \dotsb \subsetneq \ideal p_r\) eine
			Primidealkette in \(A\). Sei \(x \in \ideal p_1 \setminus \ideal p_0\). Sei \(A' \coloneqq A/\ideal p_0\),
			und sei \(x'\) das Bild von \(x \in A'\). Dann ist \(x'\) regulär,
			also \(\size(A'/(x')) \leq \size(A') - 1\).
		\item
			Sei \(\ideal m'\) das maximale Ideal in \(A'\). Dann ist \(A'/(\ideal m')^n\) homomorphes Bild von
			\(A/\ideal m^n\), also ist \(\ell(A/\ideal m^n) \ge \ell(A'/(\ideal m')^n)\), also \(\size(A) \ge \size(A')\),
			also \(\size(A'/(x')) \leq d - 1\).
		\item
			Nach Induktionsvoraussetzung ist die Länge einer Primidealkette in \(A'/(x')\) damit höchstens \(d - 1\). Das
			Bild der Kette \(\ideal p_1 \subsetneq \dotsb \subsetneq \ideal p_r\) in \(A'/(x')\) ist eine Kette der Länge
			\(r - 1\), also \(r - 1 \le d - 1\), also \(r \le d\), also \(\dim A \le d\).
		\qedhere
		\end{enumerate}
	\end{proof}
\end{frame}

\begin{frame}{Endlichkeit der Dimension}
	\begin{corollary}<+->
		Sei \(A\) ein noetherscher lokaler Ring. Dann ist \(\dim A < \infty\).
		\qed
	\end{corollary}
	\begin{visibleenv}<+->
		Die Längen von Primidealketten in \(A\) sind also beschränkt.
	\end{visibleenv}
\end{frame}

\begin{frame}{Die Höhe von Primidealen}
	\begin{definition}<+->
		Sei \(A\) ein kommutativer Ring. Die \emph{Höhe \(\height \ideal p\) eines Primideals \(\ideal p\)} von
		\(A\) ist das Supremum der Längen von Primidealketten der Form \(\ideal p_0 \subsetneq \ideal p_1 \subsetneq
		\dotsb \subsetneq \ideal p_r = \ideal p\).
	\end{definition}
	\begin{visibleenv}<+->
		Es ist also \(\height \ideal p = \dim A_{\ideal p}\).
	\end{visibleenv}
	\begin{corollary}<+->
		Ist \(A\) ein noetherscher kommutativer Ring, so hat jedes Primideal \(\ideal p\) von \(A\) endliche Höhe.
		\qed
	\end{corollary}
	\begin{visibleenv}<+->
		Die Menge der Primideale in einem noetherschen kommutativen Ring erfüllt damit die absteigende Kettenbedingung.
	\end{visibleenv}
\end{frame}

\begin{frame}{Die Tiefe von Primidealen}
	\begin{definition}<+->
		Sei \(A\) ein kommutativer Ring. Die \emph{Tiefe \(\depth \ideal p\) eine Primideals \(\ideal p\)} von
		\(A\) ist das Supremum der Längen von Primidealketten der Form \(\ideal p = \ideal p_0 \subsetneq \ideal p_1
		\subsetneq \dotsb \subsetneq \ideal p_r\).
	\end{definition}
	\begin{visibleenv}<+->
		Es ist also \(\depth \ideal p = \dim A/\ideal p\).
	\end{visibleenv}
	\begin{remark}<+->
		Selbst wenn \(A\) noethersch ist, kann die Tiefe eines  Primideals unendlich sein --- außer \(A\) ist zudem lokal.
	\end{remark}
\end{frame}

\begin{frame}{Dimension und Erzeuger primärer Ideale}
	\begin{proposition}<+->
		Sei \((A, \ideal m)\) ein noetherscher lokaler Ring der Dimension \(d\). Dann existiert ein von
		\(d\) Elementen \(x_1, \dotsc, x_d \in A\) erzeugtes \(\ideal m\)-primäres Ideal, also \(\dim A \ge \updelta(A)\).
	\end{proposition}
\end{frame}

\begin{frame}{Beweis der Proposition}
	\begin{proof}<+->
		\begin{enumerate}[<+->]
		\item<.->
			Wir konstruieren induktiv \(x_1, \dotsc, x_d \in A\), so daß jedes Primideal, welches
			\(\ideal a_i \coloneqq (x_1, \dotsc, x_i)\) enthält,
			mindestens Höhe \(i\) hat. Sei \(i > 0\) und seien \(x_1, \dotsc, x_{i - 1}\) schon konstruiert.
		\item
			Seien \(\ideal p_1, \dotsc, \ideal p_s\) die minimalen Primideale mit \(\ideal p_j \supset \ideal a_{i - 1}\) und
			\(\height \ideal p_j = i - 1\).
			Dann ist \(\ideal m \neq \ideal p_j\) für alle \(j\), da \(\height \ideal m = d > i - 1 = \height \ideal p_j\).
			Es folgt \(\ideal m \neq \bigcup\limits_{j = 1}^s \ideal p_j\), also können wir ein \(x_i \in \ideal m\) mit
			\(x_i \notin \ideal p_j\) wählen.
		\item
			Sei \(\ideal q\) ein Primideal, welches \(\ideal a_i\) enthält. Dann enthält \(\ideal q\) ein minimales Primideal
			\(\ideal p\) mit \(\ideal p \supset \ideal a_{i - 1}\). Ist \(\ideal p = \ideal p_j\) für ein \(j\),
			so folgt \(\ideal q \supsetneq \ideal p\), also \(\height \ideal q \ge i\).
		\item
			Ist dagegen \(\ideal p \neq \ideal p_j\) für alle \(j\), so ist \(\height \ideal p \ge i\),
			also \(\height \ideal q \ge i\).
		\item
			Ist schließlich \(\ideal p\) ein minimales Primideal mit \(\ideal p \supset \ideal a_d\), so hat \(\ideal p\) damit
			Höhe \(d\), also \(\ideal p = \ideal m\). Also ist \(\ideal a_d\) ein \(\ideal m\)-primäres Ideal.
			\qedhere
		\end{enumerate}
	\end{proof}
\end{frame}

\subsection{Der Dimensionssatz}

\begin{frame}{Der Dimensionssatz}
	\begin{theorem}[Dimensionssatz]<+->
		Sei \((A, \ideal m)\) ein noetherscher lokaler Ring. Dann sind folgende Größen gleich:
		\begin{enumerate}[<+->]
		\item<.->
			Das maximale Länge \(\dim A\) von Primidealketten in \(A\).
		\item
			Der Grad \(\size(A)\) des charakteristischen Polynoms \(\chi_{\ideal m}\) von \(A\).
		\item
			Die minimale Anzahl von Erzeugern \(\ideal m\)-primärer Ideale von \(A\).
			\qed
		\end{enumerate}
	\end{theorem}
\end{frame}

\begin{frame}{Beispiel zur Dimension}
	\begin{example}<+->
		Sei \(K\) ein Körper. Sei \(A \coloneqq K[X_1, \dotsc, X_n]_{\ideal m}\) der Polynomring in \(n\) Variablen über \(K\)
		lokalisiert am maximalen Ideal \(\ideal m = (X_1, \dotsc, X_n)\). Dann ist \(\Graded_{\ideal m}(A, t) \cong
		K[\bar X_1 t, \dotsc, \bar X_n t]\), wobei die \(\bar X_i\) die Bilder der \(X_i\) in \(\Graded_{\ideal m}(A, t)\) sind.
		\\
		Damit ist die Poincarésche Reihe von \(\Graded_{\ideal m}(A, t)\) durch \((1 - t)^{- n}\) gegeben, also
		\(\dim A = \size(A) = \size(\Graded_{\ideal m}(A, t)) = n\).
	\end{example}
\end{frame}

\begin{frame}{Vergleich mit der Dimension des Kotangentialraums}
	\begin{corollary}<+->
		Sei \((A, \ideal m, F)\) ein noetherscher lokaler Ring. Dann ist \(\dim A \leq \dim_F \ideal m/\ideal m^2\).
	\end{corollary}
	\begin{proof}<+->
		Seien \(x_1, \dotsc, x_s \in \ideal m\), so daß ihre Bilder in \(\ideal m/\ideal m^2\) eine Basis über \(F\) bilden.
		Nach dem Nakayamaschen Lemma erzeugen die \(x_i\) damit das Ideal \(\ideal m\). Also ist \(\dim A = \updelta(A) \le s
		= \dim_F \ideal m/\ideal m^2\).
	\end{proof}
\end{frame}

\begin{frame}{Höhe minimaler Primideale}
	\begin{corollary}<+->
		Sei \(A\) ein kommutativer noetherscher Ring. Seien \(x_1, \dotsc, x_r \in A\). Ist dann \(\ideal p\) ein
		minimales Primideal mit \(\ideal p \supset (x_1, \dotsc, x_r)\), so gilt \(\height \ideal p \le r\).
	\end{corollary}
	\begin{proof}<+->
		Indem wir von \(A\) auf \(A_{\ideal p}\) übergehen, 
		können wir davon ausgehen, daß \(A\) ein lokaler Ring mit maximalem Ideal \(\ideal p\) ist, in dem \((x_1, \dotsc,
		x_r)\) ein \(\ideal p\)-primäres Ideal ist. Damit ist \(r \ge \updelta(A) = \dim A = \height \ideal p\).
	\end{proof}
\end{frame}

\begin{frame}{Der Krullsche Hauptidealsatz}
	\begin{corollary}<+->
		Sei \(A\) ein kommutativer noetherscher Ring. Sei weiter \(x \in A\) ein reguläres Element.
		Dann gilt für jedes minimale Primideal \(\ideal p\) mit \(\ideal p \supset (x)\), daß \(\height \ideal p = 1\).
	\end{corollary}
	\begin{proof}<+->
		Nach der letzten Folgerung ist \(\height \ideal p \le 1\). Wäre \(\height \ideal p = 0\),
		so wäre \(\ideal p\) assoziiert zu \((0)\).
		Damit besteht \(\ideal p\) nur aus Nullteilern. Widerspruch.
	\end{proof}
\end{frame}

\begin{frame}{Dimension regulärer Quotienten}
	\begin{corollary}<+->
		\label{cor:dim_of_reg_quot}
		Sei \((A, \ideal m)\) ein noetherscher lokaler Ring. Sei \(x \in \ideal m\) regulär. Dann ist
		\(\dim A/(x) = \dim A - 1\).
	\end{corollary}
	\begin{proof}<+->
		\begin{enumerate}[<+->]
		\item<.->
			Sei \(d \coloneqq \dim A/(x)\). Dann ist \(d = \size(A/(x)) \le \size(A) - 1 \le \dim A - 1\). 
		\item
			Seien auf der anderen Seite \(x_1, \dotsc, x_d \in \ideal m\), deren Bilder in \(A/(x)\) ein
			\(\ideal m/(x)\)-primäres Ideal erzeugen. Dann ist \((x, x_1, \dotsc, x_d)\) ein \(\ideal m\)-primäres Ideal
			in \(A\), also \(d + 1 \ge \updelta(A) = \dim A\).
		\qedhere
		\end{enumerate}
	\end{proof}
\end{frame}

\begin{frame}{Die Dimension der Vervollständigung eines lokalen Ringes}
	\begin{corollary}<+->
		Sei \((A, \ideal m)\) ein lokaler noetherscher Ring. Sei \(\hat A\) seine \(\ideal m\)-adische Vervollständigung.
		Dann ist \(\dim A = \dim \hat A\).
	\end{corollary}
	\begin{proof}<+->
		Sei \(\hat{\ideal m}\) das maximale Ideal von \(\hat A\).
		Es ist \(A/\ideal m^n \cong \hat A/\hat{\ideal m}^n\), also \(\chi_{\ideal m} = \chi_{\hat{\ideal m}}\). Damit
		ist \(\dim A = \size(A) = \size(\hat A) = \dim \hat A\).
	\end{proof}
\end{frame}

\subsection{Parametersysteme}

\begin{frame}{Parametersysteme}
	\begin{definition}<+->
		Sei \((A, \ideal m)\) ein noetherscher lokaler Ring der Dimension \(d\). Sind dann \(x_1, \dotsc, x_d\) Erzeuger eines
		\(\ideal m\)-primären Ideals von \(A\), so heißt \((x_1, \dotsc, x_d)\) ein Parametersystem von \(A\).
	\end{definition}
\end{frame}

\begin{frame}{Unabhängigkeit der Parameter}
	\begin{proposition}<+->
		Sei \((A, \ideal m)\) ein noetherscher lokaler Ring. Sei \((x_1, \dotsc, x_d)\) ein Parametersystem für \(A\).
		Sei \(\ideal q \coloneqq (x_1, \dotsc, x_d)\) das erzeugte \(\ideal m\)-primäre Ideal.
		Ist \(f \in A[X_1, \dotsc, X_d]\) homogen vom Grad \(s\) mit \(f(x_1, \dotsc, x_d) \in \ideal q^{s + 1}\), so folgt
		\(f \in \ideal m[X_1, \dotsc, X_d]\).
	\end{proposition}
	\begin{proof}<+->
		\begin{enumerate}[<+->]
		\item<.->
			Es ist \(\alpha\colon A/\ideal q[X_1, \dotsc, X_d] \to \Graded_{\ideal q}(t), X_i \mapsto \bar x_i t\),
			wobei \(\bar x_i\) das Bild von \(x_i\) in \(\ideal q/\ideal q^2\) ist, ein surjektiver 
			Homomorphismus gewichteter Ringe. 
		\item
			Nach Voraussetzung an \(f\) ist das Bild \(\bar f\) von \(f\) im Kern von \(\alpha\). Angenommen, ein 
			Koeffizient von
			\(f\) ist eine Einheit. Dann ist \(\bar f\) regulär. Dann gilt:
			\(\size(\Graded_{\ideal q}(t)) \leq \size(A/\ideal q[X_1, \dotsc, X_d]/(\bar f))
			= \size(A/\ideal q[X_1, \dotsc, X_d]) - 1 = d - 1\).
		 \item
		 	Aber es ist \(\size(\Graded_{\ideal q}(t)) = \size(A) = d\), ein Widerspruch.
			\qedhere
		\end{enumerate}
	\end{proof}
\end{frame}

\begin{frame}{Algebraische Unabhängigkeit von Parametern}
	\begin{corollary}<+->
		Sei \(K\) ein Körper. Sei \((A, \ideal m)\) eine lokale \(K\)-Algebra, so daß \(K\) isomorph auf \(A/\ideal m\)
		abgebildet wird. Ist dann \((x_1, \dotsc, x_d)\) ein Parametersystem für \(A\), so sind die \(x_i\) algebraisch
		unabhängig über \(K\).
	\end{corollary}
	\begin{proof}<+->
		\begin{enumerate}[<+->]
		\item<.->
			Sei \(f \in K[X_1, \dotsc, X_n]\) mit \(f(x_1, \dotsc, x_d) = 0\). Angenommen, \(f \neq 0\).
		\item
			Dann können wir \(f = g + h\) schreiben, wobei \(g \neq 0\) ein homogenes Polynom ist und \(h\) echt größeren
			Grad als \(g\) hat.
		\item
			Anwenden der Proposition liefert, daß \(g\) Koeffizienten in \(\ideal m\) hat.
			Da aber \(g\) ein Polynom über \(K\) ist, folgt \(g = 0\), ein Widerspruch.
			\qedhere
		\end{enumerate}
	\end{proof}
\end{frame}

