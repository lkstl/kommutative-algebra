\section{Untermoduln und Quotientenmoduln}

\subsection{Untermoduln, Quotientenmoduln, Kerne und Kokerne}

\begin{frame}{Definition eines Untermoduls}
	Sei \(A\) ein Ring.
	\begin{definition}
		Eine Teilmenge \(M'\) eines \(A\)-Moduls \(M\) heißt \emph{Untermodul von
		\(M\)}, falls
		\begin{enumerate}[<+->]
			\item \(M'\) eine Untergruppe der abelschen Gruppe von \(M\) ist und
			\item \(M'\) abgeschlossen unter Multiplikation mit Elementen aus \(A\) ist,
			also \(a x \in M'\) für alle \(a \in A\) und \(x \in M'\).
		\end{enumerate}
	\end{definition}
	\begin{visibleenv}<+->
		Eine Teilmenge \(M'\) von \(M\) ist genau dann ein Untermodul, falls für alle
		\(x, x' \in M'\) und \(a \in A\) gilt:
		\begin{align*}
			x + x' & \in M'; & 0 & \in M'; \\
			a x & \in M'.
		\end{align*}
	\end{visibleenv}
	\begin{remark}<+->
		Durch Einschränkung der Multiplikation wird jeder Untermodul eines \(A\)-Moduls
		selbst zu einem \(A\)-Modul.
	\end{remark}
\end{frame}

\begin{frame}{Beispiele für Untermoduln}
	Sei \(A\) ein Ring.
	\begin{example}<+->
		Jeder \(A\)-Modul \(M\) besitzt die beiden trivialen Untermoduln \(\{0\}\) und
		\(M\).
	\end{example}
	\begin{example}<+->
		Sei \(A\) kommutativ. Eine Teilmenge \(\ideal a\) von \(A\) ist genau dann
		ein Ideal von \(A\), wenn \(\ideal a\) ein Untermodul von \(A\) ist,
		wobei wir \(A\) in kanonischer Weise als Modul über sich selbst auffassen.
	\end{example}
\end{frame}

\begin{frame}{Quotientenmoduln}
	Sei \(A\) ein Ring. Sei \(M'\) ein Untermodul eines \(A\)-Moduls \(M\).
	\begin{proposition}<+->
		Es gibt genau eine
		Modulstruktur auf der Menge \(M/M'\) der \(M'\)-Nebenklassen, so daß die
		kanonische Abbildung \(\pi\colon M \surjto M/M', x \mapsto [x] \coloneqq
		x + M'\) ein Homomorphismus von \(A\)-Moduln wird.
		\qed
	\end{proposition}
	\begin{definition}<+->
		Der \(A\)-Modul \(M/M'\) heißt der \emph{Quotientenmodul von \(M\) nach
		\(M'\)}.
	\end{definition}
	\begin{proposition}<+->
		Sei \(\pi\colon M \to M/M'\) die kanonische Abbildung.
		Durch \(N = \pi^{-1}(\bar N)\) wird eine bijektive, ordnungserhaltende
		Korrespondenz zwischen den Untermoduln \(N\) von \(M\) mit \(N \supset M'\)
		und den Untermoduln \(\bar N\) von \(M/M'\) gegeben.
		\qed
	\end{proposition}
\end{frame}

\begin{frame}{Kern, Bild und Kokern}
	\begin{visibleenv}<+->
		Sei \(A\) ein Ring. Sei \(\phi\colon M \to N\) ein Homomorphismus von \(A\)-
		Moduln. Es sind
		\[\ker \phi \coloneqq \{x \in M \mid \phi(x) = 0\} \subset
		M\] und \[\im \phi \coloneqq \{\phi(x) \mid x \in M\} \subset N\]
		Untermoduln von \(M\) beziehungsweise \(N\).
	\end{visibleenv}
	\begin{definition}<+->
		\begin{enumerate}[<+->]
		\item<.->
			Der Untermodul \(\ker \phi\) von \(M\) heißt der \emph{Kern von \(\phi\)}.
		\item
			Der Untermodul \(\im \phi\) von \(N\) heißt das
			\emph{Bild von \(\phi\)}.
		\item
			Der Quotientenmodul \(\coker \phi \coloneqq N/\im \phi\) heißt der
			\emph{Kokern von \(\phi\)}.
		\end{enumerate}
	\end{definition}
	\begin{proposition}<+->
		Es ist \(\coker \phi = 0\) genau dann, wenn \(\phi\) surjektiv ist.
		\qed
	\end{proposition}
\end{frame}

\begin{frame}{Der Homomorphiesatz für Moduln}
	Sei \(A\) ein Ring. Sei \(\phi\colon M \to N\) ein Homomorphismus von
	\(A\)-Moduln. 
	\begin{proposition}[Homomorphiesatz für Moduln]<+->
		Ist \(M'\) ein Untermodul von \(M\)
		mit \(M' \subset \ker \phi\), so gibt es genau
		einen Modulhomomorphismus \(\underline \phi\colon M/M' \to N,
		[x] \mapsto \phi(x)\) mit \(\ker \underline{\phi} = \ker \phi/M'\).
	\end{proposition}
	\begin{proof}<+->
		\begin{enumerate}[<+->]
		\item<.->
			Die Existenz von \(\underline \phi\) wird wie beim Homomorphiesatz für
			Ringe bewiesen.
		\item
			Es ist \([x] \in \ker{\underline\phi} \iff 
			\underline\phi([x]) = 0 \iff
			\phi(x) = 0 \iff
			x \in \ker\phi \iff [x] \in \ker\phi/M'\).
			\qedhere
		\end{enumerate}
	\end{proof}
	\begin{example}<+->
		Es existiert ein kanonischer Isomorphismus \(M/\ker \phi \to \im \phi\) von
		\(A\)-Moduln.
	\end{example}
\end{frame}

\subsection{Bezeichnungen}

\begin{frame}{Bezeichnungen für Unter- und Quotientenmoduln}
	\begin{convention}<+->
		Untermoduln von Moduln \(M, N, L, \dotsc\) bezeichnen wir häufig mit Strichen
		\(M', N', L', \dotsc\).
	\end{convention}
	\begin{convention}<+->
		Quotientenmoduln von Moduln \(M, N, L, \dotsc\) bezeichnen wir häufig mit
		Doppelstrichen \(M'', N'', L'', \dotsc\).
	\end{convention}
\end{frame}

