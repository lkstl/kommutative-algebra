\subsection{Diskrete Bewertungsringe und Dedekindsche Bereiche}

\begin{exercise}
	Sei \(A\) ein Dedekindscher Bereich. Sei \(S \subset A\) multiplikativ
	abgeschlossen. Zeige, daß \(S^{-1} A\) entweder ein Dedekindscher Bereich
	oder der Quotientenkörper von \(A\) ist.
	
	Sei jetzt \(S \neq A \setminus \{0\}\). Ist \(\ideal r\) ein nicht
	verschwindendes gebrochenes Ideal in \(A\) bzw.~\(S^{-1} A\) bezeichnen wir
	mit \([\ideal r]\) sein Bild in der Idealklassengruppe von
	\(A\) bzw.\ \(S^{-1} A\). Zeige, daß
	\(\ClassG(A) \to \ClassG(S^{-1} A), [\ideal r] \mapsto [S^{-1} \ideal r]\)
	ein surjektiver Gruppenhomomorphismus ist.
\end{exercise}

\begin{exercise}
	Sei \(A\) ein Dedekindscher Bereich. Ist \(f = a_0 + a_1 x + \dotsb + a_n
	x^n \in A[x]\) ein Polynom über \(A\), so heißt das Ideal \(\content(f)
	\coloneqq (a_0, \dotsb, a_n)\) der \emph{Inhalt von \(f\)}. Zeige das
	Gaußsche Lemma, nämlich daß \(\content(fg) = \content(f) \content(g)\) für
	\(f, g \in A[x]\).
	
	(Tip: Lokalisiere an jedem maximalen Ideal.)
\end{exercise}

\begin{exercise}
	Sei \(A\) ein Bewertungsring, welcher kein Körper ist. Zeige, daß \(A\)
	genau dann noethersch ist, wenn \(A\) ein diskreter Bewertungsring ist.
\end{exercise}

\begin{exercise}
	Sei \((A, \ideal m)\) ein lokaler Integritätsbereich, welcher kein
	Körper ist. Sei \(\ideal m\) ein Hauptideal, und sei \(\bigcap\limits_{n = 0}^\infty
	\ideal m^n = 0\). Zeige, daß \(A\) ein diskreter Bewertungsring ist.
\end{exercise}

\begin{exercise}
	Sei \(A\) ein Dedekindscher Bereich. Sei \(\ideal a\) ein Ideal in \(A\).
	Zeige, jedes Ideal im Ring \(A/\ideal a\) ein Hauptideal ist.
	
	Folgere, daß jedes Ideal in \(A\) von höchstens zwei Elementen erzeugt werden
	kann.
\end{exercise}

\begin{exercise}
	Sei \(A\) ein Dedekindscher Bereich. Seien \(\ideal a, \ideal b, \ideal c\)
	drei Ideale von \(A\). Zeige:
	\begin{enumerate}
	\item
		\(\ideal a \cap (\ideal b + \ideal c) = (\ideal a \cap \ideal b)
		+ (\ideal a \cap \ideal c)\).
	\item
		\(\ideal a + (\ideal b \cap \ideal c) = (\ideal a + \ideal b)
		\cap (\ideal a + \ideal c)\).
	\end{enumerate}
	
	(Tip: Lokalisiere.)
\end{exercise}

\begin{exercise}[Chinesischer Restsatz]
	Sei \(A\) ein Dedekindscher Bereich. Seien \(\ideal a_1, \dotsc, \ideal a_n\)
	Ideale in \(A\) und \(x_1, \dotsc, x_n \in A\). Zeige dann, daß das
	System \(x = x_i \pmod{\ideal a_i}\) von Kongruenzen genau dann eine Lösung \(x \in A\)
	in \(A\) besitzt, wenn \(x_i = x_j \pmod{\ideal a_i + \ideal a_j}\) für alle
	\(i \neq j\).
	
	(Tip: Die Aussage ist äquivalent zur Exaktheit der \(A\)-Modulsequenz
	\[
		A \xrightarrow{\phi} \bigoplus\limits_i A/\ideal a_i
		\xrightarrow{\psi} \bigoplus\limits_{i < j} A/(\ideal a_i + \ideal a_j),
	\]
	wobei die \(i\)-te Komponente von \(\phi(a)\) durch \(x + \ideal a_i\) und
	die \((i, j)\)-te Komponente von \(\psi(x_1 + \ideal a_1, \dotsc,
	x_n + \ideal a_n)\) durch \(x_i - x_j + \ideal a_i + \ideal a_j\) gegeben ist.
	
	Um zu zeigen, daß diese Sequenz exakt ist, reicht es zu zeigen, daß ihre
	Lokalisierung an jedem Primideal \(\ideal p \neq (0)\) exakt ist. Mit anderen
	Worten können wir also annehmen, daß \(A\) ein diskreter Bewertungsring ist.
	Dann ist die Aussage einfach.)
\end{exercise}

