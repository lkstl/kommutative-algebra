\section{Moduln über Dedekindschen Bereichen}

\subsection{Vorüberlegungen}

\mode<article>{In diesem Kapitel wollen wir die Moduln über einem Dedekindschen Bereich klassifizieren. Da jeder
Hauptidealbereich insbesondere ein Dedekindscher Bereich ist, ist die Klassifikation in diesem Kapitel eine
Verallgemeinerung der Klassifikation von Moduln über Hauptidealbereichen.}

\begin{frame}{Exponenten gebrochener Ideale}
	Sei \(A\) ein Dedekindscher Bereich. Wir erinnern, daß die heißt, daß \(A\) ein ganz abgeschlossener
	Integritätsbereich der Dimension \(1\) ist. Insbesondere läßt sich in \(A\) jedes nicht verschwindende
	gebrochene Ideal \(\ideal r\) eindeutig als Produkt
	\(\ideal r = \prod\limits_{\ideal p} \ideal p^{n_{\ideal p}}\)
	von (ganzzahligen) Potenzen von Primidealen von \(A\) schreiben.
	\begin{definition}<+->
		Die Zahl \(\ord_{\ideal p}(\ideal r) \coloneqq n_{\ideal p}\) heißt der \emph{Exponent von \(\ideal r\) an
		\(\ideal p\)}.
	\end{definition}
	\begin{visibleenv}<+->
		Wir setzen \(\ord_{\ideal p}((0)) \coloneqq \infty\) für alle \(\ideal p\) und
		\(\ord_{\ideal p} f = \ord_{\ideal p}(f)\) für eine Funktion \(f\).
	\end{visibleenv}
	\begin{remark}<+->
		Der Expontent \(\ord_{\ideal p} \ideal a\) von \(\ideal a\) an \(\ideal p\) ist
		gerade durch dasjenige \(n \in \set Z\) gegeben, so daß \(A_{\ideal p} \ideal a = (x^n)\)
		gilt, wobei \(x\) Erzeuger des maximalen Ideals im diskreten lokalen Bewertungsring \(A_{\ideal p}\) ist.
	\end{remark}
\end{frame}

\begin{frame}{Lösungen von Idealgleichungen}
	\begin{lemma}<+->
		Sei \(A\) ein Dedekindscher Bereich. Seien \(\ideal a, \ideal b\) zwei ganze Ideale von \(A\) mit
		\(\ideal b \neq (0)\).
		Dann existieren ein ganzes Ideal \(\ideal c\) und ein \(r \in A \setminus \{0\}\)
		mit \(\ideal a + \ideal c = (1)\) und \(\ideal b \ideal c = (r)\).
	\end{lemma}
	\begin{visibleenv}<+->
		Ist \(\ideal b\) nur ein gebrochenes Ideal, so gilt die Aussage des Hilfssatzes ebenfalls, es ist dann
		allerdings im allgemeinen \(r \in K^\units\).
	\end{visibleenv}
	\begin{proof}<+->
		\begin{enumerate}[<+->]
		\item<.->
			Seien \(\ideal a = \prod\limits_{i = 1}^t (\ideal p_i)^{a_i}\)
			und \(\ideal b = \prod\limits_{i = 1}^t (\ideal p_i)^{b_i}\)
			Primidealzerlegungen von \(\ideal a\) und \(\ideal b\).
			Wir wählen \(r_i \in \ideal p_i^{b_i} \setminus \ideal p_i^{b_i + 1}\).
		\item
			Da die \(\ideal p_i\) paarweise koprim sind, existiert ein \(r \in A\) mit \(r = r_i\)
			modulo \(\ideal p_i^{b_i + 1}\). Es folgt \(r \in \bigcup\limits_{i = 1}^t
			\ideal p_i^{b_i} = \ideal b\).
		\item
			Damit ist \((r) + \ideal a \ideal b = \ideal b\), denn der Exponent der linken Seite an einem Primideal
			\(\ideal p\) ist gerade \(b_i\).
		\item
			Setze schließlich \(\ideal c \coloneqq (r) \ideal b^{-1}\). Dann ist \(\ideal c \ideal b + \ideal a \ideal b
			= \ideal b\), also \(\ideal c + \ideal a = (1)\).
			\qedhere
		\end{enumerate}
	\end{proof}
\end{frame}

\begin{frame}{Direkte Summen gebrochener Ideale}
	\begin{lemma}<+->
		Seien \(\ideal a, \ideal b\) zwei Ideale in einem Dedekindschen Bereich \(A\). Dann existiert
		ein Isomorphismus \(\ideal a \oplus \ideal b \cong (1) \oplus \ideal a \ideal b\) von \(A\)-Moduln.
	\end{lemma}
	\begin{proof}<+->
		\begin{enumerate}[<+->]
		\item<.->
			Sind \(\ideal a, \ideal b\) ganz und koprim, so ist \(\ideal a \oplus \ideal b \to (1), (a, b) \mapsto
			a + b\) ein surjektiver Ringhomomorphismus mit Kern \(\ideal a \ideal b\).
		\item
		\item
		\end{enumerate}
	\end{proof}
\end{frame}

