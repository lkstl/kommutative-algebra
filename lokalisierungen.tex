\section{Lokalisierungen von Ringen und Moduln}

\subsection{Lokalisierung eines Ringes}

\begin{frame}{Multiplikativ abgeschlossene Teilmengen}
	\begin{definition}<+->
		Sei \(A\) ein Ring. Eine \emph{multiplikativ abgeschlossene Teilmenge von \(A\)} ist eine Teilmenge \(S \subset A\)
		mit \(1 \in S\) und \(x y \in S\) für alle \(x, y \in S\).
	\end{definition}
	\begin{visibleenv}<+->
		Eine Teilmenge \(S \subset A\) ist also genau dann multiplikativ abgeschlossen, wenn sie ein Untermonoid des
		multiplikativen Monoides von \(A\) ist.
	\end{visibleenv}
	\begin{example}<+->
		Sei \(A\) ein kommutativer Ring. Dann ist \(A \setminus \{0\}\) genau dann multiplikativ abgeschlossen,
		wenn \(A\) ein Integritätsbereich ist.
	\end{example}
	\begin{example}<+->
		Sei \(\ideal a\) ein Ideal \(A\). Dann ist \(1 + \ideal a = \{1 + x \mid x \in \ideal a\} \subset A\) multiplikativ
		abgeschlossen.
	\end{example}
\end{frame}

\begin{frame}{Kürzungsregel}
	Sei \(A\) ein kommutativer Ring. Sei \(S \subset A\) multiplikativ abgeschlossen. Wir betrachten die
	Menge der Paare \((a, s) \in A \times S\).
	\\
	Wir nennen zwei Paare \((a, s), (b, t)\) \emph{äquivalent}, wenn \((at - bs) u = 0\) für ein \(u \in S\).
	\begin{proposition}<+->
		Die so definierte Relation ist eine Äquivalenzrelation.
	\end{proposition}
	\begin{proof}<+->
		\begin{enumerate}[<+->]
		\item<.->
			Die Relation ist offensichtlich reflexiv und symmetrisch.
		\item
			Sei \((a, s)\) äquivalent zu \((b, t)\) und \((b, t)\) äquivalent zu \((c, u)\). Damit
			existieren \(v, w \in S\) mit \((a t - b s) v = (b u - c t) w = 0\).
		\item
			Multiplizieren wir die linke Gleichung mit \(uw\) und die rechte mit \(sv\), können wir \(b\)
			eliminieren und erhalten \((au - cs) tvw = 0\).
		\item
			Da \(S\) multiplikativ abgeschlossen ist, ist \(tvw \in S\). Damit sind \((a, s)\) und \((c, u)\)
			äquivalent.
			\qedhere
		\end{enumerate}
	\end{proof}
\end{frame}

\begin{frame}{Brüche}
	Sei \(A\) ein kommutativer Ring. Sei \(S \subset A\) multiplikativ abgeschlossen.
	\\Die Äquivalenzklasse
	\((a, s) \in A \times S\) eines Paares nennen wir einen \emph{Bruch}. Wir schreiben die Äquivalenzklasse
	dieses Paares als \(\frac a s\).
	\\
	Mit \(S^{-1} A\) bezeichnen wir die Menge aller dieser Brüche. Auf dieser Menge definieren wir eine Addition
	durch
	\[
		\frac a s + \frac b t = \frac{at + bs}{st}
	\]
	und eine Multiplikation durch
	\[
		\frac a s \frac b t = \frac{ab}{st}.
	\]
	\begin{proposition}<+->
		Mit den so definierten Operationen wird \(S^{-1} A\) zu einem wohldefinierten kommutativen Ring.
		\qed
	\end{proposition}
\end{frame}

\begin{frame}{Die Lokalisierung eines kommutativen Ringes}
	Sei \(A\) ein kommutativer Ring. Sei \(S\) eine multiplikativ abgeschlossene Teilmenge.
	\\
	Die Abbildung \(\iota\colon A \to S^{-1} A, a \mapsto \frac a 1\) ist ein Ringhomomorphismus.
	\begin{definition}<+->
		Der kommutative Ring \(S^{-1} A\) heißt die \emph{Lokalisierung von \(A\) nach \(S\)} und
		\(\iota\colon A \to S^{-1} A\) ihr \emph{Strukturhomomorphismus}.
	\end{definition}
	\begin{proposition}[Universelle Eigenschaft der Lokalisierung]<+->
		Sei \(\phi\colon A \to B\) ein Ringhomomorphismus, so daß \(\phi(s) \in B^\units\) für alle \(s \in S\).
		Dann existiert genau ein Ringhomomorphismus \(\psi\colon S^{-1} A \to B\) mit \(\phi = \psi \circ \iota\).
	\end{proposition}
\end{frame}

\begin{frame}{Beweis der universellen Eigenschaft der Lokalisierung}
	\begin{proof}<+->
		\begin{enumerate}[<+->]
		\item<.->
			Eindeutigkeit: Sei \(\psi\colon S^{-1} A \to B\) mit \(\psi(\frac a 1) = \phi(a)\) für alle \(a \in A\). 
			Ist weiter \(s \in S\), so gilt dann 
			\(\psi(\frac a s) = \psi(\frac a 1 (\frac s 1)^{-1}) = \psi(\frac a 1) \psi(\frac s 1)^{-1}
			= \phi(a) \phi(s)^{-1}\).
		\item
			Existenz: Es ist zu zeigen, daß \(\psi\colon S^{-1} A \to B, \frac a s \mapsto \phi(a) \phi(s)^{-1}\) wohldefiniert
			ist. Dazu sei \(\frac a s = \frac{a'}{s'}\), also \((a s' - a' s) u = 0\) für ein \(u \in S\).
		\item
			Es folgt \((\phi(a) \phi(s') - \phi(a') \phi(s)) \phi(u) = 0\).
		\item
			Da \(\phi(u) \in B^\units\), folgt \(\phi(a) \phi(s)^{-1} = \phi(a') \phi(s')^{-1}\).
			\qedhere
		\end{enumerate}
	\end{proof}
\end{frame}

\subsection{Eigenschaften der Lokalisierung}

\begin{frame}{Eigenschaften der Lokalisierung}
	\begin{proposition}<+->
		Sei \(A\) ein kommutativer Ring. Sei \(S \subset A\) multiplikativ abgeschlossen. Sei \(\iota\colon A \to S^{-1} A\)
		der Strukturhomomorphismus. Dann gilt:
		\begin{enumerate}[<+->]
		\item<.->
			Für alle \(s \in S\) ist \(\iota(s)\) eine Einheit in \(S^{-1} A\).
		\item
			Ist \(a \in A\) mit \(\iota(a) = 0\), so ist \(a s = 0\) für ein \(s \in S\).
		\item
			Jedes Element in \(S^{-1} A\) ist von der Form \(\iota(a) \iota(s)^{-1}\) mit \(a \in A\) und \(s \in S\).
			\qed
		\end{enumerate}
	\end{proposition}
\end{frame}

\begin{frame}{Charakterisierung der Lokalisierung}
	\begin{corollary}[Universelle Eigenschaft der Lokalisierung]
		Sei \(A\) ein kommutativer Ring. Sei \(S \subset A\) multiplikativ abgeschlossen.
		Sei \(\iota\colon A \to S^{-1} A\) der Strukturhomomorphismus.
		Sei \(\phi\colon A \to B\) ein Homomorphismus kommutativer Ringe mit folgenden Eigenschaften:
		\begin{enumerate}[<+->]
		\item
			Für alle \(s \in S\) ist \(\phi(s)\) eine Einheit in \(B\).
		\item
			Ist \(a \in A\) mit \(\phi(a) = 0\), so ist \(a s = 0\) für ein \(s \in S\).
		\item
			Jedes Element von \(B\) ist von der Form \(\phi(a) \phi(s)^{-1}\) mit \(a \in A\) und \(s \in S\).
		\end{enumerate}
		\begin{visibleenv}<+->
			Dann existiert ein eindeutiger Isomorphismus \(\psi\colon S^{-1} A \to B\) mit \(\phi = \psi \circ \iota\).
		\end{visibleenv}
	\end{corollary}
\end{frame}

\begin{frame}{Beweis der Folgerung aus der universellen Eigenschaft der Lokalisierung}
	\begin{proof}<+->
		\begin{enumerate}[<+->]
		\item<.->
			Es ist zu zeigen, daß der aufgrund der ersten Eigenschaft wohldefinierte Ringhomomorphismus
			\(\psi\colon S^{-1} A \to B, \frac a s \mapsto \phi(a)\phi(s)^{-1}\) ein Isomorphismus ist.
		\item
			Da jedes Element in \(B\) von der Form \(\phi(a)\phi(s)^{-1}\) ist, ist \(\psi\) offensichtlich surjektiv.
		\item
			Sei schließlich \(\frac a s \in S^{-1} A\) im Kern von \(\psi\). Damit ist insbesondere \(\phi(a) = 0\), also
			\(a t = 0\) für ein \(t \in S\). Es folgt, daß \(\frac a s = \frac 0 1 = 0 \in S^{-1} A\). Also ist \(\psi\) auch
			injektiv.
			\qedhere
		\end{enumerate}
	\end{proof}
\end{frame}

\subsection{Beispiele von Lokalisierungen}

\begin{frame}{Lokalisierungen an Primidealen}
	Sei \(A\) ein kommutativer Ring. Sei \(\ideal p\) ein Primideal in \(A\).
	\begin{example}<+->
		Die Teilmenge \(A \setminus \ideal p \subset A\) ist multiplikativ abgeschlossen.
	\end{example}
	\begin{notation}<+->
		Wir schreiben \(A_{\ideal p} \coloneqq (A \setminus \ideal p)^{-1} A\) und nennen \(A_{\ideal p}\) die
		\emph{Lokalisierung von \(A\) bei \(\ideal p\)} oder den \emph{Halm von \(A\) an \(\ideal p\)}.
	\end{notation}
	\begin{example}<+->
		Es ist \(\ideal m \coloneqq \{\frac a s \mid a \in \ideal p, s \in A \setminus \ideal p\}\) ein echtes
		Ideal im Halm \(A_{\ideal p}\).
		\\
		Ist \(\frac b t \in A_{\ideal p} \setminus \ideal m\), so ist \(b \in A \setminus \ideal p\),
		also \(\frac b t \in (A_{\ideal p})^\units\).
		\\
		Es folgt, daß \(A_{\ideal p}\) ein lokaler Ring mit maximalem Ideal \(\ideal m\) ist.
		\\
		Es ist \(\ideal m = A_{\ideal p} \ideal p\).
	\end{example}
\end{frame}

\begin{frame}{Der Quotientenkörper}
	\begin{example}<+->
		Sei \(A\) ein Integritätsbereich. Dann ist \(S \coloneqq A \setminus \{0\}\) multiplikativ
		abgeschlossen, so daß wir die Lokalisierung \(S^{-1} A\) bilden können. Da \(S\) nur reguläre Elemente (und zwar
		alle) enthält, ist \(A \to S^{-1} A, a \mapsto \frac a 1\) ein injektiver Ringhomomorphismus, so daß
		wir \(A\) als Unterring von \(S^{-1} A\) auffassen können.
		\\
		Ist \(\frac p q \in S^{-1} A \setminus \{0\}\), so ist \((\frac p q)^{-1} = \frac q p\). Damit ist \(S^{-1} A\) ein
		Körper.
		\\
		Wir nennen \(S^{-1} A\) den \emph{Quotientenkörper von \(A\)}. Es ist der kleinste Körper, welcher \(A\) als
		Unterring enthält.
	\end{example}
	\begin{example}<+->
		Der Körper \(\set Q\) der rationalen Zahlen ist der Quotientenkörper des Ringes \(\set Z\) der ganzen Zahlen.
	\end{example}
\end{frame}

\begin{frame}{Lokalisierung außerhalb von Funktionen}
	Sei \(A\) ein kommutativer Ring.
	\begin{example}<+->
		Sei \(S \subset A\) eine multiplikativ abgeschlossene Teilmenge. Dann gilt \(S^{-1} A = 0\) genau dann, wenn
		\(0 \in S\).
	\end{example}
	\begin{example}<+->
		Sei \(f \in A\). Dann ist \(\{f^n \mid n \in \set N_0\}\) eine multiplikativ abgeschlossene Teilmenge.
	\end{example}
	\begin{notation}<+->
		Wir schreiben \(A[f^{-1}] \coloneqq \{f^n \mid n \in \set N_0\}^{-1} A\) und nennen \(A[f^{-1}]\) die
		\emph{Lokalisierung von \(A\) außerhalb von \(f\)}.
	\end{notation}
\end{frame}

\begin{frame}{Spezielle Lokalisierungen}
	\begin{example}<+->
		Sei \(p\) eine Primzahl. Dann ist \(\set Z_{(p)}\) die Menge aller rationalen Zahlen \(\frac m n\), wobei
		\(n\) teilerfremd zu \(p\) ist.
	\end{example}
	\begin{example}<+->
		Sei \(f \in \set Z \setminus \{0\}\). Dann ist \(\set Z[f^{-1}]\) die Menge der rationalen Zahlen der Form
		\(\frac m {f^n}\), wobei \(n \in \set N_0\).
	\end{example}
	\begin{example}<+->
		Sei \(A \coloneqq K[x_1, \dotsc, x_n]\) der Polynomring in \(n\) Variablen über einem Körper \(K\). Für jedes
		Primideal \(\ideal p\) in \(A\) ist dann \(A_\ideal p\) der Ring derjenigen rationalen Funktionen \(\frac f g\)
		in den \(x_i\) mit \(g \notin \ideal p\).
	\end{example}
\end{frame}

\subsection{Lokalisierung von Moduln}

\begin{frame}{Konstruktion der Lokalisierung eines Moduls}
	Sei \(A\) ein kommutativer Ring. Seien \(S \subset A\) multiplikativ abgeschlossen und
	\(M\) ein \(A\)-Modul.
	\\
	Wir nennen zwei Paare \((m, s), (m', s') \in M \times S\) \emph{äquivalent}, wenn \((ms' - m's) u = 0\) für
	ein \(u \in S\).
	\begin{proposition}<+->
		Die so definierte Relation ist eine Äquivalenzrelation.
		\qed
	\end{proposition}
	\begin{visibleenv}<+->
		Die Menge der Äquivalenzklasse \(\frac m s\) der Paare \((m, s)\) bezeichnen wir mit \(S^{-1} M\). 
	\end{visibleenv}
	\begin{proposition}<+->
		Mit der offensichtlichen Definition der Modulstruktur wird \(S^{-1} M\) zu einem \(S^{-1} A\)-Modul.
		\qed
	\end{proposition}
\end{frame}

\begin{frame}{Definition der Lokalisierung eines Moduls}
	Sei \(A\) ein kommutativer Ring. Seien \(S \subset A\) multiplikativ abgeschlossen und
	\(M\) ein \(A\)-Modul.
	\\
	Vermöge des Strukturhomomorphismus \(A \to S^{-1} A\) fassen wir \(S^{-1} A\) als \(A\)-Algebra auf.
	\\
	Die Abbildung \(\iota\colon M \to (S^{-1} M)^A, m \mapsto \frac m 1\)
	ist ein Homomomorphismus von \(A\)-Moduln.
	\begin{definition}<+->
		Der \(S^{-1} A\)-Modul \(S^{-1} M\) heißt die \emph{Lokalisierung von \(M\) nach \(S\)} und
		\(\iota\colon M \to (S^{-1} M)^A\) sein \emph{Strukturhomomorphismus}.
	\end{definition}
\end{frame}

\begin{frame}{Spezielle Lokalisierungen von Moduln}
	Sei \(A\) ein kommutativer Ring. Sei \(M\) ein \(A\)-Modul. 
	\begin{notation}<+->
		Sei \(\ideal p\) ein Primideal in \(A\).
		Wir schreiben \(M_\ideal p \coloneqq (A \setminus \ideal p)^{-1} M\) und nennen den \(A_{\ideal p}\)-Modul
		\(M_{\ideal p}\) die \emph{Lokalisierung von \(M\) bei \(\ideal p\)} oder den \emph{Halm von \(M\)
		an \(\ideal p\)}.

		Das Bild eines Schnittes \(m \in M\) in \(M_{\ideal p}\) unter dem Strukturhomomorphismus \(M \to M_{\ideal p}\)
		heißt der \emph{Keim von \(m\) an \(\ideal p\)}.
	\end{notation}
	\begin{notation}<+->
		Sei \(f \in A\). Wir schreiben \(M[f^{-1}] \coloneqq \{f^n \mid n \in \set N_0\}^{-1} M\) und nennen
		\(M[f^{-1}]\) die \emph{Lokalisierung von \(M\) außerhalb von \(f\)}.
		
		Das Bild eines Schnittes \(m \in M\) in \(M[f^{-1}]\) unter dem Strukturhomomorphismus \(M \to M[f^{-1}]\)
		heißt die \emph{Einschränkung von \(m\) außerhalb von \(f\)}.
	\end{notation}
\end{frame}

\subsection{Exaktheit der Lokalisierung}

\begin{frame}{Funktorialität der Lokalisierung}
	\begin{visibleenv}<+->
		Sei \(A\) ein kommutativer Ring. Seien \(S \subset A\) multiplikativ abgeschlossen und \(\phi\colon M \to N\)
		ein Homomorphismus von \(A\)-Moduln.
		\\
		Dann ist
		\[
			S^{-1}\phi\colon S^{-1} M \to S^{-1} N, \frac m s \mapsto \frac{\phi(m)} s
		\]
		ein Homomorphismus von \(S^{-1} A\)-Moduln.
	\end{visibleenv}
	\begin{proposition}<+->
		Sei \(\psi\colon N \to P\) ein weiterer Homomorphismus von \(A\)-Moduln. Dann ist
		\(S^{-1} (\psi \circ \phi) = S^{-1} \psi \circ S^{-1} \phi\colon S^{-1} M \to S^{-1} P\).
		\qed
	\end{proposition}
\end{frame}

\begin{frame}{Exaktheit der Lokalisierung}
	\begin{proposition}<+->
		Sei \(A\) ein kommutativer Ring. Sei \(S \subset A\) multiplikativ abgeschlossen. Ist \(M' \xrightarrow{\phi} M
		\xrightarrow{\psi} M''\) exakt bei \(M\), so ist auch \(S^{-1} M' \xrightarrow{S^{-1}\phi} S^{-1} M \xrightarrow{S^{-1} \psi}
		S^{-1} M''\) exakt.
	\end{proposition}
	\begin{proof}<+->
		\begin{enumerate}[<+->]
		\item<.->
			Wegen \(\psi \circ \phi = 0\) ist auch \((S^{-1} \psi) \circ (S^{-1} \phi) = S^{-1} 0 = 0\), also
			\(\im S^{-1} \phi \subset \ker S^{-1} \psi\).
		\item
			Sei \(\frac m s \in \ker S^{-1} \psi\), also \(\frac{\psi(m)} s = 0 \in S^{-1} M''\). Damit existiert ein
			\(t \in S\) mit \(\psi(t m) = t \psi(m) = 0 \in M''\).
		\item
			Damit existiert ein \(m' \in M'\) mit \(\phi(m') = t m\). Es folgt, daß \(\frac m s = \frac{\phi(m')}{st}
			= (S^{-1} \phi)(\frac{m'}{st})\).
			\qedhere
		\end{enumerate}
	\end{proof}
\end{frame}

\begin{frame}{Lokalisierung von Untermoduln}
	\begin{example}<+->
		Sei \(A\) ein kommutativer Ring. Sei \(S\) multiplikativ abgeschlossen. Sei
		\(M'\) ein Untermodul eines \(A\)-Moduls \(M\).
		\\
		Da die Lokalisierung exakt ist, ist die Lokalisierung \(S^{-1} M' \to S^{-1} M\) der
		Inklusion \(M' \injto M\) wieder injektiv.
		\\
		Damit können wir \(S^{-1} M'\) als Untermodul von \(S^{-1} M\) ansehen.
	\end{example}
\end{frame}

\begin{frame}{Lokalisierung kommutiert mit endlichen Summen und Quotienten}
	Sei \(A\) ein kommutativer Ring. Seien \(S \subset A\) multiplikativ abgeschlossen und
	\(M\) ein \(A\)-Modul. Seien \(N, P \subset M\) Untermoduln.
	\begin{corollary}<+->
		Es gilt \(S^{-1} (N + P) = S^{-1} N + S^{-1} P \subset S^{-1} M\).
	\end{corollary}
	\begin{proof}<+->
		Folgt sofort aus den Definitionen.
	\end{proof}
	\begin{corollary}<+->
		Die \(S^{-1} A\)-Moduln \(S^{-1}(M/N)\) und \((S^{-1} M)/(S^{-1} N)\) sind isomorph.
	\end{corollary}
	\begin{proof}<+->
		Aus der Exaktheit von \(0 \to N \to M \to M/N \to 0\) folgt die
		Exaktheit von \(0 \to S^{-1} N \to S^{-1} M \to S^{-1}(M/N) \to 0\).
	\end{proof}
\end{frame}

\begin{frame}{Lokalisierung kommutiert mit endlichen Schnitten}
	\begin{proposition}<+->
		Sei \(A\) ein kommutativer Ring. Seien \(S \subset A\) multiplikativ abgeschlossen und
		\(M\) ein \(A\)-Modul. Seien \(N, P \subset M\) Untermoduln.
		Es gilt \(S^{-1} (N \cap P) = S^{-1} N \cap S^{-1} P \subset S^{-1} M\).
	\end{proposition}
	\begin{proof}<+->
		\begin{enumerate}[<+->]
		\item<.->
			Die Inklusion \(S^{-1} (N \cap P) \subset S^{-1} N \cap S^{-1} P\) ist offensichtlich.
		\item
			Seien umgekehrt \(\frac y s = \frac z t\) mit \(y \in N, z \in P\) und \(s, t \in S\). Dann
			ist \(u (t y - s z) = 0\) für ein \(u \in S\). Es folgt, daß  \(w \coloneqq uty = usz \in N \cap P\).
			Damit ist \(\frac y s = \frac w{stu} \in S^{-1} (N \cap P)\).
			\qedhere
		\end{enumerate}
	\end{proof}
\end{frame}

\subsection{Lokalisierung als Basiswechsel}

\begin{frame}{Lokalisierung als Basiswechsel}
	\begin{proposition}<+->
		Sei \(A\) ein kommutativer Ring. Seien \(S \subset A\) multiplikativ abgeschlossen und
		\(M\) ein \(A\)-Modul. Dann existiert ein eindeutiger Isomorphismus
		\(\phi \colon S^{-1} A \otimes_A M \to S^{-1} M, \frac a s \otimes m \mapsto \frac{am} s\).
	\end{proposition}
	\begin{proof}<+->
		\begin{enumerate}[<+->]
		\item<.->
			Da \(\frac{am}{s}\) bilinear in \(\frac a s\) und \(m\) ist, ist \(\phi\) nach der universellen
			Eigenschaft des Tensorproduktes wohldefiniert und eindeutig.
		\item
			Die Surjektivität von \(\phi\) ist offensichtlich.
		\item
			Sei \(\sum_i \frac{a_i}{s_i} \otimes m_i \in S^{-1} A \otimes M\) ein beliebiges Element.
			Mit \(s \coloneqq \prod_i s_i \in S\) und \(t_i \coloneqq \prod_{j \neq i} s_j \in S\) haben wir
			\(\sum_i \frac{a_i}{s_i} \otimes m_i = \frac 1 s \otimes \sum_i a_i t_i m_i\), jedes Element von
			\(S^{-1} A \otimes M\) ist also von der Form \(\frac 1 s \otimes m\).
		\item
			Sei \(\phi(\frac 1 s \otimes m) = 0\). Dann ist \(\frac m s = 0 \in S^{-1} M\), also \(t m = 0 \in M\) für ein \(t \in S\).
			Damit ist \(\frac 1 s \otimes m = \frac 1 {st} \otimes t m = 0\). Also ist
			\(\phi\) injektiv.
		\qedhere
		\end{enumerate}
	\end{proof}
\end{frame}

\begin{frame}{Flachheit der Lokalisierung}
	\begin{proposition}<+->
		Sei \(A\) ein kommutativer Ring. Sei \(S \subset A\) multiplikativ abgeschlossen. Dann ist
		\(S^{-1} A\) eine flache \(A\)-Algebra.
	\end{proposition}
	\begin{proof}<+->
		\begin{enumerate}[<+->]
		\item<.->
			Tensorieren eines \(A\)-Moduls mit \(S^{-1} A\) über \(A\) entspricht Lokalisieren dieses \(A\)-Moduls an \(S\).
		\item
			Lokalisieren ist exakt.
			\qedhere
		\end{enumerate}
	\end{proof}
\end{frame}

\begin{frame}{Tensorprodukte von Lokalisierungen}
	Sei \(A\) ein kommutativer Ring. Seien \(M, N\) zwei \(A\)-Moduln. 
	\begin{proposition}<+->
		Sei \(S \subset A\) multiplikativ abgeschlossen. Dann existiert ein eindeutiger Isomorphismus
		\(\phi\colon S^{-1} M \otimes_{S^{-1} A} \otimes S^{-1} N \to S^{-1} (M \otimes_A N),
		\frac m s \otimes \frac n t \mapsto \frac{m \otimes n}{st}\)
		von \(S^{-1} A\)-Moduln.
	\end{proposition}
	\begin{proof}<+->
		Wir haben die Isomorphismen
		\(S^{-1} M \otimes_{S^{-1} A} S^{-1} N \isoto (S^{-1} A \otimes_A M) \otimes_{S^{-1} A} (S^{-1} A \otimes_A N)
		\isoto S^{-1} A \otimes_A (M \otimes_{S^{-1} A} S^{-1} A \otimes_A N)
		\isoto S^{-1} A \otimes_A (M \otimes_A N) \isoto S^{-1} (M \otimes_A N)\).
	\end{proof}
	\begin{example}<+->
		Sei \(\ideal p\) ein Primideal in \(A\). Dann gilt für die Halme \(M_{\ideal p} \otimes_{A_{\ideal p}}
		N_{\ideal p} = (M \otimes_A N)_{\ideal p}\).
	\end{example}
\end{frame}

