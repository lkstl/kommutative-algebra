\subsection{Ringe und Ideale}

\begin{exercise}
	\label{exer:sum_unit_nilp}
	Sei \(x\) ein nilpotentes Element eines kommutativen Ringes \(A\). Zeige,
	daß \(1 + x\) eine Einheit von \(A\) ist. Folgere, daß die Summe eines
	nilpotenten Elementes mit einer Einheit wieder eine Einheit ist.
\end{exercise}

\begin{exercise}
	\label{exer:polys}
	Sei \(A\) ein kommutativer Ring und \(A[x]\) der Polynomring in der
	Variablen \(x\) über \(A\). Sei \(f = a_0 + a_1 x + \dotsb + a_m x^m
	\in A[x]\). Zeige:
	\begin{enumerate}
	\item
		Das Polynom \(f\) ist genau dann eine Einheit in \(A[x]\), wenn \(a_0\)
		eine Einheit in \(A\) und die \(a_1, \dotsc, a_m\) nilpotent sind.
		
		(Sei \(b_0 + b_1 x + \dotsb + b_n x^n \in A[x]\) eine Inverse von \(f\).
		Zeige per Induktion über \(r\), daß \(a_m^{r + 1} b_{n - r} = 0\).
		Folgere daraus, daß \(a_m\) nilpotent ist und nutze
		dann~\prettyref{exer:sum_unit_nilp}.)
	\item
		Das Polynom \(f\) ist genau dann nilpotent, wenn die \(a_0, \dotsc, a_m\)
		nilpotent sind.
	\item
		Das Polynom \(f\) ist genau dann ein Nullteiler, wenn ein \(a \in A \setminus
		\{0\}\) mit \(a f = 0\) existiert.
		
		(Sei \(g = b_0 + b_1 x + \dotsb + b_n x^n \in A[x] \setminus \{0\}\) ein
		Polynom minimalen Grades mit \(g f = 0\). Dann ist \(a_m b_n = 0\), und damit
		auch \(a_m g = 0\), denn \((a_m g) f = 0\) und \(a_m g\) hat echt kleineren
		Grad als \(g\). Folgere dann per Induktion über \(r\), daß
		\(a_{m - r} g = 0\).)
	\item
		Das Polynom \(f \in A[x]\) heißt \emph{primitiv}, wenn
		\((a_0, \dotsc, a_m) = (1)\). Sei \(g \in A[x]\) ein weiteres Polynom.
		
		Dann ist \(fg\) genau dann primitiv, wenn \(f\) und \(g\) primitiv sind.
	\end{enumerate}
\end{exercise}

\begin{exercise}
	Verallgemeinere die Aussagen der \prettyref{exer:polys} auf einen Polynomring
	\(A[x_1, \dotsc, x_n]\) in mehreren Variablen.
\end{exercise}

\begin{exercise}
	Sei \(A\) ein kommutativer Ring. Zeige, daß im Polynomring \(A[x]\) das 
	Jacobsonsche Radikal gleich dem Nilradikal ist.
\end{exercise}

\begin{exercise}
	Sei \(A\) ein kommutativer Ring. Sei \(\ps A x\) der Ring der formalen
	Potenzreihen \(f = \sum\limits_{m = 0}^\infty a_m x^m\) mit Koeffizienten in
	\(A\). Zeige:
	\begin{enumerate}
	\item
		Die Potenzreihe \(f\) ist genau dann eine Einheit in \(\ps A x\), wenn
		\(a_0\) eine Einheit in \(A\) ist.
	\item
		Ist \(f\) nilpotent, ist \(a_m\) für alle \(m \in \set N_0\) nilpotent.
		
		Gilt auch die Umkehrung? (Vergleiche mit \prettyref{exer:nilp_powerseries}.)
	\item
		Die Potenzreihe \(f\) liegt genau dann im Jacobsonschen Radikal von
		\(\ps A x\), wenn \(a_0\) im Jacobsonschen Ideal von \(A\) liegt.
	\item
		Sei \(\ideal m\) ein maximales Ideal in \(\ps A x\). Dann ist
		die Kontraktion \(\ideal m_0 \coloneqq A \cap \ideal m\) ein maximales Ideal
		in \(A\) und \(\ideal m\) ist das von \(\ideal m_0\) und \(x\) in \(\ps A x\)
		erzeugte Ideal.
	\item
		Jedes Primideal von \(A\) ist die Kontraktion eines Primideals von
		\(\ps A x\).
	\end{enumerate}
\end{exercise}

\begin{exercise}
	Sei \(A\) ein kommutativer Ring, in dem jedes nicht im Nilradikal enthaltene
	Ideal ein nicht triviales Idempotentes enthält, das heißt, ein Element
	\(e \neq 0\) mit \(e^2 = e\). Zeige, daß das Nilradikal und das Jacobsonsche
	Radikal von \(A\) übereinstimmen.
\end{exercise}

\begin{exercise}
	Sei \(A\) ein kommutativer Ring, in dem jedes Element \(x\) eine Gleichung der
	Form \(x^n = x\) für ein (von \(x\) abhängiges) \(n > 1\) erfüllt. Zeige, daß
	jedes Primideal von \(A\) maximal ist.
\end{exercise}

\begin{exercise}
	Sei \(A\) ein kommutativer Ring, welcher nicht der Nullring ist. Zeige, daß \(A\)
	ein bezüglich der Inklusion minimales Primideal besitzt.
\end{exercise}

\begin{exercise}
	\label{exer:radicals}
	Sei \(\ideal a \neq (1)\) ein echtes Ideal eines kommutativen Ringes \(A\). Zeige,
	daß \(\ideal a\) genau dann mit seinem Wurzelideal übereinstimmt, wenn
	\(\ideal a\) ein Schnitt von Primidealen ist.
\end{exercise}

\begin{exercise}
	Sei \(A\) ein kommutativer Ring mit Nilradikal \(\ideal n\). Zeige, daß folgende
	Aussagen äquivalent sind:
	\begin{enumerate}
	\item
		Der Ring \(A\) besitzt genau ein Primideal.
	\item
		Jedes Element von \(A\) ist entweder eine Einheit oder nilpotent.
	\item
		Der Quotientenring \(A/\ideal n\) ist ein Körper.
	\end{enumerate}
\end{exercise}

\begin{exercise}
	Sei \(A\) ein \emph{Boolescher Ring}, das heißt ein kommutativer Ring, in dem
	\(x^2 = x\) für alle \(x \in A\) gilt. Zeige:
	\begin{enumerate}
	\item
		Für alle \(x \in A\) gilt \(2 x = 0\).
	\item
		Jedes Primideal \(\ideal p\) von \(A\) ist maximal und \(A/\ideal p\) ist
		ein Körper mit zwei Elementen.
	\item
		Jedes endlich erzeugte Ideal von \(A\) ist ein Hauptideal.
	\end{enumerate}
\end{exercise}

\begin{exercise}
	Sei \(A\) ein lokaler kommutativer Ring. Zeige, daß \(A\) außer \(0\) und \(1\)
	keine idempotenten Elemente enthält.
\end{exercise}

\begin{exercise}
	\label{exer:ideals_of_zero_divs}
	Sei \(A\) ein kommutativer Ring. Sei \(\mathfrak S\) die Menge aller Ideale von \(A\),
	in denen jedes Element ein Nullteiler ist. Zeige, daß im Falle \(A \neq 0\) die Menge \(\mathfrak S\) bezüglich
	der Inklusion maximale Elemente besitzt und daß jedes maximale Element von \(\mathfrak S\) ein
	Primideal ist. Folgere, daß die Menge der Nullteiler von \(A\) eine Vereinigung von Primidealen
	ist.
\end{exercise}

