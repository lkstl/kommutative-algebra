\section{Ideale und Quotientenringe}

\subsection{Ideale}

\begin{frame}{Definition eines Ideals}
    \begin{definition}<+->
        Eine Teilmenge \(\ideal a\) eines Ringes \(A\) heißt
        \emph{(beidseitiges) Ideal von \(A\)}, falls
        \begin{enumerate}[<+->]
        \item<.->
            \(\ideal a\) eine Untergruppe der abelschen Gruppe von \(A\) ist und
        \item
            \(\ideal a\) abgeschlossen unter der Multiplikation mit
            Elementen aus \(A\) ist, also \(x a \in \ideal a\) und
            \(a x \in \ideal a\) für alle \(a \in \ideal a\) und \(x \in A\).
        \end{enumerate}
    \end{definition}
    \begin{remark}<+->
        Im allgemeinen ist ein Ideal kein Unterring und ein Unterring kein Ideal.
    \end{remark}
    \begin{visibleenv}<+->
        Eine Teilmenge \(\ideal a\) von \(A\) ist genau dann ein Ideal,
        falls für alle \(a, a' \in \ideal a\) und \(x \in A\) gilt:
        \begin{align*}
            a + a' & \in \ideal a; &
            0 & \in \ideal a; \\
            x a & \in \ideal a; &
            a x & \in \ideal a.
        \end{align*}
    \end{visibleenv}
\end{frame}

\begin{frame}{Das von einer Teilmenge erzeugte Ideal}
    \begin{proposition}<+->
        Ist \((\ideal a_i)_{i \in I}\) eine Familie von Idealen eines
        Ringes \(A\),
        so ist der Schnitt \(\bigcap\limits_{i \in I} \ideal a_i\) ein Ideal
        von \(A\).
        \qed
    \end{proposition}
    \begin{definition}<+->
        Sei \((f_i)_{i \in I}\) eine Familie von Elementen eines Ringes \(A\).
        Dann heißt der Schnitt \((f_i \mid i \in I)\) aller Ideale
        von \(A\), welche jedes \(f_i\) enthalten, das \emph{von
        der Familie \((f_i)_{i \in I}\) erzeugte Ideal in \(A\)}.
    \end{definition}
    \begin{proposition}<+->
        Das von einer Familie von Elementen in einem Ring \(A\) erzeugte Ideal
        ist das kleinste Ideal in \(A\), welches jedes Element der Familie
        enthält.
        \qed
    \end{proposition}
\end{frame}

\begin{frame}{Beispiele für Ideale}
    \begin{example}<+->
        Sei \((f_i)_{i \in I}\) eine Familie von Elementen in einem kommutativen
        Ring \(A\). Dann ist
        \((f_i \mid i \in I)
        = \left\{\sum\limits_{k = 1}^n a_k f_{i_k} \mid a_k \in A,
        i_k \in I\right\}\).
    \end{example}
    \begin{example}<+->
        Sei \(n \in \set Z\) eine ganze Zahl. Dann ist das Ideal \((n)\) die Menge
        der durch \(n\) teilbaren Zahlen.
    \end{example}
    \begin{example}<+->
        Das \emph{Nullideal \((0)\)} ist das kleinste Ideal in einem Ring, denn
        \((0) = \{0\}\).
    \end{example}
    \begin{example}<+->
        Das \emph{Einsideal \((1)\)} ist das größte Ideal in einem Ring \(A\), denn
        \((1) = A\).
    \end{example}
\end{frame}

\subsection{Bild und Kern}

\begin{frame}{Bilder und Urbilder unter Ringhomomorphismen}
    Sei \(\phi\colon A \to B\) ein Ringhomomorphismus.
    \begin{proposition}<+->
        \begin{enumerate}[<+->]
        \item<.->
            Für jedes Ideal \(\ideal b\) von \(B\) ist \(\ideal a \coloneqq
            \phi^{-1}(\ideal b)\) ein Ideal von \(A\).
        \item
            Für jeden Unterring \(A'\) von \(A\) ist \(\phi(A')\) ein Unterring von
            \(B\).        
        \end{enumerate}
    \end{proposition}
    \begin{proof}<+->
        \begin{enumerate}[<+->]
        \item<.->
            Als Urbild einer Untergruppe unter einem Gruppenhomomorphismus ist
            \(\ideal a\) eine Untergruppe der abelschen Gruppe von
            \(A\).
        \item
            Für \(a \in \ideal a\) und \(x \in A\) ist \(\phi(x a) = \phi(x)
            \phi(a) \in \ideal b\), also \(x a \in \ideal a\). Analog ist
            \(a x \in \ideal a\).
        \item
            Bilder von Untergruppen bzw.~-monoiden unter
            Gruppen- bzw.~Monoidhomomorphismen sind Untergruppen
            bzw.~-monoide.
            \qedhere
        \end{enumerate}
    \end{proof}
\end{frame}

\begin{frame}{Mengentheoretisches Bild und Kern eines Ringhomomorphismus}
    \begin{visibleenv}<+->
        Sei \(\phi\colon A \to B\) ein Ringhomomorphismus.
    \end{visibleenv}
    \begin{example}<.->
        Das mengentheoretische Bild \(\phi(A)\) von \(\phi\) ist ein Unterring von
        \(B\).
    \end{example}
    \begin{example}<+->
        Das Urbild \(\phi^{-1}((0)) = \{x \in A \mid \phi(x) = 0\}\) des Nullideals
        von \(B\) ist ein Ideal von \(A\).
    \end{example}    
    \begin{definition}<+->    
        Das Ideal \(\phi^{-1}((0))\) heißt der \emph{Kern \(\ker \phi\) von \(\phi\)}.
    \end{definition}   
    \begin{proposition}<+->
        Der Homomorphismus \(\phi\) ist genau dann injektiv, wenn
        \(\ker \phi = (0)\).
        \qed
    \end{proposition}
\end{frame}

\subsection{Quotientenringe}

\begin{frame}{Quotientenringe}
    \begin{proposition}<+->
        Sei \(\ideal a\) ein Ideal eines Ringes \(A\). Dann gibt es genau eine
        Ringstruktur auf der Menge \(A/\ideal a\) der
        \(\ideal a\)\nobreakdash-Nebenklassen,
        so daß die kanonische Abbildung \(\pi\colon A \surjto
        A/\ideal a, x \mapsto [x] := x + \ideal a\) ein Ringhomomorphismus ist.
    \end{proposition}
    \begin{proof}<+->
        \begin{enumerate}[<+->]
        \item<.->
            Auf \(A/\ideal a\) existiert genau eine Struktur einer abelschen
            Gruppe, so daß \(\pi\) ein Gruppenhomomorphismus abelscher Gruppen
            wird.
        \item
            Da \(\pi((x + a) (y + a')) = \pi(x y + x a' + a y + a a')
            = \pi(x y)\) für \(x, y \in A\) und
            \(a, a' \in \ideal a\), gibt es auf \(A/\ideal a\) genau eine
            Multiplikation, welche von \(\pi\) respektiert wird,
            nämlich \((x + \ideal a) (y + \ideal a) = xy + \ideal a\).
        \item
            Schließlich muß die Eins in \(A/\ideal a\) das Bild von \(1 \in A\)
            unter \(\pi\) sein.
        \item
            Da \(\pi\) surjektiv ist, folgen die Ringaxiome für \(A/\ideal a\)
            aus denen für \(A\).
            \qedhere
        \end{enumerate} 
    \end{proof}
\end{frame}

\begin{frame}{Kern des kanonischen Homomorphismus}
	Sei \(A\) ein Ring.
	\begin{example}<+->
		Seien \(\ideal a\) ein Ideal und \(\pi\colon
		A \surjto A/\ideal a\) der kanonische Homomorphismus. Dann ist
		\(\ker \pi = \ideal a\).
	\end{example}
	\begin{example}<+->
		Ist \(x \in A\) ein Element, so ist \(x = 0\) genau dann, wenn der
		kanonische Homomorphismus \(A \surjto A/(x)\) injektiv ist.
	\end{example}
\end{frame}

\begin{frame}{Ideale in Quotientenringen}
    \begin{visibleenv}<+->
        Seien \(\ideal a\) ein Ideal in einem Ring \(A\) und
        \(\pi\colon A \surjto A/\ideal a\) der kanonische Homomorphismus.
    \end{visibleenv}
    \begin{definition}<.->
        Der Ring \(A/\ideal a\) heißt der \emph{Quotientenring von \(A\) nach
        \(\ideal a\)}.
    \end{definition}
    \begin{proposition}<+->
        Durch \(\ideal x = \pi^{-1}(\bar{\ideal x})\) wird eine bijektive
        ordnungserhaltende Korrespondenz zwischen den
        Idealen \(\ideal x\) von \(A\) mit \(\ideal x \supset \ideal a\) und den
        Idealen \(\bar{\ideal x}\) von \(A/\ideal a\) gegeben.
    \end{proposition}
    \begin{proof}<+->
        \begin{enumerate}[<+->]
        \item<.->
            Ist \(\ideal y\) ein Ideal von \(A\), so ist
            \(\bar{\ideal y} \coloneqq \pi(\ideal y)\) ein Ideal von
            \(A/\ideal a\), da \(\pi\) surjektiv ist.
        \item
            Es ist \(\pi^{-1}(\bar {\ideal y}) = \ideal y + \ideal a\).
        \qedhere
        \end{enumerate}
    \end{proof}
\end{frame}

\begin{frame}{Der Homomorphiesatz für Ringe}
    \begin{proposition}<+->[Homomorphiesatz für Ringe]
        Sei \(\phi\colon A \to B\) ein Homomorphismus von Ringen. Dann gibt es
        einen Ringisomorphismus
        \(\underline\phi\colon A/\ker \phi \to \phi(A),\ [x] \mapsto \phi(x)\).
    \end{proposition}
    \begin{proof}<+->
        \begin{enumerate}
        \item<.->
            Nach dem Homomorphiesatz für abelsche Gruppen wird
            durch \([x] \mapsto \phi(x)\) ein Gruppenisomorphismus definiert.
        \item
            Da die kanonische Abbildung \(\pi\colon A \to A/\ker \phi\) surjektiv ist
            und \(\pi\) und \(\underline\phi \circ \pi = \phi\) Ringhomomorphismen sind,
            ist auch \(\underline\phi\) ein Ringhomomorphismus.
            \qedhere
        \end{enumerate}
    \end{proof}
    \begin{remark}<+->
        Ideale sind also genau diejenigen Teilmengen von Ringen, welche
        Kerne von surjektiven Homomorphismen sind.
    \end{remark}
\end{frame}

\subsection{Bezeichnungen}

\begin{frame}{Bezeichnungen von Idealen und Konventionen für Quotientenringe}
    \begin{convention}<+->
        Ideale bezeichnen wir mit kleinen deutschen Buchstaben \(\ideal a, \ideal b,
        \ideal c, \dots\).
    \end{convention}
    \begin{visibleenv}<+->
        Sei \(\ideal a\) ein Ideal in einem Ring \(A\). Sei \(x \in A\).\\
        Die Nebenklasse \([x] \in A/\ideal a\) von \(x\) bezeichnen wir häufig
        wieder mit \(x\),\\
        eventuell mit dem Zusatz "`in \(A/\ideal a\)"' oder
        "`modulo \(\ideal a\)"'.\\
        Anstelle von \([x] = 0\) sagen wir etwa, daß "`\(x = 0\) in
        \(A/\ideal a\)"'\\
        oder "`\(x = 0\) modulo \(\ideal a\)"'.
    \end{visibleenv}
\end{frame}

