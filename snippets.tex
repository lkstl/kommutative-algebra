
\begin{frame}{Definition von Primidealen}
    Sei \(A\) ein kommutativer Ring und \(\ideal p\) ein Ideal von \(A\).
    \begin{definition}<+->
        Ein Ideal \(\ideal p\) von \(A\) heißt \emph{Primideal}, falls
        \(1 \notin \ideal p\) und falls für alle \(a, b \in A\) aus
        \(a b \in \ideal p\) schon \(a \in \ideal p\) oder \(b \in \ideal p\)
        folgt.
    \end{definition}
    \begin{proposition}<+->
        Das Ideal \(\ideal p\) ist genau dann prim, wenn \(A/\ideal p\) ein 
        Integritätsbereich ist.
        \qed
    \end{proposition}
    \begin{example}<+->
        Das Nullideal von \(A\) ist genau dann prim, wenn \(A\) ein
        Integritätsbereich ist.
    \end{example}
\end{frame}

\begin{frame}{Das Spektrum eines Ringes}
    Sei \(A\) ein kommutativer Ring.
    \begin{definition}<+->
        Das \emph{Spektrum \(\Spec A\) von \(A\)} ist die Menge aller Primideale
        von \(A\).
    \end{definition}
    \begin{visibleenv}<+->
        Fassen wir Primideale als Elemente in \(\Spec A\) auf, nennen wir sie
        auch Punkte von \(\Spec A\).
    \end{visibleenv}
    
\end{frame}
