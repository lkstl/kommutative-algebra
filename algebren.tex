\section{Algebren}

\subsection{Definition von Algebren}

\begin{frame}{Definition einer kommutativen Algebra}
	Sei \(A\) ein kommutativer Ring.
	\begin{definition}<+->
		Eine \emph{kommutative \(A\)-Algebra \(B\)} ist ein kommutativer Ring \(B\) zusammen mit einem
		Ringhomomorphismus \(\phi\colon A \to B\), dem \emph{Strukturmorphismus der Algebra \(B\)}.
	\end{definition}
	\begin{remark}<+->
		Ist \(B\) eine \(A\)-Algebra, so können wir insbesondere die Skalareinschränkung
		\(B^A\) definieren. Damit ist eine Multiplikation mit Elementen aus \(A\) auf der
		\(B\) zugrundeliegenden abelschen Gruppe durch \(a b \coloneqq \phi(a) b \in B\) mit \(a \in A\)
		und \(b \in B\) definiert.
		\\
		Die so definierte Struktur ist kompatibel mit der multiplikativen Struktur auf \(B\).
	\end{remark}
\end{frame}

\begin{frame}{Beispiele von Algebren}
	\begin{example}<+->
		Sei \(K\) ein Körper und \(B \neq 0\) eine kommutative \(K\)-Algebra. Da der Strukturmorphismus in diesem
		Falle injektiv ist, können wir \(K\) kanonisch mit seinem Bild in \(B\) identifizieren.
		\\
		Damit ist eine kommutative Algebra über einem Körper nichts anderes als ein kommutativer Ring, welcher \(K\)
		als Unterring enthält.
	\end{example}
	\begin{example}<+->
		Sei \(B\) ein beliebiger kommutativer Ring. Da genau ein Ringhomomorphismus \(\set Z \to B\) existiert,
		nämlich \(n \mapsto n \cdot 1_B\), wobei \(1_B\) die Eins in \(B\) bezeichnet, wird jeder kommutativer
		Ring auf genau eine Weise zu einer \(\set Z\)-Algebra.
	\end{example}
\end{frame}

\begin{frame}{Algebrenhomomorphismen}
	Sei \(A\) ein kommutativer Ring. Seien \(B, C\) zwei kommutative \(A\)-Algebren. 
	\begin{definition}<+->
		Ein \emph{Homomorphismus
		\(\chi\colon B \to C\) von \(A\)-Algebren} ist ein Ringhomomorphismus \(\chi\colon B \to C\), welcher
		einen Homomorphismus \(\chi\colon B^A \to C^A\) von \(A\)-Moduln induziert.
	\end{definition}
	\begin{visibleenv}<+->
		Ein Ringhomomorphismus \(\chi\colon B \to C\) ist also genau dann ein Homomorphismus von \(A\)-Algebren,
		falls \(\chi(a b) = a \chi(b)\) für alle \(a \in A\) und \(b \in B\).
	\end{visibleenv}
	\begin{remark}<+->
		Seien \(\phi\colon A \to B\) und \(\psi\colon A \to C\) die beiden Strukturhomomorphismen. Ein
		Ringhomomorphismus \(\chi\colon B \to C\) ist genau dann ein Homomorphismus von \(A\)-Algebren,
		falls \(\chi \circ \phi = \psi\).	
	\end{remark}
\end{frame}

\subsection{Endliche Algebren und Algebren endlichen Typs}

\begin{frame}{Endliche Algebren}
	\begin{definition}<+->
		Sei \(A\) ein kommutativer Ring. Eine kommutative \(A\)-Algebra \(B\) heißt eine \emph{endliche \(A\)-Algebra},
		falls \(B^A\) als \(A\)-Modul endlich erzeugt ist.
	\end{definition}
	\begin{visibleenv}<+->
		Es ist \(B\) also genau dann eine endliche \(A\)-Algebra, falls endlich viele Elemente \(b_1, \dotsc, b_n \in B\)
		existieren, so daß jedes andere Element von \(B\) als eine \(A\)-Linearkombination der \(b_i\) geschrieben werden
		kann.
	\end{visibleenv}
\end{frame}

\begin{frame}{Algebren endlichen Typs}
	\begin{definition}<+->
		Sei \(A\) ein kommutativer Ring. Eine kommutative \(A\)-Algebra \(B\) heißt eine \emph{\(A\)-Algebra endlichen Typs},
		falls endlich viele Elemente \(b_1, \dotsc, b_n \in B\) existieren, so daß jedes Element von \(B\) als Polynom
		in den \(b_i\) mit Koeffizienten aus \(A\) geschrieben werden kann.
	\end{definition}
	\begin{visibleenv}<+->
		Es ist \(B\) also genau dann eine \(A\)-Algebra endlichen Typs, falls ein surjektiver \(A\)-Algebrenhomomorphismus von
		einem Polynomring \(A[x_1, \dotsc, x_n]\) auf \(B\) existiert.
	\end{visibleenv}
	\begin{example}<+->
		Ein kommutativer Ring \(B\) heißt \emph{endlich erzeugt}, falls er eine \(\set Z\)-Algebra endlichen Typs ist.
		\\
		Dies ist gleichbedeutend damit, daß endlich viele Elemente \(b_1, \dotsc, b_n\) von \(B\) existieren, so daß jedes
		Element von \(B\) als Polynom in den \(b_i\) mit ganzzahligen Koeffizienten geschrieben werden kann.
	\end{example}
\end{frame}

