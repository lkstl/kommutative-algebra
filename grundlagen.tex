\section{Grundlagen aus den Anfängervorlesungen}

\subsection{Mengen}

\begin{definition}
	Sei \(X\) eine Menge. Eine Relation \(\sim\) auf \(X\) heißt
	\begin{enumerate}
	\item
		\emph{transitiv}, falls aus \(x \sim y\) und \(y \sim z\) auch \(x \sim z\) für \(x, y, z \in X\) folgt,
	\item
		\emph{reflexiv}, falls \(x \sim x\) für \(x \in X\) gilt und
	\item
		\emph{antisymmetrisch}, falls auch \(x \sim y\) und \(y \sim x\) schon \(x = y\) für \(x, y \in X\) folgt.
	\end{enumerate}
\end{definition}

\begin{definition}
	Eine \emph{Halbordnung \(\le\) auf einer Menge \(X\)} ist eine transitive, reflexive und antisymmetrische Relation auf \(X\).
	Eine Menge \((X, \le)\) zusammen mit einer Halbordnung heißt \emph{halbgeordnete Menge}.
\end{definition}

Anstelle von einer Halbordnung wird auch der Ausdruck "`teilweise Ordnung"' benutzt oder auch einfach nur "`Ordnung"'.

\begin{example}
	Sei \((X, \le)\) eine halbgeordnete Menge. Ist \(Z \subset X\) eine Teilmenge, so definiert die Einschränkung von
	\(\le\) auf \(Z\) in kanonischer Weise eine Halbordnung auf \(Z\).
\end{example}

\begin{example}
	Sei \(Y\) eine Menge. Sei \(\mathfrak X\) ein System von Teilmengen von \(Y\), also eine Teilmenge der Potenzmenge von
	\(Y\). Dann ist die Inklusionsrelation \(\subset\) eine Halbordnung auf \(\mathfrak X\). Damit ist jedes System \(\mathfrak X\)
	von Teilmengen in natürlicher Weise eine geordnete Menge.
\end{example}

\begin{definition}
	Sei \((X, \le)\) eine halbgeordnete Menge.
	\begin{enumerate}
	\item
		Ein \emph{größtes Element \(x \in X\)} ist ein Element, so daß \(y \le x\) für alle \(y \in X\).
	\item
		Ein \emph{maximales Element \(x \in X\)} ist ein Element, so daß aus \(x \le y\) schon \(x = y\) für alle \(y \in X\) folgt.
	\item
		Eine \emph{obere Schranke \(x \in X\) einer Teilmenge \(Z \subset X\)} ist ein Element mit \(z \le x\) für alle \(z \in Z\).
	\end{enumerate}
\end{definition}

\begin{remark}
	Größte Elemente sind immer eindeutig und auch maximal. Existiert ein größtes Element, so gibt es keine weiteren maximalen
	Elemente.
	
	Ein größtes Element ist dasselbe wie eine obere Schranke der gesamten Menge.
\end{remark}

\begin{definition}
	Sei \((X, \le)\) eine halbgeordnete Menge.
	\begin{enumerate}
	\item
		Die Menge \(X\) heißt \emph{vollständig geordnet}, falls \(x \le y\) oder \(y \le x\) für alle \(x, y \in X\) gilt.
	\item
		Eine \emph{Kette \(Z\) in \(X\)} ist eine Teilmenge \(Z \subset X\), welche mit der induzierten Halbordnung vollständig
		geordnet ist, das heißt \(x \le y\) oder \(y \le x\) für alle \(x, y \in Z\).
	\end{enumerate}
\end{definition}

\begin{theorem}[Zornsches Lemma]
	Sei \(X\) eine halbgeordnete Menge. Jede Kette in \(X\) besitze eine obere Schranke in \(X\). Dann besitzt \(X\) ein
	maximales Element.
\end{theorem}

\subsection{Topologie}

\begin{definition}
		Sei \(X\) eine Menge. Eine \emph{Topologie auf \(X\)} ist eine Menge von
		Teilmengen von \(X\), den \emph{offenen Mengen} der Topologie, so daß
		\begin{enumerate}
		\item
			endliche Schnitte offener Mengen in \(X\) wieder offen sind (damit ist insbesondere die ganze Menge \(X\) als
			leerer Schnitt offen) und
		\item
			beliebige Vereinigungen offener Mengen in \(X\) wieder offen sind (damit ist insbesondere die leere Menge
			\(\emptyset\) als leere Vereinigung offen).
		\end{enumerate}
		Eine Teilmenge heißt \emph{abgeschlossene Menge} der Topologie, falls sie Komplement einer 
		offenen Menge in \(X\) ist.
\end{definition}

\begin{definition}
		Ein \emph{topologischer Raum} ist eine Menge \(X\) zusammen mit einer Topologie.
		
		Ist \(X\) ein topologischer Raum und ist \(x \in X\), so heißt eine offene Menge \(U\) von \(X\) mit \(x \in U\)
		eine \emph{offene Umgebung von \(x\) in \(X\)}.
		
		Eine \emph{Umgebung von \(x\) in \(X\)} ist eine Teilmenge \(U\) von \(X\), so daß eine offene Umgebung
		\(U'\) von \(x\) in \(X\) mit \(U' \subset U\) existiert. 
\end{definition}

\begin{example}
	Sei \(X\) eine Menge. Die Potenzmenge von \(X\) ist eine Topologie auf \(X\), die \emph{diskrete Topologie
	auf \(X\)}.
\end{example}

\begin{example}
	Sei \(X\) ein topologischer Raum. Sei \(Y\) eine Teilmenge von \(X\). Die Menge der Schnitte von \(Y\) mit den
	offenen Teilmengen von \(X\) ist eine Topologie auf \(Y\). Diese Topologie heißt die \emph{Teilraumtopologie
	von \(Y\) in \(X\)}.
	
	In Zukunft versehen wir \(Y\) immer mit dieser Topologie.
\end{example}

\begin{definition}
	Sei \(X\) eine Menge. Eine Menge \(\mathfrak U\) von Teilmengen von \(X\) heißt
	\emph{Basis einer Topologie auf \(X\)}, falls \(\mathfrak U\) abgeschlossen unter endlichen
	Schnitten ist.
\end{definition}

\begin{proposition}
	Sei \(X\) eine Menge. Ist \(\mathfrak U\) die Basis einer Topologie auf \(X\), so
	ist die Menge aller beliebigen Vereinigungen von Teilmengen in \(\mathfrak U\) in \(X\)
	eine Topologie auf \(X\). Diese Topologie heißt die durch \(\mathfrak U\) erzeugte Topologie.
    \qed
\end{proposition}

\begin{example}
	Seien \(X, Y\) zwei topologische Räumen. Dann ist die Menge aller Teilmengen der Form
	\(U \times V\) von \(X \times Y\), wobei \(U\) offen in \(X\) und \(V\) offen in \(Y\) ist,
	eine Basis einer Topologie auf \(X \times Y\). Die davon erzeugte Topologie auf \(X \times Y\)
	heißt die \emph{Produkttopologie}.
	
	In Zukunft versehen wir \(X \times Y\) immer mit dieser Topologie.
\end{example}

In Verallgemeinerung des vorherigen Beispiels wird definiert:
\begin{example}
	Sei \((X_i)_{i \in I}\) eine Familie topologischer. Dann ist die Menge aller Teilmengen von \(X \coloneqq
	\prod_{i \in I} X_i\) der Form
	\(\prod_{i \in I} U_i\) mit \(U_i \subset X_i\) offen für alle \(i \in I\) und \(U_i = X_i\) für fast alle
	\(i \in I\) die Basis einer Topologie auf \(X \times Y\). Die davon erzeugte Topologie auf \(X\) heißt die
	\emph{Produkttopologie}.
	
	In Zukunft versehen wir \(X\) immer mit dieser Topologie.
\end{example}

\begin{definition}
	Sei \(X\) ein topologischer Raum. Sei \(x \in X\). Eine Familie \(\mathfrak U\) von Teilmengen \(U\) von \(X\)
	mit \(U \ni x\) heißt \emph{Umgebungsbasis von \(x\)}, falls für alle Umgebungen \(V\) von \(x\) in \(X\)
	ein \(U \in \mathfrak U\) mit \(U \subset V\) existiert.
	
	Wir sagen, \(X\) \emph{erfülle das erste Abzählbarkeitsaxiom}, falls jeder Punkt von \(X\) eine abzählbare
	Umgebungsbasis besitzt.
\end{definition}

\begin{definition}
	Ein topologischer Raum \(X\) heißt \emph{hausdorffsch}, falls die Diagonale
	\(\{(x, x) \in X \times X \mid x \in X\}\) in \(X \times X\) abgeschlossen ist.
\end{definition}

\begin{definition}
	Sei \(X\) ein topologischer Raum. Sei \((x_n)_{n \in \set N_0}\) eine Folge in \(X\). Wir sagen, daß
	ein Element \(x \in X\) ein \emph{Grenzwert von \((x_n)\) ist}, geschrieben \(\varinjlim_{n \to \infty}\limits x_n = x\),
	falls für jede Umgebung \(U\) von \(x\) in \(X\) gilt, daß \(x_n \in U\) für \(n \gg 0\).
\end{definition}
Offensichtlich reicht es aus, sich auf Umgebungen einer Umgebungsbasis von \(x\) zu beschränken.

\begin{proposition}
	Sei \(X\) ein Hausdorffraum. Dann besitzt eine Folge höchstens einen Grenzwert.
	\qed
\end{proposition}

\begin{definition}
	Seien \(X, Y\) zwei topologische Räume. Eine Abbildung \(f\colon X \to Y\) heißt \emph{stetig}, falls
	für alle offenen Teilmengen \(V\) von \(Y\) das Urbild \(f^{-1}(U)\) in \(X\) offen ist.
	
	Eine bijektive stetige Abbildung zwischen topologischen Räumen, deren Umkehrung auch stetig ist, heißt
	\emph{Homöomorphismus}.
\end{definition}

\begin{example}
	Sei \(X\) eine Menge. Sei \(Y\) ein topologischer Raum. Versehen wir \(X\) mit der
	diskreten Topologie, so ist jede Abbildung \(f\colon X \to Y\) stetig.
\end{example}

\begin{example}
	Sei \(Y\) ein topologischer Raum. Sei \(Z\) eine Teilmenge von \(Y\). Dann ist eine Abbildung \(f\colon X \to Z\)
	von einem weiteren topologischen Raum \(X\) genau dann stetig, wenn \(f\) als Abbildung nach \(Y\) stetig ist.
\end{example}

\begin{definition}
	Sei \(X\) ein topologischer Raum. Sei \(p\colon X \to Y\) eine surjektive Abbildung in eine Menge \(Y\). Die Menge
	derjenigen Teilmengen \(V\) von \(Y\), so daß \(f^{-1}(V)\) offen in \(X\) ist, ist eine Topologie auf \(Y\), die
	\emph{Quotiententopologie bezüglich \(p\)}.
\end{definition}

\begin{example}
	Sei \(X\) ein topologischer Raum. Sei \(p\colon X \to Y\) eine surjektive Abbildung in eine Menge \(Y\), die wir
	diesbezüglich mit der Quotiententopologie versehen. Dann ist eine Abbildung \(f\colon Y \to Z\) in
	einen weiteren topologischen Raum \(Z\) genau dann stetig, wenn \(f \circ p\colon X \to Z\) stetig ist.
\end{example}

\begin{proposition}
	Ist \(A\) eine beliebige Teilmenge eines topologischen Raumes \(X\), so existiert eine kleinste Teilmenge
	\(\overline A \supset A\) von \(X\), welche abgeschlossen ist, der \emph{topologische Abschluß von \(A\)}.
	\qed
\end{proposition}

\begin{definition}
	Eine Teilmenge \(A\) eines topologischen Raumes \(X\) heißt \emph{dicht in \(X\)}, falls für den topologischen
	Abschluß \(\overline A = X\) gilt.
\end{definition}

