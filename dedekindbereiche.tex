\section{Dedekindsche Bereiche}

\subsection{Charakterisierung Dedekindscher Bereiche}

\begin{frame}{Definition Dedekindscher Bereiche}
	\begin{theorem}<+->
		Für einen eindimensionalen noetherschen Integritätsbereich \(A\) sind folgende
		Aussagen äquivalent:
		\begin{enumerate}[<+->]
		\item
			\(A\) ist ganz abgeschlossen.
		\item
			Jedes Primärideal in \(A\) ist eine Potenz eines Primideals.
		\item
			Für jedes Primideal \(\ideal p \neq (0)\) von \(A\)
			ist der Halm \(A_{\ideal p}\) ein diskreter Bewertungsring.
		\end{enumerate}
	\end{theorem}
	\begin{visibleenv}<+->
		Erfüllt \(A\) die drei äquivalenten Aussagen des Satzes, so heißt \(A\)
		ein \emph{Dedekindscher Bereich}.
	\end{visibleenv}
\end{frame}

\begin{frame}{Beweis der Charakterisierung Dedekindscher Bereiche}
	\begin{proof}<+->
		\begin{enumerate}[<+->]
		\item<.->
			Daß ein Ring ganz abgeschlossen ist, ist eine lokale Eigenschaft. Damit
			folgt die Äquivalenz der ersten mit der dritten Aussage aus den
			Charakterisierungen diskreter Bewertungsringe.
		\item
			Auch die zweite Aussage beschreibt eine lokale Eigenschaft, denn Primärideale
			und Idealpotenzen verhalten sich gut unter Lokalisierung.
			\\
			Damit folgt die Äquivalenz der zweiten mit der dritten Aussage ebenfalls aus den
			Charakterisierungen diskreter Bewertungsringe, denn in noetherschen eindimensionalen
			lokalen Ringen sind alle Ideale \(\ideal a \neq (0)\) Primärideale.
			\qedhere
		\end{enumerate}
	\end{proof}
\end{frame}

\begin{frame}{Eindeutige Faktorisierung in Dedekindschen Bereichen}
	\begin{corollary}<+->
		In einem Dedekindschen Bereich läßt sich jedes nicht verschwindende Ideal als
		eindeutiges Produkt von Primidealen schreiben.
	\end{corollary}
	\begin{proof}<+->
		\begin{enumerate}[<+->]
		\item<.->
			Jeder Dedekindsche Bereich ist ein eindimensionaler noetherscher Integritätsbereich,
			in welchem sich jedes Ideal eindeutig als Produkt von Primäridealen schreiben läßt.
		\item
			Die Primärideale in einem Dedekindschen Bereich sind alle (eindeutige) Potenzen von Primidealen.
			\qedhere
		\end{enumerate}
	\end{proof}
\end{frame}

\subsection{Beispiele Dedekindscher Bereiche}

\begin{frame}{Hauptidealbereiche}
	\begin{example}<+->
		Sei \(A\) ein Hauptidealbereich. Da jedes Ideal offensichtlich
		endlich erzeugt ist, ist \(A\) ein noetherscher Ring.
		\\
		Da in Hauptidealbereichen jedes nicht verschwindende Primideal maximal
		ist, gilt weiter \(\dim A = 1\).
		\\
		Ist \(\ideal p \neq (0)\) ein Primideal, so ist \(A_{\ideal p}\)
		ein lokaler Hauptidealbereich, nach unseren Charakterisierungen
		diskreter Bewertungsbereiche also ein solcher.
		\\
		Damit ist \(A\) ein Dedekindscher Bereich, das heißt Hauptidealbereiche
		sind spezielle Dedekindsche Bereiche.
	\end{example}
\end{frame}

\begin{frame}{Ringe ganzer Zahlen in Zahlkörpern}
	\begin{theorem}<+->
		Sei \(K\) ein Zahlkörper, das heißt eine endliche Erweiterung von
		\(\set Q\). Dann ist der Ring \(A\) der ganzen Zahlen in \(K\), das
		heißt der ganze Abschluß von \(\set Z\) in \(K\), ein Dedekindscher
		Bereich.
	\end{theorem}
	\begin{proof}<+->
		\begin{enumerate}[<+->]
		\item<.->
			Da \(\set Q\) die Charakteristik \(0\) hat, 
			ist \(K\) eine separable Erweiterung von
			\(\set Q\). Damit existiert eine Basis \(v_1, \dotsc, v_n\) von
			\(K\) über \(\set Q\) mit \(A \subset \sum\limits_i \set Z v_i\).
		\item
			Damit ist \(A\) als \(\set Z\)-Modul endlich erzeugt, also
			noethersch. Als ganzer Abschluß in \(K\) ist \(A\) selbst ganz
			abgeschlossen.
		\item
			Es bleibt zu zeigen, daß jedes Primideal \(\ideal p \neq (0)\) von
			\(A\) maximal ist: Zunächst ist \(\set Z \cap \ideal p \neq (0)\),
			da \(A\) ganz über \(\set Z\) ist. Damit ist \(\set Z \cap \ideal p\)
			maximal. Wieder weil \(A\) ganz über \(\set Z\) ist, ist damit
			auch \(\ideal p\) maximal.
			\qedhere
		\end{enumerate}
	\end{proof}
\end{frame}


