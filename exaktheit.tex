\section{Exaktheitseigenschaften des Tensorproduktes}

\subsection{Tensorprodukte und Homomorphismenmoduln}

\begin{frame}{Tensorprodukte und Homomorphismenmoduln}
	Sei \(A\) ein kommutativer Ring. Seien \(M, N, P\) drei \(A\)-Moduln.
	\\
	Sei \(\phi\colon M \otimes N \to P\) eine \(A\)-lineare Abbildung. Diese
	definiert eine \(A\)-lineare Abbildung
	\((\phi N)\colon M \to \Hom(N, P), x \mapsto (y \mapsto \phi(x \otimes y))\).
	\begin{proposition}<+->
		Die Abbildung
		\[
			\Hom(M \otimes N, P) \to \Hom(M, \Hom(N, P)), \phi \mapsto (\phi N)
		\]
		ist ein Isomorphismus von \(A\)-Moduln.
	\end{proposition}
	\begin{proof}<+->
		\begin{enumerate}[<+->]
		\item<.->
			Sei \(\psi\colon M \to \Hom(N, P)\) eine \(A\)-lineare Abbildung.
			\\
			Diese definiert eine \(A\)-lineare Abbildung
			\((N\psi)\colon M \otimes N \to P, x \otimes y \mapsto \psi(x)(y)\).
		\item
			Die Abbildung \(\psi \mapsto (N\psi)\) ist die Umkehrung der
			Abbildung \(\phi \mapsto (\phi N)\).
		\qedhere
		\end{enumerate}
	\end{proof}
\end{frame}

\subsection{Rechtsexaktheit des Tensorproduktes}

\begin{frame}{Rechtsexaktheit des Tensorproduktes}
	\begin{proposition}<+->
		Sei \(A\) ein kommutativer Ring. Sei \(E\colon M' \xrightarrow{\phi} M \xrightarrow{\psi} M'' \to 0\)
		eine exakte Sequenz von \(A\)-Moduln und \(N\) ein weiterer \(A\)-Modul. Dann ist auch die Sequenz
		\(E \otimes N\colon M' \otimes N \xrightarrow{\phi \otimes \id_N} M \otimes N \xrightarrow{\psi \otimes \id_N} M'' \otimes N
		\to 0\) exakt.
	\end{proposition}
	\begin{proof}<+->
		\begin{enumerate}[<+->]
		\item<.->
			Sei \(P\) ein beliebiger \(A\)-Modul. Aus der Exaktheit von \(E\) folgt die Exaktheit
			der Sequenz \(\Hom(E, \Hom(N, P))\).
		\item
			Diese Sequenz ist isomorph zur Sequenz \(\Hom(E \otimes N, P)\), welche damit auch exakt ist.
		\item
			Da \(P\) beliebig ist, folgt die Exaktheit von \(E \otimes N\).
			\qedhere
		\end{enumerate}
	\end{proof}
\end{frame}

\begin{frame}{Rechtsexaktheit adjungierter Funktoren}
	\begin{remark}<+->
		Sei \(A\) ein kommutativer Ring. Sei \(N\) ein \(A\)-Modul. Definieren wir
		\(T(M) \coloneqq M \otimes N\) und \(U(P) \coloneqq \Hom(N, P)\) für \(A\)-Moduln
		\(M\) und \(P\), so haben wir die Existenz eines natürlichen Isomorphismus'
		\[
			\Hom(T(M), P) = \Hom(M, U(P))
		\]
		gezeigt.
		\\
		In der Sprache der Kategorientheorie ist \(T\) damit das Linksadjungierte zu \(U\) und
		das Rechtsadjungierte zu \(T\).
		\\
		Der Beweis der letzten Proposition zeigt allgemeiner, daß jeder Funktor, welcher
		ein linksadjungierter ist, rechtsexakt ist.
		\\
		Entsprechend ist ein Funktor, welcher ein linksadjungierter ist, ein linksexakter.
	\end{remark}
\end{frame}

\begin{frame}{Flache Moduln}
	\begin{remark}<+->
		Sei \(A\) ein kommutativer Ring.
		Sei \(M' \to M \to M''\) eine exakte Sequenz von
		\(A\)-Moduln. Im allgemeinen ist dann das Tensorprodukt \(M' \otimes N \to M \otimes N
		\to M'' \otimes N\) mit einem beliebigen \(A\)-Moduln \(N\) nicht mehr exakt.
	\end{remark}
	\begin{example}<+->
		Sei die exakte Sequenz \(0 \to \set Z \xrightarrow\phi \set Z\) mit \(\phi(x) = 2 x\)
		von \(\set Z\)-Moduln gegeben.
		\\
		Sei weiter der \(\set Z\)-Modul \(N \coloneqq \set Z/(2)\) gegeben.
		Das Tensorprodukt \(0 \to \set Z \otimes N \xrightarrow{\phi \otimes \id_N} \set Z \otimes N\)
		der Sequenz mit \(N\) ist nicht exakt, denn für alle \(x \otimes y \in \set Z \otimes N\) ist
		\((\phi \otimes \id_N)(x \otimes y) = 2x \otimes y = x \otimes 2y = x \otimes 0 = 0\), also
		\(\phi \otimes \id_N = 0\). Allerdings ist \(\set Z \otimes N\) nicht der Nullmodul.
	\end{example}
\end{frame}

\subsection{Flachheit}

\begin{frame}{Flache Moduln}
	Sei \(A\) ein kommutativer Ring. Sei \(N\) ein \(A\)-Modul.
	\begin{definition}<+->
		Der \(A\)-Modul \(N\) heißt \emph{flach}, falls für jede exakte Sequenz \(E\) von \(A\)-Moduln
		auch die tensorierte Sequenz \(E \otimes N\) exakt ist.
	\end{definition}
	\begin{lemma}<+->
		Sei \(\phi \otimes \id_N\colon M' \otimes N \to M \otimes N\) für jede injektive lineare Abbildung
		\(\phi\colon M' \to M\) endlich erzeugter \(A\)-Moduln injektiv. Dann ist auch
		\(\phi \otimes \id_N\) für jede injektive lineare Abbildung \(\phi\colon M' \to M\) beliebiger
		\(A\)-Moduln injektiv.
	\end{lemma}
\end{frame}

\begin{frame}{Beweis des Hilfssatzes}
	\begin{proof}<+->
		\begin{enumerate}[<+->]
		\item<.->
			Sei also \(\phi\colon M' \to M\) eine injektive lineare Abbildung zwischen \(A\)-Moduln. Sei
			\(u = \sum x_i' \otimes y_i \in \ker (\phi \otimes \id_N)\), also \(\sum \phi(x_i') \otimes y_i = 0
			\in M \otimes N\).
		\item
			Sei \(M_0' \subset M'\) der durch die \(x_i'\) erzeugte Untermodul und sei \(u_0 = \sum x_i' \otimes y_i 
			\in M_0' \otimes N\).
		\item
			Es existiert ein endlich erzeugter Untermodul \(M_0 \subset M\) mit \(\phi(M_0') \subset M_0\), so daß
			\(\sum \phi(x_i') \otimes y_i = 0 \in M_0 \otimes N\). Damit ist also \((\phi_0 \otimes \id_N)(u_0) = 0\),
			wobei \(\phi_0 \coloneqq \phi|_{M_0'}\colon M_0' \to M_0\).
		\item 
			Da \(M_0', M_0\) endlich erzeugt sind, folgt damit nach Voraussetzung, daß \(u_0 = 0\), also \(u = 0\).
			\qedhere
		\end{enumerate}
	\end{proof}
\end{frame}

\begin{frame}{Charakterisierungen von Flachheit}
	\begin{proposition}<+->
		Sei \(A\) ein kommutativer Ring. Sei \(N\) ein \(A\)-Modul. Dann sind äquivalent:
		\begin{enumerate}[<+->]
		\item<.->
			Der \(A\)-Modul \(N\) ist flach.
		\item
			Für jede kurze exakte Sequenz \(E\colon 0 \to M' \to M \to M'' \to 0\) von \(A\)-Moduln ist die tensorierte
			Sequenz \(E \otimes N\) exakt.
		\item
			Für jede injektive \(A\)-lineare Abbildung \(\phi\colon M' \to M\) zwischen \(A\)-Moduln ist die
			Abbildung \(\phi \otimes \id_N\colon M' \otimes N \to M \otimes N\) injektiv.
		\item
			Für jede injektive \(A\)-lineare Abbildung \(\phi\colon M' \to M\) zwischen endlich erzeugten \(A\)-Moduln
			ist die Abbildung \(\phi\otimes\id_N\colon M' \otimes N \to M \otimes N\) injektiv.
		\end{enumerate}
	\end{proposition}
\end{frame}

\begin{frame}{Beweis zur Charakterisierung von Flachheit}
	\begin{proof}<+->
		\begin{enumerate}[<+->]
		\item<.->
			Die Äquivalenz der ersten beiden Aussagen folgt aus der Tatsache, daß jede lange exakte Sequenz in kurze
			exakte Sequenzen zerfällt werden kann.
		\item
			Die Äquivalenz der zweiten und dritten Aussage folgt aus der Rechtsexaktheit des Tensorproduktes.
		\item
			Die Äquivalenz der letzten beiden Aussagen folgt aus dem letzten Hilfssatz.
			\qedhere
		\end{enumerate}
	\end{proof}
\end{frame}

\begin{frame}{Flachheit von Skalarerweiterungen}
	\begin{proposition}<+->
		Sei \(\phi\colon A \to B\) ein Homomorphismus kommutativer Ringe. Ist \(M\) ein flacher \(A\)-Modul, so ist
		\(M_B\) ein flacher \(B\)-Modul.
	\end{proposition}
	\begin{proof}<+->
		Dies folgt aus den kanonischen Isomorphismen \(N \otimes_B M_B \cong N^A \otimes_A M\) für \(B\)-Moduln \(N\).
	\end{proof}
\end{frame}

