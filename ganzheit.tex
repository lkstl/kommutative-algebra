\section{Ganzheit}

\subsection{Ganze Elemente}

\begin{frame}{Definition ganzer Elemente}
	\begin{definition}<+->
		Sei \(B\) ein kommutativer Ring. Sei \(A \subset B\) ein Unterring.
		Ein Element \(x \in B\) heißt \emph{ganz über \(A\)}, falls es
		Nullstelle eines normierten Polynoms in \(A[x]\) ist, falls also
		\(x\) eine Gleichung der Form \(x^n + a_1 x^{n - 1} + \dotsb + a_n = 0\)
		mit \(a_i \in A\) erfüllt.
	\end{definition}
	\begin{visibleenv}<+->
		Insbesondere ist jedes Element aus \(A\) ganz über \(A\).
	\end{visibleenv}
	\begin{example}<+->
		Betrachte die Ringerweiterung \(\set Z \subset \set Q\). Sei eine
		rationale Zahl \(x = \frac r s\) mit \(r, s \in \set Z\) und
		\((r, s) = (1)\) ganz über \(\set Z\). Dann existieren
		\(a_i \in \set Z\) mit \(r^n + a_1 r^{n - 1} s + \dotsc + a_n s^n = 0\).
		Damit ist \(s\) Teiler von \(r^n\), also \(s = \pm 1\), also \(x \in
		\set Z\).
	\end{example}
\end{frame}

\begin{frame}{Charakterisierung ganzer Elemente}
	\begin{proposition}<+->
                \label{prop:characterization-integral-elements}
		Sei \(A \subset B\) eine Erweiterung kommutativer Ringe. Für ein
		Element \(x \in B\) sind folgende Aussagen äquivalent:
		\begin{enumerate}[<+->]
		\item
			Das Element \(x\) ist ganz über \(A\).
		\item
			Es ist \(A[x] \subset B\) als \(A\)-Modul endlich erzeugt.
		\item
			Es existiert ein Unterring \(C\) von \(B\) mit \(A[x] \subset C\),
			so daß \(C\) als \(A\)-Modul endlich erzeugt ist.
		\item
			Es existiert ein treuer \(A[x]\)-Modul \(M\), welcher als
			\(A\)-Modul endlich erzeugt ist.
		\end{enumerate}
	\end{proposition}
\end{frame}

\begin{frame}{Beweis zur Charakterisierung ganzer Elemente}
	\begin{proof}<+->
		\begin{enumerate}[<+->]
		\item<.->
			Erfüllt \(x\) die Gleichung \(x^n = - a_1 x^{n - 1} - \dotsb - 
			a_n\) mit \(a_i \in A\), so ist \(A[x]\) als \(A\)-Modul von
			\(1, x, \dotsc, x^{n - 1}\) erzeugt.
		\item
			Ist \(A[x]\) als \(A\)-Modul endlich erzeugt, so ist insbesondere
			\(C = A[x]\) ein Unterring von \(B\), welcher \(A[x]\)
			umfaßt und als \(A\)-Modul endlich erzeugt ist.
		\item
			Ist \(C \subset B\) ein \(A[x]\) umfassender Unterring, welcher
			als \(A\)-Modul endlich erzeugt ist, so ist insbesondere \(M = C\)
			ein treuer \(A[x]\)-Modul, denn aus \(y C = 0\) folgt
			insbesondere \(y = y \cdot 1 = 0\), und \(M\) ist als \(A\)-Modul
			endlich erzeugt.
		\item
			Sei \(M\) ein treuer \(A[x]\)-Modul, welcher als \(A\)-Modul
			endlich erzeugt ist. Sei \(\phi\colon M \to M, m \mapsto x m\).
			Da \(M\) als \(A\)-Modul endlich erzeugt ist, existieren
			\(a_i \in A\) mit \(\phi^n + a_1 \phi^{n -1} + \dotsc + a_n = 0\).
			Da \(M\) ein treuer \(A[x]\)-Modul ist, folgt daraus
			\(x^n + a_1 x^{n - 1} + \dotsc + a_n = 0\).
			\qedhere
		\end{enumerate}
	\end{proof}
\end{frame}

\begin{frame}{Von endlich vielen ganzen Elementen erzeugte Unterringe}
	\begin{corollary}<+->
		Sei \(A \subset B\) eine Erweiterung kommutativer Ringe. Seien
		\(x_1, \dotsc, x_n \in B\), welche jeweils ganz über \(A\) sind. Dann
		ist \(A[x_1, \dotsc, x_n]\) ein endlich erzeugter \(A\)-Modul.
	\end{corollary}
	\begin{proof}<+->
		\begin{enumerate}[<+->]
		\item<.->
			Der Fall \(n = 1\) ist ein Spezialfall der Proposition.
		\item
			Sei also \(n > 1\). Wir setzen
			\(A_r \coloneqq A[x_1, \dotsc, x_r]\). Nach Induktionsvoraussetzung
			ist \(A_{n - 1}\) als \(A\)-Modul endlich erzeugt.
		\item
			Da \(x_n\) insbesondere ganz über \(A_{n - 1}\) ist, ist
			\(A_n\) als \(A_{n - 1}\)-Modul endlich erzeugt.
		\item
			Es folgt, daß \(A_n\) auch als \(A\)-Modul endlich erzeugt ist.
			\qedhere
		\end{enumerate}
	\end{proof}
\end{frame}

\begin{frame}{Der Unterring der ganzen Elemente}
	\begin{corollary}<+->
		Sei \(A \subset B\) eine Erweiterung kommutativer Ringe. Dann ist
		die Menge \(C\) der über \(A\) ganzen Elemente von \(B\) ein Unterring
		von \(B\), welcher \(A\) enthält.
	\end{corollary}
	\begin{proof}<+->
		Seien \(x, y \in C\). Dann ist nach der letzten Folgerung \(A[x, y]\)
		ein endlich erzeugter \(A\)-Modul. Damit sind auch
		\(xy, x + y \in A[x, y]\) nach der dritten Aussage der Proposition
		ganz über \(A\). 
	\end{proof}
\end{frame}

\subsection{Ganzheit}

\begin{frame}{Ganzer Abschluß}
	\begin{definition}<+->
		Sei \(A \subset B\) eine Erweiterung kommutativer Ringe.
		\begin{enumerate}[<+->]
		\item<.->
			Der Unterring \(C\) aller über \(A\) ganzen Elemente von \(B\) heißt der
			\emph{ganze Abschluß von \(A\) in \(B\)}.
		\item
			Ist \(C = A\), so heißt \(A\) \emph{ganz abgeschlossen in \(B\)}.
		\item
			Ist \(C = B\), so heißt \(B\) \emph{ganz über \(A\)}.
		\end{enumerate}
	\end{definition}
	\begin{example}<+->
		Der Ring \(\set Z\) ist in \(\set Q\) ganz abgeschlossen.
	\end{example}
\end{frame}

\begin{frame}{Ganze Ringhomomorphismen}
	\begin{definition}<+->
		Sei \(\phi\colon A \to B\) ein Homomorphismus kommutativer Ringe, so daß
		\(B\) zu einer \(A\)-Algebra wird. Dann heißt \(\phi\) ganz und \(B\)
		eine \emph{ganze \(A\)-Algebra}, falls \(B\) ganz über dem Unterring
		\(\phi(A)\) ist.
	\end{definition}
	\begin{visibleenv}<+->
		Damit haben wir oben also gezeigt: Eine \(A\)-Algebra \(B\) ist genau
		dann endlich über \(A\), wenn sie endlich erzeugt und ganz über \(A\)
		ist.
	\end{visibleenv}	
\end{frame}

\begin{frame}{Transitivität der ganzen Abhängigkeit}
	\begin{corollary}<+->
		Seien \(A \subset B \subset C\) Erweiterungen kommutativer Ringe. 
		Ist dann \(B\) ganz über \(A\) und \(C\) ganz über \(B\), so ist
		auch \(C\) ganz über \(A\).
	\end{corollary}
	\begin{proof}<+->
		\begin{enumerate}[<+->]
		\item<.->
			Ist \(x \in C\), so existieren \(b_i \in B\) mit
			\(x^n + b_1 x^{n - 1} + \dotsb + b_n = 0\). Da die
			\(b_i\) ganz über \(A\) sind, ist \(B' = A[b_1, \dotsc, b_n]\)
			ein endlich erzeugter \(A\)-Modul.
		\item
			Da \(x\) ganz über \(B'\) ist, ist \(B'[x]\) ein endlich erzeugter
			\(B\)-Modul.
		\item
			Es folgt, daß \(B'[x]\) auch als \(A\)-Modul endlich erzeugt ist,
			so daß \(x\) nach der dritten Charakterisierung der Proposition
			endlich über \(A\) ist.
			\qedhere
		\end{enumerate}
	\end{proof}
\end{frame}

\begin{frame}{Ganze Abgeschlossenheit des ganzen Abschlusses}
	\begin{corollary}<+->
		Sei \(A \subset B\) eine Erweiterung kommutativer Ringe. Sei \(C\) der
		ganze Abschluß von \(A\) in \(B\). Dann ist \(C\) in \(B\) ganz
		abgeschlossen.
	\end{corollary}
	\begin{proof}<+->
		Sei \(x \in B\) ganz über \(C\). Da ganze Abhängigkeit transitiv ist,
		ist \(x\) damit auch ganz über \(A\). Damit ist \(x \in C\).
	\end{proof}
\end{frame}

\begin{frame}{Ganzheit in Quotienten und Lokalisierungen}
	\begin{proposition}<+->
		Sei \(A \subset B\) eine ganze Erweiterung kommutativer Ringe. Dann
		gilt:
		\begin{enumerate}[<+->]
		\item<.->
			Für jedes Ideal \(\ideal b\) von \(B\) ist \(B/\ideal b\) ganz über
			\(A/(A \cap \ideal b)\).
		\item
			Für jede multiplikativ abgeschlossene Teilmenge \(S \subset A\) ist
			\(S^{-1} B\) ganz über \(S^{-1} A\).
		\end{enumerate}
	\end{proposition}
	\begin{proof}<+->
		\begin{enumerate}[<+->]
		\item<.->
			Eine Gleichung \(x^n + a_1 x^{n - 1} + \dotsb + a_n = 0\) mit
			\(a_i \in A\) für ein \(x \in B\) können wir modulo \(\ideal b\)
			reduzieren.
		\item
			Sei \(\frac x s \in S^{-1} B\). Dann liefert die obige Gleichung
			\((\frac x s)^n + \frac{a_i} s (\frac x s)^{n - 1} + \dotsb +
			\frac{a_n} {s^n} = 0\)
			eine Ganzheitsbedingung über \(S^{-1} A\).
			\qedhere
		\end{enumerate}
	\end{proof}
\end{frame}

\subsection{Noethersche Normalisierung}

\begin{frame}{Ein schrecklicher Hilfssatz}
	\begin{lemma}<+->
		Sei \(M\) eine endliche Menge von Tupeln \(m = (m_1, \dotsc, m_n)
		\in \set N_0^n\). Dann existieren natürliche Zahlen \(w_1, \dotsc,
		w_n \in \set N\) mit \(w_n = 1\), so daß für alle
		\(m, m' \in M\) gilt: \(m \neq m' \implies
		w(m) \coloneqq
		\sum\limits_{i = 1}^n w_i m_i \neq
		w(m') = \sum\limits_{i = 1}^n w_i m_i\).
	\end{lemma}
	\begin{proof}<+->
		\begin{enumerate}[<+->]
		\item<.->
			Wir führen den Beweis nach Induktion über \(n\). Wir können
			\(n > 1\) annehmen. Ein \(m \in M\) schreiben wir als
			\(m = (m_1, m_{\ge 2})\) mit \(m_{\ge 2} = (m_2, \dotsc, m_n)\).
		\item
			Wir wenden die Induktionsvoraussetzung auf die vorkommenden
			\(m_{\ge 2}\) an und erhalten \(w_2, \dotsc, w_n \in \set N\) mit
			\(w_n = 1\).
		\item
			Wir wählen \(w_1 \in \set N\) so, daß
			\(w_1 > \sum\limits_{i = 2}^n w_i m_i\) für alle \(m \in M\).
			\qedhere
		\end{enumerate}
	\end{proof}
\end{frame}

\begin{frame}{Hilfssatz zur Noetherschen Normalisierung}
	\begin{lemma}<+->
		Sei \(K\) ein Körper. Sei \(f \in A \coloneqq K[x_1, \dotsc, x_n]\) mit
		\(f \neq 0\).
		Dann existiert ein Ringautomorphismus \(\phi\colon A \to A\) mit
		\(\phi(x_n) = x_n\), so daß
		\(
		\phi(f) = a_0 x_n^k + a_1 x_n^{k - 1} + \dotsb + a_k\)
		mit \(a_0 \in K^\units\) und \(a_1, \dotsc, a_k \in K[x_1, \dotsc, x_{n - 1}]\).
	\end{lemma}
\end{frame}

\begin{frame}{Beweis des Hilfssatzes}
	\begin{proof}<+->
		\begin{enumerate}[<+->]
		\item<.->
			Es gibt eine endliche Menge \(M\) von Tupeln \(m = (m_1, \dotsc,
			m_n) \in \set N_0^n\), so daß \(f = \sum\limits_{m \in M}
			a_m x_1^{m_1} \dotsm x_n^{m_n}\) mit \(a_m \in K^\units\). Wir
			wählen \(w_1, \dotsc, w_n \in \set N\) zu \(M\) wie im letzten
			Hilfssatz.
		\item
			Definiere \(\phi\colon A \to A\) mit \(\phi(x_i) = x_i + x_n^{w_i}\)
			für \(i < n\). Damit ist
			\(\phi(f) = f(x_1 + x_n^{w_1},
			\dotsc, x_{n - 1} + x_n^{w_{n - 1}}, x_n)\).
		\item
			Sei \(m \in M\) dasjenige Tupel, so daß \(\sum\limits_{i = 1}^n
			w_i m_i\) maximal wird. Dann ist der Term maximalen Grades in
			\(x_n\) in \(\phi(f)\) durch \(a_m x_n^{\sum w_i m_i}\) gegeben,
			und es ist \(a_m \in K^\units\).
		\qedhere
		\end{enumerate}
	\end{proof}
\end{frame}

\begin{frame}{Noethersche Normalisierung}
	\begin{proposition}<+->
		\label{prop:noether_norm}
		Seien \(K\) ein Körper und \(\ideal a \neq (1)\) ein Ideal in \(A \coloneqq K[x_1, \dotsc, x_n]\).
		Dann existieren ein Polynomring \(B \coloneqq K[y_1, \dotsc, y_n]\) und ein endlicher, injektiver
		Homomorphismus \(B \to A\) kommutativer \(K\)-Algebren und ein \(0 \leq r \leq n\), so daß
		\(B \cap \ideal a = (y_{r + 1}, \dotsc, y_{n-1})\).
		\\
		Insbesondere folgt, daß \(K[y_1, \dotsc, y_r] \to A/\ideal a\) ein endlicher, injektiver Homomorphismus
		von \(K\)-Algebren ist.
	\end{proposition}
\end{frame}

\begin{frame}{Beweis der noetherschen Normalisierung}
	\begin{proof}<+->
		\begin{enumerate}[<+->]
		\item<.->
			Wir wenden Induktion über \(n\) an. Wir können \(n \ge 1\) und \(\ideal a \neq (0)\) annehmen.
		\item
			Sei \(f \in \ideal a \setminus \{0\}\). Nach dem Hilfssatz können wir
			davon ausgehen, daß \(f = 0\) eine
			Ganzheitsbedingung für \(x_n\) über \(A' \coloneqq K[x_1, \dotsc, x_{n - 1}]\) ist.
		\item
			Nach Induktionsvoraussetzung (angewendet auf~$A'$ und das Ideal~$A' \cap \ideal a$) existiert ein endlicher, injektiver Homomorphismus \(B' \coloneqq K[y_1, \dotsc,
			y_{n - 1}] \to A'\), so daß \(B' \cap \ideal a = (y_{r + 1}, \dotsc, y_n)\) für ein \(r\). 
		\item
			Schließlich setzen wir \(B = B'[y_n] \to A = A'[x_n], y_n \mapsto f\).
			\qedhere
		\end{enumerate}
	\end{proof}
\end{frame}


