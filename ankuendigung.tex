\documentclass{algposter}

\title[Kommutative Algebra]{Vorlesung Kommutative Algebra im Wintersemester 2015/16}
\author{Marc Nieper-Wißkirchen}
\date{Wintersemester 2015/16}

\begin{document}

\begin{frame}{}
	\begin{columns}[t]
		\begin{column}{.475\linewidth}
			\begin{block}{Vorlesung Kommutative Algebra}
				\begin{description}
				\item[Dozent] Prof.~Marc Nieper-Wißkirchen
				\item[Übungsleitung] Ingo Blechschmidt, M.Sc.
				\item[Dauer] 4 SWS Vorlesung + 2 SWS Übungen
				\item[Studiengänge]
					Bachelor Mathematik, Lehramt Gymnasium, Master Mathematik
				\item[Vorkenntnisse]
					\emph{Lineare Algebra I und II}.
					\emph{Einführung in die Algebra}
				\item[Verwendbarkeit]
					Voraussetzung für Algebra-Vorlesungen im Master Mathematik
				\end{description}
			\end{block}
			\begin{block}{Zeit und Ort}
				Die Vorlesung findet im Wintersemester zu folgenden Zeiten statt:
				\begin{itemize}
				\item
					Mittwoch, 8:15--9:45, in 1009/L
				\item
					Freitag, 8:15--9:45, in 1009/L
				\end{itemize}
				Die Übung findet Mittwoch, 15:45--17:15 in 1009/L statt.
			\end{block}
			\begin{block}{Klausur}
				Die Klausur wird über zwei Zeitstunden gehen und gegen Ende des
				Wintersemesters in der vorlesungsfreien Zeit stattfinden.
			\end{block}
		\end{column}
		\begin{column}{.475\linewidth}
			\begin{block}{Inhalt}
				Inhalt der Vorlesung ist die \emph{kommutative Algebra}, also das
				Studium
				kommutativer Ringe und ihrer Moduln. Prominente Beispiele für
				kommutative Ring sind die Koordinatenringe algebraischer Varietäten
				wie zum Beispiel \(\set C[x, y, z]/(xy - z^2)\) oder die 
				Ganzheitsringe algebraischer Zahlkörper wie zum Beispiel 
				\(\set Z[i]\).
				Unter anderem werden folgende Themen in der Vorlesung behandelt:
				\begin{itemize}
				\item Moduln als stetige Familien von Vektorräumen
				\item Primideale als Verallgemeinerung von Primzahlen und 
				irreduziblen Komponenten algebraischer Varietäten
				\item Primärzerlegung von Idealen als Verallgemeinerung der
				Primfaktorzerlegung
				\item Existenz von Lösungen polynomieller Gleichungssysteme
				\item Rechnen mit Potenzreihen
				\item Was ist die Dimension eines Ringes?
				\end{itemize}
			\end{block}
			\begin{block}{Literatur}
				\begin{itemize}
				\item M.~Atiyah. I.~G.~MacDonald: \emph{Introduction to Commutative
					Algebra}
				\item M.~Nieper-Wißkirchen: \emph{Kommutative Algebra} (Skript zur Vorlesung)
				\end{itemize}
			\end{block}
		\end{column}
	\end{columns}
	\vfill
\end{frame}

\end{document}

