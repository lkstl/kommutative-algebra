\section{Nullteiler, nilpotente Elemente und Einheiten}

\subsection{Integritätsbereiche}

\begin{frame}{Reguläre Elemente und Nullteiler}
    \begin{visibleenv}<+->
        Sei \(x\) ein Element eines kommutativen Ringes \(A\).
    \end{visibleenv}
    \begin{definition}<.->
        \begin{enumerate}[<+->]
        \item<.->
            Das Element \(x\) heißt \emph{regulär}, falls für alle \(y \in A\)
            aus \(x y = 0\) schon \(y = 0\) folgt.
        \item
            Ein Element \(x\) ist ein \emph{Nullteiler}, wenn es nicht regulär ist.
        \item
            Der Ring \(A\) heißt ein \emph{Integritätsbereich}, falls \{0\} der
            einzige Nullteiler in \(A\) ist.
        \end{enumerate}
    \end{definition}
    \begin{visibleenv}<+->
        Das Element \(x\) ist also genau dann ein Nullteiler, falls ein
        \(y \in A\) mit \(y \neq 0\), aber \(x y = 0\) existiert.
    \end{visibleenv}
    \\
    \begin{visibleenv}<+->
        Der Ring \(A\) ist weiter genau dann ein Integritätsbereich,\\
        wenn \(0 \neq 1\)
        in \(A\)\\ und aus \(x y = 0\) für \(x, y \in A\) schon \(x = 0\) oder
        \(y = 0\) folgt.
    \end{visibleenv}
\end{frame}

\begin{frame}{Unterringe von Integritätsbereichen}
	\begin{proposition}<+->
		Sei \(\phi\colon A \to B\) ein injektiver Homomorphismus kommutativer Ringe.
		Ist dann \(B\) ein Integritätsbereich, so auch \(A\).
	\end{proposition}
	\begin{proof}<+->
		Sei also \(B\) ein Integritätsbereich.
		Da \(0 \neq 1\) in \(B\), muß wegen \(\phi(0) = 0\) und \(\phi(1) = 1\)
		auch \(0 \neq 1\) in \(A\) gelten.
		Seien weiter \(f, g \in A\) mit \(f g = 0\) gegeben. Es folgt
		\(\phi(fg) = 0\) in \(B\). Damit ist \(\phi(f) = 0\) oder \(\phi(g) = 0\).
		Da \(\phi\) injektiv ist, folgt dann \(f = 0\) oder \(g = 0\).
	\end{proof}
\end{frame}

\begin{frame}{Beispiele zu Integritätsbereichen}
    \begin{example}<+->
        Der Nullring ist kein Integritätsbereich, denn die Null ist im Nullring 
        regulär.  
    \end{example}
    \begin{example}<+->
        Der Ring \(\set Z\) der ganzen Zahlen ist ein Integritätsbereich.
    \end{example}
    \begin{visibleenv}<+->
        Sei \(K\) ein Körper.
    \end{visibleenv}
    \begin{example}<.->
        Der Polynomring \(K[x_1, \dotsc, x_n]\) in \(n\) Variablen über einem Körper
        \(K\) ist ein Integritätsbereich.
    \end{example}
    \begin{example}<+->
        Der Ring \(A \coloneqq K[x, y]/(x y)\) ist kein Integritätsbereich, da zum
        Beispiel \(x\) und \(y\) Nullteiler in \(A\) sind.
    \end{example}
\end{frame}

\begin{frame}{Hauptidealbereiche}
    \begin{definition}<+->
        \begin{enumerate}[<+->]
        \item<.->
            Ein Ideal \(\ideal a\) in einem kommutativen Ring \(A\) heißt
            \emph{Hauptideal}, falls \(\ideal a = (a)\) für ein \(a \in A\).
        \item
            Ein Integritätsbereich \(A\) heißt \emph{Hauptidealbereich}, falls 
            jedes Ideal in \(A\) ein Hauptideal ist.
        \end{enumerate}
    \end{definition}
    \begin{visibleenv}<+->
        Sei \(K\) ein Körper.
    \end{visibleenv}
    \begin{example}<+->
        Die Ringe \(\set Z\) und \(K[x]\) sind Hauptidealbereiche.
    \end{example}
    \begin{remark}<+->
        Für \(n \ge 2\) ist \(K[x_1, \dotsc, x_n]\) kein
        Hauptidealbereich. Das Ideal \((x_1, \dotsc, x_n)\)
        läßt sich nicht von weniger als \(n\) Elementen erzeugen.
    \end{remark}
\end{frame}

\subsection{Nilpotente Elemente}

\begin{frame}{Nilpotente Elemente}
    \begin{definition}<+->
        Ein Element \(x \in A\) eines kommutativen Ringes \(A\) heißt
        \emph{nilpotent}, falls ein \(n \in \set N_0\)
        mit \(x^n = 0\) existiert.
    \end{definition}
    \begin{example}<+->
        In jedem kommutativen Ring ist \(0\) ein nilpotentes Element.
    \end{example}
    \begin{proposition}<+->
        Ist \(A\) ein Integritätsbereich, so ist \(0\) das einzige nilpotente Element
        von \(A\).
    \end{proposition}
    \begin{proof}<+->
        Sei \(x \in A\) nilpotent und \(n \in \set N_0\) die kleinste natürliche Zahl
        mit \(x^n = 0\). Dann ist \(n \ge 1\). Aus \(x \cdot x^{n - 1} = 0\)
        und \(x^{n - 1} \neq 0\)
        folgt dann \(x = 0\).
    \end{proof}
\end{frame}

\begin{frame}{Beispiele für nilpotente Elemente}
    \begin{example}<+->
        Sei \(K\) ein Körper. Im Ring \(K[x]/(x^2)\) ist \(x\) ein nilpotentes
        Element.
    \end{example}
    \begin{example}<+->
        Sei \(\phi\colon A \to B\) ein Homomorphismus von Ringen. Sei \(a\) ein
        nilpotentes Element von \(A\). Dann ist \(\phi(a)\) ein nilpotentes Element
        von \(B\).
    \end{example}
\end{frame}

\subsection{Einheiten}

\begin{frame}{Einheiten}
    \begin{visibleenv}<+->
        Sei \(x \in A\) ein Element eines Ringes.
    \end{visibleenv}
    \begin{definition}<+->
        \begin{enumerate}[<+->]
        \item<.->
            Das Element \(x\) heißt \emph{Einheit in \(A\)}, falls \(x\) 
            Einheit des multiplikativen Monoides von \(A\) ist.
        \item<+->
            Die Untergruppe \(A^\units\) im multiplikativen Monoid von \(A\), die
            von den Einheiten in \(A\) gebildet wird, heißt die
            \emph{Einheitengruppe von \(A\)}.
        \item
            Der Ring \(A\) heißt \emph{Schiefkörper}, falls \{0\} die einzige
            Nichteinheit in \(A\) ist. Ein kommutativer Schiefkörper heißt
            \emph{Körper}.
        \end{enumerate}       
    \end{definition}
    \begin{visibleenv}<+->
        Das Element \(x\) ist also genau dann eine Einheit, falls ein \(y \in A\) mit
        \(x y = 1 = y x\) existiert, nämlich die \emph{Inverse} \(y = x^{-1}\)
        von \(x\).
    \end{visibleenv}
    \begin{example}<+->
        Der Nullring ist kein Körper, denn die Null ist im Nullring invertierbar.
    \end{example}
\end{frame}

\begin{frame}{Beispiel für Einheiten und Einheitengruppen}
    \begin{example}<+->
        Die Einheitengruppe des Ringes \(\set Z\) der ganzen Zahlen ist
        \(\set Z^\units = \{1, -1\}\). 
        Insbesondere bilden die ganzen Zahlen keinen Körper.
    \end{example}
    \begin{example}<+->
        Die Einheitengruppe des Ringes \(\set Q\) der rationalen Zahlen ist
        \(\set Q^\units = \set Q \setminus \{0\}\). Damit ist \(\set Q\) ein
        Körper.
    \end{example}
    \begin{example}<+->
        Sei \(a \in A\) ein Element eines kommutativen Ringes. In \(A[x]/(xa - 1)\)
        ist \(a\) eine Einheit mit Inverse \(x\).
    \end{example}
    \begin{example}<+->
        Seien \(a, b \in A\) Elemente eines kommutativen Ringes, so daß \(ab\)
        eine Einheit ist.
        Dann ist auch \(a\) eine Einheit mit \(a^{-1} = b (ab)^{-1}\). 
    \end{example}
\end{frame}

\begin{frame}{Beispiele und Propositionen über Einheiten}
    \begin{visibleenv}<+->
        Sei \(x \in A\) ein Element eines kommutativen Ringes \(A\).
    \end{visibleenv}
    \begin{proposition}<+->
        Das Element \(x\) ist genau dann eine Einheit, falls
        \((x) = (1)\).
    \end{proposition}
    \begin{proof}<+->
        Es ist \(1 = x^{-1} \cdot x \in (x)\) und damit \((1) \subset (x)\), also \((1) = (x)\), falls \(x\) Einheit ist.
        \\
        Die umgekehrte Implikation ist ebenso klar.
    \end{proof}
    \begin{proposition}<+->
        Ist \(x\) eine Einheit, ist \(x\) auch regulär.
    \end{proposition}
    \begin{proof}<+->
        Sei \(x \cdot y = 0\) für ein \(y \in A\). Multiplizieren mit \(x^{-1}\) von
        links liefert \(y = 0\).
    \end{proof}
\end{frame}

\subsection{Charakterisierung von Körpern}

\begin{frame}{Ein Lemma über Ideale in Körpern}
    \begin{lemma}<+->
        Sei \(A\) ein Körper. Dann besitzt \(A\) genau zwei Ideale (nämlich
        \((0)\) und \((1)\)).
    \end{lemma}
    \begin{proof}<+->
        Sei \(A\) ein Körper. Da \(0\) keine Einheit ist, ist \((0) \neq (1)\).
        Sei jetzt \(\ideal a \neq (0)\) ein Ideal in \(A\). Dann existiert ein
        \(x \in \ideal a\) mit \(x \neq 0\), also \(x \in A^\units\). Es folgt
        \((1) = (x) \subset \ideal a\), also \(\ideal a = (1)\).
    \end{proof}
\end{frame}

\begin{frame}{Ein Lemma über Ringe mit genau zwei Idealen}
    \begin{lemma}<+->
        Sei \(\phi\colon A \to B\) ein Ringhomomorphismus kommutativer Ringe.
        Besitze \(A\) genau zwei
        Ideale. Dann ist \(\phi\) genau dann injektiv, wenn \(B\) nicht der Nullring
        ist.
    \end{lemma}
    \begin{proof}<+->
        \begin{enumerate}[<+->]
        \item<.->
            Sei \(\ker \phi = (0)\). Dann ist \(\phi\) injektiv. Weiter ist
            \(1 = \phi(1) \neq 0 \in B\), da \(1 \notin (0) \subset A\), also ist
            \(B\) nicht der Nullring.
        \item
            Sei \(\ker \phi = (1)\). Dann ist \(\phi\) nicht injektiv, da
            \((1) \neq (0) \subset A\). Weiter ist \(1 = \phi(1) = 0 \in B\), also ist
            \(B\) der Nullring.
            \qedhere
        \end{enumerate}
    \end{proof}
\end{frame}

\begin{frame}{Ein Lemma über die Charakterisierung von Körpern}
    \begin{lemma}<+->
        Sei \(A\) ein kommutativer Ring, so daß jeder Ringhomomorphismus \(\phi\colon A \to B\)
        in einen weiteren kommutativen Ring \(B\) genau dann injektiv ist, wenn \(B\) nicht der
        Nullring ist. Dann ist \(A\) ein Körper.
    \end{lemma}
    \begin{proof}<+->
        Sei \(x \in A\). Dann gilt:
        \(x \in A^\units \iff A/(x) = 0 \iff \ker(A \surjto A/(x)) \neq 0
        \iff x \neq 0\).
    \end{proof}
\end{frame}

\begin{frame}{Eine Proposition über die Charakterisierung von Körpern}
    \begin{visibleenv}<+->
        Zusammengefaßt haben wir also gezeigt:
    \end{visibleenv}
    \begin{proposition}<.->
        Sei \(A\) ein kommutativer Ring. Dann sind die folgenden Aussagen äquivalent:
        \begin{enumerate}
        \item
            \(A\) ist ein Körper.
        \item
            \(A\) besitzt genau zwei Ideale (nämlich \((0)\) und \((1)\)).
        \item
            Ein Ringhomomorphismus \(A \to B\) in einen kommutativen Ring \(B\) ist genau dann
            injektiv, wenn \(B\) nicht der Nullring ist.
            \qedhere
        \end{enumerate}
    \end{proposition}
\end{frame}

